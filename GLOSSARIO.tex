\makeatletter
\renewenvironment{theindex}{%
   \@mkboth{\large\MakeLowercase{\scshape Glossário}}%
{\MakeLowercase{\large\scshape Manuel Inácio da Silva Alvarenga}}%
  \thispagestyle{plain}%
\parindent\z@
 \parskip\z@ \@plus .3\p@\relax
  \let\item\@idxitem
}{%
  \clearpage
  }
\makeatother

\part{Glossário}

\hedramarkboth{Glossário}{}

\newcommand{\Acaf}{\small\textbf{açafroado}: da cor do açafrão, amarelado}

\newcommand{\Acaso}{\small\textbf{Acaso}: figura alegorizada no poema como filho da
Fortuna, constituindo a verossimilhança da resolução rápida para alguns nós do
enredo cômico}

\newcommand{\Aiur}{\small\textbf{Aiuruoca} (sertão de): região da Capitania de Minas
Gerais, assim chamada até hoje. As referências a ``Ajuruóca'', como está grafado
na primeira edição, e aos papagaios foram valorizadas como paisagem brasileira
pelas críticas românticas e modernistas, entendida como incorporação, no poema
pombalino, de elementos da região em que o autor passou os primeiros anos.  Em
primeiro lugar, as referências a esses elementos locais, como também ao jaguar e
ao jacaré, são mediadas pelos livros que tratam matérias dessa natureza, e não
pela simulação de uma experiência direta. Além disso, no entrecho, tanto o
``Certaõ de Ajuruóca'' quanto os papagaios são elementos de um símile de
caráter cômico, baixo. Essa última circunstância mesma já bastaria para afastar
a interpretação que imputa sentimento nacional pela valorização da cor local, no
poema}

\newcommand{\Alcid}{\small\textbf{Alcides}: outro nome de Hércules; no poema, ``um copo digno de Alcides''
comicamente inverte a matéria heroica da referência mitológica; usando a tópica
comparação com o grande semideus grego em discurso de louvor, ``digno de Alcides''
representa o excesso de bebida no copo, o que comicamente definia parte do caráter
dos tipos irrisórios que o poema herói-cômico põe em cena}

\newcommand{\Amor}{\small\textbf{Amor}: o Cupido, famoso por ter em sua aljava  flechas de ouro e de prata,
que correspondem ao amor correspondido e ao amor desprezado}

\newcommand{\Anac}{\small\textbf{anacoreta}: monge eremita cristão. No poema, a Ignorância,
já travestida de Tibúrcio, faz seu personagem passar por um Anacoreta
para ganhar o crédito com o velho Amaro, que se apresenta como um
crédulo medroso de coisas sobrenaturais}

\newcommand{\Anfit}{\small\textbf{Anfitrite} (campos de Anfitrite): divindade marinha, esposa de Posídon e irmã de Tétis.
Campos de Anfitrite, isto é, o mar. Na imagem do poema: o sol já se tinha
posto detrás do mar}

\newcommand{\Aque}{\small\textbf{Aqueronte}: rio que na mitologia ficava à entrada do Hades, o reino dos mortos;
foi apropriado na \textit{Divina comédia} e em outros textos para a representação
da entrada do Inferno}

\newcommand{\Aquil}{\small\textbf{Aquilon} e \textbf{Austro}: os ventos setentrional e meridional, respectivamente,
representados mitologicamente conforme a convenção}

\newcommand{\Arist}{\small\textbf{Aristóteles} (século \textsc{iv} a.C.): filósofo grego mais importante para muitas doutrinas de
autoridades sapienciais do Catolicismo romano. Em alguns momentos, como nos séculos
em que a Companhia de Jesus prosperou em muita parte, era reconhecido como ``o Filósofo'',
às vezes como ``divino Aristóteles''. Sobre ele,  costumou"-se dizer que era,
filho de médico, nascido em Estagira em 348 a.C. Depois de haver ficado
vinte anos ao lado de Platão, fundou o Liceu onde passa a ensinar filosofia a
cidadãos gregos. Mesmo com a remoção do ensino jesuítico em todas as partes do reino,
com a política antijesuítica de Sebastião José de Carvalho e Mello, que pôs em descrédito
o termo ``peripatético'', método associado a velhos hábitos jesuíticos, até o final do século \textsc{xviii}
é recorrente o seu nome como \textit{autoridade} para as diversas artes e ciências (Ver \textbf{peripatéticos})}

\newcommand{\Aug}{\small\textbf{Augusto} (século \textsc{i} a.C.): título honorífico dado pelo senado romano a Otávio,
nomeadamente o primeiro imperador de Roma, sobrinho e filho adotivo de Júlio César}

\newcommand{\Batr}{\textit{\textbf{Batrachomyomachia}}: isto é, \textit{A batalha dos Sapos e do Rato}, é uma épica
burlesca, inventada com matéria baixa, referida na \textit{Poética} de Aristóteles.
Atribuída por alguns a Homero e por outros a Pigres, é conhecida
por parodiar inúmeras passagens da \textit{Ilíada}. A \textit{Batrachomyomachia} é aqui, portanto,
a primeira autoridade da espécie poética de que trata o discurso de Silva Alvarenga (Ver \textbf{Poéticas})}

\newcommand{\Bexig}{\small\textbf{bexigas}: nome usual para a varíola, pelo seu sintoma mais
típico, a irrupção de bolsas purulentas na pele; tem caráter muito
letal e contagioso, e deixa as cicatrizes na pele a quem sobrevive
à doença}

\newcommand{\Borea}{\small\textbf{Bóreas}: figura mitológica que representa um dos ventos
filhos de Éolo com a Aurora; ``o Norte fresco'', como é mencionado no
início do canto \textsc{ii}, é também famoso por trazer tempestades}

\newcommand{\Canic}{\small\textbf{Canícula}: como diz a nota de Silva Alvarenga, nome de constelação;
devido à sua posição, o seu aparecimento no céu, está associado ao período
de calor mais intenso no hemisfério norte}

\newcommand{\Carro}{\small\textbf{carro do Sol}: lugar comum da representação mitológica do Sol,
puxado por cavalos divinos}

\newcommand{\Casui}{\small\textbf{casuístas}: teólogos que, segundo encadeamento de premissas lógicas, examinam
casos morais da consciência do pecador. Foram postos em descrédito pelo Iluminismo
assim como pela política pombalina na reforma do ensino. Identificados ou não com o casuísmo
\textit{stricto senso},
Diana, Bonacina, Tamburino, Moia, o Sanches português e o Sanchez espanhol, Molina e
Lagarra, além do próprio Concina, que aparece na nota, são nomes de autores em teologia, jurisprudência, política,
moral, ciências e artes, enfim, tratadistas em geral que particularmente vogaram muito
no tempo dos jesuítas, mas que foram desacreditados, uns mais, outros menos, neste
fim de século \textsc{xviii} em que sai \textit{O desertor}. Na sequência do canto \textsc{v},
títulos de livros e outros autores são citados em tom depreciativo, como efeito do
que na província se lia, devido ao antigo ensino que por causa dos jesuítas ainda era
ministrado na Universidade de Coimbra, até a Reforma de 1772}

\newcommand{\Cepa}{\small\textbf{cepa}, filho de: linhagem, filho de nobre casta. No século \textsc{xviii}, 
é uma tópica da poesia
simpótica --- isto é, poesia para o banquete (\textit{symposium}) --- começar
louvando o varão jovem pela sua cepa, pela sua linhagem nas gerações de
varões ilustres das monarquias europeias}

\newcommand{\Choca}{\small\textbf{choça}: casa pobre de gente rústica. Na representação dos tipos
piores, o desconfiado Rodrigo, assim representado como um
rústico, volta com gosto para o seu casebre, porque não temia
a ira da mãe que ele lá encontraria maldizendo a escolha do filho.
Conformado à vida pior, prefere mesmo não receber as distinções que
as letras poderiam oferecer}

\newcommand{\Cipiao}{\small\textbf{Cipião}: alusão a Dom Sebastião, rei de Portugal morto em Alcazar Quibir,
na África, em referência ao general romano Emílio Cipião, o Africano, que conquistou
Cartago, cidade de África tornada célebre em relatos históricos de guerra, que foi
antigamente cabeça de um grande Império na costa de Barbeia, perto
de Túnis}

\newcommand{\Coler}{\small\textbf{colérico}: tipo  de constituição física determinada pela
bílis amarela, segundo os regimes de classificação dessa medicina}

\newcommand{\Comic}{\small\textbf{cômico} (poesia cômica, comédia):
nas poéticas do século \textsc{xviii}, cômico é a qualidade das matérias piores, a natureza
das ações, das paixões e dos costumes dos homens vis, politicamente inferiores e deformados moralmente.
São tidos por cômicos textos fingidos em estilo baixo imitando os tipos piores}

\newcommand{\Culex}{\textit{\textbf{Culex}}: poema atribuído a Virgílio que trata de assuntos inferiores.
Relata a morte de uma mosca por um pastor e o retorno desta para alertá"-lo dos
perigos do inferno. O \textit{Culex} é aí o modelo latino para a poesia épica
de matéria cômica}

\newcommand{\Dedic}{\small\textbf{dedicatória}: parte das obras em que se declaram os protetorados que
as mantêm em relações de favor previstas nas leis e no costume. Em geral, a dedicatória
é posta no princípio dos livros às vezes na forma de cartas dedicatórias, sonetos,
odes, antepostos aos poemas, às vezes como parte dos próprios poemas.
Este último uso, que é o caso de \textit{O desertor}, é respaldado pela dedicatória de
\textit{Os lusíadas} de Camões}

\newcommand{\Elvas}{\small\textbf{Elvas}: cidade nas fronteiras ao sul de Portugal, no Alentejo,
onde a nota do autor refere ter havido batalhas que constituiriam novas
aristocracias após a Restauração da Monarquia portuguesa, que aclama o herdeiro da casa
de Bragança. No poema, Gaspar perdera a espada na briga, quebrando"-a
ao errar o alvo e acabar acertando o tronco de uma oliveira.
Além de assim demonstrar inépcia no manuseio das armas, dizia que
aquela espada tinha sido herdada de gente que a usara partindo
um castelhano ao meio, nos últimos anos das guerras de Restauração contra
os espanhóis. Esse louvor dos passados heroicos o constitui um 
tipo parecido com Bertoldo, o afidalgado, porque como este gaba"-se
de antepassados ilustres, uns da Lombardia, na Itália, outros em
Elvas, no Alentejo}

\newcommand{\Empav}{\small\textbf{empavesado}: enfeitado como um pavão, emblema da vaidade; no poema,
o termo é empregado sobre o tipo afidalgado, empavesado de feitos heroicos
que não pode dizer que fez, por demonstrar"-se mais de uma vez, mas como suposto
herdeiro de antigos heroísmos não perde a arrogância com que despreza todos em redor}

\begin{comment}
% oliveira: há a palavra encômio no INTRO.tex
\textbf{encômio}: discurso de louvor, nome para diversas espécies do gênero epidítico
alto, isto é, para os discursos de elogio das matérias dignas de elogio. Nos encômios
está previsto sobretudo o encarecimento, ou amplificação, das virtudes dos objetos de
louvor; assim como em seu contrário, o vitupério, está prevista a amplificação dos
vícios dos objetos de ataque verbal.
\end{comment}

\newcommand{\Epico}{\small\textbf{épico} (poesia épica): poesia de modo misto, isto é, que imita fazendo uso da palavra ora
o poeta ora os caracteres agentes, por quem o poeta se faz passar. Tendo como modelo
sobretudo a poesia homérica mais famosa, a \textit{Ilíada} e a \textit{Odisseia},
a épica se confunde com a narrativa heroica, isto é, a imitação de matérias elevadas,
em versos heroicos, falando das \textit{res gestae}, os feitos ilustres dos reis e chefes
cujas ações arriscadas foram dignas da memória}

\newcommand{\Fabula}{\small\textbf{fábula}: até o século \textsc{xviii}, a fábula é o conjunto das ações imitadas na ficção
poética, o enredo, ou entrecho inventado pelo poeta como trama de eventos encadeados segundo
a verossimilhança e a necessidade}

\newcommand{\Fama}{\small\textbf{Fama}: em grego \textit{kléos}, finalidade dos cantos heroicos da epopeia;
no poema aparece ora alegorizada em um carro conduzido pelos ventos Austro e Aquilon, ora
mencionada como fim moral da poesia, ou melhor, como a causa de seu \textit{movere}, porque o
desejo de fama, neste sentido, deve mover os leitores e ouvintes do poema às virtudes,
por imitação e emulação das virtudes dos heróis imortalizados pela fama}

\newcommand{\Filmemo}{\small\textbf{filhas da Memória}: as filhas de Mnemosine são as nove Musas do Parnaso, que, segundo
o mito antigo e as convenções retórico"-poéticas que o apropriaram sob diversas interpretações
em diversos séculos, eram entidades ou alegorias poéticas que promoviam as artes: Clio (preside
a história), Erato (preside a lírica), Euterpe (a música), Melpômene
(a tragédia), Polímnia (os hinos), Talia (a comédia), Terpsícore (a dança), Urânia
(a astronomia) e Calíope (a eloquência)}

\newcommand{\Fisic}{\small\textbf{física}, \textbf{filosofia racional} e \textbf{história natural}: com a Reforma dos Estatutos
da Universidade, entra em Coimbra com mais força os estudos das \textit{físicas},
como se costumava dizer, e de alguns novos métodos como os de Gassendi e Descartes,
assim chamados ``filosofia racional'', e afamados em Portugal como de grande uso nos
grandes centros de sapiência na Europa. Nesse mesmo período, passa a haver em 
Portugal quem aplicasse os princípios de Newton, como o professor de matemática e
cavaleiro professo da Ordem de Cristo,
Garção Stockler, quem mencionasse Galileu, como autor da ciência nova que substituiria os
métodos dialéticos dos professores aristotélicos do tempo dos jesuítas. (ver \textbf{peripatéticos})
Cabe ressaltar que no fim do elogio da Física, a Verdade  interrompe sua enumeração
lembrando que não houve ciência que não fosse por ela reconduzida à Universidade e
que é por ela, a Verdade, que se sustentam o Estado e a Igreja, pedindo que quanto
a isso testemunhem as musas do Parnaso, isto é, a poesia e as demais belas artes
que, como o próprio poema de Silva Alvarenga, louvam e atestam as virtudes de tal
ação política do gabinete do Marquês, que aí impõe a sua representação}

\newcommand{\Fortu}{\small\textbf{Fortuna}: divindade romana reconhecida como mudável, porque
governa a roda da vida, que a um movimento faz descer os maiores e subir
os menores. Ora como representação alegórica, ora mais como noção
moral que alerta os presunçosos, no poema a fortuna, substantivo comum,
é mencionada, por exemplo, no argumento do afidalgado Bertoldo, segundo o qual,
embora pobre e já desmecerecido, o seu passado --- a má fortuna da família --- não altera
o sangue de sua linhagem. Alegoricamente, a Fortuna aparece, no canto \textsc{iv},
impaciente já com as queixas que lhe dirigia o apaixonado Rufino; razão por
que resolve que o Acaso, seu filho, fizesse Rufino encontrar e salvar a moça amarrada
ao bronco pinheiro à espera de lobos carniceiros}

\newcommand{\Galia}{\small\textbf{Gália cisalpina}: norte da Itália, ao sul dos Alpes, Lombardia, Vale do Rio Pó}

\newcommand{\Gotic}{\small\textbf{gótica escritura}: referência depreciativa geral para as coisas velhas da
biblioteca do tio de Gaspar; especificamente, refere"-se à jurisprudência gótica,
isto é, às compilações do direito visigodo, que eram ensinadas como fontes do
sistema jurídico"-português}

\newcommand{\Heroic}{\small\textbf{heroico} (poesia heroica, matéria heroica):
é a qualidade das matérias melhores, a natureza dos feitos, dos afetos e do caráter dos
heróis. Para a invenção heroica, imitam"-se as ações dignas dos \textit{melhores}, louvando os varões
ilustres em ações extraordinárias, como grandes guerras ou grandes viagens}

\newcommand{\Homer}{\small\textbf{Homero}: até o século \textsc{xviii}, Homero é referido como primeira \textit{autoridade}
da poesia pagã antiga. Por isso, e acrescidas as referências das autoridades filosóficas que
o mencionavam, Homero é modelo para tudo o que se trate da composição de versos ficcionais,
isto é, para a Poesia, em especial para a poesia épica, mas também para a tragédia (ver \textbf{Épico)} }

\newcommand{\Ignor}{\small\textbf{Ignorância}: vilã da história; é apresentada como alegoria, personificada
como uma entidade existente por si mesma, a qual se transfigura em Tibúrcio, um antigo
estudante malogrado que vivia de vender objetos usados, sempre em tabernas e envolvido
em todos os demais descaminhos que se supunham à vida estudantil}

\newcommand{\Iminat}{\small\textbf{imitação da natureza}: em tratados de poética que circularam no século
\textsc{xviii} é comum a poesia ser referida como \textit{imitação da natureza} (física ou moral).
Tanto na \textit{Retórica} como na \textit{Poética}, Aristóteles diz que a imitação
é inata no homem. Desde o século \textsc{xvi} pelo menos, a descoberta, as traduções e
apropriações da \textit{Poética} de Aristóteles puseram em evidência diversas interpretações
da Poesia como \textit{mímesis}. Basicamente a Poesia é um fingimento honesto
que pode ensinar as virtudes por diversos meios, modos e assuntos. Os assuntos
concernem à matéria dos discursos, são aquelas coisas sobre que recai a escolha
na invenção retórica. Na definição aristotélica, as coisas que se imitam são os
diversos sujeitos ou matérias da poesia, as virtudes, as histórias, as plantas,
os animais, assuntos da História, que, como arte de discurso, deveria imitar o particular
como verdadeiramente teria sucedido, enquanto a poesia deveria imitar o universal,
fingindo"-o como um particular composto como convém que seja.
A Natureza inclui as coisas humanas e as demais matérias
de que o discurso pode tratar. A Natureza das coisas que são escolhidas
para imitar poeticamente fornece modelos para fingir, com verossimilhança e com verdade,
no universal ou no particular, para utilidade e deleite dos homens que se comprazem
na imitação. No caso da poesia, a imitação é muitas vezes feita em versos
mas, conforme a \textit{Poética} de Aristóteles, não exclusivamente o verso
define a poesia, que pode ser em prosa, desde que seja imitação}
 
\newcommand{\Indo}{\small\textbf{Indo}: rio asiático (ver \textbf{Paraguai})}

\newcommand{\Indus}{\small\textbf{indústria}: no século \textsc{xviii} português, o termo é referido principalmente como destreza,
sutileza, engenho ou habilidade em uma arte; em alguns usos neste fim do século \textsc{xviii},
``indústria'' já é referido como a destreza humana nas novas técnicas de processar
manufaturas que se tornam objeto de comércio entre as nações do mundo. Neste último sentido,
mais raro inicialmente, e mais recorrente com o fim do século \textsc{xviii} e início do \textsc{xix},
será identificado com os modernos processos de manufatura que daria principalmente
ao império britânico o lugar de maior força no comércio internacional}

\newcommand{\Invoc}{\small\textbf{invocação}: artifício retórico utilizado à imitação de Homero, Hesíodo e outros.
Encenando a poesia como resultado do furor de um \textit{entusiasmós}, de uma possessão
divina, invocou"-se no início dos poemas seja uma musa, seja o conjunto delas, seja ainda
a deusa da memória,
e daí também a Virgem Maria ou os Anjos, em adaptações católicas do artifício, para que
as entidades superiores inspirassem o canto, mesmo quando era feito com arte e método (ver \textbf{filhas da Memória})}

\newcommand{\Ismae}{\small\textbf{ismaelitas}: já foram chamados mouros e agarenos, descendentes de Agar, mãe de Ismael.
Também se chamaram Sarracenos, nome que lhes deu Mafoma, como os portugueses designavam
Maomé, ou Muhamed, profeta fundador do Islã, que se presumia descendente da casta de
Sara, mulher legítima de Abraão}

\newcommand{\Jane}{\small\textbf{Janeiro}: no poema, nome próprio de um \textit{doméstico},
funcionário de hospedaria, representado como tipo traiçoeiro. O nome,
que equivale a Januário, por exemplo, já indica essa tipificação
do caráter, porque Janus é a divindade de duas caras, que,
em usos vituperantes, tem valor de hipocrisia, falsidade, pouca
confiança. Não se confunda esse uso com a natureza da divindade
que tem duas caras porque fica nas portas das cidades, desejando
boas vindas alegremente e boa viagem com tristeza}

\newcommand{\Lineu}{\small\textbf{Lineu} (Carl von Linné, 1707--1778): historiador natural sueco autor do sistema
moderno das taxionomias das espécies naturais dos seres vivos, nomeados segundo
classificações de gênero e espécie. (ver \textbf{Lucrécio})}

\newcommand{\Longob}{\small\textbf{longobardos}, ou Lombardos: povo germânico que, no tempo das assim chamadas
invasões bárbaras, ocupou o Norte da Itália (ver \textbf{Gália Cisalpina}).
Constituíram um reino que, depois de cristianizados os chefes, foi
chamado \textit{Regnum Italicum}, donde sairiam cavaleiros cruzados,
cujo mérito antigo Bertoldo, o afidalgado, requeria para si, apesar da
má fortuna}

\newcommand{\Lour}{\small\textbf{louros}: folhas de louro com que se coroavam os vencedores. Simboliza a glória
nas armas ou nas letras, distinguindo os melhores lutadores, os melhores atletas e os
melhores poetas. Os ``louros de Minerva'', como se diz no poema, representam
o reconhecimento da vitória no conhecimento, de que
Minerva é patrona. No poema, são esses os louros que os heróis perderam}

\newcommand{\Lucre}{\small\textbf{Lucrécio} (Tito Lucrécio Caro, século \textsc{i} a.C.): poeta latino, escolado na doutrina de Epicuro,
a qual expôs em verso, nos seis livros do \textit{De rerum natura} (\textit{Sobre a
Natureza das Coisas}) mencionado como fonte do ensino epicurista no mundo romano. 
Foi quase sempre lido em âmbito católico como fonte de
verdades físicas, mesmo que fossem impugnados como heresia os principais fundamentos
da doutrina de Epicuro. No ``Discurso'' de Silva Alvarenga, Lucrécio,  mencionado
ao lado de Aristóteles e logo, com as notas, ao lado
de Marcgrave e Lineu, constitui autoridade da física, isto é, em conjunto são esses
autores que sua erudição escolhe para produzir o crédito do discurso exordial sobre
a natureza e a arte do poema herói-cômico.
As autoridades dos séculos \textsc{iv} ou \textsc{i} a.C.
são compatíveis com os autores dos séculos \textsc{xvii} e \textsc{xviii}, como sistemas classificatórios
das \textit{naturalia}. A singularidade dos autores não obsta sua participação no
mesmo gênero de matéria, a \textit{physis}}

\newcommand{\Lutri}{\textit{\textbf{Lutrin}}: poema herói-cômico composto por Nicolas Boileau"-Despréaux (1636--1711).
Boileau foi uma das principais autoridades francesas da arte poética antigongórica e
antimarinista que no final do século \textsc{xvii}, desqualificou o estilo de poetas italianos
e espanhóis famosos pela acumulação de agudezas, pelas dificuldades elocutivas, pelo
excesso no emprego de figuras etc. Ficou, por isso, conhecido como ``teórico'' da
poesia dita ``neoclássica'', talvez o mais importante autor moderno para as reformas
no estilo da poesia e da eloquência portuguesa no período que ficou
conhecido como restauração das letras em Portugal, desde a publicação do \textit{Verdadeiro
método de estudar}, de Verney, no final dos anos de 1740, e depois principalmente por
efeito da política pombalina contra os hábitos e métodos empregados pelos jesuítas
e substituição por modelos prediletos dos padres oratorianos}

\newcommand{\Marc}{\small\textbf{Marcgrave} (Georg Markgraf, 1610--1648): autor, com Guillelmo Piso, de
\textit{Historia Naturalis Brasiliae} (1648), livro dedicado a Maurício de Nassau,
que representa as plantas e animais do Brasil, além dos costumes dos indígenas  (ver \textbf{Lucrécio})}

\newcommand{\Marfi}{\small\textbf{Marfisa}: personagem do \textit{Orlando Furioso}, de Ariosto, e do \textit{Orlando Enamorado}, de Boiardo.
No poema, Marfisa, que tem o peito de bronze para o Amor, só é mencionada para colocar o Cupido em cena
treinando suas vinganças contra ela e, para isso, usando os corações de Doroteia e de Cosme como alvos,
o que terá consequências penosas para ambos e para a própria companhia (ver \textbf{Amor})}

\newcommand{\Margi}{\textit{\textbf{Margites}}: na \textit{Poética}, de Aristóteles, preservam"-se alguns dos poucos
excertos que se conhecem desta sátira, referida, principalmente, como espécie
poética reconhecida pelo seu metro, jâmbico, por isso chamada poesia jâmbica.
Diz"-se que esse poema de natureza satírica, hoje perdido, foi atribuído por
Aristóteles a Homero e outros o atribuíram a Pigres, um ateniense anterior 
ao tempo de Xerxes}

\newcommand{\Marqpombal}{\small\textbf{Marquês de Pombal}: Sebastião José de Carvalho e Melo (1699--1782) foi embaixador em Londres
e em Viena, durante o reinado de Dom João \textsc{v}. Com a ascensão de Dom José \textsc{i}, foi nomeado 
ministro dos negócios estrangeiros, depois Ministro de Estado plenipotenciário. Governou
Portugal como um ditador, durante praticamente todo o reinado de D. José \textsc{i}, de 1750 a 1777.
Foi preso depois da ascensão de Dona Maria \textsc{i}. Tornou"-se Marquês de Pombal somente em 1769,
título acumulado sobre o de Conde de Oeiras, recebido em 1759 (ver ``Introdução'')}


\newcommand{\Marte}{\small\textbf{Marte}: Ares para os gregos, deus da guerra e da discórdia; conforme os
gêneros e as circunstâncias discursivas. No poema, é às vezes vituperado --- ``o homicida Marte'', no canto \textsc{iii}
--- pelos danos das \textit{tristia bella} (as tristes guerras), mas em outras circunstâncias, que previssem outros
decoros, poderia ser alegorizado para constituir o louvor na representação
de virtudes guerreiras, que  concernem às vitórias dos reis e grandes senhores,
as \textit{res gestae} (os feitos ilustres), matéria da poesia heroica}


\newcommand{\Melanc}{\small\textbf{melancólico}: tipo de constituição anímica e física que, segundo 
a fisiologia antiga, é causada pela bílis negra, que conformaria
as disposições do ânimo de quem sofre de melancolia. No poema
é sobretudo essa acepção patológica que se aplica tanto a Rodrigo,
como a Cosme, que sofrem disso por se deixarem sempre enamorar além
da medida da razão, como o poema judica mais de uma vez}

\newcommand{\Miner}{\small\textbf{Minerva}:  Palas Atena para os gregos; deusa da sabedoria, representada
com elmo e lança. Aparece no poema sempre como alegoria das ciências reformadas
em Coimbra. Assim, ``os muros de Minerva'', de onde fugiram os desertores,
são os muros da Universidade}

\newcommand{\Mocho}{\small\textbf{mocho}: ave noturna e carnívora como as corujas}

\newcommand{\Monde}{\small\textbf{Mondego}: rio de Coimbra, que nasce na serra da Estrela e desagua junto à Figueira da Foz}

\newcommand{\Morg}{\small\textbf{morgado}: herança patrimonial exclusiva do primogênito. Criado pelo tio,
Gonçalo mente descaradamente a sua própria condição para a amante}

\newcommand{\Netoi}{\small\textbf{neto imortal}: Dom José de Bragança é referido como herdeiro de Dom José \textsc{i},
que não tivera filho varão. O neto morreria uma década depois do avô, sem
deixar herdeiro, o que faria de seu irmão, o futuro Dom João \textsc{vi}, o sucessor de Dona Maria \textsc{i}}

\newcommand{\Niso}{\small\textbf{Niso} e \textbf{Euríalo}: duas personagens troianas representadas
no canto \textsc{ix} da \textit{Eneida} de Virgílio} 

\newcommand{\Ovid}{\small\textbf{Ovídio}: poeta elegíaco romano, do século \versal{I} a.C. Foi banido de Roma por Augusto,
a quem dedica seu principal livro, as \textbf{Metamorfoses}. Tanto pelas \textit{Metamorfoses},
quanto pela \textit{Arte de amar}, Ovídio foi sempre muito lido, mesmo durante a chamada
Idade Média. No costume poético em que se inscrevem os poemas de Silva Alvarenga, as
\textit{Metamorfoses} são principalmente fonte de fábulas mitológicas que podiam ser
usadas para ornamento dos poemas, não como sinal de paganismo dos autores, que eram
sempre católicos e súditos do rei de Portugal.}

\newcommand{\Paida}{\small\textbf{pai da Pátria} ou \textit{pai do Povo}: designação de Dom José \textsc{i} nos discursos
encomiásticos e em documentos oficiais}

\newcommand{\Parag}{\small\textbf{Paraguai}: rio Paraguai, chamado no soneto final de \textsc{l.f.c.s.}, ``pátrio Paraguai''
porque o autor de \textit{O desertor} era americano. Na enumeração --- Paraguai, Zaire e Indo,
os três rios são mencionados em alusão aos três continentes por onde se estendiam os domínios de Portugal.
Seguindo os influxos do Mondego, os rios constituem um emblema dos
efeitos da Reforma da Universidade em todas as dominações portuguesas}

\newcommand{\Pego}{\small\textbf{pego}: variante de pélago, referindo"-se ao mar}

\newcommand{\Perip}{\small\textbf{peripatéticos}: da escola de Aristóteles. ``Filósofo peripatético''
equivale a dizer ``filósofo aristotélico''. Com a política pombalina, o
aristotelismo português foi desqualificado, e daí que o termo ``peripatético''
apareça nas letras pombalinas em sentido pejorativo. Ter se perdido nas
questões do \textit{Peripato} é causa do fracasso acadêmico da personagem
Tibúrcio, por exemplo, que estudara no tempo dos jesuítas.
Costuma"-se usar ``peripatético'' no século \textsc{xviii} para distinguir no vitupério
os maus seguidores e a verdadeira doutrina do Filósofo, lido por muitos
santos padres da Igreja (ver \textbf{Aristóteles})}

\newcommand{\Pichel}{\small\textbf{pichel}: recipiente grosseiro, caneca}

\newcommand{\Plat}{\small\textbf{Platão} (século \textsc{v} e \textsc{vi} a.C.): filósofo referido no século \textsc{xviii} como  doutrinador dos preceitos
morais de Sócrates, como fundador da \textit{Academia} em Atenas e como mestre
de Aristóteles. Este último teria dissentido dos princípios do mestre, mas as duas
doutrinas foram harmonizadas por mais de um intérprete e comentador entre os 
Padres da Igreja. O mais célebre arranjador da tese da harmonia entre
Platão e Aristóteles, foi Boécio (século \textsc{v}--\textsc{vi} d.C.). }

\newcommand{\Poet}{\small\textbf{Poética}: as Poéticas são textos de doutrina prática que ensinam os princípios da
arte e os procedimentos técnicos que regram as várias espécies de poesia. Em textos impressos
até o fim do século \textsc{xviii}, encontram"-se referências a conceitos retórico"-poéticos gregos
e latinos que prescreviam procedimentos para os efeitos da poesia, conforme os fins
de cada gênero e de cada espécie de poemas. A primeira autoridade conhecida em Poética é
Aristóteles, que define a Poesia como imitação de caracteres, afetos e ações, mas a sua
poética só foi conhecida em âmbito europeu, entre os séculos \textsc{xv} e \textsc{xvi}. Platão fala de poesia
mas esparsamente em diversos diálogos que mencionam ora uma ora outra espécie poética,
tratando"-as como um costume. Como arte imitadora, a Poética é para Platão
produtora de inverdades e causadora  de paixões; de ambos os efeitos deveriam fugir
os filósofos de sua escola, razão pela qual a poesia em geral é recusada pela filosofia
platônica (ver \textbf{Aristóteles}, \textbf{imitação da natureza}, \textbf{Platão} e \textbf{República})}

\newcommand{\Porf}{\small\textbf{pórfido}: cor púrpura, referindo"-se ao bronze da estátua}

\newcommand{\Prosop}{\small\textbf{prosopopéia}: Alegoria em que se personificam coisas concretas,
noções morais ou coletivas.}

\newcommand{\Repub}{\textit{\textbf{República}}: livro renomado de Platão, emulado por Cícero, cujo assunto é a
constituição da ideia de cidade perfeitamente governada, representação filosófica da \textit{pólis} melhor que o possível.
Nessa cidade filosófica, mesmo os poetas que pintam os homens melhor que o possível,
como a maior parte dos poetas trágicos, não teriam seu ofício reconhecido, porque nela
não deveriam entrar as artes miméticas, bem como não entrariam os imitadores de Homero,
ainda que, para o Sócrates de Platão, Homero fosse o melhor que era possível haver para a educação ateniense.
Não são expulsos da cidade os Poetas em geral, mas conforme as espécies
discursivas que produziam. Por exemplo, a poesia que produz o riso acerca do feio, do torpe,
do desprezível, para Platão, não ensina virtudes, como outras tradições de opiniões fariam
crer, e como \textit{O Desertor} também pressupõe. Na \textit{República}, a poesia \textit{lírica},
ou \textit{mélica}, entendida como
o louvor das virtudes dos heróis, é o único tipo de canto que a cidade perfeita admitiria;
porque aí o louvor dos feitos não inclui a \textit{mímesis}, isto é, o poeta não fala pelas
personagens, alterando o próprio \textit{ethos}, mas usa sempre a voz própria (ver \textbf{imitação da natureza}, \textbf{Platão})}

\newcommand{\Romvulg}{\small\textbf{romance vulgar}: no poema, designa"-se por essa expressão um
gênero de livros, quase sempre moralidades, que narrativamente ou
não ensinavam, embora com pouca arte, os bons costumes, a boa consciência,
os perigos das paixões etc. Mais de um título de livros assim é mencionado
no poema, em tom evidentemente desqualificador sempre; são obras
que tendo sido famosas entre o vulgo logo se tornam esquecidas.
Por exemplo, no embuste da prisão, o herói lembra"-se de passagens
de romances vulgares, como \textit{Alívio de tristes} ou \textit{Cristais
d'alma} para fazer suspirar a moça enganada, Doroteia, filha do carcereiro,
para que os libertasse. A arte desses livros é vituperada no próprio poema
como áspero estilo e hiperbólicas finezas}

\newcommand{\Secch}{\textit{\textbf{Secchia Rapita}}: poema em doze cantos de autoria incerta, atribuído a Tassoni,
primeira autoridade moderna na poesia herói-cômica. O \textit{Secchia
rapita} tem por matéria heroica a guerra entre os bolonheses e os modenenses na
época do imperador Frederico \textsc{ii}. É tido por referência para a composição dos
cantos de \textit{Lutrin}, de Boileau, e de \textit{Rape of the Lock}, de Alexander Pope}

\newcommand{\Tasson}{\small\textbf{Tassoni}: Autor do livro \textit{La secchia rapita} emulado nos
séculos \textsc{xviii} e \textsc{xix}, como inventor moderno do poema herói-cômico. As
circunstâncias de sua vida são obscuras, alguns referem a ele como sendo Torquato Tasso,
autor do poema heroico \textit{Jerusalém libertada}}

\newcommand{\Termin}{\small\textbf{Termindo Sipílio}: nome de Basílio da Gama na Arcádia de Roma. No primeiro soneto
que termina \textit{O desertor}, Termindo é citado como o que cantou a liberdade dos índios;
bem entendido, a libertação dos índios é narrada em \textit{O Uraguai} como a guerra que os dizima para os
tirar da custódia dos padres da Companhia de Jesus}

\newcommand{\Tiborn}{\small\textbf{tibornas e magustos}: a nota do autor aos dois termos
explica o que sejam. O fato de estarem em nota os dois termos
é um uso análogo das notas para Aiuruoca ou Tatu: trata"-se de
coisas insignificantes para os leitores mais ilustres do poema, que
no limite era até mesmo o Marquês de Pombal. Nos versos, os dois
termos, que têm provavelmente uma circulação vulgar, devia dar
comicidade à declaração de amor de Gonçalo, que depois de falar
em laços eternos de amor, pinta a felicidade como a mulher fazendo
pão com linguiça.}

\newcommand{\Tipos}{\small\textbf{tipos}: em textos poéticos como em textos históricos, os tipos são 
inventados como \textit{ethos}, caracteres, modelos de virtudes e vícios,
fingidos com palavras, com mais ou menos harmonia, números e tropos, imitados
conforme os seus costumes, os afetos e os feitos de homens que verdadeiramente existiram,
ou que foram concebidos pelo engenho de algum poeta (Ver \textbf{imitação da natureza})}

\newcommand{\Tiria}{\small\textbf{tíria}: feminino tírio, oriundo da cidade de Tiro, famosa pela cor escarlate dos
pigmentos que produzia}

\newcommand{\Trag}{\small\textbf{trágico} (poesia trágica): poesia puramente mimética, isto é, que imita os caracteres
agindo diretamente. Tendo como modelo principalmente Ésquilo, Sófocles e Eurípedes,
a tragédia está incluída no mesmo gênero de matéria da epopeia, e suas matérias particulares
costumaram ser tiradas das narrativas homéricas, imitando também ações, afetos e caracteres
de homens melhores (\textit{aristoi}). Se por um lado pertence ao mesmo gênero da epopeia, por outro
pertence ao mesmo gênero de enunciação que a comédia, por imitar as personagens
diretamente}

\newcommand{\Ulis}{\small\textbf{Ulisses}:  nome latino de Odisseu, rei de Ítaca, herói da \textit{Odisseia}, poema atribuído
a Homero que narra as peripécias de Ulisses após a guerra de Troia, retornando para
sua casa. Enfrentando a ira de Posídon, o deus dos mares, ajudado por outras divindades,
enfrentando monstros e outros perigos, perde todos os companheiros antes de ser reconduzido
à sua pátria, onde entra sob o disfarce de um mendigo para junto ao filho retomar a
ordem e o seu poder, ameaçado pelos pretendentes de Penélope, a esposa fiel, que não cedeu
o leito e o trono na ausência do marido. Como é um poema de viagem, e não de guerra, o poema herói-cômico
de Alvarenga emula, mas comicamente, a espécie heroica da \textit{Odisseia}, assim como da \textit{Eneida}}

\newcommand{\Urag}{\textit{\textbf{O Uraguai}}: poema de Basílio da Gama que narra a guerra da aliança luso"-castelhana 
contra os Sete Povos das Missões, reduções jesuíticas juridicamente portuguesas até 1750 mas
que restaram no território que a partir do Tratado de Madrid passou a ser
possessão da Coroa espanhola. Com isso, o poema heroico em cinco cantos louva as ações
do Conde de Oeyras, de seu irmão e de seus lugares"-tenentes, na execução bélica do acordo diplomático.
Narrando a pacífica libertação da Universidade do jugo do ensino
jesuítico, \textit{O desertor} emula, mas com matéria cômica, \textit{O Uraguai} cujo
assunto é a sangrenta guerra, contra os padres jesuítas e os índios custodiados que se recusavam
a desocupar a terra, às margens do rio Uruguai, chamado Uraguai no poema de Basílio, seja por um
solecismo, seja deliberadamente pela eufonia do título.}

\newcommand{\Util}{\small\textbf{útil e agradável}: é tópica horaciana muito recorrente em textos setecentistas o
preceito do consórcio do \textit{útil} com o \textit{agradável} como definição do mais
apto na arte da poesia. Candido Lusitano assim comenta os versos de Horácio citados no
fim do ``Discurso sobre o poema herói"-cômico'': \mbox{``O Poeta,} 
pois, que quiser ter os votos de todos, dos velhos e dos moços, há de em suas obras fazer
inseparável o instrutivo do deleitoso. Esta
é toda a força do \textit{pariter} (igualmente, ao mesmo tempo): isto é, não há de instruir em um lugar, e
deleitar em outro; há de o deleite acompanhar sempre a instrução. Os que sabem a
História Romana, bem alcançam que neste verso a palavra \textit{punctum} vale o
mesmo que \textit{suffragia} [votos], sendo costume dos Romanos dar os seus votos por
pontos.'' \cite[p. 158--159]{horacio}}

\newcommand{\Veros}{\small\textbf{verossimilhança}: Em \textit{Poéticas} de cunho aristotélico, 
como esta que Silva Alvarenga repõe, que entendem a poesia como imitação da natureza,
a verossimilhança deve ser produzida a partir das qualidades naturais das matérias em geral e das marcas
acidentais das matérias particulares. Nas pessoas, imita"-se sobretudo o caráter,
o \textit{ethos}, isto é, o que na imitação faz um velho, uma moça ou um escravo, parecerem realmente um
velho, uma moça e um escravo, para que um avarento possa parecer como costumam ser os avarentos,
assim como proporcionalmente os grandes homens do passado possam parecer os autênticos portadores
das virtudes que os imortalizaram, e assim por diante conforme as longas galerias de tipos que
a poesia sempre fez encenar. Estas qualidades são referidas nas tábuas e
preceitos que ensinavam imitação poética segundo categorias como: nascimento,
condição de vida, os diversos atributos das idades, a nação, a fortuna, o engenho
(ou inclinação particular do ânimo). Para isso, seguia"-se o que se doutrinava tanto
no \textit{Fedro}, de Platão, como na \textit{Retórica}, de Aristóteles: para
bem produzir os efeitos dos discursos era preciso conhecer antes de tudo a alma
humana, que será sempre o auditório e que quase sempre será o assunto dos discursos.
No \textit{Ad Herennium}, a definição das matérias está circunscrita às coisas
que os costumes e as leis instituíram para o uso civil; assim, a verossimilhança
tem suas condições determinadas pelos ofícios no uso civil.
Daí que antes de mais nada a persuasão do auditório esteja
diretamente subordinada à paciência dos ouvintes, e daí que a arte estivesse
condicionada à adequação aos ``receptores'' das palavras poéticas (ver \textbf{imitação da natureza}, \textbf{Poéticas} e \textit{tipos}.)}

\newcommand{\Xavec}{\small\textbf{xaveco}: embarcação mourisca que ficou conhecida pelo uso na pirataria,
devido à facilidade com que permitia a abordagem graças ao seu tamanho e à
disposição  das velas; pelo sentido negativo atribuído ao fato de sua
fabricação ser moura (ver \textbf{Ismaelita}), também pode significar simplesmente
embarcações inferiores e mal aparelhadas}

\newcommand{\Zaire}{\small\textbf{Zaire}: rio africano (ver \textbf{Paraguai})}

\newcommand{\Zefir}{\small\textbf{Zéfiro}: Figura mitológica que representa um dos ventos
filhos de Éolo com a Aurora. Vento oeste}

