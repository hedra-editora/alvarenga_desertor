\chapter[Introdução, \emph{por Clara S. Santos e Ricardo M. Valle}]{Introdução}
\hedramarkboth{introdução}{clara c. santos e ricardo m. valle}


\textit{O desertor: poema heroi"-cômico}, de Manuel Inácio
da Silva Alvarenga (1749--1814), foi impresso pela primeira
vez em 1774, pela Real Oficina da Universidade de Coimbra.

Nascido em Vila Rica, ou em São João del Rey, sobre Manuel Inácio da Silva
Alvarenga disse"-se que era pardo, filho de músico, de origem pouco abastada.
Conseguiu progredir nos estudos aparentemente pelo empenho do pai e de uma
subscrição de amigos que teriam financiado sua ida ao Rio de Janeiro e depois a
Coimbra. Em Portugal, viria a se tornar amigo de Basílio da Gama, o poeta
brasileiro de \textit{O Uraguai} (1769), protegido e secretário do Marquês de
Pombal.\footnote{  Sebastião José de Carvalho e Mello ficou conhecido pelo último
e mais alto título de nobreza que recebeu em vida, concedido por decreto real em
setembro de 1769. Foi nomeado ministro de assuntos estrangeiros quando da
ascensão de Dom José \textsc{i}, em 1750. Tornando"-se ministro de Estado,
recebeu foros de plenipotenciário, isto é, privilégio de exercer como primeiro
ministro do Rei decisão sobre todos os assuntos do reino, com plenos poderes
para representar o rei no Conselho de Estado. Em 1759, recebeu o título de Conde
de Oeiras e dez anos depois o de Marquês de Pombal. }  Em 1774, \textit{O
desertor} saía à luz no momento que foi provavelmente o ápice da política do já
então Marquês de Pombal e quase às vésperas de sua queda repentina, em 1777, com
a morte de Dom José \textsc{i} e a consequente ascensão de Dona Maria
\textsc{i}.

Silva Alvarenga tinha 24 ou 25 anos e cursava o segundo ano de Direito naquela
Universidade, quando, não se sabe exatamente por que circunstâncias, o poema
teria sido mandado imprimir por Pombal. Por esta ocasião foi reconhecido como
bom poeta, e teve poesia sua integrada na pompa de inauguração da estátua
equestre de Dom José \textsc{i}, que encerrava monumentalmente a reconstrução de
Lisboa, em 1775, vinte anos após o terremoto de que até Kant falou, lá no fundo
de K\"onigsberg.  É verossímil que Silva Alvarenga tenha estado na capital do
Reino, mas aparentemente permaneceu em Portugal apenas enquanto durou seu curso
em Coimbra, entre 1773 e 1777.

De volta ao Rio de Janeiro, como advogado formado em Direito Canônico, priva com
mais de um vice"-rei, alcançando a amizade de uns e a inimizade de outro. %sic
Além de ter seu nome entre os que formaram a Arcádia Ultramarina, com o nome
acadêmico de Alcindo Palmireno, Alvarenga integrou mais de uma agremiação
literário"-científica no Rio de Janeiro.  Com a amizade do Marquês de Lavradio e
de seu sucessor, Dom Luís de Vasconcelos e Sousa, de Castelo"-Maior, nomeado
vice"-rei em 1782, Silva Alvarenga assenta uma cadeira de Retórica e Poética no
palácio do governador.  Para já evitarmos esquematismos muito apressados, é
notável que esse poeta, que surgiu tão pombalino, tenha alcançado tamanha
distinção oficial junto ao vice"-reinado justamente em 1782, no ano da morte do
``invicto Marquês'', já então exilado da Corte, depois da ascensão de Dona
Maria \textsc{i}.  Após a nomeação do Conde de Rezende, em 1790, porém, a última
sociedade de que participara seria proibida, suspeita de opiniões francesas,
afamada entre os antipatizantes como um ``\textit{club} de jacobinos''.


Com a amizade de Dom Luís de Vasconcelos, tinha adquirido a deferência do
encarregado direto do rei de Portugal. Com a inimizade do Conde de Rezende,
Alvarenga perderia primeiramente as liberdades com que o antigo governador o
distinguia.  Vale lembrar, porém, que ao cargo de vice"-rei estava
institucionalmente previsto que tinha poderes para ambas as ações --- distinguir
sujeitos particulares com prerrogativas de encargos públicos, bem como
destituí"-los, conforme o seu entendimento e vontade.  De forma correlata, o
letrado, mesmo de origem mestiça como era o caso, estava sujeito tanto ao
privilégio da distinção, como ao infortúnio da preterição por parte do superior
hierárquico. De qualquer forma, o poeta em questão tinha diploma com que se
pudesse distinguir, com ou sem o favorecimento direto da pessoa instituída como
poder local na representação política da colônia, porque advogava e, em 1790, já
adquirira fama para fazê"-lo para particulares.  Porém, perdendo a preferência da privança com o superior hierárquico, em pouco tempo perderia os direitos civis e, acusado de Inconfidência pela Devassa que o novo vice"-rei lançara sobre ele e amigos, permanece preso por mais de dois anos, entre 1794 e 1797, quando recebe indulto por decreto de Dona Maria \textsc{i}. Com isso, readquire os direitos de súdito e aparentemente segue a carreira que já cursava, sempre na cidade do Rio de Janeiro, capital da principal Colônia portuguesa na América. Antes, porém, no mesmo ano de sua saída dos cárceres da Ilha das Cobras, escreve um poema aos anos da rainha.  

Todos esses eventos biográficos, cabe frisar, estavam
institucionalmente previstos na jurisprudência portuguesa, a qual conformava as
práticas civis que encenavam os decoros da representação que soberanos e súditos
deviam manter dentro da ordem do Estado.  As partes e membros particulares da
hierarquia estavam sempre em demanda, mas as regras de precedências das
aristocracias tendiam à fixidez da \textit{lex}, especificamente a lei do
sistema jurídico português, do qual Alvarenga participava e que evidentemente
conhecia como diplomado em Direito Canônico.

No século \textsc{xix}, na posteridade que o mencionou, o livro de Silva Alvarenga
tornou"-se mais conhecido como \textit{O desertor das letras}. Com essa forma
estendida do título principal, a ele se referiu boa parte da crítica literária
do século \textsc{xix}, desde uma pequena nota bastante elogiosa que o escritor
veneziano Adrien Balbi fez ao poeta, no segundo volume do almanaque geográfico
de assuntos portugueses denominado \textit{Ensaio estatístico sobre o Reino de
Portugal e do Algarve}, impresso em 1822.  O Brasil então abrigava a sede do
bastante declinado império marítimo português, e por isso mesmo os poetas hoje
chamados ``brasileiros'', como Silva Alvarenga, não figuram no livro de Balbi
com a distinção que seria produzida de forma mais clara principalmente depois da
obra de Ferdinand Denis, o \textit{Resumo da história literária de Portugal,
seguido do resumo da história literária do Brasil}, livro importante que,
impresso em 1826, já no título ramificava uma literatura da outra. Assim,
Alvarenga é referido como poeta português que nasceu e viveu no Brasil, parte
importante do reino de Portugal, conforme a nota de Balbi, que não é demais
transcrever inteira já que ainda não foi reproduzida na íntegra pelos críticos
que a citaram:

\begin{hedraquote} Manuel Ignacio da Silva Alvarenga, membro da Arcádia,
professor de retórica no Rio de Janeiro, onde foi considerado o melhor advogado
do país. Compôs um grande número de poesias entre as quais os poemas \textit{O
desertor das letras} (\textit{le deserteur des lettres}) e \textit{Glaura} se
destacam por um merecimento real. Suas sátiras contra os vícios, a tradução em
verso português de Anacreonte, e de outras poesias [não] foram impressas. Uma
bela versificação, os pensamentos verdadeiramente filosóficos e uma crítica tão
fina quanto delicada se destacaram em todas as suas composições. Esse grande
poeta foi também um amante muito distinto da música e teve conhecimentos raros
em história natural.  Ele formou em sua casa um pequeno museu e possuiu a
biblioteca mais numerosa do Rio de Janeiro. Ela foi comprada aos seus herdeiros
e reunida à biblioteca do rei.\footnote{  ``Manoel Ignacio da Silva Alvarenga,
\textit{membre de l'Arcadia, professeur de rhétorique à Rio"-Janeiro, où il
passait pour le meilleur avocat du pays. Il a composé un grand nombre de poésies
parmi lesquelles les poèmes \textit{O desertor das letras} (le déserteur des
lettres) et le \textit{Glaura} se distinguent par un mérite réel. Ses satires
contre les vices, la traduction en vers portugais d'Anacréon, et d'autres
poésies, ont été imprimées. Une belle versification, des pensées vraiment
philosophiques, et une critique aussi fine que délicate se font remarquer dans
toutes ses compositions. Ce grand poète était aussi un amateur très"-distingué de
la musique, et avait des connaissances rares en histoire naturelle. Il s'était
formé dans sa maison un petit musée, et possédait la bibliothèque la plus
nombreuse de Rio"-Janeiro. Elle a été achetée de ses héritiers et réunie à celle
du roi.}'' (\textit{Essai statistique sur le royaume de Portugal et de l'Argarve
compare aux autres Êtats d'Europe et suivi d'un Coup d'oeil sur L'ètat actuel
des sciences, des lettres et des beaux"-arts parmi les Portuguais des deux
hémisphères}. \textsc{ii}. Paris, 1822, pp. 173--74.) } \end{hedraquote}


Suas sátiras e suas traduções de Anacreonte se tornaram célebres entre os seus
biógrafos justamente por \textit{não} terem sido impressas e terem sido assim
destruídas, logo após a sua morte, por um padre franciscano inimigo do poeta.
Com efeito, é plausível, pela própria disposição sintática do período de Balbi,
que tenha faltado a negação no sintagma ``\textit{ont été imprimées}''.  Seja
como for, com essa nota, cujas informações parecem ter sido coletadas com alguma
precisão em 1820, seis anos após a morte de Silva Alvarenga, iniciou"-se a
fortuna crítica do poeta ao menos como autor do poema heroi"-cômico que aqui
reeditamos.

Com ela, iniciava"-se também a invenção da sua personalidade literária
conveniente aos projetos nacionais e nacionalistas de constituição de uma
literatura brasileira do período colonial que mantivesse com a nova atualidade
do Império do Brasil após a Independência uma relação umbilical, ou germinal, na
justificativa de uma autonomia progressiva e crescente de um ``Espírito
nacional''.  Vale frisar que esse ``marco inicial'' da fortuna crítica do poema
não era um texto de crítica ou história literária, mas apenas um item no longo
apêndice do compêndio geográfico de Balbi.  O franco"-veneziano, escrevendo antes da Independência do Brasil, ainda denominava o reino de Dom João \textsc{vi} pela fórmula da dupla coroa portuguesa --- Portugal e Algarve ---, que foi como oficialmente se designou o reino desde os séculos \textsc{xii} e \textsc{xiii}, quando Portugal conquistara o \textit{ocidente} da Andaluzia.

Como se tratasse de um livro de atualidades, Balbi faz notas breves como esta
para quase todos os assim chamados árcades portugueses, entre outros poetas,
teólogos e oradores que tivessem alcançado algum renome naquele tempo, sendo
significativamente mais extensa sua nota sobre Bocage, ``\textit{le premier des
poètes portugais modernes}''.  Não deixa de assinalar com asterisco por exemplo
a nota sobre Tomás Antônio Gonzaga, não mencionando, porém, qualquer relação
dele com o Brasil, senão que morreu exilado em Angola e que os poemas de
\textit{Marília de Dirceu} estavam traduzidos em várias línguas.  Entre ambos,
em meio a tantos outros exemplos, o geógrafo não faz distinção de nacionalidade
literária, como até hoje se dividem ``literatura portuguesa'' e ``brasileira''.

Mas, como ficou dito, o breve texto de Balbi sobre Silva Alvarenga é bem
informado a respeito do poeta, mencionando até a transferência de sua biblioteca
pelos seus herdeiros para a Biblioteca Real, após a sua morte, o que
historicamente se demonstrou, no particular, pelas cartas e ofícios que Joaquim
Norberto encontrou e transcreveu, em nota, na sua edição das \textit{Obras
poéticas de Manuel Inácio da Silva Alvarenga}. Não deveria surpreender esse
detalhe.  Tanto a proximidade de sua morte em relação à viagem de Balbi a
Portugal, quanto a provável importância que tomou o seu nome no fim da vida, uma
vez que Alvarenga residisse desde muito naquela corte recentemente transferida,
são causas possíveis para essa relativa atenção de Adrien Balbi.  De uma forma
ou de outra, como doutor em leis, poeta já distinguido pelos maiores de Portugal
décadas antes, preso e absolvido sem pecha, o nome de Alvarenga, em 1820, era já
distinto como de varão ao menos notável; e logo seria ilustre, pela fama dos
feitos em letras para a municipalidade que improvisadamente abrigara a Corte de
Dom João \textsc{vi} e posteriormente para a nacionalidade póstera que o
reivindicaria para si.

Balbi considerava em seu \textit{Ensaio estatístico} o estado atual das
ciências, das letras e das belas"-artes entre os Portugueses dos dois
hemisférios. Neste início do século \textsc{xix}, no auge da ideologia do
progresso industrial pelo domínio das ciências da natureza, o louvor do poeta
advogado, formado em Direito Canônico na Universidade de Coimbra, produzia a
personagem como um livre"-pensador segundo modelos iluministas. Daí que
supostamente ele fosse um colecionista privado, sabedor de música, de história
natural e de cultura antiga, com ``pensamentos verdadeiramente filosóficos''. E
é possível que soubesse mesmo o seu tanto de coisas nesses específicos, menos
por que fosse um iluminista e mais provavelmente por que esses saberes fossem voga no tempo em que exerceu os ofícios das letras.  
Em meio às observações de Balbi
a respeito do desenvolvimento da hidráulica, da engenharia naval ou das ciências
agrárias em Portugal, tais atributos de Silva Alvarenga conferem interesse e
importância a esse poeta, mas não mais do que como que por um golpe de vista.  A
rigor, as citações de Lineu e de Marcgrave que se leem nas notas que o poeta faz
a \textit{O desertor} não o tornam mais parecido com Alexander von Humboldt, ou
a outros naturalistas do século \textsc{xix}.  Principalmente, isso não lhe
retira os pressupostos teológico"-políticos do Direito Canônico, que foi a alta
ciência que o fez certamente distinto como homem de leis, tanto no tempo em que
o Rio de Janeiro foi vice"-reinado, como quando passa a ser a sede da Monarquia
lusitana.  O ter escrito sátiras ou um poema de caráter misto também não supõe
autonomia literária ou liberdade intelectual, porque escreve como vassalo fiel
aos pés do trono e do conselho do Estado, porque sem tal lealdade não teria
participado das festas em torno da estátua equestre, nem, de volta à colônia,
teria se tornado lente de retórica, assentado pelo vice"-rei, para a utilidade e
deleite das ciências e das artes na capital da colônia. 

O cônego Januário da Cunha Barbosa (1780--1846), que primeiro escreveu a
\textit{vida} do poeta e teria privado com ele no Rio de Janeiro da corte
portuguesa de Dom João \textsc{vi} e Dona Maria \textsc{i}, foi membro fundador
e primeiro sócio subscrito do Instituto Histórico e Geográfico Brasileiro, além
de autor de seus primeiros estatutos. Conforme Joaquim Norberto, Cunha Barbosa
foi ``dedicado e agradecido discípulo'' de Silva Alvarenga nas aulas de
Retórica e Poética que lecionou por mandado do governador Dom Luís de
Vasconcelos e Sousa, desde o final do século \textsc{xviii}.  Depois de receber
o indulto de Dona Maria \textsc{i}, aparentemente Alvarenga retornou a essas
lições de retórica, tornadas célebres entre os círculos letrados do Império do
Brasil, quando talvez Cunha Barbosa tenha tomado lições, porque antes não teria
idade para isso.

Já no século \textsc{xix}, o famoso cônego pode ter sido aluno do autor de \textit{O
desertor}, assim como outros oradores no Império do Brasil, como Monte Alverne e
São Carlos. Contudo, Cunha Barbosa na vida de Silva Alvarenga desvia em quase
vinte anos a idade do poeta no tempo de sua morte, dizendo que faleceu com quase
oitenta anos. Joaquim Norberto publicou documentação sobre isso na referida
edição das \textit{Obras poéticas}, de 1864, o que, de certa maneira, colocava
em dúvida a prova testemunhal alegada pela relação mestre/discípulo que o poeta
e o cônego teriam firmado, embora o mesmo Norberto tenha reposto a autoridade
dessa informação, obtida por sua vez na relação também pessoal que teve com o
cônego quando foi bibliotecário submetido às suas ordens.

Segundo Januário da Cunha Barbosa, Silva Alvarenga não tinha o poema por acabado
quando este foi impresso por ordem do ministro plenipotenciário de Dom José
\textsc{i}. Assim, desde as palavras de Cunha Barbosa, no terceiro número da
Revista trimestral do Instituto Histórico e Geográfico Brasileiro --- ``o poema
heroi"-cômico intitulado o \textit{Desertor das letras}, que por ordem do Marquês
fora impresso contra a vontade de seu autor, porque ainda o não havia
suficientemente corrigido, deram"-lhe créditos de literato e o descobriram
distinto poeta'' ---, muitos compêndios de literatura e ensaios críticos sobre o
poema reafirmaram essa mesma informação como um dado objetivo, sem considerar,
por exemplo, a impropriedade política das mesmas palavras que, a rigor, opunham
a vontade do súdito e a ordem do senhor, que evidentemente não estavam e não
poderiam estar em conflito; ao menos não sem severas consequências.

É certo que a vontade nada poderia contra a ordem, porque vivia"-se sob os
princípios da \textit{monarquia absolutista}, e debaixo da \textit{ditadura
pombalina}, que foi como o historiador inglês Charles Boxer caracterizou os
modos hipercentralizadores desse ministro de Estado plenipotenciário que, por
quase trinta anos, exerceu poder muito direto sobre as decisões mais gerais e
mais particulares do reino.  Mesmo assim, desde Januário da Cunha Barbosa, a
crítica literária tem frisado que o poema foi impresso por ordem do ministro e
contra a vontade do autor, mas nada há de particularmente conclusivo em relação
a isso, além do testemunho do próprio Cunha Barbosa, que já no século
\textsc{xix} foi corrigido em relação a várias particularidades, o que obriga a
relativizar o crédito dado à informação supostamente direta.  Esse testemunho
ainda assim foi transformado em ``informação'' biobibliográfica, pela crítica e
pela historiografia literária, empenhadas desde a Independência em constituir,
corrigir e enaltecer o cânone literário do Brasil no período colonial.

Com efeito, mesmo na hipótese de que pessoalmente Silva Alvarenga o tivesse
confessado a seus alunos de Retórica e Poética, não se pode deixar de considerar
que se trate de um velho lugar comum os autores alegarem que tornaram pública
certa obra apenas por \textit{força maior} e acrescentarem ter ficado, por isso,
incompleto o trabalho da própria arte. Com isso, visam a captar a benevolência
do público em geral e a atenuar as possíveis falhas na aplicação dos preceitos,
defendendo"-se da mordacidade da crítica dos mais doutos.  Articulações de
sentido como essa abundaram em prólogos, proêmios e cartas dedicatórias, mesmo
em obras acabadíssimas, e não representaram necessariamente qualquer sinceridade
afetiva, nem humildade de artista, nem muito menos consciência crítica ou
autocrítica.

A positivação desse testemunho parece ter sido um meio de a crítica literária
brasileira justificar a perda de eficácia poética deste poema, sobretudo nos
últimos cantos, que parecem se arrastar numa elocução excessivamente prosaica
muitas vezes, depois de cantos iniciais razoavelmente bons.  Com efeito, dentro
das convenções do gênero misto em que o poema é declaradamente escrito, ao menos
os primeiros cantos devem ter tido razoável eficácia cômica pela dissociação
deliberada entre o \textit{estilo alto}, que emula poemas heroicos como
\textit{O Uraguai} e \textit{Os lusíadas}, e a \textit{matéria baixa}, que imita
tipos sórdidos e feitos indignos próprios da sátira e da comédia, exemplos
viciosos que a moralidade do poema ensina para corrigir os vícios pelo vexame.
Assim, os tipos poderiam ser imitados da \textit{Natureza} --- entendida como
\textit{natureza das coisas} e, neste específico, como as diversidades
qualificáveis da \textit{natureza humana} ---, imitando assim os engenhos
(\textit{ingenii}) dos homens particulares tipificados conforme o hábito, o
estado, a virtude etc., ou então poderiam ser imitados de tradições cômicas
gregas, romanas, italianas, francesas, portuguesas, que já tinham estilizadas e
classificadas vastas galerias de tipos, as quais constituíam repositórios para a
representação ficcional de gênero baixo, mesmo dentro de moralidades e de
circunstâncias políticas tão diversas.

Assim, o poema, cuja comicidade foi sempre posta entre aspas pela crítica
literária dos séculos \textsc{xix} e \textsc{xx}, tinha como tema heroico a ação do Marquês de
Pombal sobre o ensino na Universidade, mas relatava as ações vis de Gonçalo, o
jovem estudante que desertou da carreira das letras para ingloriamente retornar
à província de onde saíra, agora mais obscuro do que antes.  Por fraqueza de
ânimo, abandonaria os estudos movido pela Ignorância, que não tinha mais lugar
em Coimbra desde que o Marquês reformara os estatutos da Universidade, ação alta
que dava o fundamento histórico ao poema e o encomendava nas mais altas esferas
do Império marítimo dos reis de Portugal e do Algarve.

Nos primeiros anos de seu governo, ainda Sebastião José de Carvalho e Melo não
era nem conde nem marquês.  De embaixador por muitos anos na Inglaterra e depois
na Áustria, onde se casa com a prima da rainha de Portugal, é nomeado ministro
dos negócios estrangeiros, em 1750.  Com o terremoto de Lisboa, em 1755,
torna"-se a figura política mais poderosa no reino, abaixo do rei. Com o atentado
a Dom José \textsc{i} em 1758, seguido da rápida punição aos acusados, consolida
sua vitória contra as facções da nobreza antiga que faziam oposição a seu
gabinete. Os dois eventos graves alavancaram e consolidaram os êxitos do
ministro de Estado perante o rei e o reino, \mbox{levando"-o} à vitória em face dos seus
principais inimigos: de um lado, a parte da velha nobreza que o via com
desconfiança desde a sua nomeação; de outro, os padres da Companhia de Jesus,
que por decreto real foram expulsos de todos os domínios do reino em 1759, ano
em que o ministro de Estado é distinguido pelo título de Conde de Oeiras.  Neste
mesmo ano, mais de uma década antes da Reforma da Universidade, a estrondosa
expulsão dos jesuítas já resultaria numa primeira reordenação do ensino, uma vez
que a Companhia de Jesus dominava a educação básica em todas as possessões da
Coroa Portuguesa.

O poema se inicia com a invocação, pedindo à Musa que auxilie o poeta a cantar
com engenho e arte o desertor da Universidade que, ao lado de seus companheiros
de vícios, guiados pela Ignorância, retorna em viagem à província natal, onde
sem vencer nos estudos é recebido com ira pelo tio, no final das duras e
tumultuosas jornadas, que imitavam, em gênero baixo, epopeias de viagem como a
\textit{Odisseia}, a \textit{Eneida} e \textit{Os lusíadas}.  Em sua disposição
retórica, o argumento da fábula é apresentado ao mesmo tempo em que se faz a
invocação. 

\begin{verse}
Musas, cantai o desertor das letras \\
Que, depois dos estragos da Ignorância, \\
Por longos e duríssimos trabalhos, \\ 
Conduziu sempre firme os companheiros \\
Desde o loiro Mondego aos Pátrios montes. \\ 
Em vão se opõem as luzes da Verdade \\
Ao fim que já  na ideia tem proposto  \\
E em vão do Tio as iras o ameaçam. \\
\end{verse}	  

Junto à invocação já se propõe a matéria heroi"-cômica dos cantos: a renúncia de
Gonçalo aos livros e as causas de seu retorno à obscuridade em Mioselha.  Na
sequência, inicia"-se a dedicatória como em \textit{Os lusíadas}, referindo o
homenageado por perífrases, sem nomeá"-lo diretamente. Aquele engenho que,
amparado pela mão benigna do rei, alimentava as doces artes poderia ser Pombal,
mas era mais provavelmente o reitor reformador, Dom Francisco de Lemos, que,
amparado pela mão do ministro, é o ``prelado ilustre'' a quem alusivamente se
incumbe a proteção dos versos e que no final do poema aparece
esmagando o monstro da Ignorância ao pé do trono. Nos versos dedicatórios
iniciais, é a esse prelado ilustre que se perguntam as causas da deserção das
letras, já que o bispo fidalgo nascido no Brasil, além de reitor reformador da
Universidade no tempo da Reforma, é um dos autores pombalinos que redigem o
\textit{Compendio histórico do estado da Universidade de Coimbra no tempo da
invasão dos denominados jesuítas e dos estragos feitos nas ciências e nos
professores}.

O jovem Gonçalo derroca na vida por muitas causas específicas que particularizam
o seu caso, mas a moralidade das causas encenadas no enredo é, por necessidade,
geral. Assim, sua ruína, que já havia começado quando pela primeira vez não
acordou para as aulas, se precipitaria irrefreavelmente ao decidir deixar a
Universidade que se reformava.  É também causa específica de sua precipitação o
ser pouco avisado dos riscos dos comprometimentos da vida prática e, somado a
isso, ser inexperto nos assuntos de amor.  Mantém sem a permissão do tio uma
prometida noiva Narcisa, o que demonstra sua inadvertência; e, na hora de
partir, empenha uma bolsa de dinheiro como fraco compromisso, o que é fruto de
sua inexperiência.

Com este último gesto, na sequência do enredo convence imediatamente a precária
noiva, que se fingia desesperada e, assim pintada, representava"-se como uma
oportunista filha de outra. Ambas, mãe e filha, são tipificadas na vida
estudantil como outras tantas inimigas do bom estudo, desencaminhadoras de
jovens de letras, vivendo dos presentes e do dinheiro dos estudantes, mormente
os que vêm das mais distantes e das mais próximas províncias do reino.  Mas		%tirando dúvida com autores
Tibúrcio, experiente, lembrando antigos perrengues de viagem, primeiro tenta
convencer o herói de que deva levar consigo o dinheiro, depois remete a decisão
ao próprio braço, usando a força contra a mulher, que por seu lado recontava o
dinheiro muitas vezes.

Ainda que fruto da inexperiência mal advertida desse herói, que, empenhando a
bolsa, até poderia demonstrar alguma altivez de caráter, vale lembrar que na
baixeza geral da coisa que se narra, seu gesto visava a atingir a venalidade da
personagem feminina, coerentemente encenada recontando o dinheiro e batendo"-se
por ele.  Num universo jurídico em que a distribuição da justiça é herança
paterna e materna, segundo os direitos do reino, é significativo que, no momento
em que deserta da vida de estudante, Gonçalo diga à sua amante de juvenilidade
que só vai à província para receber uma herança que lhe teria deixado um
parente.  A mentira do herói na fábula baixa é cunhada, pois, sobre
prerrogativas, formas de representação, leis, costumes, etiquetas e hábitos
jurídicos e políticos que então eram fixados pelas hierarquias das coisas e dos
homens.

Entre as coisas políticas que o poema ensina, dever"-se"-ia entender com juízo que
o longo caminho de uma família súdita, por exemplo, ou mais ou menos bem
sucedida segundo a tradição familiar, podia ser posto a perder pela vida
desordenada de um único herdeiro.  É essa condição civil o que faz de Gonçalo o
herói dessa anti"-épica. Acresce que ele é, entre os seus, o mais forte e
promissor na expedição de indisciplinados; sendo, por isso, o melhor exemplo do
pior, para assim ensinar os maus efeitos da má conduta nos estudos. Do alto e
belo rapaz certamente a tradição familiar esperava grandes feitos, porque é
sempre melhor ser e andar vistoso para as virtudes políticas da representação
hierárquica. No entanto, o vistoso rapaz volta humilhado para a vara do tio,
que, como autoridade familiar, o aconselha e o espanca até que espume em sangue.
Assim, entre aconselhamentos mais ou menos virtuosos, e espancamentos sempre
muito violentos, e mais ou menos justos, são basicamente armadas as peripécias
que a torto e a direito terminam em pragmatografias de tumultos, não de grandes
combates. Como Gonçalo e seus companheiros,  Aquiles se desavem com Agamémnon e
os demais guerreiros. Na \textit{Ilíada}, Aquiles também remete a decisão ao
próprio braço  contra Agamémnon dentro do conselho dos reis. Mas, sob
entusiasmo, seu punho é interceptado pela mãe, a deusa Tétis, que pondera e o
faz decidir melhor. Coloca o moço no colo e lhe promete como mimo um tanto de
desastres para os gregos que, por isso, quase tiveram seus barcos queimados
pelos troianos. Sem presença divina de nenhuma espécie, a ignorância de Tibúrcio
parte para cima de Narcisa, provocando o tumulto que faria o forte, belo
e promissor advogado e recém desertor das letras perder parte dos dentes.

A elocução do poema é alta, imitando principalmente os versos brancos heroicos
de \textit{O Uraguai}, já então tornado célebre pela proteção do ministro de
Estado.  Essa dicção elevada --- que, nos melhores momentos, também pode fazer
lembrar parodicamente \textit{Os lusíadas} --- é provavelmente a causa da opinião
desfavorável da crítica dos séculos \textsc{xix} e \textsc{xx} que desmereceu o
caráter ``humorístico'' esperado desta anti"-épica, que frustra o leitor
anacrônico do século \textsc{xix} que teve outros modos de pensar, tanto o
chiste e a piada, como os discursos de intenção didática, lidos em jornais ou em
circulações acadêmicas do novo Império do Brasil.  O poema não integrava uma
instituição liberal, mas, sim, o corpo político da leal sujeição de Sua Alteza
Real, o Augustíssimo e Sereníssimo Rei Dom José \textsc{i}.  Numa classe de
homens letrados, sujeitos às ordens das aristocracias, nova e velhamente
inscritas nos livros do Reino, é preciso pensar que talvez a graça, maior ou
menor, deve ter havido.

O cômico estava fundado, de saída, na representação de distinções ajustadas
entre \textit{melhores} e \textit{piores}, que é como então se operavam as
\textit{Políticas} e as \textit{Éticas} de Aristóteles, tanto quanto as suas
artes \textit{Retórica} e \textit{Poética}.  A fábula cômica domina a ação do
poema, e é constituída pelas peripécias de um grupo de estudantes guiados por
Tibúrcio, personificação da Ignorância.  Mesmo que não se veja graça em mais
quase nada neste poema estudantil, a fábula é aristotelicamente cômica, isto é,
imita os piores, descreve matéria baixa, digna de opróbrio. Cômica por
definição: não faltam o acumulado de tipos socialmente inferiores ou moralmente
deformados, no lugar de heróis dignos da epopeia; a sucessão de cenas de brigas
comezinhas, com unhas e dentes, ou cheias de vilania, como a luta de três homens
contra uma mulher, em lugar de lutas famosas entre grandes guerreiros e senhores
de homens; daí os tumultos, em lugar de batalhas; e as bebedeiras, em lugar de
triunfos e banquetes em que se honrassem as vitórias da virtude.  A virtude,
contudo, era o fim a que deviam mover tanto a poesia heroica quanto a cômica,
com a diferença de que esta imitava antes de tudo os vícios para que se pudesse
fugir deles e aquela imitava principalmente as virtudes para que se as
perseguissem.

Neste sentido, no poema heroi"-cômico de Silva Alvarenga, a inglória viagem de
Gonçalo, Tibúrcio e sua trupe lastimável de maus alunos é uma fábula cômica
narrada como se fosse coisa heroica.  Com a entrada do ministro em Coimbra, a
Ignorância é expulsa e com ela todos os que viviam sob a falta de polícia nos
estudos.  Neste sentido, a ação burlesca é movida pela ação heroica do Marquês e
do seu prelado, o reitor reformador, também nascido no Brasil, homenageados
ambos no poema. Por um antigo lugar comum, o efeito da ação alta sobre a baixa
poderia traduzir"-se como a luz que, tão logo acesa, faz esconder"-se a escuridão
detrás dos móveis e objetos postos às claras.  Dessa forma, os seguidores da
Ignorância fogem da cidade e das ciências, para buscar a província, sem
merecimento de fama para prosseguir na carreira com distinção e louvor. A
escuridão reduzida às fendas e arestas mostra"-se, assim, por contraste, mais
escura na claridade do que nas trevas.

\pagebreak

\begin{verse}
Os que aprendem o nome dos autores, \\
Os que leem só o prólogo dos livros, \\
E aqueles cujo sono não perturba \\
O côncavo metal que as horas conta, \\
Seguiram as bandeiras da ignorância \\
Nos incríveis trabalhos desta empresa.
\end{verse}

Um é bruto, outro é apaixonado, outro é afidalgado, outro é furioso, outro é
dançarino, outro é um experto vendedor de objetos usados, que de mau estudante
foi se deixando ficar em Coimbra como um desencaminhador de jovens estudantes.
No catálogo dos heróis, no início do canto \textsc{ii}, apresentam"-se, além de
Tibúrcio e Gonçalo, os companheiros Cosme, Rodrigo, Bertoldo, Gaspar e Alberto,
cada qual segundo seu caráter constituindo um tipo, um vício, uma paixão, um
desvio de conduta, que os rebaixam nas hierarquias do mérito, codificadas nas
disciplinas morais lidas com Aristóteles e com Salomão, com Platão e São João
Evangelista.  Aqueles que não se levantavam ao sino, nem para a aula, nem para a
missa, e que retinham dos livros apenas o necessário para o fingimento
desonesto, os argumentos dos prólogos e o nome dos autores, todos esses com
todos os seus tipos, retornam obscuros para a própria província tendo perdido os
cabedais familiares com a vida devassa. 

Assim, o retorno à província --- a fictícia Mioselha na fábula do poema, rica em
queijos e tremoços, como a louva a Ignorância, pela voz de Tibúrcio --- não
representa qualquer elogio do campo, remanso da virtude, remédio para os vícios.
Longe de ser um louvor da vida retirada, propícia ao bom ócio estudioso e ao
feliz desengano das vaidades das Cortes e grandes cidades --- como se lê nas
tópicas do \textit{menosprezo da corte e elogio de aldeia}, muito recorrentes na
poesia pastoril que vogou no século \textsc{xviii} (e não só nele) ---, a buscada
vida na aldeia é aí mais propriamente um emblema da fuga para a obscuridade,
antiprêmio dado ao herói por sua fraca perseverança nos estudos. A causa próxima
e geral do retorno de Gonçalo e de todos os demais maus alunos foi a restituição
da luminosa Verdade, como se viu, alegoricamente recolocada no trono naquela
corte das \textit{scientiae}, por intermédio da ação divina e histórica dos
heróis verdadeiros do poema: Sebastião José de Carvalho e Melo, o ``invicto
Marquês'', que se lê no início da narração, no canto \textsc{i}, entrando
triunfalmente em Coimbra como restaurador das letras, e Francisco de Lemos, o
``Prelado fomidável'', de quem se queixa a Ignorância ao ver os eventos que
representam no poema o fim de seu império em Coimbra.  Gonçalo volta à pátria
Mioselha, abandonando os amores de Narcisa e a leitura de romances vulgares.
Agora fora definitivamente impedido de prosseguir sem esforço nos estudos,
graças à reforma da Universidade, assinados os novos \textit{Estatutos da
Universidade de Coimbra compilados debaixo da imediata e suprema inspecção de
El"-Rei D. José, nosso Senhor, pela Junta de Providência Literária, criada pelo
mesmo senhor, para a restauração das ciências e artes liberais nestes reinos e
todos os seus domínios}. 

O efeito cômico deveria estar em narrar como se fosse grande coisa, e com
palavras infladas, as bravatas irrisórias e os ânimos mesquinhos das personagens
da trama, pela dissociação entre o baixo da invenção da matéria e o alto da
elocução ornada com palavras graves dignas de grandes feitos.  O vitupério se
justifica como eficácia didática que exorta à virtude, uma vez que representa as
coisas dignas de vergonha com as mesmas palavras e proporcionais figuras com que
se louvam os grandes feitos. É por isso análoga da epopeia, mas inverte o valor
das matérias, das ações, dos exemplos, dos caracteres dos heróis, trocando a
lâmina do melhor metal herdado na paternidade das armas por paus e pedras que se
tomam da beira das ruas, entre gente sem herança. Ao contraste do melhor
humilha"-se o pior. Esse efeito deveria produzir o discernimento do certo e do
errado, finalidade do \textit{movere} que o poema encena, como moralidade que
deve instruir e convencer o entendimento do leitor, que rindo deleita"-se, e
deleitando"-se aprende o mau exemplo que se deve evitar.

Em sua disposição retórica, após a invocação, a proposição e a dedicatória,
inicia"-se a narração (\textit{narratio}) como um \textit{epibatérion}, isto é,
um discurso laudatório sobre a entrada triunfal do Marquês de Pombal em Coimbra
para restaurar os Estatutos da Universidade.  Louvam"-se a chegada e os feitos do
Marquês, encomiando"-o a partir da exposição das boas qualidades das ciências que
o acompanham para alegoricamente restaurar os princípios da Verdade que deveriam
nela reinar. Por ser retórica em sua invenção (\textit{inventio}), a descrição
das matérias --- sejam os homens em ação (\textit{pragmatografias}), sejam o
caráter e o costume deles (\textit{etopeias}), sejam as cidades e lugares
(\textit{corografias}), entre outras, seguem ordenamentos previstos na execução
do discurso ---, é produzida como quadros verbais que retoricamente louvam ou
vituperam seu objeto, segundo as convenções da descrição (\textit{ecfrasis}).
Seja a descrição de feitos dignos de memória, seja a descrição de coisas torpes
e infames, conforme os preceitos, as palavras devem pintar na fantasia com vivas
cores, de modo a tornar vívidos para o leitor os episódios narrados na invenção
da fábula.  Por isso, quando Tibúrcio fala de Mioselha, a partir do lugar comum
do elogio da aldeia revertido pelo caráter cômico da invenção, sabe"-se que o
poeta pode lançar mão da origem da cidade, da antiguidade dela, da situação
atual, das realizações virtuosas de seus varões ilustres, do esplendor da
paisagem e do clima, da fertilidade do solo, da alegria dos viventes, das
comidas saborosas que os aguardam etc. Como nesta moral os maus estudantes são
desviados do bom caminho pelos gostos mais baixos, não é decerto casual que, além
das festas de romaria, seja a abundância de queijos e mais víveres que componha
o argumento de Tibúrcio para persuadir Gonçalo da felicidade que encontraria na
província, longe do peso dos estudos.

Assim, procedimentos previstos moderam os efeitos gerais das representações
poéticas.  As descrições dos tipos, das cidades, dos combates, dos enganos e
desenganos consideram seres particulares conforme sistemas gerais e específicos
de classificação dos seres.  Num sistema que, sendo retórico, devia tornar os
ouvintes dóceis e atentos ao dizer, a personificação dos hábitos dos
companheiros que seguem os enganos da Ignorância é inscrita em regras de
representação compartilhadas pelo leitor presumido, o que auxiliava a produção
do artifício, como uma máquina discursiva em que os versos deveriam formar uma
unidade para o entendimento dos leitores que ali reconhecem costumes políticos e
poéticos da representação. Assim, as personagens são tipos, não representam
individualidades psicológicas. Por isso, o catálogo, ou enumeração, das figuras
justapõem as representações individuais sem que os caracteres de uns e de outros
sejam comparados entre si. Por isso também, cada episódio contém uma unidade que
lhe é própria, geral e particularmente, sem ter com o todo da trama, que é uma
ação bastante simples, mais relação de necessidade do que  a disposição funcional dos encadeamentos básicos entre um evento e outro. Assim é que, no canto
\textsc{iii}, é introduzido uma personagem nova, Rufino, apenas porque é amante
da filha do carcereiro, ao mesmo tempo que a Ignorância alegorizada via que os
desertores, seus últimos súditos, estavam perdidos nas montanhas e em pouco
tempo acabariam presos pela multidão de que fugiam, furiosa pelos maus feitos
que no regresso aqueles estudantes já tinham deixado pelo caminho. 

%, como quando Tibúrcio invade o quarto de Amaro
%fingindo ser um fantasma ao mesmo tempo em que Doroteia segue para a cela de
%Gonçalo. % ou, em casos mais simples, como quando ao mato espesso no qual os
%jovens se escondem se pode atribuir qualidades semelhantes às da floresta onde
%Rufino sofre por desencanto de Doroteia.
%E, neste caso, como o lugar tenha alguma similitude em afeto com a pessoa, pode também
%o poeta inventar uma similitude entre o lugar e o caráter daquele que chora, que bebe ou que boceja.
%Em tratados cosmográficos no \textsc{xvi} (aqueles que dão conta da descrição de todas as coisas
%que podem existir debaixo do Sol, desde o \textit{Gênesis}) é comum a comparação de objetos
%particulares com partes do corpo humano. Por isso, quando se descreve a geografia do
%mundo deve"-se buscar semelhanças com a cabeça de um homem, assim como quando se descreve
%um lugar em particular (uma cidade, por exemplo) deve"-se buscar semelhanças com partes
%do homem, um olho ou uma orelha, por exemplo.

Como é um poema anterior ao romantismo e à constituição da literatura como
mercado editorial, o poema talvez nem devesse passar por ``apreciações
estéticas'' no ``crivo'' da crítica literária.  Neste fim do século
\textsc{xviii} português em que se formou o poeta, mesmo no fogo cruzado das
reformas pombalinas, ``crítica'' talvez devesse ser entendida como o uso do
juízo segundo as leis naturais de Deus e segundo as leis positivas e os costumes
dos homens, conforme os direitos de cada Estado político.  No tempo da Real Mesa
Censória, que desde 1768 unificava as mesas eclesiásticas e a mesa civil em
Portugal, não poderia haver jornalismo de profissão, muito menos exercício de
``crítica literária'' no sentido moderno, no qual a livre iniciativa individual
se entendesse intelectualmente apta para avaliar as obras de Poesia pressupondo
a autonomia delas.  Antes da formação da noção de literatura como campo
autonomizado e antes da instituição das artes como exercício da livre reflexão
do juízo subjetivo mais ou menos consciente de realidades sociais que a
subjetividade autonomizada hoje julga poder imprimir e exprimir como arte,
``crítica'' era o ato de uma faculdade da alma cujo fim era discernir o acerto
do erro, o justo do injusto, o belo do feio, e assim por diante.\footnote{ 
Essas outras formas com que se definiu a palavra ``crítica'', não pressupunham o
uso kantiano que a reinstauraria sem teologia para outras formas de compreender
a alma ou os limites do conhecimento humano, nem poderiam supor os usos
marxistas e positivistas em cujo discurso provavelmente se cunhou a acepção mais
corrente atual, em enunciados que, no âmbito da crítica literária, por exemplo
passaram a inferir ``crítica social'' em textos alegórico"-devocionais ou
satíricos de Gil Vicente a Gregório de Matos, para só citar os mais próximos e
conhecidos hoje.} Especificamente ``crítica literária'', só poderia ser
entendida como a \textit{crisis}, isto é, o exercício do \textit{entendimento}
(ou \textit{intellectum}) --- a mais alta faculdade inculcada por Deus na alma
humana --- sobre as ``coisas das letras'' (\textit{litteratura}), dentro das
instituições poéticas e oratórias que o uso ordenava conforme a fins que não
seriam jamais fechados na contemplação da própria arte.\footnote{ São
inumeráveis os desenvolvimentos da opinião kantiana acerca da definição das
belas artes na \textit{Crítica do juízo}, como aquelas em que o fim não se
encontra fora dela mesma.} Para o século em que sai \textit{O desertor}, tanto
os poemas mais obviamente \textit{didáticos} quanto os poemas mais aparentemente
\textit{fúteis} (dois adjetivos que seriam usados para caracterizar o poema
heroi"-cômico) tiveram fins morais e políticos que excediam as bordas imaginárias
da própria arte. Mas tratava"-se de fins constituídos por princípios que a
própria arte, como técnica, definia para si, segundo a tópica horaciana segundo
a qual o melhor poeta será aquele que ao agradável unir o útil, de modo que
ensine e deleite ao mesmo tempo.

O poema heroi"-cômico não é uma reação à poesia épica e cavalheiresca, como já se
disse na crítica literária que falou deste poema. Ao contrário, vai no mesmo
sentido destas, ainda que seja uma espécie de poesia \textit{menor}, por
definição.  Não obstante, exorta à virtude fazendo falar a Verdade. No poema de
Silva Alvarenga, a Verdade fala, literalmente, na forma de uma alegoria sonhada
como num afresco que imitasse um episódio mítico grego. Como define o próprio
poeta no ``Discurso sobre o poema heroi"-cômico'' que introduz \textit{O
desertor}, trata"-se de narrar heroicamente eventos cômicos, o que realça a
baixeza das ações. Em suas palavras, ``o poema chamado heroi"-cômico [\ldots{}] é a
imitação de uma ação cômica heroicamente tratada.'' No seu poema, a celebridade
da fama que a poesia heroica cantava é contrastada com a irrisão da fuga indigna
para o ostracismo, como se a moral da história assim ensinasse, sacudindo o dedo
com um irônico \textit{memento}: lembre"-se, homem, os \textit{grandes} cantos
que se farão a respeito de sua má inclinação na vida, trocando a fama por
infâmia, a dignidade por indignação, a memória pelo desprezo da posteridade. A
fábula  heroicamente escrita faz os que perseguiam os maus costumes lembrarem"-se
de que não haveria menção de sua vida nem mesmo na posteridade da família. 

A obscuridade na carreira e o esquecimento da posteridade são o que a fábula
cômica ameaça ao mau letrado, a quem a História não irá assinalar como varão,
nem deles se escreveriam \textit{vidas} a inventar os pormenores do verossímil
particular, que constituiriam a posteridade de sua fama, nem os encômios iriam
comemorar o monumento e o documento de suas obras; muito menos irão tornar"-se
modelos imitáveis de alguma coisa, ou suas palavras continuarão a mover as
gerações após a sua morte. Homens letrados como Vieira e Gracián provavelmente
cursaram a carreira sob o \textit{desejo}, digamos assim, desta última
possibilidade; talvez também um advogado e diz"-que excelente poeta, Gregório de
Matos e Guerra, é provável que tenha sido parcialmente salvo da obscuridade,
graças às \textit{vidas} que dele deixaram, provavelmente não sem interesse,
algumas tradições de compiladores, que as inventaram segundo a convenção do
gênero.  Silva Alvarenga participa de um sistema meritório mais ou menos
codificado assim ainda, com algumas particularidades, pois é bom lembrar que
Pombal nos anos de 1750, em nome da povoação dos domínios da cristandade		\EP
portuguesa, dera altura de português em quase tudo a colonos de comprovada
utilidade nos serviços de Sua Alteza Real, confirmados na lealdade da Monarquia
Lusitana e da Santa Fé Católica. Além disso, encoraja as uniões matrimoniais
entre colonos e libertas convertidas, em nome do universalismo cristão que o
reino constituía como positivação política.  Na condição política de filho de
africanos, como quiseram alguns, ou mais provavelmente de indígenas americanos
por parte de mãe, ser advogado é decerto uma possibilidade jurídica recente para
a família que alcançou colocar Silva Alvarenga na Universidade de Coimbra, por
exemplo.

Nas duas edições impressas em vida do autor (1774 e provavelmente 1788), o poema
de Silva Alvarenga saiu com o título composto \textit{O desertor: poema
herói"-cômico}. Contudo, foi comumente referido como \textit{O desertor das
%MANTER AS PARTICULARIDADES SÓ AQUI.  
Letras}, desde Balbi, talvez porque assim se declarasse mais especificamente o
argumento do poema, talvez porque assim tivesse soado bem. Mas certamente, para
um poema heroi"-cômico, \textit{O desertor} é um título suficiente para declarar o
seu teor, pela natureza do desvio que o sentido bélico do termo já representa.
De qualquer maneira, a fórmula \textit{o desertor das letras}, que quase se
tornou mais conhecida que o título original, foi retirada do primeiro verso do
poema. Ali, como se viu, o poeta invoca a Musa e já enuncia a matéria do poema
--- ``Musa, cantai o desertor das letras'' --- seguindo daí em elocução camoniana
a súmula dos principais sucessos do anti"-herói que protagoniza a narrativa.  A
elocução é alta e grave como n'\textit{Os lusíadas}, mas em tom paródico, uma
vez que a gravidade da dicção não corresponde decididamente à baixeza da
matéria, cuja depreciação só a palavra \textit{desertor} já bastaria para
declarar. No primeiro verso da \textit{Ilíada} --- ``Canta"-me, ó Deusa, a ira
funesta de Aquiles Pelida'' ---, o poeta pede à divindade que inspire o canto e
imediatamente propõe a matéria do poema nomeando o herói (Aquiles), sua
ascendência (filho de Peleu) e a paixão que o tipifica na qualidade de bravo
guerreiro, sendo o destempero de sua ira a causa dos maiores danos ocorridos em
Troia.

Neste sentido, pela inversão paródica, o primeiro verso de \textit{O desertor}
emula a primeira autoridade grega da poesia narrativa (diegética) de
matéria heroica, a epopeia; mas não declara de imediato o nome do herói infame.
Na maior parte das doutrinas épicas que até o século \textsc{xviii} tiveram
circulação, a epopeia devia imortalizar o nome dos heróis, tornando duradoura a
sua \textit{kléos}, isto é, a sua \textit{fama}, juntamente com a de seus pais e
de seus filhos.  Em perspectivas cristãs europeias, que supunham por exemplo a
guerra justa, e em perspectivas monárquicas e aristocráticas, que ordenavam o
Estado sob parâmetros da hierarquia militar, a poesia heroica da epopeia
homérica deveria ensinar aos moços as virtudes de guerreiro --- constância,
disciplina, temperança e principalmente a coragem.  O bom guerreiro deveria ter
coragem porque a fama, que, mormente em regimes patriarcais monárquicos,
legitimava o mando herdado pelo tronco familiar, era explicada como decorrência
dos perigos por que passaram os mais antigos senhores da terra e que as
narrativas antigas teriam conservado com este fim. A ira mesma, embora fosse
classificada como uma paixão da alma e por isso devesse ser subjugada pelo
intelecto conforme doutrinas da alma vigentes, podia ser funesta por suas
consequências, mas, como alguns moralistas da época expuseram, essa paixão
poderia servir de ornamento da bravura, aumentando o seu efeito. Segundo esse
raciocínio,  desde que se evitasse o extremo que leva a impiedades excessivas,
como as de Aquiles irado, a ira pode tornar mais forte a coragem, desde que
dirigida a causas tidas por justas.  Daí que a ira de Gaspar ou de Tibúrcio não
possa ser qualificada senão como baixeza de caráter, porque a dirigem ora a um
velho que fala coisas justas, ora a uma moça que por eles tinha sido enganada.
Assim, também entendia"-se inversamente o herói de \textit{O desertor}
apresentado pela palavra que, dando só ela título ao poema, já o desqualificava
pelo vício fraco da \textit{covardia}, que faz um mau guerreiro debandar dos
trabalhos da guerra, como analogamente Gonçalo e companhia desertam dos esforços
do estudo.

Aproximadamente assim ensinavam tanto doutrinas poéticas como doutrinas
políticas acerca da origem do poder hereditário dos senhores bem como acerca das
razões políticas da narrativa heroica.  Ao preferir invocar a Musa, o poema
também declara a emulação da \textit{Eneida} de Virgílio, que canta as altas
peripécias do pio Eneias e que no oitavo verso enuncia --- ``\textit{Musa, mihi
causas memora}'' (``Musa, relembra"-me as causas''). Com efeito, como na
\textit{Eneida}, ao invés de pedir à Deusa da memória, Mnemosine, o poeta de
\textit{O desertor} invoca as Musas, sem nomear em momento algum, como era
costume, qual delas seria a que dominava no específico da sua arte. A ocultação
do nome da divindade podia estar prevista nos usos da arte, mas num gênero  deliberadamente misto como o heroi"-cômico a Musa não indicada
principalmente mantém dúbia a natureza da espécie poética, já
também por isso cômica. Com isso, pode fingir a semelhança com a arte solene de
Calíope, mas logo demonstra o riso indolor de Talia, isto é, efetivamente o
poema pertence à arte da comédia, pela matéria que inventa, e à arte da epopeia,
pelo estilo e modo com que inventa.

%, ..que com que equivoca, entre Calíope, musa da epopeia,
%e Talia, musa da comédia, os gêneros poéticos já indicados
%no subtítulo ``poema heroi"-cômico'', 
%Imita consequentemente diversos autores ..o herói e o vício forte
%que o caracteriza como irascível

Na \textit{Odisseia}, Palas Atena se transforma num velho sábio e vai até o
jovem e ajuizado Telêmaco, para que tome a função paterna, assuma os trabalhos
da necessidade e com coragem vá buscar notícia do pai. De forma análoga, a
Ignorância, em \textit{O desertor}, transforma"-se num antiquário que vivia em
Coimbra, porque havia frustrado a carreira nas letras, e vai até Gonçalo
dissuadi"-lo de seguir nos estudos, porque são inúteis os esforços das letras e,
com as luzes das reformas recentes, já não havia mais ali as liberdades para o
gozo dos gostos baixos que antes se proliferavam como vícios.  Como na poesia
heroica a virtude capital é a coragem, na fábula do poema heroi"-cômico, ao
contrário, só era preciso a Gonçalo vencer o medo do tio, que mantinha os seus
estudos e o haveria de espancar quando retornasse sem tomar grau.

O herói dessa antiépica devia dedicar"-se às leis e às ciências, à teologia, à
jurisprudência, mesmo à história que deleita mas ensina, como a boa poesia. Em
vez disso, lia maus romances, \textit{litteratura} muito lidas e mal reputadas
na preceptiva da arte. Mesmo sendo letras vulgares, representam a moral, por isso vendidas na conformidade da lei. Mas eram livros mal reputados,
como livros de pouca ciência e pior arte, que mais \textit{deleitavam} do que
\textit{utilizavam}, como se dizia, o que não convinha às boas letras. Na tese
que o poema encena, essa tipificação do mau estudante e seu mau hábito de
leitura eram efeitos das dificuldades dos maus livros que se ministravam na
Universidade no tempo dos jesuítas.  Na lógica do poema, os estudantes começavam
a desertar das letras quando se entregavam a leituras mais doces do que
docentes, mais agradáveis do que úteis para causar virtude. E a causa disso, na
propaganda pombalina, era também a dificuldade desnecessária das questões dos
padres chamados peripatéticos, como os famosos Sanches e Molina.

Ivan Teixeira, em seu estudo sobre a poesia nos círculos de mecenato pombalino,
reproduz uma gravura pombalina onde se veem os padres da Companhia de Jesus
derrubando uma árvore do conhecimento. Nos seus galhos e ramos, viam"-se as
figuras dos maiores doutores da Igreja, como São Jerônimo, Jasão de Nores, Tomás
de Aquino, Santo Agostinho, o Venerável Beda, São Gregório Magno, entre outros,
e sobre eles o Espírito Santo, o sagrado coração e a imagem de Deus pai.  O
fruto das obras desses santos padres é o que o título da gravura refere como
\textit{O trabalho perdido}, isto é, o estrago orquestrado pelos padres gerais
da Companhia de Jesus, que o gabinete pombalino faria passar por uma república
dentro da República e, por isso, perigosa para a soberania do Rei e de seus
herdeiros no reino. 

Essa árvore de autoridades das maiores ciências para o catolicismo pombalino
aparece sendo serrada justamente por Sanches e Molina, assim indicados em lema
na gravura.  Ambos são teólogos assim chamados casuístas, comentadores da
\textit{traditio} de doutrina que a Universidade ensinava. São também os mesmos
vituperados entre outros na estante do tio, no canto \textsc{v}, de \textit{O
desertor}.  Com efeito, o Concílio de Trento continuava publicado e ensinado em
Portugal, enquanto a maior parte dos autores associáveis ao assim chamado
Iluminismo, principalmente francês, continuavam proibidos no reino, como também
ao contrário muitos autores jesuítas mais respeitados durante o auge da segunda
escolástica continuavam a ser modelos exemplares de virtude e de doutrina para o
ensino, como é o caso dos santos fundadores da Companhia de Jesus e de outros
nomes de jesuítas célebres como Francisco Suarez e Antônio Vieira. 
%% por outras
%razões não são reconhecidos como dignos na ciência de parte do que ensinam
%Gracián e Tesauro, porque a comum opinião acerca do bom estilo na escrita das
%cartas, dos ofícios, o que recaía sobre o estilo dos livros de história fingida
%ou verdadeira, sobre os livros de desengano amoroso, nos louvores de feitos
%ilustres etc.

A Ignorância, a vilã da trama cômica, é expulsa de Coimbra por efeito daquela
ação heroica que restituía a Verdade ao seu trono na velha instituição de
ensino.  A Verdade é uma espécie de divindade patronal do poema, que se
representa pela emulação de Minerva, a Palas Atena dos poemas de Homero.  Com
efeito, a Verdade aparece em sonho ao herói da fábula cômica que já derrocava
para em vão tentar convencê"-lo a decidir"-se por melhores feitos, para obter
melhor fama. Mas o entendimento já estava decidido pelo mau caminho, e as
circunstâncias levavam sempre o herói a outras piores circunstâncias, até que
chegam à província. Na biblioteca do tio, veem"-se grandes livros de um século
antes. Não é só a literatura seiscentista vituperada, é o que de melhor o tio
pôde juntar no seu tempo. Mas as reformas do ensino e o próprio tempo fizeram
muitos daqueles livros descorarem de sua autoridade antiga. Perdida a carreira,
Gonçalo agora não teria mais do que isso. Uma biblioteca de aldeia, com todas as
marcas do tempo, para eventualmente tornar"-se um rábula, advogado sem diploma
que, no contraste das luzes que o poema forja, é a mais escura obscuridade.

Boécio e Tomás de Aquino, padres da Igreja dos primeiros e dos últimos tempos da
Idade Média, eram no século \textsc{xviii} distintos como santos ou beatos, entre
outras razões pela doutrina que retiraram sobretudo do ``divino
Aristóteles''.\footnote{ Essa adaptação de Aristóteles à teologia católica
refeita sobretudo ao longo do século \textsc{xviii} por grandes autoridades
eclesiásticas, mormente jesuítas, ficou conhecida como segunda escolástica, cujo
fim era fortalecer os fundamentos filosóficos que embasaram as cláusulas do
Concílio de Trento, defendidas contra a heresia pelo Tribunal do Santo Ofício da
Inquisição.} Com o fim da era jesuítica no ensino português, a reforma pombalina
da Universidade representou também uma relativização do aristotelismo que se
ensinava na Universidade até o século \textsc{xviii}. Com isso, porém, não se
chegou a efetivamente destituir a autoridade de Aristóteles como o mais
importante filósofo grego para a doutrina católica, que permanecia sendo a
religião do rei e do reino. Aristóteles assim havia sido considerado muitas
vezes desde a primeira escolástica do século \textsc{xiii} e, muito antes dela,
desde alguns dos primeiros doutores da Igreja dos séculos \textsc{v} e
\textsc{vi} d.C.  Sem nem de longe ferir a autoridade de nomes como esses, e
entre tantos outros nomes tão ilustres, mais de uma vez o poema de Silva
Alvarenga faz alusão aos maus métodos dos peripatéticos, que é o nome com que
são designados os seguidores de Aristóteles. Contudo, ali se fala mal
principalmente das apostilas e cadernos, das antologias e compilações, feitos
por professores portugueses que ensinavam lógica aristotélica por métodos que
então foram postos em descrédito.  Nem por isso Silva Alvarenga deixa de inserir
antes do poema um ``Discurso sobre o poema heroi"-cômico'' que começa justamente
citando Aristóteles como respeitada autoridade no ensino das regras da arte
poética e dos fins morais, gerais e particulares que se perfazem na leitura da
poesia em geral e do poema heroi"-cômico em específico.

Como se viu, o tempo que ficou conhecido pela posteridade como ``período
pombalino'' teve início em 1750, um ano após o nascimento de Manuel Inácio da
Silva Alvarenga, e, por ocasião da impressão de seu poema, Pombal acabava de
levar a termo, em 1772, a célebre Reforma da Universidade de Coimbra. A
principal instituição de ensino portuguesa havia sido posta em descrédito,
segundo a propaganda pombalina, por culpa dos membros da Companhia de Jesus.
Desde o século \textsc{xvi}, os jesuítas a geriram e nos últimos tempos teriam
deixado medrar maus hábitos entre professores e estudantes, consequência dos
métodos antiquados que usavam, sempre segundo a opinião que a política de Pombal
fez imperar. 

Desta mesma opinião politicamente produzida, \textit{O desertor},
esse exercício poético de estudante comprometido com as novas diretrizes da
Universidade, é uma peça casualmente estratégica. Alfredo Bosi entendeu essa
posição como a de um típico militante ilustrado. Talvez, menos do que isso, a
posição de Manuel Inácio da Silva Alvarenga seja a de um bom súdito.  Ao lado de
muitas outras obras que tiveram o mesmo comprometimento, \textit{O desertor}
integrou o que Ivan Teixeira chamou ``mecenato pombalino'', caracterizando com
essa expressão o agrupamento de poetas, artistas, juristas, eruditos,
professores etc, em torno de Sebastião José de Carvalho e Melo, empenhados no
louvor de seu governo.  Acordes em perpetuar em monumentos de memória as
reformas implementadas, trataram"-nas como o nascimento de uma nova era, ou como
o renascimento de uma idade áurea antiga, que se vestia por exemplo como um novo
século de Augusto, cantado por outros Virgílios e Horácios lusitanos, que
obviamente não eram iluministas, nem poderiam simpatizar com opiniões tão
perigosas, para o mundo católico e monárquico de que eram parte.

Em contraste com essa nova Idade do Ouro, assim pintada em prosa e verso, o
jesuitismo reduzido a tipos sórdidos, vilões de comédia, ou alegorizado como
Hipocrisia, Abuso, Ignorância, Monstro de mil olhos foi representado como
trevas. Com isso, a historiografia literária quis equivocar sombras barrocas,
renascidas de trevas medievais, provavelmente para imputar iluminismo na
política de Estado do Marquês e de seu séquito de letrados.  Fugindo destas
positivações metafóricas de antigos artifícios, já que luzes e sombras aí são só
metáforas, podemos dizer que, para a comum opinião sob Pombal, a gestão
jesuítica da Universidade teria imposto a seus currículos velhos métodos,
baseados fundamentalmente na leitura católica da lógica aristotélica.


Desde os séculos \textsc{xix} e \textsc{xx}, com o fim de enquadrar o Marquês de
Pombal no Iluminismo europeu, foi recorrente a interpretação historiográfica que
quis dar à expulsão dos jesuítas um caráter anticlerical, como se suas reformas
visassem a atingir o clero português e assim seu governo tivesse uma posição
assimilável à de outros monarcas e outros ministros que ficaram conhecidos como
``déspotas esclarecidos'', interpretados como parte, mesmo que contraditória, de
um movimento geral da Ilustração.  Contudo, a luta institucional do governo
pombalino é quase que exclusivamente dirigida à Companhia de Jesus nas pessoas
de seus membros atuais, que obtiveram poder provavelmente pelo favorecimento
institucional, nas dependências de Dom João \textsc{v}, falecido em 1750.
Acresce que os lugares institucionais anteriormente ocupados pelos jesuítas,
sobretudo os relativos à educação, viriam a ser dados, quase sempre, também a
padres, mas agora preferencialmente os oratorianos, ordem religiosa de origem
francesa, ascética em sua doutrina de vida, como a dos inacianos, e
misteriosamente interessada na vida política, como aqueles, sobretudo na
instrução dos homens também.

O Portugal de \textit{O desertor} não se tornava mais ilustrado nem se laicizava
além da conta e das tradições jurídicas conhecidas.  A disputa representou"-se
como uma querela institucional que teve o tamanho que teve.  Não precisaria ser
reinventada como imbuída de significados transistóricos, que organizam o
trânsito do Espírito, das ``manifestações culturais'' e das ``mentalidades'' das
épocas.

Todo o Estado português continuou a ter muitos clérigos em seus postos mais ou
menos altos, tanto nos Conselhos do Estado, quanto nas instituições de ensino e
na Mesa Censória dos livros impressos no reino.  As figuras mais típicas do
assim chamado Iluminismo português, como Luís Antônio Verney e Francisco José
Freire, eram igualmente clérigos que tinham o Concílio de Trento como a
verdadeira e grande restauração moderna das ciências, o que torna muito estreito
o que de Iluminismo o pensamento sem dúvida \textit{ilustrado} desses homens
pode ter representado; mesmo porque ``ilustrado'' é um termo que em português
sempre significou culto, erudito, cheio de ciências, sendo a Teologia, antes
como neste século \textsc{xviii} ibérico, a mais alta das ciências.

Conhecedor dos sistemas de gênero e espécies que desde Aristóteles operavam, de
várias formas, a invenção, a disposição e a elocução da poesia \textit{em
geral}, isto é, \textit{enquanto gênero} de imitação, conhecedor também dos
sistemas jurídicos civis e eclesiásticos que regiam a Monarquia portuguesa, o
poeta estava longe de pertencer a um ``\textit{club} de jacobinos'', como se lê
nas acusações da Devassa. Aliás, ``\textit{club} de jacobinos'' foi
provavelmente uma agudeza vituperante que põe em evidência algumas posições de
uma cena política coetânea. O \textit{club} é uma sociedade pacífica de pares
que se distinguem mutuamente como \textit{socii} (sócios).  Grafada à maneira de
ingleses, \textit{club}, a sociedade de pares é representada, no vitupério, como
coisa de anglicanos, gente que por mais polida que fosse era sectária do credo
decretado herético havia dois séculos pelas mais altas cúrias eclesiásticas que
a Monarquia lusitana acatava integralmente. A \textit{club} junta"-se o adjetivo
\textit{jacobino}, o que haveria de pior e mais horroroso em termos de impiedade
política laica, no ponto de vista, ou melhor, segundo a ética portuguesa de
então.  Assim, a infâmia que recaía sobre Alvarenga se indiciava por meio desta
prova: em ditos mordazes falava"-se do grupo de Silva Alvarenga como um conluio
de sujeitos mistos de revolucionários franceses e anglicanos hereges reunidos em
sociedade aparentemente pacífica.  A \textit{Devassa} que Silva Alvarenga
enfrenta acusa"-o de \textit{francesia}, associando"-o a opiniões revolucionárias
francesas, isto é, à opinião política que sustentava gente desqualificada nas
hierarquias políticas do reino ocupando ilegitimamente o lugar do rei. Para a
contemporaneidade portuguesa de Silva Alvarenga na década de 1790, certamente o
perigo disso sentia"-se como enorme, mas não se marcavam esses eventos
particulares que hoje são \textit{a Revolução Francesa} senão como um distúrbio
assimilável às sublevações que as histórias antigas e modernas nunca deixaram de
contar.

\section{Sobre a edição}

O texto desta edição foi estabelecido a partir da edição de 1774,
utilizando o exemplar do Instituto de Estudos Brasileiros (\versal{ieb}-\versal{usp}).
Para decisões específicas, valemo"-nos também da edição de Joaquim
Norberto, de 1864, da edição feita por Ronald Polito, de 2003, e
das indicações da tese de Francisco Topa (ver, abaixo, Bibliografia).

Foi feita a atualização ortográfica, mesmo dos nomes próprios. Mantiveram"-se
apenas as maiúsculas, porque muitas vezes são empregadas para dar sentido
alegórico a conceitos abstratos, como são os casos emblemáticos para o poema
em questão das palavras ``Verdade'' e ``Ignorância'', que em quase todas as
ocorrências representam"-se personificadas em ação alegórica.
Conforme observa Ronald Polito, diversas palavras aparecem grafadas com maiúsculas,
``no entanto não são homogêneos os critérios da primeira edição'' [p. 61].
Contudo, muitas vezes também ``arrieiros'', ``estudantes'', entre outros substantivos
comuns em uso aparentemente simples são grafados com maiúsculas sem critério identificável.
Sendo assim, atendemos a argumentação de Francisco Topa: ``em atenção ao \textit{usus
scribendi} do autor e aos hábitos da época \textit{é possível} conservar maiúsculas
não justificáveis gramaticalmente, atendendo também ao seu possível valor expressivo'' [p. 19].
Como não oferecem dificuldade para a leitura, mantivemos as maiúsculas,
também para relativizar o esquematismo que define alegorização de conceitos por
meio de maiúsculas. É possível que maiúsculas também indiquem ênfases para a
\textit{pronuntiatio}, já que, na época, estava prevista a \textit{performance},
ou representação do poema heroi"-cômico, assim como se fazia encenação pública do
poema heroico. Por esse aspecto residual do uso que o poema teve, achamos interessante
a reprodução das maiúsculas e minúsculas conforme a edição de 1774. % PRECISO REVISAR AS MAIÚSCULAS COM O fIND.

Não obstante tudo isso, preferimos atualizar a pontuação para facilitar a leitura moderna.
Por se tratar de um texto narrativo, a pontuação retórica, provavelmente também
seguindo critérios da pronunciação, dificulta a fluência da leitura silenciosa
deste texto cuja unidade para o entendimento já é, de saída, bastante difícil, seja pelos cortes bruscos presentes no poema, que talvez indiquem a precariedade da composição do entrecho, 
seja por estar inscrito em outro registro de representação ficcional, diverso por exemplo
da narrativa de romance em prosa, com a qual, desde o século \versal{xix}, tendemos a estar mais
acostumados.
Neste mesmo sentido, preferimos inserir aspas para sinalizar falas diretas das personagens,
que não são muitas e não devem ser confundidas com as apóstrofes da voz heroi"-cômica que
narra e que, vez e outra, interpela fantasticamente as próprias personagens, como estava
previsto na convenção da poesia épica em geral.
Tais apóstrofes mantivemos sem alteração mais do que a atualização já referida da
pontuação, algumas vezes trocando exclamação por interrogação.
A pontuação sempre que possível não foi inserida para não fechar as possibilidades
abertas de leitura. Principalmente, foram trocados os sinais de dois pontos declamatórios,
que se sucediam por exemplo nas enumerações, indicando a disposição da matéria em orações
correlatas. Nestes casos foram trocados por vírgula ou ponto"-e-vírgula. Nos símiles homéricos,
isto é, nas comparações extensas, que são abundantes por conta do gênero do poema,
mantivemos os dois pontos marcando os dois hemistíquios da analogia. As notas apostas ao poema são de Silva Alvarenga, por isso optamos pela composição de um glossário de termos poéticos, históricos, biográficos e geográficos que se encontra ao fim do volume. Ainda para apoiar a leitura do texto inserimos antes de cada canto um argumento, com a súmula da ação que irá transcorrer.



\section*{Bibliografia}

\begin{Parskip}
\textsc{alvarenga}, Manuel Inácio da Silva. \emph{Obras poéticas de Manoel
Ignacio da Silva Alvarenga (Alcindo Palmireno) collegidas, annotadas, e precedidas do juízo crítico dos escritores nacionais e
estrangeiros e de uma notícia sobre o autor e suas obras e acompanhadas de documentos históricos}, org. J. Norberto de Souza.
Paris/ Rio de Janeiro: Garnier Irmãos, 1864. Brasilia Bibliotheca dos Melhores Auctores Nacionaes Antigos e Modernos:
Silva Alvarenga.

\_\_\_\_\_\_. \emph{O desertor: poema herói-cômico}. Coimbra: Na real
officina da Universidade, 1774.

\_\_\_\_\_\_. \emph{O desertor: poema herói-cômico}, org. Ronald Polito.
Campinas: Editora da Unicamp, 2003.

\textsc{alvear}, D. A. \& \textsc{dávila}, D. J. Herrera. \emph{Coleccion de tratados
breves y metodicos de Ciencias, Literatura y Artes: Biografia
Antigua}. Sevilla: Imprensa de D. Mariano Caro, 1829.

\textsc{ambrósio}, Renato. \emph{De rationibus exordiendi: os princípios da
história em Roma}. Associação Editorial Humanitas – Fapesp,
2005.

\textsc{aristóteles}. \emph{Rettorica et poetica d’Aristotile. Tradotte di greco
in lingua vulgare Fiorentina da Bernardo Segni Gentilh’huomo,
\& Academico Fiorentino}. Vinegia: per Bartholomeo detto
l’Imperador, \& Francesco suo genero, 1551.

\textsc{balbi}, Adrien. \emph{Essai Statistique sur le royaume de Portugal et
D’Algarve, comparé aux autres états de l’Europe, et suivi d’un
coup d’oeil sur l’état actuel des sciences, des lettres et des beaux-arts parmi les Portugais des deux hémisphères}, vol. \textsc{ii}. Paris: Rey et Gravier, 1822.

\textsc{barbosa}, Januário da Cunha. \emph{Revista trimensal de História e 49
Geografia ou Jornal do Instituto Histórico e Geográfico Brasileiro, fundado no Rio de Janeiro sob os auspícios da Sociedade
Auxiliadora da Indústria Nacional, debaixo da imediata proteção
de S. M. I. O Senhor D. Pedro \textsc{ii}}., vol. \textsc{iii}. \textsc{ihgb}, 1841.

\textsc{bluteau}, Rafael. \emph{Vocabulario Portuguez e Latino, aulico, anatomico, architectonico, bellico, botanico, brasilico, comico, critico,
chimico etc.} Coimbra: no Collegio das Artes da Companhia de
\textsc{jesu}, 1712. Autorizado com Exemplos dos Melhores Escritores
Portugueses e Latinos e offerecido a El Rey de Portugal D. Joao
V pelo padre D. Raphael Bluteau Clerigo Regular, Doutor na
Sagrada Theologia, Pregador da Rainha de Inglaterra Henriqueta Maria de França, e Calificador no Sagrado Tribunal da
Inquisição de Lisboa.

\textsc{boxer}, Charles. \emph{O império marítimo português. 1415--1825}. São
Paulo: Companhia das Letras, 2008. Tradução Anna Olga de
Barros Barreto.

\textsc{camões}, Luís Vaz. \emph{Os Lusíadas}. Em casa de Antonio Gonçalvez
Impressor, 1572. Com priuilegio Real. Impresso em Lisboa,
com licença da sancta Inquisição, \& do Ordinario: em casa de
Antonio Gõnçaluez Impressor.

\textsc{candido}, Antonio. \emph{Formação da literatura brasileira}, vol. \textsc{ii}. São
Paulo: Itatiaia/Edusp, 1975.

\_\_\_\_\_\_. “Os poetas da Inconfidência.” \textsc{ix} \textsc{Anuário da Inconfidência} (1993): 130--137.


\textsc{coleridge}, Henry Nelson. \emph{Introductions to the Study of the
Greek Classic Poets. Designed principally for the use of Young
persons at School and College. Part \textsc{i}: General Introduction. Homer.} London: John Murray, Albemarle Street, 1834.

\textsc{foucault}, Michel. \emph{O que é um autor?} Lisboa: Passagens/ Nova
Vega, 2006, 6 ed. Prefácio de José A. Bragança de Miranda e
Antonio Fernando Novais.

\textsc{hansen}, João Adolfo. \emph{A sátira e o engenho. Gregório de Matos e
a Bahia do século \textsc{xvii}}. São Paulo: Companhia das Letras, 1989.

\_\_\_\_\_\_. “Autor”, in: Jobim, José Luís. (Org.). \emph{Palavras da
crítica}. São Paulo: Imago, 1992.

\textsc{horácio}. \emph{Arte Poetica de Q. Horacio Flacco, Traduzida, e illustrada em Portuguez por Candido Lusitano}. Lisboa: Na Officina
Rollandiana, com Licença da Real Meza Censória, 1778.

\textsc{jesus}, Frei Rafael de. \emph{Primeiro volume da 18ª parte da “Monarchia Lusitana”}, vol. \textsc{i}. Coimbra: Biblioteca Geral da Universidade de Coimbra, 1958.

\textsc{junta de providência literária}. \emph{Compêndio histórico do estado
da Universidade de Coimbra no tempo da invasão dos denominados jesuítas e dos estragos feitos nas sciencias e nos professores, e
diretores que regiam pelas maquinações, e publicações dos novos
Estatutos e por eles fabricados}. Na Régia Oficina Typographica,
\textsc{mdcclxxi}.

\textsc{lucrecio}, Tito. \emph{A natureza das coisas, poema de Tito Lucrécio
Caro}. Traduzido do original latino para verso portuguez por
Antonio José de Lima Leitão. Lisboa: Typographia de Jorge
Ferreira de Matos, 1851.

\textsc{mesnardière}, Jules de la. \emph{La Poetique de Jules de la Mesnardiere}. Paris: Antoine de Sommaville, 1639.

\textsc{minturno}. \emph{L’Arte Poetica del Signor Minturno Nella quale si
contengono i preccetti Eroici, Tragici, Comici, Satirici, e d’ogni
altra Poesia: con la dottrina De’Sonetti, Canzoni, ed ogni forte
di Rime Toscane, dove s’insegna il modo, che tenne il Petrarca nelle sue opere. E si dichiara a’suoi luoghi tutto quel, che da 51
Aristotele, Orazio, ed altri Autori greci, e Latini è stato scritto
per ammaestramento de’Poeti}. Napoli: Stamperia di Gennaro
Muzio, erede di Michele Luigi con Licenza de Superiori, 1725.

\textsc{peixoto}, Ignacio José de Alvarenga. \emph{Obras poéticas de Ignacio
José de Alvarenga Peixoto colligidas, annotadas precedidas do
juízo crítico dos escriptores nacionaes e estrangeiros e de uma
noticia sobre o autor e suas obras com documentos históricos}, org.
J. Norberto de Souza. Rio de Janeiro: Garnier, 1865.

\textsc{real academia española}. \emph{Dicionario de la lengua castellana,
en que se explica el verdadero sentido de las voces, su naturaleza
y calidad, com las phrases o modos de hablar, los proverbios o
refranes, y otras cosas convenientes al uso de la lengua}. Imprenta
de la Real Academia Española, por los herederos de Francisco
de Hierro, 1734.

\textsc{reis}, Francisco Sotero dos. \emph{Curso de literatura portuguesa e
brasileira}. Maranhão, \textsc{mdccclxvii}.

\textsc{rengifo}, Ivan Diaz. \emph{Arte poética española, con una fertilissima
silva de consonantes comunes, propios, esdruxulos, y reflexos, y
un divino estimulo del amor de Dios}. Madrid: por la viuda de
Alonso Martin, 1628.

\textsc{silva}, António José da. \emph{Esopaida ou vida de Esopo}. Coimbra:
Acta Universitatis Coninbrigensis, 1979.

\textsc{silva}, J. M. Pereira da. \emph{Parnaso brasileiro ou Selecção de poesias
dos melhores poetas brasileiros desde o descobrimento do Brasil
precedida de uma introdução histórica e biográfica sobre a literatura brasileira}, vol. \textsc{i}. Rio de Janeiro: Eduardo e Henrique
Laemmert, 1843.

\_\_\_\_\_\_. \emph{Plutarco Brasileiro}, vol. \textsc{ii}. Rio de Janeiro: Eduardo e
Henrique Laemmert, 1847.

\textsc{spinelli}, Miguel. \emph{Caminhos de Epicuro}. São Paulo: Edições
Loyola, s.d.

\textsc{teixeira}, Ivan. \emph{Mecenato pombalino e poesia neoclássica}. São
Paulo: Edusp, 1999.

\textsc{topa}, Francisco. \emph{Para uma edição crítica da obra do árcade Brasileiro Silva Alvarenga – Inventário sistemático dos seus textos e
publicação de novas versões, dispersos e inéditos}. Porto: mimeo,
1998.

\textsc{weinberg}, Bernard. “From Aristotle to Pseudo-Aristotle.” \emph{Comparative Literature} 5 (1953): 97--104. Duke University Press on
Behalf of the University of Oregon.
\end{Parskip}

%****

%É comum lembrar que
%o governo de Pombal reivindicasse para o rei Dom José \textsc{i} o direito de fazer
%bispos é uma querela que, no século \textsc{xii}, Dom Afonso Henriques, o Fundador do Reino, emulado
%pela representação política de Dom José \textsc{i}, chamado ``pai da pátria''
%no encômio.\footnote{ Ivan Teixeira reúne um vasto material encomiástico em
%torno de Sebastião José e consequentemente sobre Dom José \textsc{i}. } 
%O poder dos reis de exercer poder sobre o bispado do reino é uma prática que
%o papa proibiu a Afonso Henriques, o qual a reivindicava como legítima.
%A ordenação real de bispos não deixa de ser sem dúvida uma séria pendência
%judicial, pertinente ao Direito Canônico e logo ao Direito Natural, não é
%um ``traço de Iluminismo'', uma ``marca da época'', um sinal de que Portugal
%participava de certa história do Espírito, mesmo quando disfarçado das mais
%variadas formas. Práticas inglesas adaptadíssimas à prática
%institucional dos estabelecimentos dinásticos cujo centro estava assentado
%às margens do Tejo e de outros ribeiros daquela ponta da Península Europeia.
%Inglaterra que de inimiga entra no trato português que não se É impossível não
%sorrir pensando que as cartas expedidas pelo gabinete do Embaixador português
%fossem ridicularizadas por serem por exemplo \textit{barocas}, isso, que
%assim ficou sendo depois como uma positividade reconhecível que se \textit{manifesta}
%como a face do Espírito ou como a conjunção da materialidade. É interessante
%pensar que as ``cartas de Pombal'', isto é, muito antes disso, as cartas da
%\textit{representação} portuguesa na Inglaterra eram escritas segundo uma arte
%retórica ali enquadrada como uma eloquência antiquada, cujo estilo muitas vezes
%asiático, ou ciceroneano foi interpretado como redundante e excessivo tornava"-se
%ali uma arte velha, correspondente a uma ciência antiquada. %

%Essa contemporaneidade, súdita dos herdeiros de Dom João \textsc{iv}, não poderia crer senão
%que seria na França tudo seria conduzido a um reestabelecimento da ordem, segundo
%a melhor forma de governo que, segundo a tratadística política aristotélica, só
%poderia ser a monárquica. É significativo que a palavra democracia, constante em
%estatutos das Sociedades que Alvarenga compôs, seja alegada,
%nos autos do processo, como indício das opiniões que ali se trocavam. Mais significativa
%é a resposta do acusado que situa semanticamente a palavra pela data do seu uso,
%que sendo anterior às calamidades políticas mais recentes indicava outra acepção,
%que há muito era tirada justamente da trilogia aristotélica das formas de governo,
%que se lia principalmente no livro da \textit{Política}, de Aristóteles, cujos princípios
%estavam implicados nos livros da \textit{Retórica} e das \textit{Éticas}.
%A coisa toda tem de ser pensada como um quadro que contemporaneamente se pintava
%do levante violento que tinham sofrido os herdeiros dos herdeiros de Carlos Magno,
%nos eventos de 1789 e 1791, e seguintes. 
%A anglicização da política portuguesa na segunda metade do século \textsc{xviii}
%corresponde à imitação pombalina de hábitos políticos ingleses, isto é,
%trata"-se de uma representação política que recompila regulações específicas
%de decoros políticos diversos (o bom estilo das cartas, do \textit{sermo},
%isto é, da conversação civil, e dos demais gêneros de discurso, daí também o 
%bom estilo do ensino da poesia, que é pasto para a eloquência, que é exercício
%para futuros engenhos, os bons limites para a amplificação nos discursos
%epidíticos a pessoas particulares do reino, que estava proibida de inventar
%sem fundamento tradições de mérito,
%os preceitos para a composição dos caracteres das representações, as regras
%para a ostentação das precedências familiares nas aparições públicas, os
%novos métodos de ensinar desde a medicina até a ciência dos princípios das
%ciências, seja a Teologia seja a Metafísica, seja uma no interior da outra,
%conforme os currículos etc.). Tudo isso se remoça, com as vogas protocolares,
%desde que os novos usos não firam os estatutos sem os poder alterar. 
%

%Nesse
%sentido os decoros exigiriam mais ou menos declaradamente a necessidade de andar
%à moda. O professor Tibúrcio, que na juventude quis seguir carreira, mas provavelmente
%perdeu"-se nas questões da filosofia peripatética, cujo uso em Coimbra \textit{O
%desertor} vitupera, encarna a Ignorância alegorizada que talvez nunca tenha
%vestido à moda.

%para os despachos da Coroa e de suas depedências as leis e os saberes do reino, em primeiro lugar
%pondo a diante os estatutos irredutíveis das leis do reino. Além disso,
%mantém as folhas de pagamento mais ou menos inalteradas, mas revê estatutos
%que são passíveis de serem revisados por suas forças, como a Reforma da Universidade,
%das leis do Comércio. 
