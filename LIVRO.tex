\documentclass[showtrims,12pt,conselho,spreadimages]{memoir}

\usepackage[largepost]{hedraoptions} %% << %%%%%%%%%%%%%%%%
\usepackage[baruch]{hedrastyles}
\usepackage[xetex,chicagofootnotes]{tipografia}
\usepackage[standart,compontinhos]{toc}
\usepackage{hedraextra}
\usepackage{penalidades}
\usepackage{graficos}
\usepackage{hedralogo}
\usepackage{hifensextras}
\usepackage{fichatecnica}
\usepackage[standart]{aparatos}
\usepackage{tabelas}
\usepackage{versos}
\usepackage{gitrevisioninfo}

\newcommand{\forceindent}{\leavevmode{\parindent=1,4em\indent}}

\usepackage[switch,modulo,pagewise]{lineno}
\newfontfamily\Libertine{Linux Libertine O}
\renewcommand\linenumberfont{\Libertine\normalfont\small}

\linespread{1.15}

\usepackage{endnotes}
\renewcommand{\notesname}{Notas}

\usepackage{makeidx,hedraindex}  % cria índice
\makeindex	

%\counterwithin*{endnote}{part}
%\counterwithin*{endnote}{chapter}

\let\latexchapter\chapter
\makeatletter
\renewcommand\enoteheading{%
  \setcounter{secnumdepth}{-2}
  \latexchapter*{\notesname\markboth{NOTAS}{}}
  \mbox{}\par\vskip-\baselineskip
  \let\@afterindentfalse\@afterindenttrue
}
\makeatother
%\usepackage{fancyhdr}
%\pagestyle{fancy}
%\setlength{\headheight}{9mm}
%\fancyhf{}
%\fancyhead[R]{\thepage}
%\renewcommand{\headrulewidth}{0pt}

%\lhead[\fancyplain{}]{}
%\chead[\fancyplain{}]{}
%\rhead[\fancyplain{}]{\cnvt{\thepage} -- \thepage}

%\newcommand*{\cnvt}[1]{\the\numexpr#1-1\relax}

%\fancypagestyle{chapter}{
%\pagestyle{fancy}
%\setlength{\headheight}{5mm}
%\fancyhf{}
%\fancyhead[R]{\thepage}
%\renewcommand{\headrulewidth}{0pt}}


\usepackage{footmisc}

\renewcommand*\footnoterule{}
%\fancyhf[RO]{\cnvt{\thepage} -- \thepage}
%\fancyfoot{}
%\renewcommand{\headrulewidth}{0pt}
%\renewcommand{\footrulewidth}{0pt}}

\usepackage{fontspec}

%\usepackage{Formular}
\newfontfamily\Formular{Formular-Regular}[
BoldFont = Formular-Bold.otf]	

%--------------------------------------------ALTERAR DISTÃNCIA ENTRE TÍTULO DO SUMÁRIO E CAPÍTULOS
%\addtocontents{toc}{\vskip-15pt}
%--------------------------------------------
\usepackage{afterpage}

\newcommand\blankpage{%
    \null
    \thispagestyle{empty}%
    \addtocounter{page}{0}%
    \newpage}



%\usepackage{imakeidx} 
%\makeindex[program=xindy, options=-C utf8 -L portuguese]
%\newcommand\gobbleone[1]{}
%\newcommand*{\seeonly}[2]{\ (\emph{\seename} #1)}
%\newcommand*{\also}[2]{\emph{cf.} #1}
%\newcommand{\Also}[2]{\emph{See also} #1}
%\renewcommand\indexname{Índice onomástico}
%\makeindex[intoc]

\setcounter{tocdepth}{0}
\setcounter{secnumdepth}{-2} 
%\linespread{1.08}
\usepackage{commands}

\usepackage{setspace}

\makeatletter
\newenvironment{Parskip}{%
   \par
   \parskip=0.3\baselineskip \advance\parskip by 0pt plus 2pt
   \parindent=\z@
   \def\@listI{\leftmargin\leftmargini
      \topsep\z@ \parsep\parskip \itemsep\z@}
   \let\@listi\@listI
   \@listi
   \def\@listii{\leftmargin\leftmarginii
      \labelwidth\leftmarginii\advance\labelwidth-\labelsep
      \topsep\z@ \parsep\parskip \itemsep\z@}
   \def\@listiii{\leftmargin\leftmarginiii
       \labelwidth\leftmarginiii\advance\labelwidth-\labelsep
       \topsep\z@ \parsep\parskip \itemsep\z@}
   \partopsep=\z@
}{\par}
\makeatother

\makeatletter
\newenvironment{myParskip}{%
   \par
   \parskip=0.2\baselineskip \advance\parskip by 0pt plus 2pt
   \parindent=\z@
   \def\@listI{\leftmargin\leftmargini
      \topsep\z@ \parsep\parskip \itemsep\z@}
   \let\@listi\@listI
   \@listi
   \def\@listii{\leftmargin\leftmarginii
      \labelwidth\leftmarginii\advance\labelwidth-\labelsep
      \topsep\z@ \parsep\parskip \itemsep\z@}
   \def\@listiii{\leftmargin\leftmarginiii
       \labelwidth\leftmarginiii\advance\labelwidth-\labelsep
       \topsep\z@ \parsep\parskip \itemsep\z@}
   \partopsep=\z@
}{\par}
\makeatother

\newcommand{\mystar}{{\fontfamily{lmr}\selectfont$\star$}}

%\makeatletter
%\renewcommand{\@chapapp}{}% Not necessary...
%\newenvironment{chapquote}[2][2em]
%  {\setlength{\@tempdima}{#1}%
%   \def\chapquote@author{#2}%
%   \parshape 1 \@tempdima \dimexpr\textwidth-2\@tempdima\relax%
%   \itshape}
%  {\par\scriptsize\hfill-- \chapquote@author\hspace*{\@tempdima}\par\bigskip}
%\makeatother

%\newcommand\Chapter[2]{\chapter
%  [#1\hfil\hbox{}\protect\linebreak{\itshape#1}]%
%  {#1\\[2ex]\Large\itshape#2}%
%}

\begin{document}

%!TEX root=./LIVRO.tex
\chapter[Introdução, \emph{por Clara S. Santos e Ricardo M. Valle}]{Introdução}
\hedramarkboth{introdução}{clara c. santos e ricardo m. valle}

\begin{flushright}
\textsc{clara c. santos\\ricardo m. valle}
\end{flushright}

\noindent{}\textit{O desertor: poema heroi"-cômico}, de Manuel Inácio
da Silva Alvarenga (1749--1814), foi impresso pela primeira
vez em 1774, pela Real Oficina da Universidade de Coimbra.

Nascido em Vila Rica, ou em São João del Rey, sobre Manuel Inácio da Silva
Alvarenga disse"-se que era pardo, filho de músico, de origem pouco abastada.
Conseguiu progredir nos estudos aparentemente pelo empenho do pai e de uma
subscrição de amigos que teriam financiado sua ida ao Rio de Janeiro e depois a
Coimbra. Em Portugal, viria a se tornar amigo de Basílio da Gama, o poeta
brasileiro de \textit{O Uraguai} (1769), protegido e secretário do Marquês de
Pombal.\footnote{  Sebastião José de Carvalho e Mello ficou conhecido pelo último
e mais alto título de nobreza que recebeu em vida, concedido por decreto real em
setembro de 1769. Foi nomeado ministro de assuntos estrangeiros quando da
ascensão de Dom José \textsc{i}, em 1750. Tornando"-se ministro de Estado,
recebeu foros de plenipotenciário, isto é, privilégio de exercer como primeiro
ministro do Rei decisão sobre todos os assuntos do reino, com plenos poderes
para representar o rei no Conselho de Estado. Em 1759, recebeu o título de Conde
de Oeiras e dez anos depois o de Marquês de Pombal. }  Em 1774, \textit{O
desertor} saía à luz no momento que foi provavelmente o ápice da política do já
então Marquês de Pombal e quase às vésperas de sua queda repentina, em 1777, com
a morte de Dom José \textsc{i} e a consequente ascensão de Dona Maria
\textsc{i}.

Silva Alvarenga tinha 24 ou 25 anos e cursava o segundo ano de Direito naquela
Universidade, quando, não se sabe exatamente por que circunstâncias, o poema
teria sido mandado imprimir por Pombal. Por esta ocasião foi reconhecido como
bom poeta, e teve poesia sua integrada na pompa de inauguração da estátua
equestre de Dom José \textsc{i}, que encerrava monumentalmente a reconstrução de
Lisboa, em 1775, vinte anos após o terremoto de que até Kant falou, lá no fundo
de K\"onigsberg.  É verossímil que Silva Alvarenga tenha estado na capital do
Reino, mas aparentemente permaneceu em Portugal apenas enquanto durou seu curso
em Coimbra, entre 1773 e 1777.

De volta ao Rio de Janeiro, como advogado formado em Direito Canônico, priva com
mais de um vice"-rei, alcançando a amizade de uns e a inimizade de outro. %sic
Além de ter seu nome entre os que formaram a Arcádia Ultramarina, com o nome
acadêmico de Alcindo Palmireno, Alvarenga integrou mais de uma agremiação
literário"-científica no Rio de Janeiro.  Com a amizade do Marquês de Lavradio e
de seu sucessor, Dom Luís de Vasconcelos e Sousa, de Castelo"-Maior, nomeado
vice"-rei em 1782, Silva Alvarenga assenta uma cadeira de Retórica e Poética no
palácio do governador.  Para já evitarmos esquematismos muito apressados, é
notável que esse poeta, que surgiu tão pombalino, tenha alcançado tamanha
distinção oficial junto ao vice"-reinado justamente em 1782, no ano da morte do
``invicto Marquês'', já então exilado da Corte, depois da ascensão de Dona
Maria \textsc{i}.  Após a nomeação do Conde de Rezende, em 1790, porém, a última
sociedade de que participara seria proibida, suspeita de opiniões francesas,
afamada entre os antipatizantes como um ``\textit{club} de jacobinos''.


Com a amizade de Dom Luís de Vasconcelos, tinha adquirido a deferência do
encarregado direto do rei de Portugal. Com a inimizade do Conde de Rezende,
Alvarenga perderia primeiramente as liberdades com que o antigo governador o
distinguia.  Vale lembrar, porém, que ao cargo de vice"-rei estava
institucionalmente previsto que tinha poderes para ambas as ações --- distinguir
sujeitos particulares com prerrogativas de encargos públicos, bem como
destituí"-los, conforme o seu entendimento e vontade.  De forma correlata, o
letrado, mesmo de origem mestiça como era o caso, estava sujeito tanto ao
privilégio da distinção, como ao infortúnio da preterição por parte do superior
hierárquico. De qualquer forma, o poeta em questão tinha diploma com que se
pudesse distinguir, com ou sem o favorecimento direto da pessoa instituída como
poder local na representação política da colônia, porque advogava e, em 1790, já
adquirira fama para fazê"-lo para particulares.  Porém, perdendo a preferência da privança com o superior hierárquico, em pouco tempo perderia os direitos civis e, acusado de Inconfidência pela Devassa que o novo vice"-rei lançara sobre ele e amigos, permanece preso por mais de dois anos, entre 1794 e 1797, quando recebe indulto por decreto de Dona Maria \textsc{i}. Com isso, readquire os direitos de súdito e aparentemente segue a carreira que já cursava, sempre na cidade do Rio de Janeiro, capital da principal Colônia portuguesa na América. Antes, porém, no mesmo ano de sua saída dos cárceres da Ilha das Cobras, escreve um poema aos anos da rainha.  

Todos esses eventos biográficos, cabe frisar, estavam
institucionalmente previstos na jurisprudência portuguesa, a qual conformava as
práticas civis que encenavam os decoros da representação que soberanos e súditos
deviam manter dentro da ordem do Estado.  As partes e membros particulares da
hierarquia estavam sempre em demanda, mas as regras de precedências das
aristocracias tendiam à fixidez da \textit{lex}, especificamente a lei do
sistema jurídico português, do qual Alvarenga participava e que evidentemente
conhecia como diplomado em Direito Canônico.

No século \textsc{xix}, na posteridade que o mencionou, o livro de Silva Alvarenga
tornou"-se mais conhecido como \textit{O desertor das letras}. Com essa forma
estendida do título principal, a ele se referiu boa parte da crítica literária
do século \textsc{xix}, desde uma pequena nota bastante elogiosa que o escritor
veneziano Adrien Balbi fez ao poeta, no segundo volume do almanaque geográfico
de assuntos portugueses denominado \textit{Ensaio estatístico sobre o Reino de
Portugal e do Algarve}, impresso em 1822.  O Brasil então abrigava a sede do
bastante declinado império marítimo português, e por isso mesmo os poetas hoje
chamados ``brasileiros'', como Silva Alvarenga, não figuram no livro de Balbi
com a distinção que seria produzida de forma mais clara principalmente depois da
obra de Ferdinand Denis, o \textit{Resumo da história literária de Portugal,
seguido do resumo da história literária do Brasil}, livro importante que,
impresso em 1826, já no título ramificava uma literatura da outra. Assim,
Alvarenga é referido como poeta português que nasceu e viveu no Brasil, parte
importante do reino de Portugal, conforme a nota de Balbi, que não é demais
transcrever inteira já que ainda não foi reproduzida na íntegra pelos críticos
que a citaram:

\begin{hedraquote} Manuel Ignacio da Silva Alvarenga, membro da Arcádia,
professor de retórica no Rio de Janeiro, onde foi considerado o melhor advogado
do país. Compôs um grande número de poesias entre as quais os poemas \textit{O
desertor das letras} (\textit{le deserteur des lettres}) e \textit{Glaura} se
destacam por um merecimento real. Suas sátiras contra os vícios, a tradução em
verso português de Anacreonte, e de outras poesias [não] foram impressas. Uma
bela versificação, os pensamentos verdadeiramente filosóficos e uma crítica tão
fina quanto delicada se destacaram em todas as suas composições. Esse grande
poeta foi também um amante muito distinto da música e teve conhecimentos raros
em história natural.  Ele formou em sua casa um pequeno museu e possuiu a
biblioteca mais numerosa do Rio de Janeiro. Ela foi comprada aos seus herdeiros
e reunida à biblioteca do rei.\footnote{  ``Manoel Ignacio da Silva Alvarenga,
\textit{membre de l'Arcadia, professeur de rhétorique à Rio"-Janeiro, où il
passait pour le meilleur avocat du pays. Il a composé un grand nombre de poésies
parmi lesquelles les poèmes \textit{O desertor das letras} (le déserteur des
lettres) et le \textit{Glaura} se distinguent par un mérite réel. Ses satires
contre les vices, la traduction en vers portugais d'Anacréon, et d'autres
poésies, ont été imprimées. Une belle versification, des pensées vraiment
philosophiques, et une critique aussi fine que délicate se font remarquer dans
toutes ses compositions. Ce grand poète était aussi un amateur très"-distingué de
la musique, et avait des connaissances rares en histoire naturelle. Il s'était
formé dans sa maison un petit musée, et possédait la bibliothèque la plus
nombreuse de Rio"-Janeiro. Elle a été achetée de ses héritiers et réunie à celle
du roi.}'' (\textit{Essai statistique sur le royaume de Portugal et de l'Argarve
compare aux autres Êtats d'Europe et suivi d'un Coup d'oeil sur L'ètat actuel
des sciences, des lettres et des beaux"-arts parmi les Portuguais des deux
hémisphères}. \textsc{ii}. Paris, 1822, pp. 173--74.) } \end{hedraquote}


Suas sátiras e suas traduções de Anacreonte se tornaram célebres entre os seus
biógrafos justamente por \textit{não} terem sido impressas e terem sido assim
destruídas, logo após a sua morte, por um padre franciscano inimigo do poeta.
Com efeito, é plausível, pela própria disposição sintática do período de Balbi,
que tenha faltado a negação no sintagma ``\textit{ont été imprimées}''.  Seja
como for, com essa nota, cujas informações parecem ter sido coletadas com alguma
precisão em 1820, seis anos após a morte de Silva Alvarenga, iniciou"-se a
fortuna crítica do poeta ao menos como autor do poema heroi"-cômico que aqui
reeditamos.

Com ela, iniciava"-se também a invenção da sua personalidade literária
conveniente aos projetos nacionais e nacionalistas de constituição de uma
literatura brasileira do período colonial que mantivesse com a nova atualidade
do Império do Brasil após a Independência uma relação umbilical, ou germinal, na
justificativa de uma autonomia progressiva e crescente de um ``Espírito
nacional''.  Vale frisar que esse ``marco inicial'' da fortuna crítica do poema
não era um texto de crítica ou história literária, mas apenas um item no longo
apêndice do compêndio geográfico de Balbi.  O franco"-veneziano, escrevendo antes da Independência do Brasil, ainda denominava o reino de Dom João \textsc{vi} pela fórmula da dupla coroa portuguesa --- Portugal e Algarve ---, que foi como oficialmente se designou o reino desde os séculos \textsc{xii} e \textsc{xiii}, quando Portugal conquistara o \textit{ocidente} da Andaluzia.

Como se tratasse de um livro de atualidades, Balbi faz notas breves como esta
para quase todos os assim chamados árcades portugueses, entre outros poetas,
teólogos e oradores que tivessem alcançado algum renome naquele tempo, sendo
significativamente mais extensa sua nota sobre Bocage, ``\textit{le premier des
poètes portugais modernes}''.  Não deixa de assinalar com asterisco por exemplo
a nota sobre Tomás Antônio Gonzaga, não mencionando, porém, qualquer relação
dele com o Brasil, senão que morreu exilado em Angola e que os poemas de
\textit{Marília de Dirceu} estavam traduzidos em várias línguas.  Entre ambos,
em meio a tantos outros exemplos, o geógrafo não faz distinção de nacionalidade
literária, como até hoje se dividem ``literatura portuguesa'' e ``brasileira''.

Mas, como ficou dito, o breve texto de Balbi sobre Silva Alvarenga é bem
informado a respeito do poeta, mencionando até a transferência de sua biblioteca
pelos seus herdeiros para a Biblioteca Real, após a sua morte, o que
historicamente se demonstrou, no particular, pelas cartas e ofícios que Joaquim
Norberto encontrou e transcreveu, em nota, na sua edição das \textit{Obras
poéticas de Manuel Inácio da Silva Alvarenga}. Não deveria surpreender esse
detalhe.  Tanto a proximidade de sua morte em relação à viagem de Balbi a
Portugal, quanto a provável importância que tomou o seu nome no fim da vida, uma
vez que Alvarenga residisse desde muito naquela corte recentemente transferida,
são causas possíveis para essa relativa atenção de Adrien Balbi.  De uma forma
ou de outra, como doutor em leis, poeta já distinguido pelos maiores de Portugal
décadas antes, preso e absolvido sem pecha, o nome de Alvarenga, em 1820, era já
distinto como de varão ao menos notável; e logo seria ilustre, pela fama dos
feitos em letras para a municipalidade que improvisadamente abrigara a Corte de
Dom João \textsc{vi} e posteriormente para a nacionalidade póstera que o
reivindicaria para si.

Balbi considerava em seu \textit{Ensaio estatístico} o estado atual das
ciências, das letras e das belas"-artes entre os Portugueses dos dois
hemisférios. Neste início do século \textsc{xix}, no auge da ideologia do
progresso industrial pelo domínio das ciências da natureza, o louvor do poeta
advogado, formado em Direito Canônico na Universidade de Coimbra, produzia a
personagem como um livre"-pensador segundo modelos iluministas. Daí que
supostamente ele fosse um colecionista privado, sabedor de música, de história
natural e de cultura antiga, com ``pensamentos verdadeiramente filosóficos''. E
é possível que soubesse mesmo o seu tanto de coisas nesses específicos, menos
por que fosse um iluminista e mais provavelmente por que esses saberes fossem voga no tempo em que exerceu os ofícios das letras.  
Em meio às observações de Balbi
a respeito do desenvolvimento da hidráulica, da engenharia naval ou das ciências
agrárias em Portugal, tais atributos de Silva Alvarenga conferem interesse e
importância a esse poeta, mas não mais do que como que por um golpe de vista.  A
rigor, as citações de Lineu e de Marcgrave que se leem nas notas que o poeta faz
a \textit{O desertor} não o tornam mais parecido com Alexander von Humboldt, ou
a outros naturalistas do século \textsc{xix}.  Principalmente, isso não lhe
retira os pressupostos teológico"-políticos do Direito Canônico, que foi a alta
ciência que o fez certamente distinto como homem de leis, tanto no tempo em que
o Rio de Janeiro foi vice"-reinado, como quando passa a ser a sede da Monarquia
lusitana.  O ter escrito sátiras ou um poema de caráter misto também não supõe
autonomia literária ou liberdade intelectual, porque escreve como vassalo fiel
aos pés do trono e do conselho do Estado, porque sem tal lealdade não teria
participado das festas em torno da estátua equestre, nem, de volta à colônia,
teria se tornado lente de retórica, assentado pelo vice"-rei, para a utilidade e
deleite das ciências e das artes na capital da colônia. 

O cônego Januário da Cunha Barbosa (1780--1846), que primeiro escreveu a
\textit{vida} do poeta e teria privado com ele no Rio de Janeiro da corte
portuguesa de Dom João \textsc{vi} e Dona Maria \textsc{i}, foi membro fundador
e primeiro sócio subscrito do Instituto Histórico e Geográfico Brasileiro, além
de autor de seus primeiros estatutos. Conforme Joaquim Norberto, Cunha Barbosa
foi ``dedicado e agradecido discípulo'' de Silva Alvarenga nas aulas de
Retórica e Poética que lecionou por mandado do governador Dom Luís de
Vasconcelos e Sousa, desde o final do século \textsc{xviii}.  Depois de receber
o indulto de Dona Maria \textsc{i}, aparentemente Alvarenga retornou a essas
lições de retórica, tornadas célebres entre os círculos letrados do Império do
Brasil, quando talvez Cunha Barbosa tenha tomado lições, porque antes não teria
idade para isso.

Já no século \textsc{xix}, o famoso cônego pode ter sido aluno do autor de \textit{O
desertor}, assim como outros oradores no Império do Brasil, como Monte Alverne e
São Carlos. Contudo, Cunha Barbosa na vida de Silva Alvarenga desvia em quase
vinte anos a idade do poeta no tempo de sua morte, dizendo que faleceu com quase
oitenta anos. Joaquim Norberto publicou documentação sobre isso na referida
edição das \textit{Obras poéticas}, de 1864, o que, de certa maneira, colocava
em dúvida a prova testemunhal alegada pela relação mestre/discípulo que o poeta
e o cônego teriam firmado, embora o mesmo Norberto tenha reposto a autoridade
dessa informação, obtida por sua vez na relação também pessoal que teve com o
cônego quando foi bibliotecário submetido às suas ordens.

Segundo Januário da Cunha Barbosa, Silva Alvarenga não tinha o poema por acabado
quando este foi impresso por ordem do ministro plenipotenciário de Dom José
\textsc{i}. Assim, desde as palavras de Cunha Barbosa, no terceiro número da
Revista trimestral do Instituto Histórico e Geográfico Brasileiro --- ``o poema
heroi"-cômico intitulado o \textit{Desertor das letras}, que por ordem do Marquês
fora impresso contra a vontade de seu autor, porque ainda o não havia
suficientemente corrigido, deram"-lhe créditos de literato e o descobriram
distinto poeta'' ---, muitos compêndios de literatura e ensaios críticos sobre o
poema reafirmaram essa mesma informação como um dado objetivo, sem considerar,
por exemplo, a impropriedade política das mesmas palavras que, a rigor, opunham
a vontade do súdito e a ordem do senhor, que evidentemente não estavam e não
poderiam estar em conflito; ao menos não sem severas consequências.

É certo que a vontade nada poderia contra a ordem, porque vivia"-se sob os
princípios da \textit{monarquia absolutista}, e debaixo da \textit{ditadura
pombalina}, que foi como o historiador inglês Charles Boxer caracterizou os
modos hipercentralizadores desse ministro de Estado plenipotenciário que, por
quase trinta anos, exerceu poder muito direto sobre as decisões mais gerais e
mais particulares do reino.  Mesmo assim, desde Januário da Cunha Barbosa, a
crítica literária tem frisado que o poema foi impresso por ordem do ministro e
contra a vontade do autor, mas nada há de particularmente conclusivo em relação
a isso, além do testemunho do próprio Cunha Barbosa, que já no século
\textsc{xix} foi corrigido em relação a várias particularidades, o que obriga a
relativizar o crédito dado à informação supostamente direta.  Esse testemunho
ainda assim foi transformado em ``informação'' biobibliográfica, pela crítica e
pela historiografia literária, empenhadas desde a Independência em constituir,
corrigir e enaltecer o cânone literário do Brasil no período colonial.

Com efeito, mesmo na hipótese de que pessoalmente Silva Alvarenga o tivesse
confessado a seus alunos de Retórica e Poética, não se pode deixar de considerar
que se trate de um velho lugar comum os autores alegarem que tornaram pública
certa obra apenas por \textit{força maior} e acrescentarem ter ficado, por isso,
incompleto o trabalho da própria arte. Com isso, visam a captar a benevolência
do público em geral e a atenuar as possíveis falhas na aplicação dos preceitos,
defendendo"-se da mordacidade da crítica dos mais doutos.  Articulações de
sentido como essa abundaram em prólogos, proêmios e cartas dedicatórias, mesmo
em obras acabadíssimas, e não representaram necessariamente qualquer sinceridade
afetiva, nem humildade de artista, nem muito menos consciência crítica ou
autocrítica.

A positivação desse testemunho parece ter sido um meio de a crítica literária
brasileira justificar a perda de eficácia poética deste poema, sobretudo nos
últimos cantos, que parecem se arrastar numa elocução excessivamente prosaica
muitas vezes, depois de cantos iniciais razoavelmente bons.  Com efeito, dentro
das convenções do gênero misto em que o poema é declaradamente escrito, ao menos
os primeiros cantos devem ter tido razoável eficácia cômica pela dissociação
deliberada entre o \textit{estilo alto}, que emula poemas heroicos como
\textit{O Uraguai} e \textit{Os lusíadas}, e a \textit{matéria baixa}, que imita
tipos sórdidos e feitos indignos próprios da sátira e da comédia, exemplos
viciosos que a moralidade do poema ensina para corrigir os vícios pelo vexame.
Assim, os tipos poderiam ser imitados da \textit{Natureza} --- entendida como
\textit{natureza das coisas} e, neste específico, como as diversidades
qualificáveis da \textit{natureza humana} ---, imitando assim os engenhos
(\textit{ingenii}) dos homens particulares tipificados conforme o hábito, o
estado, a virtude etc., ou então poderiam ser imitados de tradições cômicas
gregas, romanas, italianas, francesas, portuguesas, que já tinham estilizadas e
classificadas vastas galerias de tipos, as quais constituíam repositórios para a
representação ficcional de gênero baixo, mesmo dentro de moralidades e de
circunstâncias políticas tão diversas.

Assim, o poema, cuja comicidade foi sempre posta entre aspas pela crítica
literária dos séculos \textsc{xix} e \textsc{xx}, tinha como tema heroico a ação do Marquês de
Pombal sobre o ensino na Universidade, mas relatava as ações vis de Gonçalo, o
jovem estudante que desertou da carreira das letras para ingloriamente retornar
à província de onde saíra, agora mais obscuro do que antes.  Por fraqueza de
ânimo, abandonaria os estudos movido pela Ignorância, que não tinha mais lugar
em Coimbra desde que o Marquês reformara os estatutos da Universidade, ação alta
que dava o fundamento histórico ao poema e o encomendava nas mais altas esferas
do Império marítimo dos reis de Portugal e do Algarve.

Nos primeiros anos de seu governo, ainda Sebastião José de Carvalho e Melo não
era nem conde nem marquês.  De embaixador por muitos anos na Inglaterra e depois
na Áustria, onde se casa com a prima da rainha de Portugal, é nomeado ministro
dos negócios estrangeiros, em 1750.  Com o terremoto de Lisboa, em 1755,
torna"-se a figura política mais poderosa no reino, abaixo do rei. Com o atentado
a Dom José \textsc{i} em 1758, seguido da rápida punição aos acusados, consolida
sua vitória contra as facções da nobreza antiga que faziam oposição a seu
gabinete. Os dois eventos graves alavancaram e consolidaram os êxitos do
ministro de Estado perante o rei e o reino, \mbox{levando"-o} à vitória em face dos seus
principais inimigos: de um lado, a parte da velha nobreza que o via com
desconfiança desde a sua nomeação; de outro, os padres da Companhia de Jesus,
que por decreto real foram expulsos de todos os domínios do reino em 1759, ano
em que o ministro de Estado é distinguido pelo título de Conde de Oeiras.  Neste
mesmo ano, mais de uma década antes da Reforma da Universidade, a estrondosa
expulsão dos jesuítas já resultaria numa primeira reordenação do ensino, uma vez
que a Companhia de Jesus dominava a educação básica em todas as possessões da
Coroa Portuguesa.

O poema se inicia com a invocação, pedindo à Musa que auxilie o poeta a cantar
com engenho e arte o desertor da Universidade que, ao lado de seus companheiros
de vícios, guiados pela Ignorância, retorna em viagem à província natal, onde
sem vencer nos estudos é recebido com ira pelo tio, no final das duras e
tumultuosas jornadas, que imitavam, em gênero baixo, epopeias de viagem como a
\textit{Odisseia}, a \textit{Eneida} e \textit{Os lusíadas}.  Em sua disposição
retórica, o argumento da fábula é apresentado ao mesmo tempo em que se faz a
invocação. 

\begin{verse}
Musas, cantai o desertor das letras \\
Que, depois dos estragos da Ignorância, \\
Por longos e duríssimos trabalhos, \\ 
Conduziu sempre firme os companheiros \\
Desde o loiro Mondego aos Pátrios montes. \\ 
Em vão se opõem as luzes da Verdade \\
Ao fim que já  na ideia tem proposto  \\
E em vão do Tio as iras o ameaçam. \\
\end{verse}	  

Junto à invocação já se propõe a matéria heroi"-cômica dos cantos: a renúncia de
Gonçalo aos livros e as causas de seu retorno à obscuridade em Mioselha.  Na
sequência, inicia"-se a dedicatória como em \textit{Os lusíadas}, referindo o
homenageado por perífrases, sem nomeá"-lo diretamente. Aquele engenho que,
amparado pela mão benigna do rei, alimentava as doces artes poderia ser Pombal,
mas era mais provavelmente o reitor reformador, Dom Francisco de Lemos, que,
amparado pela mão do ministro, é o ``prelado ilustre'' a quem alusivamente se
incumbe a proteção dos versos e que no final do poema aparece
esmagando o monstro da Ignorância ao pé do trono. Nos versos dedicatórios
iniciais, é a esse prelado ilustre que se perguntam as causas da deserção das
letras, já que o bispo fidalgo nascido no Brasil, além de reitor reformador da
Universidade no tempo da Reforma, é um dos autores pombalinos que redigem o
\textit{Compendio histórico do estado da Universidade de Coimbra no tempo da
invasão dos denominados jesuítas e dos estragos feitos nas ciências e nos
professores}.

O jovem Gonçalo derroca na vida por muitas causas específicas que particularizam
o seu caso, mas a moralidade das causas encenadas no enredo é, por necessidade,
geral. Assim, sua ruína, que já havia começado quando pela primeira vez não
acordou para as aulas, se precipitaria irrefreavelmente ao decidir deixar a
Universidade que se reformava.  É também causa específica de sua precipitação o
ser pouco avisado dos riscos dos comprometimentos da vida prática e, somado a
isso, ser inexperto nos assuntos de amor.  Mantém sem a permissão do tio uma
prometida noiva Narcisa, o que demonstra sua inadvertência; e, na hora de
partir, empenha uma bolsa de dinheiro como fraco compromisso, o que é fruto de
sua inexperiência.

Com este último gesto, na sequência do enredo convence imediatamente a precária
noiva, que se fingia desesperada e, assim pintada, representava"-se como uma
oportunista filha de outra. Ambas, mãe e filha, são tipificadas na vida
estudantil como outras tantas inimigas do bom estudo, desencaminhadoras de
jovens de letras, vivendo dos presentes e do dinheiro dos estudantes, mormente
os que vêm das mais distantes e das mais próximas províncias do reino.  Mas		%tirando dúvida com autores
Tibúrcio, experiente, lembrando antigos perrengues de viagem, primeiro tenta
convencer o herói de que deva levar consigo o dinheiro, depois remete a decisão
ao próprio braço, usando a força contra a mulher, que por seu lado recontava o
dinheiro muitas vezes.

Ainda que fruto da inexperiência mal advertida desse herói, que, empenhando a
bolsa, até poderia demonstrar alguma altivez de caráter, vale lembrar que na
baixeza geral da coisa que se narra, seu gesto visava a atingir a venalidade da
personagem feminina, coerentemente encenada recontando o dinheiro e batendo"-se
por ele.  Num universo jurídico em que a distribuição da justiça é herança
paterna e materna, segundo os direitos do reino, é significativo que, no momento
em que deserta da vida de estudante, Gonçalo diga à sua amante de juvenilidade
que só vai à província para receber uma herança que lhe teria deixado um
parente.  A mentira do herói na fábula baixa é cunhada, pois, sobre
prerrogativas, formas de representação, leis, costumes, etiquetas e hábitos
jurídicos e políticos que então eram fixados pelas hierarquias das coisas e dos
homens.

Entre as coisas políticas que o poema ensina, dever"-se"-ia entender com juízo que
o longo caminho de uma família súdita, por exemplo, ou mais ou menos bem
sucedida segundo a tradição familiar, podia ser posto a perder pela vida
desordenada de um único herdeiro.  É essa condição civil o que faz de Gonçalo o
herói dessa anti"-épica. Acresce que ele é, entre os seus, o mais forte e
promissor na expedição de indisciplinados; sendo, por isso, o melhor exemplo do
pior, para assim ensinar os maus efeitos da má conduta nos estudos. Do alto e
belo rapaz certamente a tradição familiar esperava grandes feitos, porque é
sempre melhor ser e andar vistoso para as virtudes políticas da representação
hierárquica. No entanto, o vistoso rapaz volta humilhado para a vara do tio,
que, como autoridade familiar, o aconselha e o espanca até que espume em sangue.
Assim, entre aconselhamentos mais ou menos virtuosos, e espancamentos sempre
muito violentos, e mais ou menos justos, são basicamente armadas as peripécias
que a torto e a direito terminam em pragmatografias de tumultos, não de grandes
combates. Como Gonçalo e seus companheiros,  Aquiles se desavem com Agamémnon e
os demais guerreiros. Na \textit{Ilíada}, Aquiles também remete a decisão ao
próprio braço  contra Agamémnon dentro do conselho dos reis. Mas, sob
entusiasmo, seu punho é interceptado pela mãe, a deusa Tétis, que pondera e o
faz decidir melhor. Coloca o moço no colo e lhe promete como mimo um tanto de
desastres para os gregos que, por isso, quase tiveram seus barcos queimados
pelos troianos. Sem presença divina de nenhuma espécie, a ignorância de Tibúrcio
parte para cima de Narcisa, provocando o tumulto que faria o forte, belo
e promissor advogado e recém desertor das letras perder parte dos dentes.

A elocução do poema é alta, imitando principalmente os versos brancos heroicos
de \textit{O Uraguai}, já então tornado célebre pela proteção do ministro de
Estado.  Essa dicção elevada --- que, nos melhores momentos, também pode fazer
lembrar parodicamente \textit{Os lusíadas} --- é provavelmente a causa da opinião
desfavorável da crítica dos séculos \textsc{xix} e \textsc{xx} que desmereceu o
caráter ``humorístico'' esperado desta anti"-épica, que frustra o leitor
anacrônico do século \textsc{xix} que teve outros modos de pensar, tanto o
chiste e a piada, como os discursos de intenção didática, lidos em jornais ou em
circulações acadêmicas do novo Império do Brasil.  O poema não integrava uma
instituição liberal, mas, sim, o corpo político da leal sujeição de Sua Alteza
Real, o Augustíssimo e Sereníssimo Rei Dom José \textsc{i}.  Numa classe de
homens letrados, sujeitos às ordens das aristocracias, nova e velhamente
inscritas nos livros do Reino, é preciso pensar que talvez a graça, maior ou
menor, deve ter havido.

O cômico estava fundado, de saída, na representação de distinções ajustadas
entre \textit{melhores} e \textit{piores}, que é como então se operavam as
\textit{Políticas} e as \textit{Éticas} de Aristóteles, tanto quanto as suas
artes \textit{Retórica} e \textit{Poética}.  A fábula cômica domina a ação do
poema, e é constituída pelas peripécias de um grupo de estudantes guiados por
Tibúrcio, personificação da Ignorância.  Mesmo que não se veja graça em mais
quase nada neste poema estudantil, a fábula é aristotelicamente cômica, isto é,
imita os piores, descreve matéria baixa, digna de opróbrio. Cômica por
definição: não faltam o acumulado de tipos socialmente inferiores ou moralmente
deformados, no lugar de heróis dignos da epopeia; a sucessão de cenas de brigas
comezinhas, com unhas e dentes, ou cheias de vilania, como a luta de três homens
contra uma mulher, em lugar de lutas famosas entre grandes guerreiros e senhores
de homens; daí os tumultos, em lugar de batalhas; e as bebedeiras, em lugar de
triunfos e banquetes em que se honrassem as vitórias da virtude.  A virtude,
contudo, era o fim a que deviam mover tanto a poesia heroica quanto a cômica,
com a diferença de que esta imitava antes de tudo os vícios para que se pudesse
fugir deles e aquela imitava principalmente as virtudes para que se as
perseguissem.

Neste sentido, no poema heroi"-cômico de Silva Alvarenga, a inglória viagem de
Gonçalo, Tibúrcio e sua trupe lastimável de maus alunos é uma fábula cômica
narrada como se fosse coisa heroica.  Com a entrada do ministro em Coimbra, a
Ignorância é expulsa e com ela todos os que viviam sob a falta de polícia nos
estudos.  Neste sentido, a ação burlesca é movida pela ação heroica do Marquês e
do seu prelado, o reitor reformador, também nascido no Brasil, homenageados
ambos no poema. Por um antigo lugar comum, o efeito da ação alta sobre a baixa
poderia traduzir"-se como a luz que, tão logo acesa, faz esconder"-se a escuridão
detrás dos móveis e objetos postos às claras.  Dessa forma, os seguidores da
Ignorância fogem da cidade e das ciências, para buscar a província, sem
merecimento de fama para prosseguir na carreira com distinção e louvor. A
escuridão reduzida às fendas e arestas mostra"-se, assim, por contraste, mais
escura na claridade do que nas trevas.

\begin{verse}
Os que aprendem o nome dos autores, \\
Os que leem só o prólogo dos livros, \\
E aqueles cujo sono não perturba \\
O côncavo metal que as horas conta, \\
Seguiram as bandeiras da ignorância \\
Nos incríveis trabalhos desta empresa.
\end{verse}

Um é bruto, outro é apaixonado, outro é afidalgado, outro é furioso, outro é
dançarino, outro é um experto vendedor de objetos usados, que de mau estudante
foi se deixando ficar em Coimbra como um desencaminhador de jovens estudantes.
No catálogo dos heróis, no início do canto \textsc{ii}, apresentam"-se, além de
Tibúrcio e Gonçalo, os companheiros Cosme, Rodrigo, Bertoldo, Gaspar e Alberto,
cada qual segundo seu caráter constituindo um tipo, um vício, uma paixão, um
desvio de conduta, que os rebaixam nas hierarquias do mérito, codificadas nas
disciplinas morais lidas com Aristóteles e com Salomão, com Platão e São João
Evangelista.  Aqueles que não se levantavam ao sino, nem para a aula, nem para a
missa, e que retinham dos livros apenas o necessário para o fingimento
desonesto, os argumentos dos prólogos e o nome dos autores, todos esses com
todos os seus tipos, retornam obscuros para a própria província tendo perdido os
cabedais familiares com a vida devassa. 

Assim, o retorno à província --- a fictícia Mioselha na fábula do poema, rica em
queijos e tremoços, como a louva a Ignorância, pela voz de Tibúrcio --- não
representa qualquer elogio do campo, remanso da virtude, remédio para os vícios.
Longe de ser um louvor da vida retirada, propícia ao bom ócio estudioso e ao
feliz desengano das vaidades das Cortes e grandes cidades --- como se lê nas
tópicas do \textit{menosprezo da corte e elogio de aldeia}, muito recorrentes na
poesia pastoril que vogou no século \textsc{xviii} (e não só nele) ---, a buscada
vida na aldeia é aí mais propriamente um emblema da fuga para a obscuridade,
antiprêmio dado ao herói por sua fraca perseverança nos estudos. A causa próxima
e geral do retorno de Gonçalo e de todos os demais maus alunos foi a restituição
da luminosa Verdade, como se viu, alegoricamente recolocada no trono naquela
corte das \textit{scientiae}, por intermédio da ação divina e histórica dos
heróis verdadeiros do poema: Sebastião José de Carvalho e Melo, o ``invicto
Marquês'', que se lê no início da narração, no canto \textsc{i}, entrando
triunfalmente em Coimbra como restaurador das letras, e Francisco de Lemos, o
``Prelado fomidável'', de quem se queixa a Ignorância ao ver os eventos que
representam no poema o fim de seu império em Coimbra.  Gonçalo volta à pátria
Mioselha, abandonando os amores de Narcisa e a leitura de romances vulgares.
Agora fora definitivamente impedido de prosseguir sem esforço nos estudos,
graças à reforma da Universidade, assinados os novos \textit{Estatutos da
Universidade de Coimbra compilados debaixo da imediata e suprema inspecção de
El"-Rei D. José, nosso Senhor, pela Junta de Providência Literária, criada pelo
mesmo senhor, para a restauração das ciências e artes liberais nestes reinos e
todos os seus domínios}. 

O efeito cômico deveria estar em narrar como se fosse grande coisa, e com
palavras infladas, as bravatas irrisórias e os ânimos mesquinhos das personagens
da trama, pela dissociação entre o baixo da invenção da matéria e o alto da
elocução ornada com palavras graves dignas de grandes feitos.  O vitupério se
justifica como eficácia didática que exorta à virtude, uma vez que representa as
coisas dignas de vergonha com as mesmas palavras e proporcionais figuras com que
se louvam os grandes feitos. É por isso análoga da epopeia, mas inverte o valor
das matérias, das ações, dos exemplos, dos caracteres dos heróis, trocando a
lâmina do melhor metal herdado na paternidade das armas por paus e pedras que se
tomam da beira das ruas, entre gente sem herança. Ao contraste do melhor
humilha"-se o pior. Esse efeito deveria produzir o discernimento do certo e do
errado, finalidade do \textit{movere} que o poema encena, como moralidade que
deve instruir e convencer o entendimento do leitor, que rindo deleita"-se, e
deleitando"-se aprende o mau exemplo que se deve evitar.

Em sua disposição retórica, após a invocação, a proposição e a dedicatória,
inicia"-se a narração (\textit{narratio}) como um \textit{epibatérion}, isto é,
um discurso laudatório sobre a entrada triunfal do Marquês de Pombal em Coimbra
para restaurar os Estatutos da Universidade.  Louvam"-se a chegada e os feitos do
Marquês, encomiando"-o a partir da exposição das boas qualidades das ciências que
o acompanham para alegoricamente restaurar os princípios da Verdade que deveriam
nela reinar. Por ser retórica em sua invenção (\textit{inventio}), a descrição
das matérias --- sejam os homens em ação (\textit{pragmatografias}), sejam o
caráter e o costume deles (\textit{etopeias}), sejam as cidades e lugares
(\textit{corografias}), entre outras, seguem ordenamentos previstos na execução
do discurso ---, é produzida como quadros verbais que retoricamente louvam ou
vituperam seu objeto, segundo as convenções da descrição (\textit{ecfrasis}).
Seja a descrição de feitos dignos de memória, seja a descrição de coisas torpes
e infames, conforme os preceitos, as palavras devem pintar na fantasia com vivas
cores, de modo a tornar vívidos para o leitor os episódios narrados na invenção
da fábula.  Por isso, quando Tibúrcio fala de Mioselha, a partir do lugar comum
do elogio da aldeia revertido pelo caráter cômico da invenção, sabe"-se que o
poeta pode lançar mão da origem da cidade, da antiguidade dela, da situação
atual, das realizações virtuosas de seus varões ilustres, do esplendor da
paisagem e do clima, da fertilidade do solo, da alegria dos viventes, das
comidas saborosas que os aguardam etc. Como nesta moral os maus estudantes são
desviados do bom caminho pelos gostos mais baixos, não é decerto casual que, além
das festas de romaria, seja a abundância de queijos e mais víveres que componha
o argumento de Tibúrcio para persuadir Gonçalo da felicidade que encontraria na
província, longe do peso dos estudos.

Assim, procedimentos previstos moderam os efeitos gerais das representações
poéticas.  As descrições dos tipos, das cidades, dos combates, dos enganos e
desenganos consideram seres particulares conforme sistemas gerais e específicos
de classificação dos seres.  Num sistema que, sendo retórico, devia tornar os
ouvintes dóceis e atentos ao dizer, a personificação dos hábitos dos
companheiros que seguem os enganos da Ignorância é inscrita em regras de
representação compartilhadas pelo leitor presumido, o que auxiliava a produção
do artifício, como uma máquina discursiva em que os versos deveriam formar uma
unidade para o entendimento dos leitores que ali reconhecem costumes políticos e
poéticos da representação. Assim, as personagens são tipos, não representam
individualidades psicológicas. Por isso, o catálogo, ou enumeração, das figuras
justapõem as representações individuais sem que os caracteres de uns e de outros
sejam comparados entre si. Por isso também, cada episódio contém uma unidade que
lhe é própria, geral e particularmente, sem ter com o todo da trama, que é uma
ação bastante simples, mais relação de necessidade do que  a disposição funcional dos encadeamentos básicos entre um evento e outro. Assim é que, no canto
\textsc{iii}, é introduzido uma personagem nova, Rufino, apenas porque é amante
da filha do carcereiro, ao mesmo tempo que a Ignorância alegorizada via que os
desertores, seus últimos súditos, estavam perdidos nas montanhas e em pouco
tempo acabariam presos pela multidão de que fugiam, furiosa pelos maus feitos
que no regresso aqueles estudantes já tinham deixado pelo caminho. 

%, como quando Tibúrcio invade o quarto de Amaro
%fingindo ser um fantasma ao mesmo tempo em que Doroteia segue para a cela de
%Gonçalo. % ou, em casos mais simples, como quando ao mato espesso no qual os
%jovens se escondem se pode atribuir qualidades semelhantes às da floresta onde
%Rufino sofre por desencanto de Doroteia.
%E, neste caso, como o lugar tenha alguma similitude em afeto com a pessoa, pode também
%o poeta inventar uma similitude entre o lugar e o caráter daquele que chora, que bebe ou que boceja.
%Em tratados cosmográficos no \textsc{xvi} (aqueles que dão conta da descrição de todas as coisas
%que podem existir debaixo do Sol, desde o \textit{Gênesis}) é comum a comparação de objetos
%particulares com partes do corpo humano. Por isso, quando se descreve a geografia do
%mundo deve"-se buscar semelhanças com a cabeça de um homem, assim como quando se descreve
%um lugar em particular (uma cidade, por exemplo) deve"-se buscar semelhanças com partes
%do homem, um olho ou uma orelha, por exemplo.

Como é um poema anterior ao romantismo e à constituição da literatura como
mercado editorial, o poema talvez nem devesse passar por ``apreciações
estéticas'' no ``crivo'' da crítica literária.  Neste fim do século
\textsc{xviii} português em que se formou o poeta, mesmo no fogo cruzado das
reformas pombalinas, ``crítica'' talvez devesse ser entendida como o uso do
juízo segundo as leis naturais de Deus e segundo as leis positivas e os costumes
dos homens, conforme os direitos de cada Estado político.  No tempo da Real Mesa
Censória, que desde 1768 unificava as mesas eclesiásticas e a mesa civil em
Portugal, não poderia haver jornalismo de profissão, muito menos exercício de
``crítica literária'' no sentido moderno, no qual a livre iniciativa individual
se entendesse intelectualmente apta para avaliar as obras de Poesia pressupondo
a autonomia delas.  Antes da formação da noção de literatura como campo
autonomizado e antes da instituição das artes como exercício da livre reflexão
do juízo subjetivo mais ou menos consciente de realidades sociais que a
subjetividade autonomizada hoje julga poder imprimir e exprimir como arte,
``crítica'' era o ato de uma faculdade da alma cujo fim era discernir o acerto
do erro, o justo do injusto, o belo do feio, e assim por diante.\footnote{ 
Essas outras formas com que se definiu a palavra ``crítica'', não pressupunham o
uso kantiano que a reinstauraria sem teologia para outras formas de compreender
a alma ou os limites do conhecimento humano, nem poderiam supor os usos
marxistas e positivistas em cujo discurso provavelmente se cunhou a acepção mais
corrente atual, em enunciados que, no âmbito da crítica literária, por exemplo
passaram a inferir ``crítica social'' em textos alegórico"-devocionais ou
satíricos de Gil Vicente a Gregório de Matos, para só citar os mais próximos e
conhecidos hoje.} Especificamente ``crítica literária'', só poderia ser
entendida como a \textit{crisis}, isto é, o exercício do \textit{entendimento}
(ou \textit{intellectum}) --- a mais alta faculdade inculcada por Deus na alma
humana --- sobre as ``coisas das letras'' (\textit{litteratura}), dentro das
instituições poéticas e oratórias que o uso ordenava conforme a fins que não
seriam jamais fechados na contemplação da própria arte.\footnote{ São
inumeráveis os desenvolvimentos da opinião kantiana acerca da definição das
belas artes na \textit{Crítica do juízo}, como aquelas em que o fim não se
encontra fora dela mesma.} Para o século em que sai \textit{O desertor}, tanto
os poemas mais obviamente \textit{didáticos} quanto os poemas mais aparentemente
\textit{fúteis} (dois adjetivos que seriam usados para caracterizar o poema
heroi"-cômico) tiveram fins morais e políticos que excediam as bordas imaginárias
da própria arte. Mas tratava"-se de fins constituídos por princípios que a
própria arte, como técnica, definia para si, segundo a tópica horaciana segundo
a qual o melhor poeta será aquele que ao agradável unir o útil, de modo que
ensine e deleite ao mesmo tempo.

O poema heroi"-cômico não é uma reação à poesia épica e cavalheiresca, como já se
disse na crítica literária que falou deste poema. Ao contrário, vai no mesmo
sentido destas, ainda que seja uma espécie de poesia \textit{menor}, por
definição.  Não obstante, exorta à virtude fazendo falar a Verdade. No poema de
Silva Alvarenga, a Verdade fala, literalmente, na forma de uma alegoria sonhada
como num afresco que imitasse um episódio mítico grego. Como define o próprio
poeta no ``Discurso sobre o poema heroi"-cômico'' que introduz \textit{O
desertor}, trata"-se de narrar heroicamente eventos cômicos, o que realça a
baixeza das ações. Em suas palavras, ``o poema chamado heroi"-cômico [\ldots{}] é a
imitação de uma ação cômica heroicamente tratada.'' No seu poema, a celebridade
da fama que a poesia heroica cantava é contrastada com a irrisão da fuga indigna
para o ostracismo, como se a moral da história assim ensinasse, sacudindo o dedo
com um irônico \textit{memento}: lembre"-se, homem, os \textit{grandes} cantos
que se farão a respeito de sua má inclinação na vida, trocando a fama por
infâmia, a dignidade por indignação, a memória pelo desprezo da posteridade. A
fábula  heroicamente escrita faz os que perseguiam os maus costumes lembrarem"-se
de que não haveria menção de sua vida nem mesmo na posteridade da família. 

A obscuridade na carreira e o esquecimento da posteridade são o que a fábula
cômica ameaça ao mau letrado, a quem a História não irá assinalar como varão,
nem deles se escreveriam \textit{vidas} a inventar os pormenores do verossímil
particular, que constituiriam a posteridade de sua fama, nem os encômios iriam
comemorar o monumento e o documento de suas obras; muito menos irão tornar"-se
modelos imitáveis de alguma coisa, ou suas palavras continuarão a mover as
gerações após a sua morte. Homens letrados como Vieira e Gracián provavelmente
cursaram a carreira sob o \textit{desejo}, digamos assim, desta última
possibilidade; talvez também um advogado e diz"-que excelente poeta, Gregório de
Matos e Guerra, é provável que tenha sido parcialmente salvo da obscuridade,
graças às \textit{vidas} que dele deixaram, provavelmente não sem interesse,
algumas tradições de compiladores, que as inventaram segundo a convenção do
gênero.  Silva Alvarenga participa de um sistema meritório mais ou menos
codificado assim ainda, com algumas particularidades, pois é bom lembrar que
Pombal nos anos de 1750, em nome da povoação dos domínios da cristandade		\EP
portuguesa, dera altura de português em quase tudo a colonos de comprovada
utilidade nos serviços de Sua Alteza Real, confirmados na lealdade da Monarquia
Lusitana e da Santa Fé Católica. Além disso, encoraja as uniões matrimoniais
entre colonos e libertas convertidas, em nome do universalismo cristão que o
reino constituía como positivação política.  Na condição política de filho de
africanos, como quiseram alguns, ou mais provavelmente de indígenas americanos
por parte de mãe, ser advogado é decerto uma possibilidade jurídica recente para
a família que alcançou colocar Silva Alvarenga na Universidade de Coimbra, por
exemplo.

Nas duas edições impressas em vida do autor (1774 e provavelmente 1788), o poema
de Silva Alvarenga saiu com o título composto \textit{O desertor: poema
herói"-cômico}. Contudo, foi comumente referido como \textit{O desertor das
%MANTER AS PARTICULARIDADES SÓ AQUI.  
Letras}, desde Balbi, talvez porque assim se declarasse mais especificamente o
argumento do poema, talvez porque assim tivesse soado bem. Mas certamente, para
um poema heroi"-cômico, \textit{O desertor} é um título suficiente para declarar o
seu teor, pela natureza do desvio que o sentido bélico do termo já representa.
De qualquer maneira, a fórmula \textit{o desertor das letras}, que quase se
tornou mais conhecida que o título original, foi retirada do primeiro verso do
poema. Ali, como se viu, o poeta invoca a Musa e já enuncia a matéria do poema
--- ``Musa, cantai o desertor das letras'' --- seguindo daí em elocução camoniana
a súmula dos principais sucessos do anti"-herói que protagoniza a narrativa.  A
elocução é alta e grave como n'\textit{Os lusíadas}, mas em tom paródico, uma
vez que a gravidade da dicção não corresponde decididamente à baixeza da
matéria, cuja depreciação só a palavra \textit{desertor} já bastaria para
declarar. No primeiro verso da \textit{Ilíada} --- ``Canta"-me, ó Deusa, a ira
funesta de Aquiles Pelida'' ---, o poeta pede à divindade que inspire o canto e
imediatamente propõe a matéria do poema nomeando o herói (Aquiles), sua
ascendência (filho de Peleu) e a paixão que o tipifica na qualidade de bravo
guerreiro, sendo o destempero de sua ira a causa dos maiores danos ocorridos em
Troia.

Neste sentido, pela inversão paródica, o primeiro verso de \textit{O desertor}
emula a primeira autoridade grega da poesia narrativa (diegética) de
matéria heroica, a epopeia; mas não declara de imediato o nome do herói infame.
Na maior parte das doutrinas épicas que até o século \textsc{xviii} tiveram
circulação, a epopeia devia imortalizar o nome dos heróis, tornando duradoura a
sua \textit{kléos}, isto é, a sua \textit{fama}, juntamente com a de seus pais e
de seus filhos.  Em perspectivas cristãs europeias, que supunham por exemplo a
guerra justa, e em perspectivas monárquicas e aristocráticas, que ordenavam o
Estado sob parâmetros da hierarquia militar, a poesia heroica da epopeia
homérica deveria ensinar aos moços as virtudes de guerreiro --- constância,
disciplina, temperança e principalmente a coragem.  O bom guerreiro deveria ter
coragem porque a fama, que, mormente em regimes patriarcais monárquicos,
legitimava o mando herdado pelo tronco familiar, era explicada como decorrência
dos perigos por que passaram os mais antigos senhores da terra e que as
narrativas antigas teriam conservado com este fim. A ira mesma, embora fosse
classificada como uma paixão da alma e por isso devesse ser subjugada pelo
intelecto conforme doutrinas da alma vigentes, podia ser funesta por suas
consequências, mas, como alguns moralistas da época expuseram, essa paixão
poderia servir de ornamento da bravura, aumentando o seu efeito. Segundo esse
raciocínio,  desde que se evitasse o extremo que leva a impiedades excessivas,
como as de Aquiles irado, a ira pode tornar mais forte a coragem, desde que
dirigida a causas tidas por justas.  Daí que a ira de Gaspar ou de Tibúrcio não
possa ser qualificada senão como baixeza de caráter, porque a dirigem ora a um
velho que fala coisas justas, ora a uma moça que por eles tinha sido enganada.
Assim, também entendia"-se inversamente o herói de \textit{O desertor}
apresentado pela palavra que, dando só ela título ao poema, já o desqualificava
pelo vício fraco da \textit{covardia}, que faz um mau guerreiro debandar dos
trabalhos da guerra, como analogamente Gonçalo e companhia desertam dos esforços
do estudo.

Aproximadamente assim ensinavam tanto doutrinas poéticas como doutrinas
políticas acerca da origem do poder hereditário dos senhores bem como acerca das
razões políticas da narrativa heroica.  Ao preferir invocar a Musa, o poema
também declara a emulação da \textit{Eneida} de Virgílio, que canta as altas
peripécias do pio Eneias e que no oitavo verso enuncia --- ``\textit{Musa, mihi
causas memora}'' (``Musa, relembra"-me as causas''). Com efeito, como na
\textit{Eneida}, ao invés de pedir à Deusa da memória, Mnemosine, o poeta de
\textit{O desertor} invoca as Musas, sem nomear em momento algum, como era
costume, qual delas seria a que dominava no específico da sua arte. A ocultação
do nome da divindade podia estar prevista nos usos da arte, mas num gênero  deliberadamente misto como o heroi"-cômico a Musa não indicada
principalmente mantém dúbia a natureza da espécie poética, já
também por isso cômica. Com isso, pode fingir a semelhança com a arte solene de
Calíope, mas logo demonstra o riso indolor de Talia, isto é, efetivamente o
poema pertence à arte da comédia, pela matéria que inventa, e à arte da epopeia,
pelo estilo e modo com que inventa.

%, ..que com que equivoca, entre Calíope, musa da epopeia,
%e Talia, musa da comédia, os gêneros poéticos já indicados
%no subtítulo ``poema heroi"-cômico'', 
%Imita consequentemente diversos autores ..o herói e o vício forte
%que o caracteriza como irascível

Na \textit{Odisseia}, Palas Atena se transforma num velho sábio e vai até o
jovem e ajuizado Telêmaco, para que tome a função paterna, assuma os trabalhos
da necessidade e com coragem vá buscar notícia do pai. De forma análoga, a
Ignorância, em \textit{O desertor}, transforma"-se num antiquário que vivia em
Coimbra, porque havia frustrado a carreira nas letras, e vai até Gonçalo
dissuadi"-lo de seguir nos estudos, porque são inúteis os esforços das letras e,
com as luzes das reformas recentes, já não havia mais ali as liberdades para o
gozo dos gostos baixos que antes se proliferavam como vícios.  Como na poesia
heroica a virtude capital é a coragem, na fábula do poema heroi"-cômico, ao
contrário, só era preciso a Gonçalo vencer o medo do tio, que mantinha os seus
estudos e o haveria de espancar quando retornasse sem tomar grau.

O herói dessa antiépica devia dedicar"-se às leis e às ciências, à teologia, à
jurisprudência, mesmo à história que deleita mas ensina, como a boa poesia. Em
vez disso, lia maus romances, \textit{litteratura} muito lidas e mal reputadas
na preceptiva da arte. Mesmo sendo letras vulgares, representam a moral, por isso vendidas na conformidade da lei. Mas eram livros mal reputados,
como livros de pouca ciência e pior arte, que mais \textit{deleitavam} do que
\textit{utilizavam}, como se dizia, o que não convinha às boas letras. Na tese
que o poema encena, essa tipificação do mau estudante e seu mau hábito de
leitura eram efeitos das dificuldades dos maus livros que se ministravam na
Universidade no tempo dos jesuítas.  Na lógica do poema, os estudantes começavam
a desertar das letras quando se entregavam a leituras mais doces do que
docentes, mais agradáveis do que úteis para causar virtude. E a causa disso, na
propaganda pombalina, era também a dificuldade desnecessária das questões dos
padres chamados peripatéticos, como os famosos Sanches e Molina.

Ivan Teixeira, em seu estudo sobre a poesia nos círculos de mecenato pombalino,
reproduz uma gravura pombalina onde se veem os padres da Companhia de Jesus
derrubando uma árvore do conhecimento. Nos seus galhos e ramos, viam"-se as
figuras dos maiores doutores da Igreja, como São Jerônimo, Jasão de Nores, Tomás
de Aquino, Santo Agostinho, o Venerável Beda, São Gregório Magno, entre outros,
e sobre eles o Espírito Santo, o sagrado coração e a imagem de Deus pai.  O
fruto das obras desses santos padres é o que o título da gravura refere como
\textit{O trabalho perdido}, isto é, o estrago orquestrado pelos padres gerais
da Companhia de Jesus, que o gabinete pombalino faria passar por uma república
dentro da República e, por isso, perigosa para a soberania do Rei e de seus
herdeiros no reino. 

Essa árvore de autoridades das maiores ciências para o catolicismo pombalino
aparece sendo serrada justamente por Sanches e Molina, assim indicados em lema
na gravura.  Ambos são teólogos assim chamados casuístas, comentadores da
\textit{traditio} de doutrina que a Universidade ensinava. São também os mesmos
vituperados entre outros na estante do tio, no canto \textsc{v}, de \textit{O
desertor}.  Com efeito, o Concílio de Trento continuava publicado e ensinado em
Portugal, enquanto a maior parte dos autores associáveis ao assim chamado
Iluminismo, principalmente francês, continuavam proibidos no reino, como também
ao contrário muitos autores jesuítas mais respeitados durante o auge da segunda
escolástica continuavam a ser modelos exemplares de virtude e de doutrina para o
ensino, como é o caso dos santos fundadores da Companhia de Jesus e de outros
nomes de jesuítas célebres como Francisco Suarez e Antônio Vieira. 
%% por outras
%razões não são reconhecidos como dignos na ciência de parte do que ensinam
%Gracián e Tesauro, porque a comum opinião acerca do bom estilo na escrita das
%cartas, dos ofícios, o que recaía sobre o estilo dos livros de história fingida
%ou verdadeira, sobre os livros de desengano amoroso, nos louvores de feitos
%ilustres etc.

A Ignorância, a vilã da trama cômica, é expulsa de Coimbra por efeito daquela
ação heroica que restituía a Verdade ao seu trono na velha instituição de
ensino.  A Verdade é uma espécie de divindade patronal do poema, que se
representa pela emulação de Minerva, a Palas Atena dos poemas de Homero.  Com
efeito, a Verdade aparece em sonho ao herói da fábula cômica que já derrocava
para em vão tentar convencê"-lo a decidir"-se por melhores feitos, para obter
melhor fama. Mas o entendimento já estava decidido pelo mau caminho, e as
circunstâncias levavam sempre o herói a outras piores circunstâncias, até que
chegam à província. Na biblioteca do tio, veem"-se grandes livros de um século
antes. Não é só a literatura seiscentista vituperada, é o que de melhor o tio
pôde juntar no seu tempo. Mas as reformas do ensino e o próprio tempo fizeram
muitos daqueles livros descorarem de sua autoridade antiga. Perdida a carreira,
Gonçalo agora não teria mais do que isso. Uma biblioteca de aldeia, com todas as
marcas do tempo, para eventualmente tornar"-se um rábula, advogado sem diploma
que, no contraste das luzes que o poema forja, é a mais escura obscuridade.

Boécio e Tomás de Aquino, padres da Igreja dos primeiros e dos últimos tempos da
Idade Média, eram no século \textsc{xviii} distintos como santos ou beatos, entre
outras razões pela doutrina que retiraram sobretudo do ``divino
Aristóteles''.\footnote{ Essa adaptação de Aristóteles à teologia católica
refeita sobretudo ao longo do século \textsc{xviii} por grandes autoridades
eclesiásticas, mormente jesuítas, ficou conhecida como segunda escolástica, cujo
fim era fortalecer os fundamentos filosóficos que embasaram as cláusulas do
Concílio de Trento, defendidas contra a heresia pelo Tribunal do Santo Ofício da
Inquisição.} Com o fim da era jesuítica no ensino português, a reforma pombalina
da Universidade representou também uma relativização do aristotelismo que se
ensinava na Universidade até o século \textsc{xviii}. Com isso, porém, não se
chegou a efetivamente destituir a autoridade de Aristóteles como o mais
importante filósofo grego para a doutrina católica, que permanecia sendo a
religião do rei e do reino. Aristóteles assim havia sido considerado muitas
vezes desde a primeira escolástica do século \textsc{xiii} e, muito antes dela,
desde alguns dos primeiros doutores da Igreja dos séculos \textsc{v} e
\textsc{vi} d.C.  Sem nem de longe ferir a autoridade de nomes como esses, e
entre tantos outros nomes tão ilustres, mais de uma vez o poema de Silva
Alvarenga faz alusão aos maus métodos dos peripatéticos, que é o nome com que
são designados os seguidores de Aristóteles. Contudo, ali se fala mal
principalmente das apostilas e cadernos, das antologias e compilações, feitos
por professores portugueses que ensinavam lógica aristotélica por métodos que
então foram postos em descrédito.  Nem por isso Silva Alvarenga deixa de inserir
antes do poema um ``Discurso sobre o poema heroi"-cômico'' que começa justamente
citando Aristóteles como respeitada autoridade no ensino das regras da arte
poética e dos fins morais, gerais e particulares que se perfazem na leitura da
poesia em geral e do poema heroi"-cômico em específico.

Como se viu, o tempo que ficou conhecido pela posteridade como ``período
pombalino'' teve início em 1750, um ano após o nascimento de Manuel Inácio da
Silva Alvarenga, e, por ocasião da impressão de seu poema, Pombal acabava de
levar a termo, em 1772, a célebre Reforma da Universidade de Coimbra. A
principal instituição de ensino portuguesa havia sido posta em descrédito,
segundo a propaganda pombalina, por culpa dos membros da Companhia de Jesus.
Desde o século \textsc{xvi}, os jesuítas a geriram e nos últimos tempos teriam
deixado medrar maus hábitos entre professores e estudantes, consequência dos
métodos antiquados que usavam, sempre segundo a opinião que a política de Pombal
fez imperar. 

Desta mesma opinião politicamente produzida, \textit{O desertor},
esse exercício poético de estudante comprometido com as novas diretrizes da
Universidade, é uma peça casualmente estratégica. Alfredo Bosi entendeu essa
posição como a de um típico militante ilustrado. Talvez, menos do que isso, a
posição de Manuel Inácio da Silva Alvarenga seja a de um bom súdito.  Ao lado de
muitas outras obras que tiveram o mesmo comprometimento, \textit{O desertor}
integrou o que Ivan Teixeira chamou ``mecenato pombalino'', caracterizando com
essa expressão o agrupamento de poetas, artistas, juristas, eruditos,
professores etc, em torno de Sebastião José de Carvalho e Melo, empenhados no
louvor de seu governo.  Acordes em perpetuar em monumentos de memória as
reformas implementadas, trataram"-nas como o nascimento de uma nova era, ou como
o renascimento de uma idade áurea antiga, que se vestia por exemplo como um novo
século de Augusto, cantado por outros Virgílios e Horácios lusitanos, que
obviamente não eram iluministas, nem poderiam simpatizar com opiniões tão
perigosas, para o mundo católico e monárquico de que eram parte.

Em contraste com essa nova Idade do Ouro, assim pintada em prosa e verso, o
jesuitismo reduzido a tipos sórdidos, vilões de comédia, ou alegorizado como
Hipocrisia, Abuso, Ignorância, Monstro de mil olhos foi representado como
trevas. Com isso, a historiografia literária quis equivocar sombras barrocas,
renascidas de trevas medievais, provavelmente para imputar iluminismo na
política de Estado do Marquês e de seu séquito de letrados.  Fugindo destas
positivações metafóricas de antigos artifícios, já que luzes e sombras aí são só
metáforas, podemos dizer que, para a comum opinião sob Pombal, a gestão
jesuítica da Universidade teria imposto a seus currículos velhos métodos,
baseados fundamentalmente na leitura católica da lógica aristotélica.


Desde os séculos \textsc{xix} e \textsc{xx}, com o fim de enquadrar o Marquês de
Pombal no Iluminismo europeu, foi recorrente a interpretação historiográfica que
quis dar à expulsão dos jesuítas um caráter anticlerical, como se suas reformas
visassem a atingir o clero português e assim seu governo tivesse uma posição
assimilável à de outros monarcas e outros ministros que ficaram conhecidos como
``déspotas esclarecidos'', interpretados como parte, mesmo que contraditória, de
um movimento geral da Ilustração.  Contudo, a luta institucional do governo
pombalino é quase que exclusivamente dirigida à Companhia de Jesus nas pessoas
de seus membros atuais, que obtiveram poder provavelmente pelo favorecimento
institucional, nas dependências de Dom João \textsc{v}, falecido em 1750.
Acresce que os lugares institucionais anteriormente ocupados pelos jesuítas,
sobretudo os relativos à educação, viriam a ser dados, quase sempre, também a
padres, mas agora preferencialmente os oratorianos, ordem religiosa de origem
francesa, ascética em sua doutrina de vida, como a dos inacianos, e
misteriosamente interessada na vida política, como aqueles, sobretudo na
instrução dos homens também.

O Portugal de \textit{O desertor} não se tornava mais ilustrado nem se laicizava
além da conta e das tradições jurídicas conhecidas.  A disputa representou"-se
como uma querela institucional que teve o tamanho que teve.  Não precisaria ser
reinventada como imbuída de significados transistóricos, que organizam o
trânsito do Espírito, das ``manifestações culturais'' e das ``mentalidades'' das
épocas.

Todo o Estado português continuou a ter muitos clérigos em seus postos mais ou
menos altos, tanto nos Conselhos do Estado, quanto nas instituições de ensino e
na Mesa Censória dos livros impressos no reino.  As figuras mais típicas do
assim chamado Iluminismo português, como Luís Antônio Verney e Francisco José
Freire, eram igualmente clérigos que tinham o Concílio de Trento como a
verdadeira e grande restauração moderna das ciências, o que torna muito estreito
o que de Iluminismo o pensamento sem dúvida \textit{ilustrado} desses homens
pode ter representado; mesmo porque ``ilustrado'' é um termo que em português
sempre significou culto, erudito, cheio de ciências, sendo a Teologia, antes
como neste século \textsc{xviii} ibérico, a mais alta das ciências.

Conhecedor dos sistemas de gênero e espécies que desde Aristóteles operavam, de
várias formas, a invenção, a disposição e a elocução da poesia \textit{em
geral}, isto é, \textit{enquanto gênero} de imitação, conhecedor também dos
sistemas jurídicos civis e eclesiásticos que regiam a Monarquia portuguesa, o
poeta estava longe de pertencer a um ``\textit{club} de jacobinos'', como se lê
nas acusações da Devassa. Aliás, ``\textit{club} de jacobinos'' foi
provavelmente uma agudeza vituperante que põe em evidência algumas posições de
uma cena política coetânea. O \textit{club} é uma sociedade pacífica de pares
que se distinguem mutuamente como \textit{socii} (sócios).  Grafada à maneira de
ingleses, \textit{club}, a sociedade de pares é representada, no vitupério, como
coisa de anglicanos, gente que por mais polida que fosse era sectária do credo
decretado herético havia dois séculos pelas mais altas cúrias eclesiásticas que
a Monarquia lusitana acatava integralmente. A \textit{club} junta"-se o adjetivo
\textit{jacobino}, o que haveria de pior e mais horroroso em termos de impiedade
política laica, no ponto de vista, ou melhor, segundo a ética portuguesa de
então.  Assim, a infâmia que recaía sobre Alvarenga se indiciava por meio desta
prova: em ditos mordazes falava"-se do grupo de Silva Alvarenga como um conluio
de sujeitos mistos de revolucionários franceses e anglicanos hereges reunidos em
sociedade aparentemente pacífica.  A \textit{Devassa} que Silva Alvarenga
enfrenta acusa"-o de \textit{francesia}, associando"-o a opiniões revolucionárias
francesas, isto é, à opinião política que sustentava gente desqualificada nas
hierarquias políticas do reino ocupando ilegitimamente o lugar do rei. Para a
contemporaneidade portuguesa de Silva Alvarenga na década de 1790, certamente o
perigo disso sentia"-se como enorme, mas não se marcavam esses eventos
particulares que hoje são \textit{a Revolução Francesa} senão como um distúrbio
assimilável às sublevações que as histórias antigas e modernas nunca deixaram de
contar.

\section{Sobre a edição}

O texto desta edição foi estabelecido a partir da edição de 1774,
utilizando o exemplar do Instituto de Estudos Brasileiros (\versal{ieb}-\versal{usp}).
Para decisões específicas, valemo"-nos também da edição de Joaquim
Norberto, de 1864, da edição feita por Ronald Polito, de 2003, e
das indicações da tese de Francisco Topa (ver, abaixo, Bibliografia).

Foi feita a atualização ortográfica, mesmo dos nomes próprios. Mantiveram"-se
apenas as maiúsculas, porque muitas vezes são empregadas para dar sentido
alegórico a conceitos abstratos, como são os casos emblemáticos para o poema
em questão das palavras ``Verdade'' e ``Ignorância'', que em quase todas as
ocorrências representam"-se personificadas em ação alegórica.
Conforme observa Ronald Polito, diversas palavras aparecem grafadas com maiúsculas,
``no entanto não são homogêneos os critérios da primeira edição'' [p. 61].
Contudo, muitas vezes também ``arrieiros'', ``estudantes'', entre outros substantivos
comuns em uso aparentemente simples são grafados com maiúsculas sem critério identificável.
Sendo assim, atendemos a argumentação de Francisco Topa: ``em atenção ao \textit{usus
scribendi} do autor e aos hábitos da época \textit{é possível} conservar maiúsculas
não justificáveis gramaticalmente, atendendo também ao seu possível valor expressivo'' [p. 19].
Como não oferecem dificuldade para a leitura, mantivemos as maiúsculas,
também para relativizar o esquematismo que define alegorização de conceitos por
meio de maiúsculas. É possível que maiúsculas também indiquem ênfases para a
\textit{pronuntiatio}, já que, na época, estava prevista a \textit{performance},
ou representação do poema heroi"-cômico, assim como se fazia encenação pública do
poema heroico. Por esse aspecto residual do uso que o poema teve, achamos interessante
a reprodução das maiúsculas e minúsculas conforme a edição de 1774. % PRECISO REVISAR AS MAIÚSCULAS COM O fIND.

Não obstante tudo isso, preferimos atualizar a pontuação para facilitar a leitura moderna.
Por se tratar de um texto narrativo, a pontuação retórica, provavelmente também
seguindo critérios da pronunciação, dificulta a fluência da leitura silenciosa
deste texto cuja unidade para o entendimento já é, de saída, bastante difícil, seja pelos cortes bruscos presentes no poema, que talvez indiquem a precariedade da composição do entrecho, 
seja por estar inscrito em outro registro de representação ficcional, diverso por exemplo
da narrativa de romance em prosa, com a qual, desde o século \versal{xix}, tendemos a estar mais
acostumados.
Neste mesmo sentido, preferimos inserir aspas para sinalizar falas diretas das personagens,
que não são muitas e não devem ser confundidas com as apóstrofes da voz heroi"-cômica que
narra e que, vez e outra, interpela fantasticamente as próprias personagens, como estava
previsto na convenção da poesia épica em geral.
Tais apóstrofes mantivemos sem alteração mais do que a atualização já referida da
pontuação, algumas vezes trocando exclamação por interrogação.
A pontuação sempre que possível não foi inserida para não fechar as possibilidades
abertas de leitura. Principalmente, foram trocados os sinais de dois pontos declamatórios,
que se sucediam por exemplo nas enumerações, indicando a disposição da matéria em orações
correlatas. Nestes casos foram trocados por vírgula ou ponto"-e-vírgula. Nos símiles homéricos,
isto é, nas comparações extensas, que são abundantes por conta do gênero do poema,
mantivemos os dois pontos marcando os dois hemistíquios da analogia. As notas apostas ao poema são de Silva Alvarenga, por isso optamos pela composição de um glossário de termos poéticos, históricos, biográficos e geográficos que se encontra ao fim do volume. Ainda para apoiar a leitura do texto inserimos antes de cada canto um argumento, com a súmula da ação que irá transcorrer.


\pagebreak

\begin{bibliohedra}
\tit{alvarenga}, Manuel Inácio da Silva. \emph{Obras poéticas de Manoel
Ignacio da Silva Alvarenga (Alcindo Palmireno) collegidas, annotadas, e precedidas do juízo crítico dos escritores nacionais e
estrangeiros e de uma notícia sobre o autor e suas obras e acompanhadas de documentos históricos}, org. J. Norberto de Souza.
Paris/ Rio de Janeiro: Garnier Irmãos, 1864. Brasilia Bibliotheca dos Melhores Auctores Nacionaes Antigos e Modernos:
Silva Alvarenga.

\tit{\_\_\_\_\_\_.} \emph{O desertor: poema herói-cômico}. Coimbra: Na real
officina da Universidade, 1774.

\tit{\_\_\_\_\_\_.} \emph{O desertor: poema herói-cômico}, org. Ronald Polito.
Campinas: Editora da Unicamp, 2003.

\tit{alvear}, D. A. \& \textsc{dávila}, D. J. Herrera. \emph{Coleccion de tratados
breves y metodicos de Ciencias, Literatura y Artes: Biografia
Antigua}. Sevilla: Imprensa de D. Mariano Caro, 1829.

\tit{ambrósio}, Renato. \emph{De rationibus exordiendi: os princípios da
história em Roma}. Associação Editorial Humanitas – Fapesp,
2005.

\tit{aristóteles}. \emph{Rettorica et poetica d’Aristotile. Tradotte di greco
in lingua vulgare Fiorentina da Bernardo Segni Gentilh’huomo,
\& Academico Fiorentino}. Vinegia: per Bartholomeo detto
l’Imperador, \& Francesco suo genero, 1551.

\tit{balbi}, Adrien. \emph{Essai Statistique sur le royaume de Portugal et
D’Algarve, comparé aux autres états de l’Europe, et suivi d’un
coup d’oeil sur l’état actuel des sciences, des lettres et des beaux-arts parmi les Portugais des deux hémisphères}, vol. \textsc{ii}. Paris: Rey et Gravier, 1822.

\tit{barbosa}, Januário da Cunha. \emph{Revista trimensal de História e 49
Geografia ou Jornal do Instituto Histórico e Geográfico Brasileiro, fundado no Rio de Janeiro sob os auspícios da Sociedade
Auxiliadora da Indústria Nacional, debaixo da imediata proteção
de S. M. I. O Senhor D. Pedro \textsc{ii}}., vol. \textsc{iii}. \textsc{ihgb}, 1841.

\tit{bluteau}, Rafael. \emph{Vocabulario Portuguez e Latino, aulico, anatomico, architectonico, bellico, botanico, brasilico, comico, critico,
chimico etc.} Coimbra: no Collegio das Artes da Companhia de
\textsc{jesu}, 1712. Autorizado com Exemplos dos Melhores Escritores
Portugueses e Latinos e offerecido a El Rey de Portugal D. Joao
V pelo padre D. Raphael Bluteau Clerigo Regular, Doutor na
Sagrada Theologia, Pregador da Rainha de Inglaterra Henriqueta Maria de França, e Calificador no Sagrado Tribunal da
Inquisição de Lisboa.

\tit{boxer}, Charles. \emph{O império marítimo português. 1415--1825}. São
Paulo: Companhia das Letras, 2008. Tradução Anna Olga de
Barros Barreto.

\tit{camões}, Luís Vaz. \emph{Os Lusíadas}. Em casa de Antonio Gonçalvez
Impressor, 1572. Com priuilegio Real. Impresso em Lisboa,
com licença da sancta Inquisição, \& do Ordinario: em casa de
Antonio Gõnçaluez Impressor.

\tit{candido}, Antonio. \emph{Formação da literatura brasileira}, vol. \textsc{ii}. São
Paulo: Itatiaia/Edusp, 1975.

\tit{\_\_\_\_\_\_.} “Os poetas da Inconfidência.” \textsc{ix} \textsc{Anuário da Inconfidência} (1993): 130--137.


\tit{coleridge}, Henry Nelson. \emph{Introductions to the Study of the
Greek Classic Poets. Designed principally for the use of Young
persons at School and College. Part \textsc{i}: General Introduction. Homer.} London: John Murray, Albemarle Street, 1834.

\tit{foucault}, Michel. \emph{O que é um autor?} Lisboa: Passagens/ Nova
Vega, 2006, 6 ed. Prefácio de José A. Bragança de Miranda e
Antonio Fernando Novais.

\tit{hansen}, João Adolfo. \emph{A sátira e o engenho. Gregório de Matos e
a Bahia do século \textsc{xvii}}. São Paulo: Companhia das Letras, 1989.

\tit{\_\_\_\_\_\_.} “Autor”, in: Jobim, José Luís. (Org.). \emph{Palavras da
crítica}. São Paulo: Imago, 1992.

\tit{horácio}. \emph{Arte Poetica de Q. Horacio Flacco, Traduzida, e illustrada em Portuguez por Candido Lusitano}. Lisboa: Na Officina
Rollandiana, com Licença da Real Meza Censória, 1778.

\tit{jesus}, Frei Rafael de. \emph{Primeiro volume da 18ª parte da “Monarchia Lusitana”}, vol. \textsc{i}. Coimbra: Biblioteca Geral da Universidade de Coimbra, 1958.

\tit{junta de providência literária}. \emph{Compêndio histórico do estado
da Universidade de Coimbra no tempo da invasão dos denominados jesuítas e dos estragos feitos nas sciencias e nos professores, e
diretores que regiam pelas maquinações, e publicações dos novos
Estatutos e por eles fabricados}. Na Régia Oficina Typographica,
\textsc{mdcclxxi}.

\tit{lucrecio}, Tito. \emph{A natureza das coisas, poema de Tito Lucrécio
Caro}. Traduzido do original latino para verso portuguez por
Antonio José de Lima Leitão. Lisboa: Typographia de Jorge
Ferreira de Matos, 1851.

\tit{mesnardière}, Jules de la. \emph{La Poetique de Jules de la Mesnardiere}. Paris: Antoine de Sommaville, 1639.

\tit{minturno}. \emph{L’Arte Poetica del Signor Minturno Nella quale si
contengono i preccetti Eroici, Tragici, Comici, Satirici, e d’ogni
altra Poesia: con la dottrina De’Sonetti, Canzoni, ed ogni forte
di Rime Toscane, dove s’insegna il modo, che tenne il Petrarca nelle sue opere. E si dichiara a’suoi luoghi tutto quel, che da 51
Aristotele, Orazio, ed altri Autori greci, e Latini è stato scritto
per ammaestramento de’Poeti}. Napoli: Stamperia di Gennaro
Muzio, erede di Michele Luigi con Licenza de Superiori, 1725.

\tit{peixoto}, Ignacio José de Alvarenga. \emph{Obras poéticas de Ignacio
José de Alvarenga Peixoto colligidas, annotadas precedidas do
juízo crítico dos escriptores nacionaes e estrangeiros e de uma
noticia sobre o autor e suas obras com documentos históricos}, org.
J. Norberto de Souza. Rio de Janeiro: Garnier, 1865.

\tit{real academia española}. \emph{Dicionario de la lengua castellana,
en que se explica el verdadero sentido de las voces, su naturaleza
y calidad, com las phrases o modos de hablar, los proverbios o
refranes, y otras cosas convenientes al uso de la lengua}. Imprenta
de la Real Academia Española, por los herederos de Francisco
de Hierro, 1734.

\tit{reis}, Francisco Sotero dos. \emph{Curso de literatura portuguesa e
brasileira}. Maranhão, \textsc{mdccclxvii}.

\tit{rengifo}, Ivan Diaz. \emph{Arte poética española, con una fertilissima
silva de consonantes comunes, propios, esdruxulos, y reflexos, y
un divino estimulo del amor de Dios}. Madrid: por la viuda de
Alonso Martin, 1628.

\tit{silva}, António José da. \emph{Esopaida ou vida de Esopo}. Coimbra:
Acta Universitatis Coninbrigensis, 1979.

\tit{silva}, J. M. Pereira da. \emph{Parnaso brasileiro ou Selecção de poesias
dos melhores poetas brasileiros desde o descobrimento do Brasil
precedida de uma introdução histórica e biográfica sobre a literatura brasileira}, vol. \textsc{i}. Rio de Janeiro: Eduardo e Henrique
Laemmert, 1843.

\tit{\_\_\_\_\_\_.} \emph{Plutarco Brasileiro}, vol. \textsc{ii}. Rio de Janeiro: Eduardo e
Henrique Laemmert, 1847.

\tit{spinelli}, Miguel. \emph{Caminhos de Epicuro}. São Paulo: Edições
Loyola, s.d.

\tit{teixeira}, Ivan. \emph{Mecenato pombalino e poesia neoclássica}. São
Paulo: Edusp, 1999.

\tit{topa}, Francisco. \emph{Para uma edição crítica da obra do árcade Brasileiro Silva Alvarenga – Inventário sistemático dos seus textos e
publicação de novas versões, dispersos e inéditos}. Porto: mimeo,
1998.

\tit{weinberg}, Bernard. “From Aristotle to Pseudo-Aristotle.” \emph{Comparative Literature} 5 (1953): 97--104. Duke University Press on
Behalf of the University of Oregon.
\end{bibliohedra}

%****

%É comum lembrar que
%o governo de Pombal reivindicasse para o rei Dom José \textsc{i} o direito de fazer
%bispos é uma querela que, no século \textsc{xii}, Dom Afonso Henriques, o Fundador do Reino, emulado
%pela representação política de Dom José \textsc{i}, chamado ``pai da pátria''
%no encômio.\footnote{ Ivan Teixeira reúne um vasto material encomiástico em
%torno de Sebastião José e consequentemente sobre Dom José \textsc{i}. } 
%O poder dos reis de exercer poder sobre o bispado do reino é uma prática que
%o papa proibiu a Afonso Henriques, o qual a reivindicava como legítima.
%A ordenação real de bispos não deixa de ser sem dúvida uma séria pendência
%judicial, pertinente ao Direito Canônico e logo ao Direito Natural, não é
%um ``traço de Iluminismo'', uma ``marca da época'', um sinal de que Portugal
%participava de certa história do Espírito, mesmo quando disfarçado das mais
%variadas formas. Práticas inglesas adaptadíssimas à prática
%institucional dos estabelecimentos dinásticos cujo centro estava assentado
%às margens do Tejo e de outros ribeiros daquela ponta da Península Europeia.
%Inglaterra que de inimiga entra no trato português que não se É impossível não
%sorrir pensando que as cartas expedidas pelo gabinete do Embaixador português
%fossem ridicularizadas por serem por exemplo \textit{barocas}, isso, que
%assim ficou sendo depois como uma positividade reconhecível que se \textit{manifesta}
%como a face do Espírito ou como a conjunção da materialidade. É interessante
%pensar que as ``cartas de Pombal'', isto é, muito antes disso, as cartas da
%\textit{representação} portuguesa na Inglaterra eram escritas segundo uma arte
%retórica ali enquadrada como uma eloquência antiquada, cujo estilo muitas vezes
%asiático, ou ciceroneano foi interpretado como redundante e excessivo tornava"-se
%ali uma arte velha, correspondente a uma ciência antiquada. %

%Essa contemporaneidade, súdita dos herdeiros de Dom João \textsc{iv}, não poderia crer senão
%que seria na França tudo seria conduzido a um reestabelecimento da ordem, segundo
%a melhor forma de governo que, segundo a tratadística política aristotélica, só
%poderia ser a monárquica. É significativo que a palavra democracia, constante em
%estatutos das Sociedades que Alvarenga compôs, seja alegada,
%nos autos do processo, como indício das opiniões que ali se trocavam. Mais significativa
%é a resposta do acusado que situa semanticamente a palavra pela data do seu uso,
%que sendo anterior às calamidades políticas mais recentes indicava outra acepção,
%que há muito era tirada justamente da trilogia aristotélica das formas de governo,
%que se lia principalmente no livro da \textit{Política}, de Aristóteles, cujos princípios
%estavam implicados nos livros da \textit{Retórica} e das \textit{Éticas}.
%A coisa toda tem de ser pensada como um quadro que contemporaneamente se pintava
%do levante violento que tinham sofrido os herdeiros dos herdeiros de Carlos Magno,
%nos eventos de 1789 e 1791, e seguintes. 
%A anglicização da política portuguesa na segunda metade do século \textsc{xviii}
%corresponde à imitação pombalina de hábitos políticos ingleses, isto é,
%trata"-se de uma representação política que recompila regulações específicas
%de decoros políticos diversos (o bom estilo das cartas, do \textit{sermo},
%isto é, da conversação civil, e dos demais gêneros de discurso, daí também o 
%bom estilo do ensino da poesia, que é pasto para a eloquência, que é exercício
%para futuros engenhos, os bons limites para a amplificação nos discursos
%epidíticos a pessoas particulares do reino, que estava proibida de inventar
%sem fundamento tradições de mérito,
%os preceitos para a composição dos caracteres das representações, as regras
%para a ostentação das precedências familiares nas aparições públicas, os
%novos métodos de ensinar desde a medicina até a ciência dos princípios das
%ciências, seja a Teologia seja a Metafísica, seja uma no interior da outra,
%conforme os currículos etc.). Tudo isso se remoça, com as vogas protocolares,
%desde que os novos usos não firam os estatutos sem os poder alterar. 
%

%Nesse
%sentido os decoros exigiriam mais ou menos declaradamente a necessidade de andar
%à moda. O professor Tibúrcio, que na juventude quis seguir carreira, mas provavelmente
%perdeu"-se nas questões da filosofia peripatética, cujo uso em Coimbra \textit{O
%desertor} vitupera, encarna a Ignorância alegorizada que talvez nunca tenha
%vestido à moda.

%para os despachos da Coroa e de suas depedências as leis e os saberes do reino, em primeiro lugar
%pondo a diante os estatutos irredutíveis das leis do reino. Além disso,
%mantém as folhas de pagamento mais ou menos inalteradas, mas revê estatutos
%que são passíveis de serem revisados por suas forças, como a Reforma da Universidade,
%das leis do Comércio. 

\part[O desertor: poema herói-cômico]{O desertor\break poema herói-cômico} 

\chapter{Discurso sobre o poema herói"-cômico}

A Imitação da Natureza, em que consiste toda força da Poesia, 				\index{\Iminat}
é o meio mais eficaz para mover e deleitar os homens; 					\index{\Util}
porque estes têm um inato amor à imitação, harmonia e ritmo.
Aristóteles, que bem tinha estudado a origem das paixões, assim o afirma no 
cap. 4 da \textit{Poética}. 							\index{\Arist}
Este inato amor foi o que logo ao princípio ensinou a imitar o Canto das Aves; ele 
depois foi o inventor da Flauta e da Poesia, como felizmente exprimiu Lucrécio
no liv. \versal{I}, v.~1378.								\index{\Lucre} \index{\Poet}


\begin{verse}
At liquidas avium voces imitarier ore \\
Ante fuit multo, quam l\ae via carmina cantu \\
Concelebrare homines possent, auresque \qb{}juvare. \\
Et Zephyri cava per calamorum sibila \qb{}primum \\
Agrestes docuere cavas instare cicutas.\footnote{ Segue uma tradução
setecentista dos versos: ``\textit{Das aves o terníssimo gorgeio/ Os homens
imitar co'a voz tentavam,/ Muito antes que cantando eles soubessem/ Articular
os versos sonorosos,/ Que hoje tanto os ouvidos nos encantam./ O silvo que dos
zéfiros se ouvia/ No oco das canas suscitar"-lhes pôde/ Dos cálamos agrestes a 
lembrança./ Pouco a pouco depois os sons maviosos/ Foi espalhando a cítara,
pulsada/ Por quem com doce voz a par lhe ia/ Nos bosques, selvas, brenhas, que
aos pastores,/ Por mudas solidões, por longos ócios,/ Da harmonia as primeiras
lições deram}.'' [p. 172] [N. do org.] }\\[10pt] 				\index{\Zefir}
\end{verse}

O prazer que nos causam todas as artes imitadoras
é a mais segura prova deste princípio. Mas assim 
como o sábio Pintor para mover a compaixão não 						%gloss imitação
representa um quadro alegre e risonho; também o hábil 
Poeta deve escolher para a sua imitação ações 
conducentes ao fim que se propõe. Por isso o Épico, 					\index{\Epico}
que pretende inspirar a admiração e o amor da virtude, 
imita uma ação na qual possam aparecer brilhantes o valor,
a piedade, a constância, a prudência, o amor da Pátria, 
a veneração dos Príncipes, o respeito das Leis e os 
sentimentos da humanidade. O Trágico, que por meio do terror 				\index{\Trag}
e da compaixão deseja purgar o que há de mais violento em  
nossas paixões, escolhe ação onde possa ver"-se o horror 
do crime acompanhado da infâmia, do temor, do remorso, da desesperação 
e do castigo; enquanto o Cômico acha nas ações vulgares 				\index{\Comic}
um dilatado campo à irrisão, com que repreende os vícios.

Qual destas imitações consegue mais depressa o seu fim
é difícil o julgar; sendo tão diferentes os caracteres, 
como as inclinações; mas quase sempre o coração humano,
regido pelas leis do seu amor próprio, é mais fácil em ouvir 
a censura dos vícios, do que o louvor das virtudes alheias.

O poema chamado Herói"-cômico, porque abraça ao mesmo
tempo uma e outra espécie de poesia, é a imitação 
de uma ação cômica heroicamente tratada. 
Este Poema pareceu monstruoso aos Críticos mais escrupulosos; 
porque se não pode (dizem eles) assinar o seu verdadeiro caráter. 
Isto é mais uma nota pueril, do que bem fundada crítica; 
pois a mistura do heroico e do cômico não envolve a
contradição, que se acha na Tragicomédia, onde o terror 
e o riso mutuamente se destroem. 							\index{\Comic}\index{\Heroic}

Não obsta a autoridade de Platão referida por muitos; 					\index{\Plat}
porque quando este Filósofo, no Diálogo 3 de sua \textit{República}, 			\index{\Trag}
parece dizer que são incompatíveis duas diversas imitações, 
fala expressamente dos Autores Trágicos e Cômicos, que jamais
serão perfeitos em ambas. 								\index{\Comic} \index{\Plat} \index{\Repub} \index{\Trag}

Esta Poesia não foi desconhecida dos Antigos.  Homero daria				\index{\Homer}
mais de um modelo digno da sua mão, se o tempo, que
respeitou a \textit{Batracomiomaquia}, deixasse chegar a nós o seu			\index{\Batr}
\textit{Margites}, de que fala								\index{\Margi}
Aristóteles no cap. 4 da \textit{Poética}, dizendo que este poema tinha com a		\index{\Arist}
Comédia a mesma relação que a \textit{Ilíada} com a Tragédia. O
\textit{Culex}, ou seja de Virgílio, ou de outro qualquer, não contribui 
pouco para confirmar a sua antiguidade.							\index{\Culex} \index{\Margi} \index{\Poet} \index{\Trag} %gloss cômico

Muitos são os poemas herói"-cômicos modernos. A \textit{Secchia
rapita} de Tassoni é para os Italianos o mesmo que o \textit{Lutrin}	
de Boileau para os Franceses, e o \textit{Hudibraz} de Butler e o
\textit{Rape of the Lock} de Pope para os Ingleses.					\index{\Secch} \index{\Lutri} \index{\Tasson} %gloss, Boileau

Uns sujeitaram o poema herói"-cômico a todos os preceitos da 
Epopeia e quiseram que só diferisse pelo cômico da ação,				\index{\Comic} 
e misturaram o ridículo e o sublime de tal sorte que servindo um 
de realce a outro, fizeram aparecer novas belezas em ambos os 
gêneros. Outros omitindo ou talvez desprezando algumas regras
abriram novos caminhos à sua engenhosa fantasia e mostraram
disfarçada com inocentes graciosidades a crítica mais 
insinuante, como M.~Gresset no seu \textit{Ververt}.

Não faltou quem tratasse comicamente uma ação heroica; 
mas esta imitação não foi tão bem recebida, ainda que a 
Paródia da \textit{Eneida}, de Scarron, possa servir de modelo. 

É desnecessário trazer à memória a autoridade e o 
sucesso de tão ilustres Poetas para justificar o Poema 
herói"-cômico, quando não há quem duvide, que ele, 
porque imita, move e deleita: e porque mostra ridículo o 				\index{\Util}
vício, e amável a Virtude, consegue o fim da verdadeira poesia.

\begin{verse}
Omne tulit punctum, qui miscuit utile dulci\footnote{ Quem sabe pois
tecer ação, que instrua, juntamente agrade.}\\[10pt]

\hfill Horat. \textit{Poet}. v.~342 
\end{verse}

\medskip

\begin{verse}
Discit enim citius, meminitque libentius illud,\\[-15pt]
Quod quis deridet, quam quod probat, ac \qb{}veneratur. 
\\[10pt]

\hfill Horat. \textit{Epist}., 1, \versal{ii}. v. 262
\end{verse}

\pagebreak
\paginabranca

\mbox{}\vfill
\thispagestyle{empty}
\noindent Argumento do Canto \versal{i}
\medskip

{\footnotesize\noindent Invocação das Musas -- Dedicatória -- Chegada triunfal do	\index{\Dedic} \index{\Invoc}
Marquês de Pombal a Coimbra -- Memória do período áureo do reino de Portugal		\index{\Marqpombal}
interrompido pela morte de Dom Sebastião -- 
Apresentação da Ignorância, personificação dos		
hábitos do ensino jesuítico na Universidade -- Divulgação da Reforma pombalina pelo rio Mondego --		\index{\Monde}
A Ignorância lamenta seu Império perdido -- A 		
Preguiça e a Ociosidade alegorizadas como colegas
de pensão da Ignorância -- Transfiguração 			
da Ignorância em Tibúrcio, antiquário que vivia em
Coimbra -- Apresentação do herói, Gonçalo, o Desertor das Letras -- Conselhos da
Ignorância -- O letrado frustrado e vendedor de objetos usados vitupera as disciplinas
do estudo, mimetizando a linguagem difícil do método escolástico
-- Desmerece a carreira das letras naqueles
tempos, que já não conferiam distinção e obrigavam a uma longa carreira nas partes
distantes do Império -- Exorta Gonçalo a voltar para Mioselha, rever o tio, fazendo o louvor da
aldeia -- Tibúrcio constitui a companhia de desertores das letras -- Imprecação da Velha
Guiomar, mãe de Narcisa -- Ira de Narcisa, amante de Gonçalo, com a partida
do amante -- Gonçalo a consola com uma bolsa de dinheiro e a desculpa de uma nova herança
-- Tibúrcio, experiente, lembra ao herói as 
despesas da viagem -- Não valendo os seus argumentos e sua encenação, Tibúrcio
usa o braço arrancando"-o de Narcisa -- Briga na pensão -- O povo se ajunta
com paus e pedras -- Narcisa termina com a bolsa, recontando o dinheiro -- Guiomar,
insatisfeita de a filha ter sido rejeitada, planeja fazer prender a Gonçalo -- Rodrigo
o avisa das maldições da velha e o aconselha a fugir -- Partida de Coimbra.}

\chapter{Canto \versal{I}} 

\begin{verse}
Musas, cantai o Desertor das letras\\
Que, depois dos estragos da Ignorância\footnote{ Depois de abolidos os velhos estatutos pela criação da nova universidade.}, \\		\index{\Ignor}
Por longos e duríssimos trabalhos, \\
Conduziu sempre firme os companheiros, \\
Desde o louro Mondego aos Pátrios montes. \\		\index{\Monde}
Em vão se opõem as luzes da Verdade \\
Ao fim que já na ideia tem proposto \\
E em vão do Tio as iras o ameaçam. \\[10pt]

E tu, que à sombra duma mão benigna,\\
Gênio da Lusitânia, no teu seio\\
De novo alentas as amáveis Artes;\\
Se ao surgir do letargo vergonhoso\\
Não receias pisar da Glória a estrada,\\
Dirige o meu batel, que as velas solta,\\
O porto deixa e rompe os vastos mares,\\
De perigosas Sirtes povoados. %gloss Sirtes

Quais seriam as causas, quais os meios\\
Por que Gonçalo renuncia os livros?\\
Os conselhos e indústrias da Ignorância\\	\index{\Ignor}		\index{\Indus}
O fizeram curvar ao peso enorme\\
De tão difícil e arriscada empresa.\\
E tanto pode a rústica progênie\footnote{ Virg. \AE n., 1.~\versal{I}:
\textit{``Tant\ae ne anmis c\oe lestibus ir\ae !''}. Despréaux no canto \versal{I} do
\textit{Lutrin}: \textit{``Tant de fiel ente"-t-il dans l'âme des dévots!''}.} 
\\[10pt]		\index{\Lutri}

A vós, por quem a Pátria altiva enlaça\\
Entre as penas vermelhas e amarelas\\
Honrosas palmas e sagrados louros,\\		\index{\Lour}
Firme coluna, escudo impenetrável\\
Aos assaltos do Abuso e da Ignorância,\\			\index{\Ignor}
A vós pertence o proteger meus versos.\\
Consenti que eles voem sem receio\\
Vaidosos de levar o vosso nome\\
Aos apartados climas onde chegam\\
Os ecos imortais da Lusa glória. \\[10pt]


Já o invicto Marquês com régia pompa\footnote{ O Ilustríssimo e
Excelentíssimo Senhor Marquês de Pombal entrou em Coimbra como Plenipotenciário		
e Lugar"-tenente de Sua Majestade Fidelíssima para a criação da Universidade em
22 de setembro de 1772.}\\		\index{\Marqpombal}
Da risonha Cidade avista os muros.\\
Já toca a larga ponte em áureo coche.\\
Ali junta a brilhante Infantaria,\\
Ao rouco som de música guerreira,\\
Troveja por espaços; a Justiça,\\
Fecunda mãe da Paz e da Abundância,\\
Vem a seu lado; as Filhas da Memória,\\			\index{\Filmemo}
Digna, imortal coroa lhe oferecem,\\
Prêmio de seus trabalhos; as Ciências\\ 
Tornam com ele aos ares do Mondego,\\		\index{\Monde}
E a Verdade entre júbilos o aclama\\
Restaurador do seu Império antigo.\\	
Brilhante luz, paterna liberdade,\\
Vós, que fostes num dia sepultadas,\\
C'o bravo Rei nos campos de Marrocos\footnote{ O Senhor Rei D.~Sebastião
ficou em África no ano de 1578, e se perdeu com ele a liberdade
Portuguesa, de donde nasceram as funestas consequências que até agora se fizeram
sentir.},\\
Quando traidoras, ímpias mãos o armaram,\\
Vítima ilustre da ambição alheia,\\
Tornai, tornai a nós. Da régia estirpe\\
Renasce o vingador da antiga afronta\footnote{ O Sereníssimo Senhor D.~José
Príncipe herdeiro.}:\\
Assim o novo Cipião crescia\footnote{ Públio Cornélio Cipião vingou a
morte de seu Pai e Tio destruindo Cartago.}\\		\index{\Cipiao}
Para terror da bárbara Cartago.\\
Possam meus olhos ver o Ismaelita\footnote{Os Mouros são descendentes
de Ismael filho de Agar.}\\	\index{\Ismae}
Nadar em sangue e pálido de susto\\
Fugir da morte e mendigar cadeias;\\
E amontoando Luas sobre alfanges\\ %\footnote{ LUas sobre alfanges.}
Formar degraus ao Trono Lusitano.\\
Dissiparam"-se as trevas horrorosas\\
Que os belos horizontes assombravam\\
E a suspirada luz nos aparece.\\
Tal depois que, raivoso e sibilante,\\
Sobre o carro da Noite o Euro açoita\footnote{ Euro, o vento
vulgarmente chamado l'Este. Boótes, constelação na cauda da Ursa, ou a Guarda.}\\
Os tardios cavalos do Boótes\footnote{ Juvenal, \textit{Sat}. \versal{V}, v. 23.:
\textit{Frigida circumagunt pigri Sarraça Boot\ae}.},\\
E insulta as terras e revolve os mares,\\
Raia a manhã serena entre douradas\\
E brancas nuvens; ri"-se o Céu e a Terra:\\
O Vento dorme e as Horas vigilantes,\\
Abrem ao claro Sol a azul campanha. \\[10pt]


A soberba Ignorância entanto observa,\\			\index{\Ignor}
E se confunde ao ver o próprio trono\\
Abalar"-se e cair; o seu ruído\\
Redobra os ecos nos opostos vales\\
E o Mondego feliz ao mar undoso\\			\index{\Monde}
Leva alegre a notícia, porque chegue\\
Das suas praias aos confins da Terra.\\
Ela abatida e só não acha abrigo,\\
E desta sorte em seu temor suspira. \\[10pt]


``Verei eu sepultar"-se entre ruínas\\
O meu reino, o meu nome e a minha glória,\\
Depois de ser temida, e respeitada?\\
Pobre resto de míseros vassalos\\
Não há mais que esperar. Já fui rainha:\\
Já fostes venturosos: não soframos\\
As injúrias que o vulgo nos prepara,\\
Injúrias mais cruéis do que a desgraça.\\
Deixemos para sempre estes terríveis\\
Climas de mágoa, susto, horror e estrago.\\
Mostrai"-me algum lugar desconhecido,\\
Onde oculta repouse até que possa\\
Tomar de quem me ofende alta vingança.\\
Mas onde se um Prelado formidável\footnote{ O Ilustríssimo e
Excelentíssimo Senhor Bispo de Coimbra, Reitor e Reformador da Universidade.},\\
Esse Argos\footnote{ Fingiu a fábula ser Pastor de Tessália, que tinha
cem olhos, a quem Juno deu a guardar Io, filha de Ínaco, Rei dos Argivos.} que
me assusta vigilante\\			\index{\Fabula}
Ao lugar mais remoto estende a vista?\\
Monstros do cego abismo em meu socorro\\
Empenhai o poder do vosso braço;\\
Que se entre os homens me faltar asilo,\\
Ao triste vão dos ásperos rochedos,\\
Onde o Tenaro escuro e cavernoso\footnote{ Promontório de Lacônia,
onde há uma cova profundíssima, que os antigos chamaram a porta do Inferno.
Virg., \textit{Georg.}, liv. \versal{iv}, v. 467: \textit{T\ae narias etiam fauces alta
ostia Ditis}.}\\
Da morada sombria as portas abre,\\
Irei chorar meus dias sem ventura:\\
Irei''\ldots Assim falando misturava\\
Gemidos e soluços que sufocam\\
Dentro do peito a voz e umedecia\\
C'o pranto amargo a face descorada.\\
Mas logo, serenando o rosto aflito,\\
Corre por entre sustos e esperanças\\
Ao caro abrigo do fiel Gonçalo.\\
A sonolenta, a pigra Ociosidade\\
Por esta vez deixou de acompanhá"-la:\\
E a lânguida Preguiça forcejando\\
Pôde apenas segui"-la com os olhos. \\[10pt]


Toma a forma dum célebre Antiquário\\
Sebastianista acérrimo, incansável,\\
Libertino com capa de devoto.\\
Tem macilento o rosto, os olhos vivos,\\
Pesado o ventre, o passo vagaroso.\\
Nunca trajou à moda: uma casaca\\
Da cor da noite o veste e traz pendentes\\
Largos canhões do tempo dos Afonsos.\\
Dizem que o tempo da mais bela idade\\
Consagrou às questões do Peripato.\\ 	\index{\Perip}
Já viu passar dez lustros e experiente\\
Sabe enredos urdir e pôr"-se em salvo.\\
Entra por toda a parte e em toda a parte\\
é conhecido o nome de Tibúrcio. \\[10pt]


Gonçalo que foi sempre desejoso\\
Da mais bela instrução, lia e relia\\
Ora os longos acasos de Rosaura\footnote{ \textit{Carlos} e
\textit{Rosaura}, \textit{Constante Florinda}, e \textit{Carlos Magno} são
romances muito conhecidos.},\\ 
Ora as tristes desgraças de Florinda,\\ 	\index{\Romvulg}
E sempre se detinha com mais gosto\\
Na cova Tristifeia e na passagem\\
Da perigosa ponte de Mantible.\\
Repetia de cor de Albano as queixas\\
Chamando a Damiana injusta, ingrata;\\
Quando Tibúrcio apaixonado e triste\\
Ralhando entrou. ``Que espera tu dos livros?\\
Crês que ainda apareçam grandes homens\\
Por estas invenções com que se apartam\\
Da profunda ciência dos antigos?\\
Morreram as \textit{postilas} e os \textit{Cadernos}:\\
Caiu de todo a \textit{Ponte}\footnote{ O método escolástico. Quem
conheceu a lógica peripatética, não ignora qual seja esta ponte.} e se acabaram\\		\index{\Perip}
As \textit{distinções} que tudo defendiam,\\
E o \textit{ergo}, que fará saudade a muitos!\\
Noutro tempo dos Sábios era a língua\\
\textit{Forma}, e mais \textit{forma}: tudo enfim se acaba,\\
Ou se muda em pior. Que alegres dias\\
Não foram os de Maio quando a estrada\\
Se enchia de Arrieiros e Estudantes!\\
Ó tempo alegre e bem"-aventurado!\\
Que fácil era então o azul Capelo,\\
Adornado de franjas e alamares,\\
O rico anel e flutuante borla,\\
Honra e fortuna que chegava a todos!\\
Hoje é grande a carreira e serão raros\\
Os que se atrevam a tocar a meta.\\
Ah Gonçalo! Gonçalo! que mais vale\\
Tirar co'a própria mão no fértil Souto\\
Moles castanhas do espinhoso ouriço!\\
Quanto é doce ao voltar da Primavera\\
O saboroso mel no louro favo!\\
Ó alegre e famosa Mioselha,\\
Fértil em queijos, fértil em tramoços!\\
Só lá de romaria em romaria\\
Podes viver feliz e descansado.\\ %\footnote{nOTA VOCABULAR + PROVÍNCIA.}
Quem te obriga a levar sobre os teus ombros\\
O desmedido peso, que te espera?\\
Não tenhas do bom Tio algum receio:\\
Comigo irás, bem sabes quanto posso.\\
Se te envergonhas de ser só, descansa;\\
Fiel parente, amigo inseparável,\\
Eu farei que, abraçando o mesmo exemplo,\\
Muitos se apressem a seguir teus passos.'' \\[10pt]


Assim falava, quando um ar de riso\\
Apareceu no rosto de Gonçalo.\\
Tudo o que se deseja se acredita;\\
Nem há quem o seu gosto desaprove.\\
Ele, porque já traz no pensamento\\
Poupar"-se dos estudos à fadiga,\\
Não vacila na escolha e se aproveita\\
Da feliz ocasião que lhe assegura\\
O meditado fim de seus desejos. \\[10pt]


Convocam"-se os heróis e deliberam\\
Em pleno consistório onde Gonçalo\\
Silêncio pede e assim a todos fala.\\
``Heróis, a quem uma alma livre anima,\\
Que desprezando as Artes e as Ciências,\\
Ides buscar da Pátria no regaço,\\
Longe da sujeição e da fadiga,\\
Doce descanso, amável liberdade:\\
Se algum de vós (o que eu não creio) ainda\\
Tem na alma o vão desejo dos estudos,\\
Levante o dedo ao alto.'' Uns para os outros\\ 
Olharam de repente e de repente\\
Rouco e brando sussurro ao ar se espalha:\\
Qual nos bosques de Tempe\footnote{ Lugar de Tessália célebre pela
amenidade dos seus bosques.}, ou nas frondosas\\
Margens que banham o plácido Mondego,\\			\index{\Monde}
Costuma ouvir"-se o Zéfiro suave,\\		\index{\Zefir}
Quando meneia os álamos sombrios.\\
Nenhum alçou a mão e a Ignorância\\			\index{\Ignor}
Pareceu consolar"-se imaginando\\
Sonhadas glórias de futuro império. \\[10pt]


Dispõe"-se a companhia e se aparelha\\
Para partir antes que o Sol desate\\
Sobre a Terra orvalhada as tranças d'ouro.\\
Tibúrcio tudo apronta. Mas Janeiro\\ 		\index{\Jane}
Loquaz, traidor, doméstico inimigo\\
Voa de casa em casa publicando\\
Da forte esquadra a próxima partida. \\[10pt]


Guiomar, velha que há muito que insensível\\
às delícias do amor, aferrolhando\\
Emagrece nos míseros cuidados\\
Da faminta ambição e é na Cidade\\
Uma ave de rapina que entre as unhas\\
Leva tudo o que encontra aos ermos cumes\\
Da escalvada montanha onde a festejam\\
Co'a boca aberta os ávidos filhinhos:\\
Triste agora e infeliz ouve e se assusta\\
Das notícias cruéis que o Moço espalha.\\
``Ó Ama desgraçada! Ó dia infausto!\\ 
Agora que esperava mais sossego\\
Principiam de novo os meus trabalhos!''\\
Estas e outras palavras arrancava\\
Do peito descontente, enquanto a Filha\\
Amorosa e sagaz estuda os meios\\
Com que possa deter o ingrato amante.\\ 	\index{\Tipos}
Faz ajuntar de partes mil à pressa\\
Cordões e anéis e a pedra reluzente\\
Que os olhos desafia: os seus cabelos,\\
Que desconhecem o toucado, empasta\\
Co'a cheirosa pomada: a Mãe se lembra\\
Da própria mocidade e lhe vai pondo\\
Com a trêmula mão vermelhas fitas.\\
Simples noiva da aldeia que ao mover"-se\\
Teme perder o desusado adorno\\
Nunca formou mais vagarosa os passos.\\
Narcisa chega entre raivosa e triste,\\
E fingido"-se esquecer"-se da mantilha\\
Para mostrar"-se irada, desta sorte\\
Em alta voz lhe fala. ``Será certo\\
Que pretendes fugir, e que me deixas\\
Infeliz, enganada, e descontente?\\
Assim faltas cruel, pérfido, ingrato\\
Dum longo amor aos ternos juramentos?\\ 
Não disseste mil vezes\ldots{} mas que importa\\
Que os meus males recorde? enfim, perjuro,\\
As tuas vãs promessas me enganaram.\\
Justiça pedirei ao Céu e ao Mundo:\\ %footnote mulher ensandecida
O mundo tem prisões, o Céu tem raios.'' \\[10pt]


Falava e o Herói, que arrasta ainda\\
D'um incômodo amor os duros ferros,\\
Parece vacilar quando Tibúrcio\\
Dá conselhos a um, a outro ameaça,\\
Pondo irados os olhos em Narcisa.\\
Diz"-lhe que em vão suspira, que em vão chora\\
E que sempre tiveram as mulheres,\\
Para enganar aos míseros amantes,\\
As lágrimas no rosto, o riso na alma.\\
Gonçalo, então, que o seu dever conhece,\\
Dá provas de valor e de prudência.\\
``Ouve, Narcisa bela,'' (lhe dizia)\\
``Serena a tua dor e os teus queixumes;\\
O teu pranto me move, injusto pranto,\\
Que o meu constante amor de ingrato acusa.\\
Sossega: a nova herança dum morgado\\ 		\index{\Morg}
É quem me chama, a ausência será breve.\\
Tempo depois virá que em doces laços\\
Eterno amor as nossas almas prenda,\\
E então farás tibornas\footnote{ Comida feita de pão e azeite novo.}
e magustos\footnote{ Castanhas assadas e vinho.}.\\		\index{\Tiborn}
Nem sempre cobre o mar a longa praia:\\
Nem sempre o vento com furor raivoso\\
Do robusto pinheiro o tronco açoita.'' \\[10pt]


Acaba de falar e lhe oferece\\
A leve bolsa, que Narcisa aceita,\\
Como penhor sincero de amizade,\\
Bolsa que deve ser na dura ausência\\
Breve consolação de tristes mágoas. \\[10pt]


O experto Amigo, que se mostra em tudo\\
Companheiro fiel, c'os olhos tristes,\\
Pondera os longos e ásperos caminhos:\\
Lembra funestas noites de estalagem,\\
E adverte, em vão, que ao menos por cautela\\
Deve fazer"-lhe a bolsa companhia.\\
Deixando enfim inúteis argumentos\\
Remete a decisão ao próprio braço.\\
Não se esquecem das unhas, nem dos dentes,\\
Armas que a todos deu a Natureza.\\
Ouvem"-se pela casa em som confuso\\
As troncadas injúrias e os queixumes.\\
Assim dois cães se o hóspede imprudente\\
Lança da mesa os ossos esburgados\\
Prontos avançam: duma e doutra parte,\\
Se vê firme o valor; mordem"-se e rosnam,\\ 
Mas não cessa a contenda. Amigo e amante,\\
Que farias, Gonçalo, em tanto aperto?\\
Concorre a plebe e o férvido tumulto\\
Vai pelas negras fúrias conduzido\\ 
Despertando nos peitos a desordem.\\
Ninguém sabe por quê, mas todos gritam.\\
Já voam as cadeiras pelos ares:\\
Pedras e paus de longe se arremessam.\\
E se a cândida Paz com rosto alegre\\
Serenou as desgraças deste dia,\\
Os teus dentes, intrépido Gonçalo,\\
Viste voar em negro sangue envoltos. \\[10pt]


Torna alegre Narcisa, e cinco vezes\\
Abriu a bolsa e numerou a prata.\\
Fez diversas porções que num momento\\
Tornou a confundir: não doutra sorte\\
O menino impaciente e cobiçoso,\\
Quando alcança o que há muito lhe negavam,\\
Repara, volta, move, ajunta, espalha,\\
E neste giro o seu prazer sustenta. \\[10pt] 


Entanto, a mãe que já por experiência\\
Os enganos conhece mais ocultos\\
Busca novos pretextos de vingança,\\
Fingindo torpes e horrorosos crimes.\\
E espera ouvir gemer em poucas horas\\
O mancebo infeliz em prisão dura.\\
Mas Rodrigo que ouviu o rumor vago,\\
À pressa chega, e desta sorte fala. \\[10pt]


``Que desgraças te esperam! foge, foge,\\
Gonçalo, enquanto há tempo: gente armada\\
Vem logo contra ti. Guiomar convoca\\
Todo o poder do mundo: um só momento\\
Não percas, caro amigo; os companheiros\\
Com alvoroço esperam. Ah, deixemos,\\
Deixemos duma vez estas paredes,\\
Onde c'o próprio sangue escrita deixas\\
De teu trágico amor a breve história.\\		
É já outro o Mondego: a liberdade\\			\index{\Monde}
Destes campos fugiu e só ficaram\\
A dura sujeição e o triste estudo.\\
Enfim hei de apartar"-me desta sorte?\\ 
Ó sempre tristes, sempre amargos sejam\\
Os teus últimos dias, velha infame.''\\
Gonçalo sim chorando, monta e parte. \\[10pt]
\end{verse}
\pagebreak 

\thispagestyle{empty}
\movetoevenpage
\mbox{}\vfill
\thispagestyle{empty}
\noindent Argumento do Canto \versal{ii}
\medskip

{\footnotesize\noindent Catálogo e descrição dos tipos que formam a companhia de desertores, conforme os respectivos vícios:  \index{\Tipos}
Gonçalo, o mais destro, é um jovem que poderia ter sido promissor, mas
não aguentou os esforços das letras; Tibúrcio, a Ignorância, leva um lenço pardo amarrado
a um galho, como a bandeira da companhia; Cosme é enamorado; Rodrigo é
rústico; Bertoldo se diz fidalgo de antiga linhagem improvável; Gaspar
é iracundo; Alberto, o alegre, em Coimbra aplicou"-se às festas --
Chegada à estalagem -- Comem, bebem, brindam e brigam sobre mesas de tosco pinho --
Rodrigo brinda à vitória da viagem, imitando convenções da poesia de banquete; Tibúrcio diz palavras
lascivas para Rodrigo, também conforme foi costume em banquetes -- Fúria de Rodrigo -- Briga na estalagem --
Exortação do velho Ambrósio: pergunta"-lhes se aquilo era aprendido com as letras e os convoca à moderação;
narra o próprio exemplo demonstrando na sua miséria presente o efeito dos
seus vícios de juventude; exorta os jovens a retornarem para os estudos,  do contrário,
os amaldiçoa -- Gaspar, ofendido, ataca o velho -- Gaspar e Gonçalo armados
atacam a multidão que invadira o estabelecimento aos gritos do velho --
Levante geral contra os desertores -- Fuga dos companheiros através do mato espesso.}

\chapter{Canto \versal{ii}}
%\setlinenum{0}

\begin{verse}

Com largo passo longe do Mondego,\\			\index{\Monde}
Alegre a forte gente caminhava.\\
Gonçalo excede a todos na estatura,\\
Na força, no valor e na destreza.\\
Sobre um magro jumento se escarrancha\\
Tibúrcio, e já dum ramo de salgueiro\\
Desata ao Norte fresco que assobia\\
Por vistoso estandarte um lenço pardo.\\ 
Cosme, infeliz e sempre namorado\\
Sem ser correspondido, vai saudoso,\\
Ama e não sabe a quem: vive penando\\
E se consola só porque imagina\\
Que tem de conseguir melhor ventura.\\ 
Rodrigo, que de todos desconfia,\\
é de índole grosseira e gênio bruto,\\
Não conhece os perigos, nem os teme:\\
Melancólico sempre, vai por gosto\\			\index{\Melanc}
Viver na choça aonde foi criado,\\ 		\index{\Choca}
Qual o Tatu, que o destro Americano\footnote{ Lin. \textit{Sys. nat.}, \textit{Zool.}, edic. 10, t.~\versal{I}, p. 50. \textit{Dasypus}.}\\
Vivo prendeu e em vão depois se cansa\\
Por fazê"-lo doméstico, que sempre\\
Temeroso nas conchas se recolhe\\
E parece fugir à luz do dia.\\ 
Também vinha Bertoldo e traz consigo\\
Carunchosos papéis por onde afirma\\
Vir do sétimo Rei dos Longobardos\footnote{ Povos de Escandinávia e Pomerânia, que se apoderaram 
da parte da Gália Cisalpina em 568.}.\\		\index{\Galia}	\index{\Longob}
Grita contra as riquezas, a Fortuna\\		\index{\Fortu}
Segundo o que ele diz não muda o sangue:\\
Pisa com força o chão e empavesado\\			\index{\Empav}
De ações que ele não pode chamar suas\\
Aos outros trata com feroz desprezo.\\ 
Iracundo Gaspar, que te enfureces\\
No jogo e quando perdes não duvidas\\
Meter a mão à ferrugenta espada,\\
Tu não ficaste: as noites sobre os livros\\
Não queres suportar, porque não temes\\
Da já viúva mãe as frouxas iras.\\
Nem tu, Alberto, alegre e desejado\\
Nas vistosas funções das romarias,\\
Que és vivo, pronto e ágil, e nos bailes\\
Tens fama de engraçado e garganteias\\
Co'a viola na mão trocando as pernas.\\ 
Os que aprendem o nome dos autores,\\
Os que leem só o prólogo dos livros,\\
E aqueles cujo sono não perturba\\
O côncavo metal que as horas conta,\\
Seguiram as bandeiras da ignorância\\
Nos incríveis trabalhos desta empresa. \\[10pt] 


O Sol já sobre os campos de Anfitrite\\		\index{\Anfit}
Inclina o carro e as nuvens carregadas\\	\index{\Carro}
Importunos chuveiros ameaçam\\
Quando a velha estalagem os recebe. \\[10pt]


Mesa de tosco pinho se povoa\\
De negras azeitonas e salgado\\
Queijo que estima a gente que mais bebe.\\
De um lado e de outro lado se levantam\\
Pichéis e copos em que o vinho abunda.\\
Corriam para aqui desafiados\\
Rodrigo, o triste, e o glutão Tibúrcio.\\
Este instante fatal é que decide\\
Da dúbia sorte dos heróis, cobrindo\\
Um de eterna vergonha, outro de glória. \\[10pt]


A feia Noite que aborrece as luzes,\\
Desce dos altos montes com mais pressa\\
Por ver este combate e afugentada\\
Pela sombria luz de uma candeia\\
De longe observa o novo desafio.\\
Um e outro, ocupando as mãos, e a boca\\
Avidamente a devorar começa:\\ 
Assim esse animal grosseiro e pingue,\\
Que de alpestres bolotas se sustenta,\\
À pressa come e tendo uma nos dentes,\\
Noutra tem o desejo e noutra a vista.\\
Rodrigo, quase certo da vitória,\\
Co'as mãos ambas levanta um grande copo,\\
Copo digno de Alcides, e à saúde\\ 			\index{\Alcid}	%\footnote{Hércules}
De todos os famosos Desertores\\
De uma vez esgotou: então Tibúrcio,\\
Cheio de nobre ardor, fechando os olhos\\ 
Toma um largo pichel e assim lhe fala. \\[10pt]	\index{\Pichel}


``Vasilha da minha alma, tu que guardas\\
A alegria dos homens no teu seio,\\
E tu, filho da cepa generoso,\\		\index{\Cepa}
Se estimas e recebes os meus votos,\\
Derrama sobre mim os teus encantos.''\\
Já tinha dito muito e enquanto bebe\\
Voa a cega Discórdia que se nutre\\
De sangue e de vingança e sobre os copos\\
Três vezes sacudiu as negras asas.\\
Viam"-se já, nos lívidos semblantes,\\
A raiva sanguinosa, a má tristeza.\\
A Noite, a quem o Acaso favorece,\\		\index{\Acaso}
Estende a fusca mão e a luz abafa.\\
Veloz passa o furor de peito em peito,\\
Perturba os corações e inspira o ódio. \\[10pt]


Só tu, Gonçalo, descrever puderas\\
Os terríveis estragos desta noite,\\
Tu, que posto debaixo duma banca\\
(Por manchar as mãos no sangue amigo),\\
Sentiste pela casa e pelos ares\\
Rolar os pratos e tinir os copos.\\
Range os dentes Gaspar e pelo escuro\\
Não acerta co'a espada, nem co'a porta:\\
Quando Ambrósio, que tinha envelhecido\\
Da Estalagem na mísera oficina,\\
Co'a candeia na mão assim falava.\\
``É crível, que entre vós jamais se encontre\\
Um gênio dócil, sério e moderado?\\
Isto deveis às letras? respondei"-me,\\
Ou insultai também os meus cabelos\\
Da triste e longa idade embranquecidos.\\
Julgais acaso que o saber se infunde\\
Deixando o nosso nome assinalado\\
Pelos muros e portas na Estalagem?\\
Ó néscia mocidade! é necessário\\
Muito tempo sofrer, gastando a vista\\
Na contínua lição e sobre os livros\\
Passar do frio Inverno as longas noites.\\
E quando já tivésseis conseguido\\
De tão bela carreira os dignos prêmios,\\
Muito pouco sabeis, se inda vos falta\\
Essa grande Arte de viver no mundo,\\
Essa, que em todo o estado nos ensina\\
A ter moderação, honra e prudência.\\
Eu também já na flor da mocidade\\
Varri co'a minha capa o pó da sala:\\
Eu também fui \textit{rancho da carqueja}\footnote{ Esta companhia de Estudantes cometeu muitos crimes e foi dispersa e castigada.},\\
Digno de fama e digno de castigo.\\
Era então como vós. Jamais os livros\\
Me deveram cuidado e me alegrava\\
Das noturnas empresas, dos distúrbios:\\
Os dias se passavam quase inteiros\\
Nos jogos, nos passeios, nas intrigas,\\
Que fomentam os ódios e as vinganças.\\
Por isso estou no seio da miséria,\\
Por isso arrasto uma infeliz velhice,\\
Sem honra, sem proveito, sem abrigo.\\
Tempo feliz da alegre mocidade!\\
Hoje encurvado sobre a sepultura,\\
Eu choro em vão de vos haver perdido!''\\
Assim suspira e geme e continua.\\
``Conservai, sempre firme, na memória,\\
De um velho desgraçado o triste exemplo,\\
E aprendei a ser bons, que a vossa idade\\
As indignas ações não justifica.\\
Mas se vós desprezais os meus conselhos,\\
Nunca gozeis o prêmio dos estudos:\\
Aflições e trabalhos vos oprimam,\\
Enquanto o mar das Índias vos espera.'' \\[10pt] %\footnote{mar das índias e moral da história}


Então Gaspar, tomando o caso em brio,\\
Aceso de ira com valor responde,\\
Traça o capote e tira pela espada.\\
O velho grita e foge: às suas vozes,\\
De rústico um povo se enfurece,\\
E toma as armas e bradando avança.\\
Qual nos imensos e profundos mares\\
O voraz Tubarão entre o cardume\\
De argentadas Sardinhas; elas fogem,\\
Deixam o campo, e nada lhe resiste:\\
Assim Gonçalo, a quem já todos temem,\\
Faz espalhar a turba que o rodeia,\\
E só deixa a quem foge de encontrá"-lo. \\[10pt]


Gaspar, que o rosto nunca viu ao medo,\\
A todos desafia e não perdoa\\
De uma oliveira ao carcomido tronco,\\
Que ele julga broquel impenetrável,\\
Vendo estalar da sua espada a folha. \\[10pt]


Da noite a densa névoa favorece.\\
Receosos de nova tempestade,\\
Salvam as vidas os Heróis fugindo\\
Por entre o mato espesso. Ouvem ao longe\\
Da vingativa plebe a voz irada.\\
À clara luz das pinhas resinosas\footnote{ Costumam os rústicos acender de noite as pinhas.}\\
Aparecem as foices e aparecem\\
Chuços, cacheiras, trancas e machados.\\
Levanta"-se o clamor e a crua guerra,\\
Que o sangue dos mortais derrama e bebe,\\
Gira por toda a parte e move as armas.\\
Entanto a valerosa companhia,\\
Amparada da sombra feia e triste,\\
Voa por longo espaço sobre as asas\\
Do pálido terror. Não de outra sorte\\
Rasos xavecos de piratas Mouros,\\		\index{\Xavec}
Quando aos ecos do bronze fulminante\\
Vêem tremular as vencedoras Quinas\\
Sobre a possante Nau, que oprime os mares,\\
Fogem à vela e remo, e não descansam\\
Sem ter beijado as Argelinas praias.\\
Ouvem"-se  então diversos sentimentos.\\
Chora Gaspar de se não ter vingado,\\
E ainda aqui colérico assevera\\		\index{\Coler}
Que a não faltar"-lhe a espada não fugira.\\
Espada, que ao romper as linhas d'Elvas\footnote{ Gloriosa batalha,
que ganhou D.~Antonio Luiz de Menezes, Excelentíssimo Conde de Castanhede, no
ano de 1658. A este herói também se deve o triunfo de Montes Claros.},\\		\index{\Elvas}
Se dos velhos Avós não mente a história,\\
Abriu de meio a meio um Castelhano. \\[10pt] 


Teme Bertoldo que o encontre o Povo\\ 
E no meio daquela escuridade\\
Chega"-se aos mais com pânico receio.\\
Cosme, quase insensível aos perigos\\
E aos amargos momentos desta noite,\\
Aproveita o silêncio, o sítio, a hora,\\
Para chorar saudades sem motivo.\\
Só Gonçalo pensava cuidadoso\\
Em salvar os aflitos companheiros:\\
Assim o astuto assolador de Troia\footnote{ Ulisses, cujos companheiros foram transformados por Circe. Homr. \textit{Odiss.}, \versal{I}, \versal{X}, v.~238.},\\ 	\index{\Ulis}
Quando os Gregos heróis ouviu cerdosos\\
Grunhir nos bosques da encantada Circe,\\
Ou quando viu a detestável mesa\footnote{ Polifemo devorou dois Gregos
em presença de Ulisses. \textit{Odyss.}, \versal{i}. \versal{ix}, v.289.}\\	\index{\Ulis}
Na vasta cova do Ciclope horrendo.\\
``Onde estarás fiel e caro amigo?''\\
(Dizia o condutor da estulta gente)\\
``Se tu me faltas como irei meter"-me\\
Nas mãos dum Tio rústico, inflexível?\\
Voltarei? mas ó Céus! quem me assegura\\
Que essa velha cruel, nefanda harpia\\
Não tenha urdido algum funesto engano?\\
E se o Povo indignado e ofendido\\
Nos vem seguindo, e ao surgir da Aurora\\
Neste inculto deserto\ldots{}  Céu piedoso,\\
Longe, longe de nós tão graves danos.'' \\[10pt]


Gonçalo assim falava e vigilante\\
Tristes horas passou até que o dia\\
Apareceu entre rosadas nuvens\\
Sobre as altas montanhas do horizonte. \\[10pt]

\end{verse}
\pagebreak

\mbox{}\vfill
\thispagestyle{empty}
\noindent Argumento do Canto \versal{iii}
\medskip


{\footnotesize\noindent A fama leva no vento os louvores do rei de Portugal pela restauração dos
estatutos em Coimbra -- Louvor do edifício da Universidade -- Semelhante
à fama, a infâmia leva as incógnitas notícias, como em bandos de papagaios
-- Espalha"-se assim a murmuração indignada contra a companhia de Gonçalo --
A multidão ainda busca o grupo de estudantes -- Descrição do \textit{locus
horrendus} em que dorme Rufino, que vive nas brenhas daquela serra por causa
de um amor não correspondido -- Vendo que os vassalos de seu novo reino
estavam perdidos nas montanhas e preocupada com o destino deles, que acabariam
presos, a Ignorância aparece em sonho ao triste Rufino na forma de sua amada
Doroteia, filha do velho Amaro, guardador da cadeia local -- No sonho,
Doroteia faz promessas de amor a Rufino e o aconselha a mudar"-se junto com
a companhia de desertores, para guiá"-los até Mioselha -- Iludido, Rufino
acorda confiante quanto à predição da Ignorância -- Levados pelo acaso,
os desertores encontram Rufino tão logo ele acorda -- Gonçalo solenemente
pede a orientação ao jovem habitador daquelas brenhas -- Rufino se apresenta,
refere o próprio sofrimento e o presságio que o animava a viajar -- Abandonando
a caverna onde queria enterrar o amor não correspondido, assume o serviço de
guiar os desertores --  Seguem viagem -- Quando Gonçalo pensa que o bom sucesso 
da viagem estava garantido, o grupo é repentinamente cercado pelo povo armado
de foices e outras armas de serviços mecânicos -- Descrição de combate entre
a multidão furiosa e a companhia de desertores -- Na briga, diante de um jovem
conhecido como gigante Ferrabrás, Gonçalo tropeça e cai -- Gaspar tenta em vão
salvar o grupo mas perde a espada -- Os desertores das letras são levados presos.} \enlargethispage{\baselineskip}


\chapter{Canto \versal{iii}}
%\setlinenum{0}


\begin{verse}

A Fama sobre o carro transparente,\\		\index{\Fama}
Que arrastam através do espaço imenso\\
O sonoro Aquilon e o veloz Austro\footnote{ Aquilon vento setentrional, e Austro meridional.},\\		\index{\Aquil}
Cantava o caro nome, a imortal glória\\
Do Augusto Pai do Povo. Entre milhares\\ \index{\Aug} 	\index{\Paida}
De ações dignas dum Rei, Europa admira\\
O soberbo Edifício levantado\\
Que o saudoso Mondego abraça e adora:\\ 		\index{\Monde}
Edifício que o tempo doravante\\
Vê de longe, rodeia, teme e foge:\\
Que sustenta em firmíssimas colunas\\
De ciência imortal o Régio Trono. \\[10pt]


Se longe da feroz barbaridade\\
Os olhos abre a forte Lusitânia,\\
Grande Rei, esta ação é toda vossa. \\[10pt]


Entanto a Fama heroica vão seguindo\\		\index{\Fama}
As velozes e incógnitas notícias,\\
Que trazem e que levam os sucessos\\
De país em país, de clima em clima.\\
Elas voam em turba, enchendo os ares\\
Dos ecos dissonantes a que atendem\\
Crédulas velhas e homens ociosos.\\
Qual no fértil Sertão da Aiuruoca\footnote{ Aiuruoca na língua dos índios		
soa o mesmo que \textit{casa de papagaios}. Este vasto país nas minas do Rio das Mortes
é abundantíssimo destas aves.}\\		\index{\Aiur}
Vaga nuvem de verdes Papagaios,\\
Que encobre a luz do Sol e que em seus gritos\\
É semelhante a um povo amotinado:\\
Assim vão as Notícias e estas vozes\\
Pelo campo entre os rústicos semeiam. \\[10pt]


Gente inexperta, alegre e sem cuidados,\\
Fero esquadrão que os vossos campos tala,\\
Vem destruindo as terras e os lugares.\\
O povo indócil, cego e receoso,\\
Que as funestas palavras acredita,\\
Toma os caminhos e os oiteiros cobre.\\
Por onde irás, intrépido Gonçalo,\\
Que escapes ao furor da plebe armada?\\
Mas já os desgraçados companheiros\\
Desciam por incógnitas veredas\\
Para o fundo dum vale cavernoso,\\
Que o Zêrere veloz lavando insulta\footnote{ Este pequeno e arrebatado
rio perde o nome no Tejo, e faz a maior parte do seu curso por penhascos inacessíveis.}\\
Co'as turvas águas do gelado Inverno.\\
Há um lugar nunca dos homens visto,\\
Na raiz de dois montes sobranceiros.\\
Suam as frias e musgosas pedras,\\
Que dos altos cabeços penduradas\\
Ameaçam ruína há tempo imenso.\\
Jamais do Cão feroz o ardor maligno\footnote{ A constelação chamada Canícula.}\\		\index{\Canic}
Desfez a neve eterna destas grutas.\\
Árvores, que se firmam sobre a rocha,\\
Famintas de sustento, à terra enviam\\
As tortas e longuíssimas raízes.\\
Pendentes caracóis co'a frágil concha\\
Adornam as abóbadas sombrias.\\
Neste lugar se esconde temerosa\\
A Noite envolta em longo e negro manto\\
Ao ver do Sol os lúcidos cavalos,\\
Fúnebre, eterno abrigo aos tristes mochos,\\		\index{\Mocho}
Às velhas, às fatídicas corujas,\\
Que com medonha voz gemendo aumentam\\
O rouco som do rio acantilado. \\[10pt]


Rufino por seu mal sempre extremoso\\
E sempre escarnecido, suspirando\\
Aqui se entrega ao pálido ciúme,\\
De um puro amor ingrata recompensa.\\
Contam que nestas hórridas cavernas\\
De míseras angústias rodeado,\\
Vinha exalar os últimos suspiros\\
Queixando"-se do Amor e da Fortuna.\\		\index{\Amor} \index{\Fortu}
Entre os braços do sono repousava,\\
Este infeliz já de chorar cansado;\\
Quando a inquieta Ignorância, que se aflige\\		\index{\Ignor}
De ver nestas montanhas escabrosas\\
Os tímidos amigos, em que funda\\
De novo império a única esperança,\\
Por que Rufino os acompanhe e guie\\
À pingue e suspirada Mioselha,\\
Que é de tantos heróis Pátria famosa,\\
Finge o rosto da bela Doroteia,\\
Doroteia a mais nova, a mais humana,\\
De quantas filhas teve o velho Amaro.\\
Ela a roca na cinta, as mãos no fuso,\\
Em sonhos lhe aparece e mais corada\\
Que a rosa na manhã da Primavera\\
A falar principia. ``Se até agora\\
Ingrata me mostrei a teus amores,\\
Se inconstante e perjura me chamaste,\\
Dá"-me nomes mais doces e ouve atento\\
De uma alma amante a confissão sincera.\\
Sempre te amei e espero ver unidos\\
Os nossos corações em fortes laços\\
Do casto amor que o Céu não desaprova.\\
Mas eu sem nada mais, que a lã, que fio,\\
Tu rico só de afetos e palavras,\\
Onde iremos que a sórdida miséria\\
Não seja em nossos males companheira?\\
Vai"-te e longe de mim segue a ventura,\\
Que firme te hei de ser em toda a idade.\\
Do velho Afonso o triste e pobre filho,\\
Pela dura madrasta afugentado,\\
Também deixou a suspirada Pátria,\\
E veio em poucos anos o mais rico\\
Dos bens imensos que o Brasil encerra.\\
Vês tu quanto cresceu que não cabendo\\
No paterno casal, ergue as paredes,\\
Até chegar ao Céu que testemunha\\
A ditosa união com que ele paga\\
O firme amor da venturosa Ulina?\\
Vai pois, Rufino meu, que muitas vezes\\
Muda"-se a terra e muda"-se a Fortuna.'' \\[10pt] 		\index{\Fortu}


Assim falando os braços lhe oferece.\\
Ó que instante feliz, se Amor perverso,\\		\index{\Amor}
Dos últimos favores sempre avaro,\\
Não firmasse esta sombra de ventura\\
Sobre as asas de um sonho lisonjeiro!\\
Desperta o triste e desgostoso amante,\\
E não duvida que a pressaga imagem\\
Noutro lugar tesouros lhe promete.\\
Futuros bens na ideia se apresentam,\\
E ele crê possuí"-los. Ó dos homens\\
Contínuo delirar sem fundamento!\\
Que bela e fácil se nos pinta a posse\\
Dum incógnito bem, que desejamos! \\[10pt]


Já se ajuntava o esquadrão famoso\\
Pela mesma Ignorância conduzido,\\			\index{\Ignor}
E Gonçalo primeiro assim falando,\\
Os mais em roda todos escutavam. \\[10pt]


``Benigno habitador de incultas brenhas,\\ 
Se um desgraçado errante e peregrino\\
Dentro em tua alma a compaixão desperta,\\
Os meus passos dirige, antes que a fome\\
Com ímpia mão nos deixe frio pasto\\
Às bravas feras, às famintas aves.'' \\[10pt] 


Falava ainda: alguns estremeceram,\\
Outros amargo pranto derramaram.\\
Da boca de Rufino todos pendem.\\
Ele os lânguidos olhos levantando\\
Já do longo chorar enfraquecidos,\\
Estas vozes soltou do rouco peito.\\         
``Que Fortuna cruel, maligna, incerta\\			\index{\Fortu}
Vos trouxe a penetrar o intacto abrigo\\
Destes lugares ermos e escabrosos?\\
Vós em mim achareis amigo e guia:\\
Que pode dar alguma vez socorro\\
Um desgraçado a outro desgraçado.\\
Duros casos de amor me conduziram\\
A acabar nesta gruta os tristes dias;\\
Mas hoje volto por feliz presságio\\
A tentar noutra parte a desventura.'' \\[10pt]


Acaba de falar movendo os passos\\
Pelo torcido vão das nuas pedras.\\
Todos o seguem com trabalho imenso. \\[10pt]


Depois que largo tempo caminharam\\
Por ásperas montanhas, aparecem\\
Ao longe a estrada e o lugar vizinho.\\
Qual a nau sofredora das tormentas,\\
Que, depois de tocar o porto amigo,\\
Sente fugir"-lhe as arenosas praias,\\
E dos hórridos ventos açoitada\\
Volta a lutar c'o pélago profundo:\\
Assim Gonçalo, quando ver espera\\
Tranquilo fim de míseros trabalhos,\\
O povo o cerca e dos confusos gritos\\
As montanhas ao longe retumbaram.\\
Vós, ó Musas, dizei como a Discórdia\\
Com o negro tição que acende os peitos,\\
Mostra o rosto de sangue e pó coberto,\\
Seguindo os passos do homicida Marte.\\		\index{\Marte}
Aqui não aparecem refulgentes\\
Escudos de aço e bronze triplicado,\\
Não assombram a testa dos guerreiros\\
Flutuantes penachos que ameaçam,\\
Como tu viste, ó Troia, ante os teus muros;\\
Mas o valor intrépido aparece\\
A peito descoberto. O povo armado\\
De choupas, longos paus e curvas foices,\\
É semelhante a um bosque de pinheiros,\\
Que o fogo devorou, deixando nuas\\
As elevadas pontas. Animoso\\ 
Dispõe Gonçalo a forma de batalha\\
Posto na frente: à sua voz a um tempo\\
Todos avançam, todos se aproveitam\\
Das perigosas e terríveis armas,\\
Que o terreno oferece em larga cópia.\\
Voa a cega Desordem e aparece\\
No meio do combate. Por um lado\\
Gaspar se opõe arremessando pedras\\
Com força tal que atroam os ouvidos.\\
Gonçalo doutra parte invicto e forte\\
Abre com ferro agudo amplo caminho.\\
Já pendia a balança da vitória\\
Contra a tímida gente que se espalha;\\
Quando chega atrevido Brás, o forte.\\
(Gigante Ferrabrás lhe chama o povo\\
Pela enorme estatura e força incrível)\\
Ergue a pesada maça sem trabalho,\\
Qual nos montes de Lerne o fero Alcides\footnote{ Lerne, lago de Acaia, onde Hércules matou a Hidra.}:\\	\index{\Alcid}
Gonçalo evita a morte com destreza.\\
Ele renova os formidáveis golpes;\\
Mas o irado mancebo ao desviar"-se\\
Tropeça e cai. Neste arriscado instante\\
Serias morto, intrépido Gonçalo,\\
Se Gaspar com um rochedo áspero e rombo\\
Não atalhasse do inimigo a fúria,\\
Quebrando"-lhe com golpe repentino\\
Ambas as canas do direito braço.\\
Rangem os ossos e a terrível maça\\
Caindo sobre a terra ao longe soa.\\
Torna a juntar"-se a fugitiva plebe,\\
E o prudente Gonçalo que deseja\\
Mostrar o seu valor noutros perigos,\\
Finge"-se morto: a turba irada o pisa,\\
Mas ele não se move. Contra todos\\
Então Gaspar em cólera se acende:\\
Ameaça, derriba, ataca e fere;\\
Até que já sem forças, rodeado\\
Vê de seus companheiros os opróbrios. \\[10pt]


Soa nas costas dos heróis valentes\\
O duro azambujeiro e são levados\\
Ao som terrível de insultantes gritos\\
Para a escura prisão que os esperava.\\
Gonçalo, o bom Gonçalo as mãos atadas,\\
Os olhos para o chão, porque era terno\\
Não refreou o compassivo pranto.\\
A par dele Bertoldo em vão lamenta\\
A falta de respeito que devia\\
Rústica plebe ao neto de Alarico\footnote{ Alarico, Rei dos Godos, que
alcançou muitas vitórias contra os Romanos no tempo de Honório.}\\
Com vagaroso passo todos marcham,\\
Como as ovelhas por caminho estreito.\\
Tal depois da ruína de um Quilombo\footnote{ Fortificação de escravos
rebelados, que muitas vezes se fazem temidos pelas suas hostilidades.}\\
Vem a indômita plebe da Etiópia,\\
Quando rico dos louros da vitória\\			\index{\Lour}
O velho Chagas sempre valeroso\footnote{ Este famoso Índio foi dos que
mais se assinalaram nas ocasiões de ataques contra os escravos.}\\
Cobre o fuzil da pele da Guariba\footnote{ Guariba, espécie de mono, 
cuja pele serve aos viajantes dos sertões para livrar o fuzil da umidade, e
costumam estes homens forrar"-se com a pele dos animais que matam. Pode ver"-se
M. Buffon no tom. \versal{iv}, edic. de 4 vol., p. 378. Lin., \textit{Sys. nat. anim},
ed. 10, tom. \versal{i}, p. 26, \textit{Paniscus}, \textit{Marcgrave}, 226.},\\		\index{\Marc}
E forra o largo peito c'os despojos\\
Da malhada Pantera\footnote{ Lin. \textit{Sys. nat. anim.}, ed.~10, p.~41. \textit{Pardus}.} e do escamoso\\
Jacaré\footnote{ Crocodilo brasiliense. \textit{Marcgrave}, 242. Lin.		
\textit{Sys. nat.}, p. 200,  \textit{Crocodilus}.} nadador, que infesta as
águas. \\[10pt] 	\index{\Marc}  %para sobre a edição}

\end{verse}

\pagebreak
\thispagestyle{empty}

\movetoevenpage
\mbox{}\vfill
\thispagestyle{empty}
\noindent Argumento do Canto \versal{iv}
\medskip

{\footnotesize\noindent Episódios na prisão onde o velho Amaro é carcereiro -- Tibúrcio faz passar"-se por um
monge eremita que se hospeda com Amaro -- Embuste sobre Doroteia: primeiro
Tibúrcio, fingindo"-se monge, e depois Marcela, uma vidente, lançam o falso
presságio de que Doroteia é a prometida de Gonçalo, conforme combinado com todos,
inclusive com o herói -- Como parte do embuste, Gonçalo deixa palavras apaixonadas num papel 
-- Esperanças de Doroteia sobre o bem vindo esposo -- Trazendo vinho e presunto aos
presos, a moça leva a resposta que Gonçalo e companhia esperavam: promete soltá"-los
na calada da noite com as chaves do pai -- Sonho  de Gonçalo na cadeia: a Verdade,
a Justiça e a Paz se apresentam alegorizadas lado a lado -- No sonho, louvam"-se os
progressos da Reforma da Universidade -- Mesmo sentindo o suave efeito da Verdade, 
Gonçalo denega do seu apelo e de sua advertência -- Doroteia, enganada pelos companheiros
e pela vidente, planeja libertar os prisioneiros -- Ao fazê"-lo, tropeça e acorda o pai,
mas Tibúrcio improvisa um fantasma com o lençol do carcereiro, fingindo ser seu pai
que o vinha buscar do outro mundo -- Enquanto o velho treme de medo, Doroteia liberta os companheiros
e o presumido esposo a quem se entrega, fazendo sofrer Rufino, enganado pela Ignorância,
e também Cosme, enganado pelo Amor -- Fuga dos amigos pela floresta --  Combate entre os amigos por		\index{\Amor}
conta de Doroteia, que notara tarde o embuste que sofrera e pelo qual tivera culpa -- Doroteia
tenta matar Gonçalo com a espada do herói -- Fim da luta, com a imobilização violenta de Doroteia
por três dos companheiros.}


\chapter{Canto \versal{iv}}
%\setlinenum{0}

\begin{verse}

Tibúrcio, que nas guerras da estalagem\\
Soube abrandar os inimigos peitos,\\
Pondo"-se como em êxtase profundo\\
Com os olhos no Céu e as mãos no peito,\\
Vem a empenhar a força das intrigas.\\
Que não farás, intrépida Ignorância,\\			\index{\Ignor}
Por libertar os tristes prisioneiros! \\[10pt]


Tem o cuidado das ferradas portas,\\
Amaro, vigilante inexorável;\\
Mas crédulo e medroso; e tem ouvido\\
Não sem horror pela calada noite\\
Grasnar nos ares e mugir nos campos\\
Feias bruxas e vagos lobisomens.\\
Com ele o Antiquário se acredita\\
Por um devoto e santo Anacoreta,\\		\index{\Anac}
Que passa os breves dias deste mundo\\
Entre os rigores duma austera vida.\\
Amaro, que se fia de aparências,\\
Para nutrir o frágil penitente\\
Vai degolando os patos e as galinhas.\\
Entanto (quem dissera!) a própria filha\\
Inocente era o móvel deste enredo,\\
Seu nome é Doroteia e no semblante\\
Gênio se lhe descobre inquieto e leve.\\
E como estes momentos preciosos\\
Não se devem perder, depois que a fome\\
Afugentou do estômago vazio,\\
Com branda voz em tom de profecia,\\
Humildade afetando assim começa. \\[10pt]


``Pois tanta caridade usais comigo\\
O Senhor, que reparte os seus tesouros,\\ 
Vos encherá de mil prosperidades.\\
A vossa filha\ldots{} mas convém que eu cale\\
Os segredos que o Céu me comunica.\\
Inda vereis nascer, entre riquezas,\\
Os venturosos netos, doce arrimo\\
Aos fracos dias da caduca idade.''\\
O velho então co'as lágrimas nos olhos\\
Assim falou: ``ó filho Abençoado,\\
Que pela débil voz já me pareces\\
Habitador do Céu, quanto consolas\\
As pecadoras cãs que te estão vendo!\\
Assim talvez seria o meu Leandro,\\
Se as bexigas em flor o não roubassem!\\	\index{\Bexig}
Dez anos tinha, quando a morte avara\\
Cortou co'a dura mão seus tenros dias.''\\
Então suspira e segue passo a passo\\
A longa enfermidade; e enquanto narra,\\
Aparece Marcela, conhecida\\
Entre todas as velhas por mais sábia\\
Em penetrar, olhando para os dedos\footnote{ Esta superstição tem tido
grande uso, vulgarmente \textit{dizer a buena dicha}.},\\
Tudo quanto já dantes lhe contaram.\\
Sobre o pequeno pau a que se encosta,\\
Ela vem debruçada pouco a pouco,\\
O semblante enrugado, os olhos fundos,\\
Contra o nariz oposta a barba aguda:\\
Os dois últimos dentes balanceiam\\
Com pestífero alento, que respira.\\
Em segredo lhe mostra Doroteia\\
A esquerda mão por que ela decifrasse\\
As confusas palavras de Tibúrcio.\\
Ela observa e, depois de mil trejeitos,\\
Franzindo a testa, arcando as sobrancelhas,\\
Com voz trêmula e fraca assim dizia. \\[10pt]


``Ó que grande ventura o Céu te guarda!\\
Por esposo terás um cavalheiro\\
Que te ama e te deseja. Mas ai triste!\\
Em vão chora infeliz o terno amante\\
Nesta escura prisão desconhecido\\
Por casos de Fortuna. Criai filhos,\\			\index{\Fortu}
Ó desgraçadas mães, para que um dia\\
Longe de vós padeçam mil trabalhos!''\\
Aqui suspira a boa velha e chora.\\
Duas vezes começa e depois fala.\\
``O seu nome é Gonçalo: é rico e nobre,\\
E mancebo gentil, robusto e louro.''\\
Estas e outras palavras lhe dizia,\\
E Doroteia já se sente amante,\\
Excogitando os mais seguros meios\\
De abrir a porta e dar"-lhe a liberdade.\\
Na molesta prisão, o novo engano,\\
De imperceptível arte pronto efeito,\\
Sabe o Herói e assim consigo fala.\\
``Ó amigo tão raro como a Fênix,\\
Que podendo deixar"-me entre estes ferros,\\
Vens encher"-me de alívios e esperanças!''\\
Valentes expressões em crespa frase,\\
Que ao \textit{Alívio de tristes}\footnote{ Romance vulgar.} rouba a glória,\\		\index{\Romvulg}
Pensando, felizmente ressuscita\\
Aquelas hiperbólicas finezas,\\
Que em seus escritos prodigou Gerardo\footnote{ Gerardo de Escobar fez
uma obra que intitulou \textit{Cristais d'alma}, cheia de ridículas
hipérboles.},\\ %\textbf{Inserir aqui uma nota sobre este uso do vitupério.}. 
Num pequeno papel como convinha\\
A triste e desgraçado prisioneiro,\\
Viu Doroteia as letras amorosas,\\
Que os ditos confirmaram de Marcela;\\
E dois grandes presuntos, que jaziam\\
Intactos na despensa do bom velho,\\
Vão levar a resposta, acompanhados\\
Do roxo néctar, que dissipa os males.\\
Mensageira fiel, então afirma,\\
Que virá Doroteia abrir"-lhe as portas\\
Nas horas em que o plácido sossego\\
Dos cansados mortais os olhos cerra.\\
Gonçalo espera tímido e confuso,\\
Vem"-lhe à memória o seu antigo afeto;\\
Qual leve sombra, escuta, arde e deseja\\
Sentir no coração novas cadeias. \\[10pt]


Já com a fria mão a noite escura\\
Entre o miúdo orvalho derramava\\
Papoilas soporíferas, que inspiram\\
O brando sono e o doce esquecimento.\\
Reina o vago silêncio que acompanha\\
De amor furtivo os trágicos transportes.\\
Gonçalo então, cansada a fantasia\\
Sobre os meios e os fins de seus projetos,\\ %FANTASIA, MEIOS E FINS
Pouco a pouco se esquece, e pouco a pouco\\
Cerra os olhos, boceja, dorme e sonha.\\
Quando voa do leito, onde deixava\\
Nos braços do Descanso ao Pai da Pátria,\\ \index{\Paida} %\Dom José I	
A brilhante Verdade, e lhe aparece\\		%esse "e lhe" soa estranho
Numa nuvem azul bordada d'ouro.\\
A Deusa ocupa o meio, um lado, e outro\\
A severa Justiça, a Paz ditosa. \\[10pt]


``Benignos Céus, enchei meus puros votos:\\
Fazei que esta celeste companhia,\\
Como do terno Avô rodeia o trono\footnote{ O Augusto e Fidelíssimo Rei de Portugal.},\\		\index{\Aug}
De seu Neto imortal orne a Coroa\footnote{ O Sereníssimo príncipe
herdeiro.}.'' \\[10pt] 		\index{\Netoi}


Gonçalo viu e pondo as mãos nos olhos\\
Receia e teme de encarar as luzes.\\
``Abre os olhos, mortal,'' (assim lhe fala\\
Do claro Céu a preciosa filha)\\
``Abre os olhos, verás como se eleva\\
Do meu nascente Império, a nova glória\footnote{ A Universidade de Coimbra novamente criada.}.\\
Esses muros, que a pérfida Ignorância\\			\index{\Ignor}
Infamou temerária com seus erros,\\
Cobertos hão de ser em poucos dias\\
Com eternos sinais de meus triunfos.\\
Eu sou quem de intrincados Labirintos\footnote{ A Filosofia Racional
sem os enredos dos silogismos Peripatéticos.}\\	\index{\Perip}
Pôs em salvo a Razão ilesa e pura.\\		\index{\Fisic}
Eu abri aos mortais os 
meus tesouros\footnote{ A física.}:\\	\index{\Fisic}
Fiz chegar aos seus olhos quanto esconde\footnote{ A história natural.}\\		\index{\Fisic}
No seio imenso a fértil Natureza.\\
Pode uma destra mão por mim guiada\\
Descrever o caminho dos Planetas;\\
O mar descobre as causas do seu fluxo;\\
A Terra\ldots{} mas que digo? Que ciência\\
Não fiz tornar às margens do Mondego,\\		\index{\Monde}
Ou dentre os braços da Latina Gente\footnote{ Os ótimos e famosos
Professores, que El Rei Fidelíssimo atraiu de diversas partes da Europa.},\\
Ou dos belos países, cujas praias\\
O mar azul por toda a parte lava?\\
Se são firmes por mim o Estado, a Igreja,\\
Se é no seio da paz feliz o Povo,\\
Dizei"-o vós, ó Ninfas do Parnaso,\\
Ilustres, imortais, vós que ditastes\\
As poderosas leis a vez primeira,\\
Vós que ouvistes da lira de Mercúrio\\
Os úteis meios de alongar a vida.\\
Eu vejo renascer um Povo ilustre\\
Nas armas e nas letras respeitado.\\
O seu nome vai já de boca em boca\\
A tocar os limites do Universo.\\
O pacífico Rei lhe traz os dias\\
Dignos de Manuel\footnote{ O senhor rei D.~Manuel, chamado o Feliz.},
dignos de Augusto.\\			\index{\Aug}
E tu enquanto a Pátria se levanta,\\
Sacudindo os vestidos empoados\\
Co'a cinza vil dum ócio entorpecido,\\
Enquanto corre a mocidade alegre\\
A colher louros ávidos de glória,\\			\index{\Lour}
Serás o frouxo, o estúpido, o insensível?\\
Sacrificas o nome, a honra, a Pátria\\
Aos moles dias de uma vida escura?\\
Cego, errado mortal, vê que te enganas.''\\
Disse: e cerrada a nuvem luminosa,\\
Estremece Gonçalo; foge o sono;\\
Por toda a parte lança incerto a vista,\\
Busca assustado, mas já nada encontra.\\
As mesmas impressões em seus sentidos\\
Vivas imagens pintam e não sabe\\
Se então dormia, ou se inda agora sonha.\\
Sente a suave força da Verdade;\\
Mas recusa abraçá"-la. Triste sorte\\
D'alma infeliz que ao erro se acostuma! \\[10pt]


Entanto sem receio o Velho dorme,\\
E a filha vem as sombras apalpando\\
Com as chaves na mão; e quantas vezes\\
Segue, vacila e para, e lhe parece\\
Ouvir a voz do Pai; escuta e treme;\\
Move os passos, tropeça e ao ruído\\
Acorda Amaro e grita. Ela se apressa,\\
E torna a tropeçar. Aqui Tibúrcio\\
Em casos repentinos pronto e destro\\
Em um lençol se embrulha e corre ao leito\\
Onde jazia o Velho espavorido,\\
Que cuida que vê bruxas e fantasmas:\\
Então lhe diz em tom medonho: ``Ó filho,\\
Ingrato filho, que dum Pai te esqueces!\\
Que mal, que mal cumpriste os meus legados!\\
Hoje comigo irás\ldots{}'' Ao Velho o medo\\
Corre as medulas dos cansados ossos;\\
A voz lhe falta, eriça"-se o cabelo.\\
Entanto as portas Doroteia abrindo\\
(Amor a fez intrépida) abraçava\\		\index{\Amor}
O prometido esposo; ele se apressa,\\
Acorda os miserandos companheiros,\\
Que se alegram, deixando solitárias\\
As vagas sombras da prisão funesta.\\
Passa o resto da noite entre temores\\
Amaro, quanto pode o prejuízo! \\[10pt]


Apenas matizava a branca aurora\\
Da Tíria cor o véu açafroado,\\ 		\index{\Acaf}	\index{\Tiria}
Quando o Velho ao través da luz escassa\\
Viu abertas as portas. ``Doroteia,\\
Doroteia onde estás?'' Assim clamava,\\
E entregue à sua dor consulta os olhos\\
Do profeta que pronto a pôr"-se em marcha\\
Com rosto de candura e de inocência\\
Brandamente o consola. ``O Céu, Amigo,\\
Tudo faz por melhor e muitas vezes\\
Com trabalhos cruéis aos bons aflige.''\\
Disse e deixando ao Pai desconsolado,\\
Caminha na esperança de encontrar"-se\\
C'o valente esquadrão dos fugitivos.\\
O Sol já com seus raios luminosos\\
Tinha roubado às folhas dos arbustos\\
O frio gelo do noturno orvalho.\\
Eis à sombra de fúnebre arvoredo\\
Rufino, o melancólico, chorando.\\		\index{\Melanc}
``Quem és, que em tua mágoa inconsolável,\\
Pareces abalar estas montanhas?''\\
Compassivo, pergunta o Antiquário.\\
E depois de chorar por largo tempo,\\
Estas vozes o triste lhe tornava.\\
``Eu sou aquele amante sem ventura,\\
Sempre extremoso e sempre escarnecido,\\
Sofredor das ingratas esquivanças\\
Que vi (ai dura vista!) face a face\\
Do tardo Desengano o feio rosto.\\
Ah Doroteia, um sonho lisonjeiro\\
Meus dias dilatou para que agora\\
Te visse em outros braços insultando\\
O meu fiel amor? Ó noite infausta,\\
Noite terrível, noite acerba e dura!\\
Quanto eu fora feliz se a tua sombra\\
Eternamente os olhos me cobrisse!'' \\[10pt]


Tibúrcio, que já tudo penetrava,\\
Do caminho se informa e dos lugares,\\
Por onde fora a incerta companhia,\\
Que em tanto risco o seu conselho espera. \\[10pt]


Não distante se eleva antigo bosque\\
Horroroso por fama: já nos tempos,\\
Em que torrente Bárbara saindo\footnote{ A irrupção dos bárbaros foi no século \versal{V}.}\\
Do seio da Meótis inundava\\                   %\Meotis
As províncias de Europa, aqui se via\\
Arruinado Templo. Os vivedouros\\
Ciprestes se levantam sobre os pinhos;\\
Heras e madressilvas enlaçadas\\
Ali fazem curvar a crespa rama\\
Dos velhos e infrutíferos carrascos.\\
Três fontes misturando as puras águas\\
Mansamente se envolvem e oferecem\\
À vista cobiçosa os alvos seixos\\
E os verdes limos que no fundo nascem.\\
Os amigos fiéis aqui se encontram.\\
Qual, em noite funesta e pavorosa,\\
Perdido caminhante que receia\\
Achar em cada passo um precipício,\\
Se acaso a dúbia luz divisa ao longe\\
A esperança renasce e de alegria\\
Sente pular o coração no peito:\\
Assim o Desertor, constante e forte,\\
Ao ver o companheiro que prudente\\
Sabe evitar e prevenir os males.\\
Eles se reconhecem e derramam\\
De alegria e ternura o doce pranto.\\
Ó vínculos do sangue e da amizade!\\
Menos unidos viu o Lácio antigo\\
Aos dois Troianos, que uma cega noite\footnote{ Niso e Euríalo. \textit{Virg}.},\\		\index{\Niso}
Espalhando o terror no campo adverso,\\
Levou às turvas margens de Aqueronte.\\		\index{\Aque}
Gonçalo se retira pelo bosque;\\
Com ele vai Tibúrcio e mil projetos\\
Formavam sobre o fim da grande empresa;\\
E a muito fácil e infeliz donzela\\
Do seu profeta o rosto e a voz conhece,\\
E pensa e teme de se achar culpada. \\[10pt]


Então o Amor, que na sonora aljava\\		\index{\Amor}
Esconde setas de mortal veneno,\\
E setas doutro ardor mais grato e puro,\\
Fazia escolha das terríveis armas,\\
Para vingar"-se da cruel Marfisa:\\			\index{\Marfi}
Marfisa ingrata, pérfida, inconstante,\\	\index{\Marfi}
Peito de bronze, a quem a natureza\\
Não formou para ternos sentimentos.\\
E por ver se os seus tiros correspondem\\
Sempre fiéis à mão e ao desejo,\\
Faz no teu peito, ó Doroteia, o alvo,\\
As forças prova e a destreza ensaia.\\
Encurva o arco ebúrneo, solta e voa\\
Sequiosa de sangue a ponta aguda\\
Tinta no Averno. Ao golpe inevitável\\
Tremeu o coração e um vivo lume\\
Nos olhos aparece: do seu braço\\
Admira a força Amor. Vai outra seta\\		\index{\Amor}
Ao brando peito incauto e descoberto\\
Do mancebo infeliz. A vez primeira\\
Soube de amor o namorado Cosme.\\
Que violenta paixão pode encobrir"-se!\\
Os olhos falam: seguem as palavras;\\
E depois o delírio. O tempo é surdo\\
Aos votos dos amantes. Eles viam\\
Crescer ditoso em rápidos momentos\\
De uma nova esperança o belo fruto;\\
Mas Gonçalo a favor dos arvoredos\\
Oculto chega, para e ceva as iras.\\
Tal pode ver"-se o rápido Jaguará\footnote{ Marcgrave \textit{Hist. Brasil.}, p. 235.}\\ \index{\Marc}
Do fértil Ingaí\footnote{ Rio da América, nas Minas do Rio das Mortes.}
nos vastos campos,\\
Se tem defronte o cervo temeroso.\\
Encolhe"-se torcendo a hirsuta cauda,\\
Tenta, vigia, espera e lambe os beiços\\
Formando o salto sobre a incauta presa.\\
Cegos amantes, aprendei agora\\
Os perigos da nímia confiança.\\
O zeloso Gonçalo investe, acodem\\
Os companheiros duma e doutra parte.\\
Triste ruído! pedras contra pedras\\
Ali se despedaçam; ao seu lado\\
Acha Cosme a Rodrigo, acha a Bertoldo.\\
Enquanto dura o férvido combate,\\
Doroteia, que vê sem uso a espada,\\
De que o Herói em fúria se não lembra,\\
(Que não farás, Amor, tu que transformas\\		\index{\Amor}
Uma donzela num feroz guerreiro!)\\
Desembainha; a Morte insaciável\\
Lhe afia o gume e o furor sanguíneo\\
Ergue e dirige o ferro; já pendente\\
Sobre Gonçalo o golpe, salta e chega\\
O amigo a tempo de salvar"-lhe a vida,\\
Pelos braços a aperta e neles grava\\
Roxos sinais dos dedos. Em derrota,\\
Correm os três e o campo desamparam.\\
O mísero, infeliz e novo amante\\
As negras fúrias levam que despertam\\
No aflito coração desesperado\\
Ciúme, raiva, amor, ódio e vingança.\\
Assim o invicto domador dos monstros\footnote{ Hércules, que recebeu
de Dejanira o vestido tinto no sangue do centauro Nesso, e agitado das Fúrias se
lançou no fogo.},\\
Quando por mão da crédula consorte\\
Recebeu o vestido envenenado\\
No sangue infausto do biforme Nesso,\\
Os rochedos e os montes abalava\\ 
Soaram os seus fúnebres gemidos\\
Por longo tempo nas Ismárias grutas\footnote{ Ismaro, monte de Trácia.}.\\
Valentes e indiscretos vencedores\\
Tarde conhecereis e muito tarde,\\
Que um amigo ultrajado é perigoso. \\[10pt]


Para soltar os oprimidos braços\\
Doroteia se empenha; mas Tibúrcio,\\
Lançando a esquerda mão à ruiva trança,\\
A fez voltar, torcendo"-lhe o pescoço,\\
Ao claro Céu a vista ameaçante.\\
Gaspar o ferro dentre as mãos lhe arranca;\\
Este um braço sustenta, outro Gonçalo,\\
E ela presa e sem forças grita e geme.\\
Não doutra sorte o touro da Chamusca\footnote{ Todos sabem que desta
vila são bravíssimos os touros.},\\
Quando três cães o cercam atrevidos,\\
Dois pendem das orelhas e um da cauda.\\
A cornígera testa em vão sacode:\\
Contra a terra se arroja a um lado e outro,\\
E depois que não pode defender"-se,\\
Mugindo exala a indômita fereza. \\[10pt]
\end{verse}

\pagebreak
\thispagestyle{empty}

\movetoevenpage
\mbox{}\vfill
\thispagestyle{empty}
\noindent Argumento do Canto \versal{V}
\medskip

{\footnotesize\noindent Conselho dos heróis sobre o destino de Doroteia prisioneira --
Sofrimento de Doroteia -- Os efeitos da imprudência no Amor			\index{\Amor}
que leva além dos limites da razão -- Para que Rufino parasse de
queixar"-se da Fortuna, o Acaso, filho dela, conduz o amante não					\index{\Acaso}		\index{\Fortu}
correspondido aonde Doroteia ficara amarrada -- Chegada dos companheiros a Mioselha --
Por ciúmes de Gonçalo, Cosme vai ao tio contar as causas da derrota nos estudos 
-- Gaspar determina que os amigos se abriguem na casa de sua mãe --
Jantam levemente -- Tibúrcio chora de fome e lamenta o tempo em
que tivera uma ocupação como tesoureiro de uma irmandade -- Descrição da biblioteca
do tio de Gaspar, que se gabava de livros que para o estilo da Universidade restaurada
eram já obras de mau gosto -- Amaro segue o grupo que lhe roubara a filha -- O povo
cerca a casa -- A Ignorância fala pela boca de Gonçalo, exortando os amigos
a mais uma vez lutar contra a multidão -- Gonçalo precipita um vaso grande
do alto do sobrado sobre a multidão, fazendo muitos feridos -- Começa assim a penúltima
pragmatografia de guerra, descrição de homens em ação bélica -- Os desertores
fogem por um postigo em que Tibúrcio fica entalado -- Encontro de Gonçalo com o tio,
cujas iras se anunciaram no primeiro canto do poema -- O tio apela para que Gonçalo
volte aos estudos -- Gonçalo denega -- Pragmatografia final da surra de pau que Gonçalo
leva do tio -- A Ignorância goza o Império que ainda tem sobre aqueles espíritos fracos
condenados ao ostracismo na província -- Peroração: a voz poética do poema
herói"-cômico pede que as mesmas mãos e cetro que a expulsaram de Coimbra confinem
o monstro da Ignorância definitivamente nas montanhas mais ermas onde o céu
não cansa de lançar raios. }


\chapter{Canto \versal{V}} 
%\setlinenum{0}

\begin{verse}

Alto conselho aqui se faz, aonde,\\
Infeliz Doroteia, o teu destino\\
Cruel e dúbio dum só voto pende.\\
Dos três heróis discordam as sentenças.\\
Um deseja que fique em liberdade\\
E do Pai ultrajado exposta às iras.\\
Inexorável outro pensa e julga\\
Que a sua morte deve dar exemplo,\\
Que encha d'horror as pérfidas amantes.\\
Gonçalo, que era o único ofendido,\\
Consulta o coração e se enternece.\\
Mas o ardente Ciúme, que se alegra\\
De pintar como crimes horrorosos,\\
Inocentes ações, então lhe mostra\\
A feia Ingratidão e o torpe Engano.\\
A Vingança cruel e o vil Desprezo,\\
Ainda mais terrível que a Vingança,\\
Ganham do coração ambas as portas.\\
Mimosa Doroteia, e como ficas\\
Co'as mãos ligadas a um pinheiro bronco\\
Sem outra companhia que os teus males!\\
É este o prêmio, filhas namoradas,\\
Este o prêmio de Amor, quando imprudente\\		\index{\Amor}
Os termos passa que a razão prescreve.\\
De quando em quando um ai do peito arranca,\\
Que ao longe os tristes magoados Ecos\\
Desperta e faz sentir os duros troncos.\\
E espera sem defesa (sorte ingrata!)\\
Que a devorem os lobos carniceiros.\\
Assim ligada aos ásperos rochedos\\
A filha de Cefeu\footnote{ Andrômeda foi exposta a um monstro marinho.
Ovi, \textit{Metamorphose.}} ao mar lançava\\
A temerosa vista e lhe parece\\
A cada instante ver surgir das ondas\\
A verde espalda do marinho monstro. \\[10pt]


Sem esposo, sem pai, sem liberdade,\\
Mísera Doroteia chora e geme.\\
``Ai, Marcela cruel, que me enganaste\\
Com teus belos, fantásticos agouros!\\
Queira o Céu que outras lágrimas sem fruto\\
Mil vezes tresdobradas te consumam\\
Os encovados olhos! Que inda a Morte\\
Às tuas vozes surda correr deixe\\
Piorando em seu curso vagaroso\\
Os momentos de dor e de amargura!'' \\[10pt]


Assim falava. A leve Fantasia\\
Com as cores mais vivas lhe apresenta,\\
De escarpados rochedos no alto cume,\\
O palácio da cândida Inocência,\\
Cercado de funestos precipícios.\\
Ó morada feliz, onde não torna\\
Quem uma vez rodou entre as ruínas!\\
Giram no plano do elevado monte\\
Cruas dores, remorsos devorantes,\\
As três Irmãs, a Peste, a Fome, a Guerra,\\
O pálido Receio, o Crime, a Morte,\\
As Fúrias e as Harpias, que se envolvem\\
No turbilhão dos míseros cuidados. \\[10pt]


Então, de tantas lágrimas movida\\
A mãe soberba do propício Acaso,\\      \index{\Acaso}
A mudável Fortuna e já cansada\\           \index{\Fortu}
De ouvir as tristes queixas de Rufino,\\
Tais palavras ao filho dirigia. \\[10pt]


``Esse amante infeliz que em vão suspira,\\
Ache a dita uma vez e enxugue o pranto.''\\
Acaba de falar e ao mesmo tempo\\
Rufino para o bosque se encaminha,\\
E o Acaso o conduz por entre as sombras\\		\index{\Acaso}
Da pavorosa Noite que já desce.\\
À rouca voz da mísera donzela\\
Palpita o coração: o Amor e o Susto\\			\index{\Amor}
Quiméricas imagens lhe afiguram;\\
Mas ele chega: o próprio crime e o pejo\\
Cobrem de roxas nuvens o semblante\\
De Doroteia ao ver"-se ainda amada\\
Por aquele que foi há poucas horas\\
Alvo de seus insultos e desprezos.\\
A mole vista, as lágrimas em fio,\\
Que aos corações indômitos abrandam,\\
Que fariam num peito namorado?\\
Tu lhe ensinas c'o fraco rendimento\\
Os meios de vencer. Ó sete vezes\\
Venturoso Rufino, se ela um dia\\
Não quiser renovar os seus triunfos\\
E medir a fraqueza do teu peito\\
Pelo grande poder das suas armas! \\[10pt]


Depois de longa e trabalhosa marcha,\\
Cansado de sofrer enfim respira\\
O Desertor e mostra aos companheiros\\
Os conhecidos montes. Fuma ao longe\\
A fértil Mioselha e pouco a pouco\\
Os oiteiros e as casas aparecem. \\[10pt]


Tibúrcio, que uma antiga e voraz fome\\
Sofreu nestes aspérrimos trabalhos,\\
Com gosto espera de afogá"-la em vinho,\\
E já se apressa alegre e transportado.\\
Qual o novilho que perdeu nos bosques\\
A doce vista do rebanho amigo,\\
E depois devagar a noite e o dia\\
Por vales sem caminho a Mãe conhece,\\
Alegre salta e berra e por momentos\\
Espera umedecer entre carícias\\
C'o leite represado a boca ardente. \\[10pt]


Mas Cosme, que conserva na memória\\
As passadas injúrias, por vingar"-se,\\
Ao Tio de Gonçalo, narra as causas\\
Da funesta derrota. Determina\\
Gaspar que os fatigados companheiros\\
Achem na própria casa um doce abrigo.\\
De os ver a Mãe se aflige; mas espera\\
Que obrigados da fome se retirem.\\
Leve foi o Jantar, mais leve a Ceia,\\
E Tibúrcio com pena assim chorava\\
Os dias, em que fora Tesoureiro\\
Duma rica e devota Confraria.\\
``Ó santa Ocupação, tu nunca viste\\
A magra mão da pálida Miséria,\\
Que os fracos membros do mendigo apalpa.\\
Sem trabalho em teus próvidos Celeiros\\
A ditosa Abundância se recolhe.\\
Se torno a possuir"-te, quantas vezes\\
Dos cuidados tenazes e importunos\\
Lavarás a minha alma nas perenes\\
Purpúreas fontes do espremido cacho!'' \\[10pt]


Mostra Gaspar vaidoso a livraria,\\
Donde o Tio Doutor sermões tirava.\\
Mau Gosto, que à razão não dás ouvidos,\\
Vem numerar as obras, que ditaste:\\
Seja a última vez e eu te asseguro\\
Que não vejas fumar nos teus altares\\
Do Gênio Português jamais o incenso. \\[10pt]


Geme infeliz a carunchosa Estante\\
C'o peso de indulgentes \textit{Casuístas}\footnote{ Pode ver"-se o
que deles diz Concina, \textit{Appar. ad Theol. christ.}, c.~\versal{iv}, cap. 5.},\\		\index{\Casui}
\textit{Dianas, Bonacinas, Tamburinos,}\\
\textit{Moias, Sanches, Molinas e Lagarras}.\\ %\footnote
Criminosa Moral, que em surdo ataque\\
Fez nos muros da Igreja horrível brecha,\\
Moral, que tudo encerra e tudo inspira,\\
Menos o puro amor que a Deus se deve.\\
Aparecei famosa \textit{Academia}\\
\textit{De humildes e ignorantes}, \textit{Eva e ave},\\
\textit{Báculo pastoral}, e \textit{Flos sanctorum},\\
E vós, ó \textit{Teoremas predicáveis}\footnote{ Coleção de sermões.},\\
Não tomeis o lugar, que é bem devido\\
Ao \textit{Kess}, ao \textit{Bem Ferreira}, ao \textit{Baldo}, ao \textit{Pegas},\\
Grão"-Mestre de forenses subterfúgios.\\ %\footnote
Aqui Tibúrcio vê o amado \textit{Aranha},\\
O \textit{Reis}, o bom \textit{Supico} e os dois
\textit{Suares}\footnote{ Lusitano e Granatense.}:\\ %footnte
Dum lado o \textit{Sol nascido no Ocidente},\\
E a \textit{Mística Cidade}, doutro lado,\\
Cedem ao pó e à roedora traça.\\
Por cima o \textit{Lavatório da consciência},\\
\textit{Peregrino da América}, os \textit{Segredos}\\
\textit{Da Natureza}, a \textit{Fênix renascida,}\\
\textit{Lenitivos da dor} e os \textit{Olhos de água}\footnote{ Obra
que tem este título -- Fluxo Breve, desengano perene, que o Pégaso da Morte
abriu no monte da contemplação em nove olhos de água para refrescar a alma das
securas do espírito etc. Todas as obras nomeadas neste lugar são conhecidas, e
quando o não fossem bastaria ver os títulos para julgar do seu merecimento, e da
barbaridade do século em que foram escritas. Talvez não sejam estas as mais
extravagantes à vista do \textit{Chrysol Seraphico, da Tuba concionatoria,
Syntagma comparistico, Primavera Sagrada etc}.}:\\
Por baixo está de \textit{São Patrício a cova},\\
A \textit{Imperatriz Porcina} e quantos \textit{Autos}\\
A miséria escreveu do Limoeiro\footnote{ A cadeia pública da Corte.}\\
Para entreter os cegos e os rapazes.\\
Rudes montões de Gótica escritura,\\		\index{\Gotic}
Quanto cheirais aos séculos de barro!\\
Falta ainda uma Estante; mas Amaro\\
Seguindo os passos da roubada filha\\
Caminha aflito e de encontrar receia\\
O valente esquadrão que procurava.\\
Tanto a fama das bélicas proezas\\
O seu nome fazia respeitado! \\[10pt]


Que novas desventuras se preparam!\\
O povo cerca da Viúva as portas;\\
Quando a triste Ignorância, que deseja\\			\index{\Ignor}
Arrancar dentre os ásperos perigos\\
Aos seus Heróis, por boca de Gonçalo\\
Começou a falar. ``Se tantas vezes\\
Mais que heroico valor tendes mostrado,\\		\index{\Heroic}
É este o campo, ide a cortar os louros\\		\index{\Lour}
Para cingir a vencedora frente.\\
Não se diga que fostes oprimidos\\
Por fraca e rude plebe; este combate\\
Não se pode evitar: só dois caminhos\\
Em tanto aperto aos olhos se oferecem.\\
Escolhei ou a Índia, ou a Vitória.'' \\[10pt]


Disse, e depois abrindo uma janela,\\
Arroja de improviso sobre o povo\\
De informe barro uma espantosa talha.\\
Seco trovão que faz gemer os Polos\\
Quando vomitam as pesadas nuvens\\
Do oculto seio a negra tempestade,\\
Não causa mais pavor: ao golpe horrendo\\
Muitos feridos, muitos assombrados\\
Mancham de negro pó as mãos e o rosto.\\
Amaro anima aos seus e enquanto voam\\
Contra a janela mil pesados seixos,\\
(Que novo estratagema!) o Antiquário\\
Finge da capa um vulto, que aparece\\
De quando em quando, com que atrai as \qb{}armas,\\
Que hão de servir depois para a defesa. \\[10pt]


Novo furor os corações acende.\\
Qual a grossa saraiva ao sopro horrível\\
Do Bóreas turbulento embravecido\\		\index{\Borea}
As searas derrota, os troncos despe,\\
E o triste lavrador contempla e chora\\
A perdida esperança de seus frutos:\\
Assim de pedras vaga e densa nuvem\\
Sai da janela a devastar o campo:\\
As que arroja o Herói já se distinguem\\
Pelo som entre as mais, já pelo estrago.\\
A confusão e o susto ao mesmo instante\\
Pelo povo se espalha: então Gonçalo\\
Valeroso saiu por um postigo;\\
Depois Gaspar; o intrépido Tibúrcio,\\
Metendo o braço e a cabeça, clama\\
Que o não deixem ficar naquele estado.\\
O Herói as mãos firmando na orelha\\
Ainda mais o aperta e deixa exposto\\
Da plebe ao riso, à cólera de Amaro.\\
Quantas vezes Tibúrcio desejaste\\
Não ser de grosso peito e largo ventre! \\[10pt]


O Desertor enfim cansado chega\\
À presença do Tio formidável,\\
E a teimosa Ignorância, que se aferra\\			\index{\Ignor}
E que afirma somente porque afirma,\\
O coração de novo lhe endurece.\\
A sofrer o trabalho dos estudos\\
O Tio o anima e roga e ameaça,\\
Mas o Herói inflexível só responde,\\
Que não há de mudar do seu projeto.\\
Não é mais firme a carrancuda roca,\\
Com que Sintra\footnote{ Serra, que acaba na foz do Tejo com nome do
cabo da Roca.} soberba enfreia os mares;\\
Nem tu, ó Pão de Açúcar\footnote{ Grande rochedo na barra da baía do
Rio de Janeiro.}, namorado\\
Da formosa Cidade, Velho e forte,\\
Que dás repouso às nuvens e te avanças\\
Por defendê"-la do furor das ondas. \\[10pt]


Então falando o Tio em torpes crimes,\\
E em furtadas Donzelas, ergue irado\\
Co'a mão inda robusta o pau grosseiro,\\
E a paixão desabafa: a longa idade\\
Proíbe"-lhe o correr; mas não proíbe\\
Que o pau com força ao longe o acompanhe.\\
Ai, Gonçalo infeliz, que dura estrela\\
Maligna cintilou quando nasceste!\\
Depois de mil trabalhos insofríveis,\\
Onde o gosto esperavas e o sossego,\\
Viste nascer estragos e ruínas.\\
Assim depois dos últimos combates,\\
Que as margens do Escamandro \qb{}ensanguentaram,\\
O Rei potente\footnote{ Agamenon, que voltando do cerco de Troia foi
assassinado por Egisto.} d'Argos e Micenas,\\
Esperando abraçar saudoso os Lares,\\
Abraça o ferro duma mão traidora.\\
Fechadas tem o experto Tio as portas:\\
Volta Gonçalo, encontra novos golpes\\
E jaz enfim  por terra. Ferve o sangue\\
Da boca e dos ouvidos; sem acordo,\\
Apenas se conhece que inda vive;\\
Mas tem glória de trazer consigo\\
A derrotada estúpida Ignorância.\\			\index{\Ignor}
Ela reina em seu peito e se contenta\\
De ter roubado aos muros de Minerva\\		\index{\Miner}
De fracos Cidadãos o preço inútil. \\[10pt]


Goza, Monstro orgulhoso, o antigo Império\\
Sobre espíritos baixos que te adoram;\\
Enquanto à vista dum Prelado ilustre,\\
Prudente, Pio, Sábio, e Justo e Firme\\ % assim mesma a pontuaçao e as conjunções
Defensor das Ciências que renascem,\\
Puras as águas cristalinas correm\\
A fecundar os aprazíveis campos.\\
Brotam as flores e aparecem frutos.\\
Que hão de encurvar com próprio peso os \qb{}ramos\\
Nos belos dias da estação dourada.\\
Possa a robusta mão, que o Cetro empunha,\\
Lançar"-te num lugar tão desabrido,\\
Que te sejam amáveis os rochedos\footnote{ Os montes Acroceraunos de
Epiro, onde frequentemente caem raios.}\\
Onde os coriscos de contínuo chovem. \\[10pt]
\end{verse}

\pagebreak

\chapter{Soneto}
%\setlinenum{0}

\begin{verse}


A Terra oprima pórfido, luzente\\		\index{\Porf}
E brilhante metal, que ao Céu erguidos\\
Os altos feitos mostrem esculpidos\\
Do Rei que mais amou a Lusa Gente. \\[10pt]


Esteja aos Régios pés Dragão potente,\\
Que tanto os povos teve espavoridos,\\
C'os tortuosos colos suspendidos\\
No gume cortador da espada ardente. \\[10pt]


Juntas as castas filhas da Memória,\\		\index{\Filmemo}
As brancas asas sobre o Trono abrindo,\\
Assombrem a dourada e muda História. \\[10pt]


Ao Índio livre já cantou Termindo.\\		\index{\Termin}
Que falta, Grande Rei, à tua Glória,\\
Se os louros de Minerva canta Alcindo? \\[10pt]		\index{\Lour}		\index{\Miner}

\vspace*{2em}\versal{E.~G.~P.}
\end{verse}

\pagebreak
\oneside
\chapter{Soneto}
%\setlinenum{0}

\begin{verse}

Enquanto o Grande Rei co'a mão potente\\
Quebra os grilhões do Erro e da Ignorância,\\			\index{\Ignor}
E enquanto firma, com igual constância,\\
À Ciência imortal, Trono luzente, \\[10pt]


Nova Musa de clima diferente\\
Canta do Pai da Pátria a vigilância,\\		\index{\Paida}
Vingando a Mãe das luzes, da arrogância\\
Com que a despreza o estúpido indolente. \\[10pt]


O Monstro de mil bocas sem sossego,\\
Que a Glória de José vai repetindo\\
Ou sobre a Terra ou sobre o imenso Pego: \\[10pt]		\index{\Pego}


Com ela o nome levará de Alcindo\\
Desde a invejada margem do Mondego\\		\index{\Monde}
Ao pátrio Paraguai, ao Zaire, ao Indo. \\[10pt]		\index{\Indo}	\index{\Parag}	\index{\Zaire}

\vspace*{2em}\versal{L.~F.~C.~S.}
\end{verse}

\twoside

\makeatletter
\renewenvironment{theindex}{%
   \@mkboth{\large\MakeLowercase{\scshape Glossário}}%
{\MakeLowercase{\large\scshape Manuel Inácio da Silva Alvarenga}}%
  \thispagestyle{plain}%
\parindent\z@
 \parskip\z@ \@plus .3\p@\relax
  \let\item\@idxitem
}{%
  \clearpage
  }
\makeatother

\part{\textsc{glossário}}

\hedramarkboth{Glossário}{}

\newcommand{\Acaf}{\small\textbf{açafroado}: da cor do açafrão, amarelado}

\newcommand{\Acaso}{\small\textbf{Acaso}: figura alegorizada no poema como filho da
Fortuna, constituindo a verossimilhança da resolução rápida para alguns nós do
enredo cômico}

\newcommand{\Aiur}{\small\textbf{Aiuruoca} (sertão de): região da Capitania de Minas
Gerais, assim chamada até hoje. As referências a ``Ajuruóca'', como está grafado
na primeira edição, e aos papagaios foram valorizadas como paisagem brasileira
pelas críticas românticas e modernistas, entendida como incorporação, no poema
pombalino, de elementos da região em que o autor passou os primeiros anos.  Em
primeiro lugar, as referências a esses elementos locais, como também ao jaguar e
ao jacaré, são mediadas pelos livros que tratam matérias dessa natureza, e não
pela simulação de uma experiência direta. Além disso, no entrecho, tanto o
``Certaõ de Ajuruóca'' quanto os papagaios são elementos de um símile de
caráter cômico, baixo. Essa última circunstância mesma já bastaria para afastar
a interpretação que imputa sentimento nacional pela valorização da cor local, no
poema}

\newcommand{\Alcid}{\small\textbf{Alcides}: outro nome de Hércules; no poema, ``um copo digno de Alcides''
comicamente inverte a matéria heroica da referência mitológica; usando a tópica
comparação com o grande semideus grego em discurso de louvor, ``digno de Alcides''
representa o excesso de bebida no copo, o que comicamente definia parte do caráter
dos tipos irrisórios que o poema heroi-cômico põe em cena}

\newcommand{\Amor}{\small\textbf{Amor}: o Cupido, famoso por ter em sua aljava  flechas de ouro e de prata,
que correspondem ao amor correspondido e ao amor desprezado}

\newcommand{\Anac}{\small\textbf{anacoreta}: monge eremita cristão. No poema, a Ignorância,
já travestida de Tibúrcio, faz seu personagem passar por um Anacoreta
para ganhar o crédito com o velho Amaro, que se apresenta como um
crédulo medroso de coisas sobrenaturais}

\newcommand{\Anfit}{\small\textbf{Anfitrite} (campos de Anfitrite): divindade marinha, esposa de Posídon e irmã de Tétis.
Campos de Anfitrite, isto é, o mar. Na imagem do poema: o sol já se tinha
posto detrás do mar}

\newcommand{\Aque}{\small\textbf{Aqueronte}: rio que na mitologia ficava à entrada do Hades, o reino dos mortos;
foi apropriado na \textit{Divina comédia} e em outros textos para a representação
da entrada do Inferno}

\newcommand{\Aquil}{\small\textbf{Aquilon} e \textbf{Austro}: os ventos setentrional e meridional, respectivamente,
representados mitologicamente conforme a convenção}

\newcommand{\Arist}{\small\textbf{Aristóteles} (século \textsc{iv} a.C.): filósofo grego mais importante para muitas doutrinas de
autoridades sapienciais do Catolicismo romano. Em alguns momentos, como nos séculos
em que a Companhia de Jesus prosperou em muita parte, era reconhecido como ``o Filósofo'',
às vezes como ``divino Aristóteles''. Sobre ele,  costumou"-se dizer que era,
filho de médico, nascido em Estagira em 348 a.C. Depois de haver ficado
vinte anos ao lado de Platão, fundou o Liceu onde passa a ensinar filosofia a
cidadãos gregos. Mesmo com a remoção do ensino jesuítico em todas as partes do reino,
com a política antijesuítica de Sebastião José de Carvalho e Mello, que pôs em descrédito
o termo ``peripatético'', método associado a velhos hábitos jesuíticos, até o final do século \textsc{xviii}
é recorrente o seu nome como \textit{autoridade} para as diversas artes e ciências (Ver \textbf{peripatéticos})}

\newcommand{\Aug}{\small\textbf{Augusto} (século \textsc{i} a.C.): título honorífico dado pelo senado romano a Otávio,
nomeadamente o primeiro imperador de Roma, sobrinho e filho adotivo de Júlio César}

\newcommand{\Batr}{\textit{\textbf{Batrachomyomachia}}: isto é, \textit{A batalha dos Sapos e do Rato}, é uma épica
burlesca, inventada com matéria baixa, referida na \textit{Poética} de Aristóteles.
Atribuída por alguns a Homero e por outros a Pigres, é conhecida
por parodiar inúmeras passagens da \textit{Ilíada}. A \textit{Batrachomyomachia} é aqui, portanto,
a primeira autoridade da espécie poética de que trata o discurso de Silva Alvarenga (Ver \textbf{Poéticas})}

\newcommand{\Bexig}{\small\textbf{bexigas}: nome usual para a varíola, pelo seu sintoma mais
típico, a irrupção de bolsas purulentas na pele; tem caráter muito
letal e contagioso, e deixa as cicatrizes na pele a quem sobrevive
à doença}

\newcommand{\Borea}{\small\textbf{Bóreas}: figura mitológica que representa um dos ventos
filhos de Éolo com a Aurora; ``o Norte fresco'', como é mencionado no
início do canto \textsc{ii}, é também famoso por trazer tempestades}

\newcommand{\Canic}{\small\textbf{Canícula}: como diz a nota de Silva Alvarenga, nome de constelação;
devido à sua posição, o seu aparecimento no céu, está associado ao período
de calor mais intenso no hemisfério norte}

\newcommand{\Carro}{\small\textbf{carro do Sol}: lugar comum da representação mitológica do Sol,
puxado por cavalos divinos}

\newcommand{\Casui}{\small\textbf{casuístas}: teólogos que, segundo encadeamento de premissas lógicas, examinam
casos morais da consciência do pecador. Foram postos em descrédito pelo Iluminismo
assim como pela política pombalina na reforma do ensino. Identificados ou não com o casuísmo
\textit{stricto senso},
Diana, Bonacina, Tamburino, Moia, o Sanches português e o Sanchez espanhol, Molina e
Lagarra, além do próprio Concina, que aparece na nota, são nomes de autores em teologia, jurisprudência, política,
moral, ciências e artes, enfim, tratadistas em geral que particularmente vogaram muito
no tempo dos jesuítas, mas que foram desacreditados, uns mais, outros menos, neste
fim de século \textsc{xviii} em que sai \textit{O desertor}. Na sequência do canto \textsc{v},
títulos de livros e outros autores são citados em tom depreciativo, como efeito do
que na província se lia, devido ao antigo ensino que por causa dos jesuítas ainda era
ministrado na Universidade de Coimbra, até a Reforma de 1772}

\newcommand{\Cepa}{\small\textbf{cepa}, filho de: linhagem, filho de nobre casta. No século \textsc{xviii}, 
é uma tópica da poesia
simpótica --- isto é, poesia para o banquete (\textit{symposium}) --- começar
louvando o varão jovem pela sua cepa, pela sua linhagem nas gerações de
varões ilustres das monarquias europeias}

\newcommand{\Choca}{\small\textbf{choça}: casa pobre de gente rústica. Na representação dos tipos
piores, o desconfiado Rodrigo, assim representado como um
rústico, volta com gosto para o seu casebre, porque não temia
a ira da mãe que ele lá encontraria maldizendo a escolha do filho.
Conformado à vida pior, prefere mesmo não receber as distinções que
as letras poderiam oferecer}

\newcommand{\Cipiao}{\small\textbf{Cipião}: alusão a Dom Sebastião, rei de Portugal morto em Alcazar Quibir,
na África, em referência ao general romano Emílio Cipião, o Africano, que conquistou
Cartago, cidade de África tornada célebre em relatos históricos de guerra, que foi
antigamente cabeça de um grande Império na costa de Barbeia, perto
de Túnis}

\newcommand{\Coler}{\small\textbf{colérico}: tipo  de constituição física determinada pela
bílis amarela, segundo os regimes de classificação dessa medicina}

\newcommand{\Comic}{\small\textbf{cômico} (poesia cômica, comédia):
nas poéticas do século \textsc{xviii}, cômico é a qualidade das matérias piores, a natureza
das ações, das paixões e dos costumes dos homens vis, politicamente inferiores e deformados moralmente.
São tidos por cômicos textos fingidos em estilo baixo imitando os tipos piores}

\newcommand{\Culex}{\textit{\textbf{Culex}}: poema atribuído a Virgílio que trata de assuntos inferiores.
Relata a morte de uma mosca por um pastor e o retorno desta para alertá"-lo dos
perigos do inferno. O \textit{Culex} é aí o modelo latino para a poesia épica
de matéria cômica}

\newcommand{\Dedic}{\small\textbf{dedicatória}: parte das obras em que se declaram os protetorados que
as mantêm em relações de favor previstas nas leis e no costume. Em geral, a dedicatória
é posta no princípio dos livros às vezes na forma de cartas dedicatórias, sonetos,
odes, antepostos aos poemas, às vezes como parte dos próprios poemas.
Este último uso, que é o caso de \textit{O desertor}, é respaldado pela dedicatória de
\textit{Os lusíadas} de Camões}

\newcommand{\Elvas}{\small\textbf{Elvas}: cidade nas fronteiras ao sul de Portugal, no Alentejo,
onde a nota do autor refere ter havido batalhas que constituiriam novas
aristocracias após a Restauração da Monarquia portuguesa, que aclama o herdeiro da casa
de Bragança. No poema, Gaspar perdera a espada na briga, quebrando"-a
ao errar o alvo e acabar acertando o tronco de uma oliveira.
Além de assim demonstrar inépcia no manuseio das armas, dizia que
aquela espada tinha sido herdada de gente que a usara partindo
um castelhano ao meio, nos últimos anos das guerras de Restauração contra
os espanhóis. Esse louvor dos passados heroicos o constitui um 
tipo parecido com Bertoldo, o afidalgado, porque como este gaba"-se
de antepassados ilustres, uns da Lombardia, na Itália, outros em
Elvas, no Alentejo}

\newcommand{\Empav}{\small\textbf{empavesado}: enfeitado como um pavão, emblema da vaidade; no poema,
o termo é empregado sobre o tipo afidalgado, empavesado de feitos heroicos
que não pode dizer que fez, por demonstrar"-se mais de uma vez, mas como suposto
herdeiro de antigos heroísmos não perde a arrogância com que despreza todos em redor}

\begin{comment}
% oliveira: há a palavra encômio no INTRO.tex
\textbf{encômio}: discurso de louvor, nome para diversas espécies do gênero epidítico
alto, isto é, para os discursos de elogio das matérias dignas de elogio. Nos encômios
está previsto sobretudo o encarecimento, ou amplificação, das virtudes dos objetos de
louvor; assim como em seu contrário, o vitupério, está prevista a amplificação dos
vícios dos objetos de ataque verbal.
\end{comment}

\newcommand{\Epico}{\small\textbf{épico} (poesia épica): poesia de modo misto, isto é, que imita fazendo uso da palavra ora
o poeta ora os caracteres agentes, por quem o poeta se faz passar. Tendo como modelo
sobretudo a poesia homérica mais famosa, a \textit{Ilíada} e a \textit{Odisseia},
a épica se confunde com a narrativa heroica, isto é, a imitação de matérias elevadas,
em versos heroicos, falando das \textit{res gestae}, os feitos ilustres dos reis e chefes
cujas ações arriscadas foram dignas da memória}

\newcommand{\Fabula}{\small\textbf{fábula}: até o século \textsc{xviii}, a fábula é o conjunto das ações imitadas na ficção
poética, o enredo, ou entrecho inventado pelo poeta como trama de eventos encadeados segundo
a verossimilhança e a necessidade}

\newcommand{\Fama}{\small\textbf{Fama}: em grego \textit{kléos}, finalidade dos cantos heroicos da epopeia;
no poema aparece ora alegorizada em um carro conduzido pelos ventos Austro e Aquilon, ora
mencionada como fim moral da poesia, ou melhor, como a causa de seu \textit{movere}, porque o
desejo de fama, neste sentido, deve mover os leitores e ouvintes do poema às virtudes,
por imitação e emulação das virtudes dos heróis imortalizados pela fama}

\newcommand{\Filmemo}{\small\textbf{filhas da Memória}: as filhas de Mnemosine são as nove Musas do Parnaso, que, segundo
o mito antigo e as convenções retórico"-poéticas que o apropriaram sob diversas interpretações
em diversos séculos, eram entidades ou alegorias poéticas que promoviam as artes: Clio (preside
a história), Erato (preside a lírica), Euterpe (a música), Melpômene
(a tragédia), Polímnia (os hinos), Talia (a comédia), Terpsícore (a dança), Urânia
(a astronomia) e Calíope (a eloquência)}

\newcommand{\Fisic}{\small\textbf{física}, \textbf{filosofia racional} e \textbf{história natural}: com a Reforma dos Estatutos
da Universidade, entra em Coimbra com mais força os estudos das \textit{físicas},
como se costumava dizer, e de alguns novos métodos como os de Gassendi e Descartes,
assim chamados ``filosofia racional'', e afamados em Portugal como de grande uso nos
grandes centros de sapiência na Europa. Nesse mesmo período, passa a haver em 
Portugal quem aplicasse os princípios de Newton, como o professor de matemática e
cavaleiro professo da Ordem de Cristo,
Garção Stockler, quem mencionasse Galileu, como autor da ciência nova que substituiria os
métodos dialéticos dos professores aristotélicos do tempo dos jesuítas. (ver \textbf{peripatéticos})
Cabe ressaltar que no fim do elogio da Física, a Verdade  interrompe sua enumeração
lembrando que não houve ciência que não fosse por ela reconduzida à Universidade e
que é por ela, a Verdade, que se sustentam o Estado e a Igreja, pedindo que quanto
a isso testemunhem as musas do Parnaso, isto é, a poesia e as demais belas artes
que, como o próprio poema de Silva Alvarenga, louvam e atestam as virtudes de tal
ação política do gabinete do Marquês, que aí impõe a sua representação}

\newcommand{\Fortu}{\small\textbf{Fortuna}: divindade romana reconhecida como mudável, porque
governa a roda da vida, que a um movimento faz descer os maiores e subir
os menores. Ora como representação alegórica, ora mais como noção
moral que alerta os presunçosos, no poema a fortuna, substantivo comum,
é mencionada, por exemplo, no argumento do afidalgado Bertoldo, segundo o qual,
embora pobre e já desmecerecido, o seu passado --- a má fortuna da família --- não altera
o sangue de sua linhagem. Alegoricamente, a Fortuna aparece, no canto \textsc{iv},
impaciente já com as queixas que lhe dirigia o apaixonado Rufino; razão por
que resolve que o Acaso, seu filho, fizesse Rufino encontrar e salvar a moça amarrada
ao bronco pinheiro à espera de lobos carniceiros}

\newcommand{\Galia}{\small\textbf{Gália cisalpina}: norte da Itália, ao sul dos Alpes, Lombardia, Vale do Rio Pó}

\newcommand{\Gotic}{\small\textbf{gótica escritura}: referência depreciativa geral para as coisas velhas da
biblioteca do tio de Gaspar; especificamente, refere"-se à jurisprudência gótica,
isto é, às compilações do direito visigodo, que eram ensinadas como fontes do
sistema jurídico"-português}

\newcommand{\Heroic}{\small\textbf{heroico} (poesia heroica, matéria heroica):
é a qualidade das matérias melhores, a natureza dos feitos, dos afetos e do caráter dos
heróis. Para a invenção heroica, imitam"-se as ações dignas dos \textit{melhores}, louvando os varões
ilustres em ações extraordinárias, como grandes guerras ou grandes viagens}

\newcommand{\Homer}{\small\textbf{Homero}: até o século \textsc{xviii}, Homero é referido como primeira \textit{autoridade}
da poesia pagã antiga. Por isso, e acrescidas as referências das autoridades filosóficas que
o mencionavam, Homero é modelo para tudo o que se trate da composição de versos ficcionais,
isto é, para a Poesia, em especial para a poesia épica, mas também para a tragédia (ver \textbf{Épico)} }

\newcommand{\Ignor}{\small\textbf{Ignorância}: vilã da história; é apresentada como alegoria, personificada
como uma entidade existente por si mesma, a qual se transfigura em Tibúrcio, um antigo
estudante malogrado que vivia de vender objetos usados, sempre em tabernas e envolvido
em todos os demais descaminhos que se supunham à vida estudantil}

\newcommand{\Iminat}{\small\textbf{imitação da natureza}: em tratados de poética que circularam no século
\textsc{xviii} é comum a poesia ser referida como \textit{imitação da natureza} (física ou moral).
Tanto na \textit{Retórica} como na \textit{Poética}, Aristóteles diz que a imitação
é inata no homem. Desde o século \textsc{xvi} pelo menos, a descoberta, as traduções e
apropriações da \textit{Poética} de Aristóteles puseram em evidência diversas interpretações
da Poesia como \textit{mímesis}. Basicamente a Poesia é um fingimento honesto
que pode ensinar as virtudes por diversos meios, modos e assuntos. Os assuntos
concernem à matéria dos discursos, são aquelas coisas sobre que recai a escolha
na invenção retórica. Na definição aristotélica, as coisas que se imitam são os
diversos sujeitos ou matérias da poesia, as virtudes, as histórias, as plantas,
os animais, assuntos da História, que, como arte de discurso, deveria imitar o particular
como verdadeiramente teria sucedido, enquanto a poesia deveria imitar o universal,
fingindo"-o como um particular composto como convém que seja.
A Natureza inclui as coisas humanas e as demais matérias
de que o discurso pode tratar. A Natureza das coisas que são escolhidas
para imitar poeticamente fornece modelos para fingir, com verossimilhança e com verdade,
no universal ou no particular, para utilidade e deleite dos homens que se comprazem
na imitação. No caso da poesia, a imitação é muitas vezes feita em versos
mas, conforme a \textit{Poética} de Aristóteles, não exclusivamente o verso
define a poesia, que pode ser em prosa, desde que seja imitação}
 
\newcommand{\Indo}{\small\textbf{Indo}: rio asiático (ver \textbf{Paraguai})}

\newcommand{\Indus}{\small\textbf{indústria}: no século \textsc{xviii} português, o termo é referido principalmente como destreza,
sutileza, engenho ou habilidade em uma arte; em alguns usos neste fim do século \textsc{xviii},
``indústria'' já é referido como a destreza humana nas novas técnicas de processar
manufaturas que se tornam objeto de comércio entre as nações do mundo. Neste último sentido,
mais raro inicialmente, e mais recorrente com o fim do século \textsc{xviii} e início do \textsc{xix},
será identificado com os modernos processos de manufatura que daria principalmente
ao império britânico o lugar de maior força no comércio internacional}

\newcommand{\Invoc}{\small\textbf{invocação}: artifício retórico utilizado à imitação de Homero, Hesíodo e outros.
Encenando a poesia como resultado do furor de um \textit{entusiasmós}, de uma possessão
divina, invocou"-se no início dos poemas seja uma musa, seja o conjunto delas, seja ainda
a deusa da memória,
e daí também a Virgem Maria ou os Anjos, em adaptações católicas do artifício, para que
as entidades superiores inspirassem o canto, mesmo quando era feito com arte e método (ver \textbf{filhas da Memória})}

\newcommand{\Ismae}{\small\textbf{ismaelitas}: já foram chamados mouros e agarenos, descendentes de Agar, mãe de Ismael.
Também se chamaram Sarracenos, nome que lhes deu Mafoma, como os portugueses designavam
Maomé, ou Muhamed, profeta fundador do Islã, que se presumia descendente da casta de
Sara, mulher legítima de Abraão}

\newcommand{\Jane}{\small\textbf{Janeiro}: no poema, nome próprio de um \textit{doméstico},
funcionário de hospedaria, representado como tipo traiçoeiro. O nome,
que equivale a Januário, por exemplo, já indica essa tipificação
do caráter, porque Janus é a divindade de duas caras, que,
em usos vituperantes, tem valor de hipocrisia, falsidade, pouca
confiança. Não se confunda esse uso com a natureza da divindade
que tem duas caras porque fica nas portas das cidades, desejando
boas vindas alegremente e boa viagem com tristeza}

\newcommand{\Lineu}{\small\textbf{Lineu} (Carl von Linné, 1707--1778): historiador natural sueco autor do sistema
moderno das taxionomias das espécies naturais dos seres vivos, nomeados segundo
classificações de gênero e espécie. (ver \textbf{Lucrécio})}

\newcommand{\Longob}{\small\textbf{longobardos}, ou Lombardos: povo germânico que, no tempo das assim chamadas
invasões bárbaras, ocupou o Norte da Itália (ver \textbf{Gália Cisalpina}).
Constituíram um reino que, depois de cristianizados os chefes, foi
chamado \textit{Regnum Italicum}, donde sairiam cavaleiros cruzados,
cujo mérito antigo Bertoldo, o afidalgado, requeria para si, apesar da
má fortuna}

\newcommand{\Lour}{\small\textbf{louros}: folhas de louro com que se coroavam os vencedores. Simboliza a glória
nas armas ou nas letras, distinguindo os melhores lutadores, os melhores atletas e os
melhores poetas. Os ``louros de Minerva'', como se diz no poema, representam
o reconhecimento da vitória no conhecimento, de que
Minerva é patrona. No poema, são esses os louros que os heróis perderam}

\newcommand{\Lucre}{\small\textbf{Lucrécio} (Tito Lucrécio Caro, século \textsc{i} a.C.): poeta latino, escolado na doutrina de Epicuro,
a qual expôs em verso, nos seis livros do \textit{De rerum natura} (\textit{Sobre a
Natureza das Coisas}) mencionado como fonte do ensino epicurista no mundo romano. 
Foi quase sempre lido em âmbito católico como fonte de
verdades físicas, mesmo que fossem impugnados como heresia os principais fundamentos
da doutrina de Epicuro. No ``Discurso'' de Silva Alvarenga, Lucrécio,  mencionado
ao lado de Aristóteles e logo, com as notas, ao lado
de Marcgrave e Lineu, constitui autoridade da física, isto é, em conjunto são esses
autores que sua erudição escolhe para produzir o crédito do discurso exordial sobre
a natureza e a arte do poema heroi-cômico.
As autoridades dos séculos \textsc{iv} ou \textsc{i} a.C.
são compatíveis com os autores dos séculos \textsc{xvii} e \textsc{xviii}, como sistemas classificatórios
das \textit{naturalia}. A singularidade dos autores não obsta sua participação no
mesmo gênero de matéria, a \textit{physis}}

\newcommand{\Lutri}{\textit{\textbf{Lutrin}}: poema heroi-cômico composto por Nicolas Boileau"-Despréaux (1636--1711).
Boileau foi uma das principais autoridades francesas da arte poética antigongórica e
antimarinista que no final do século \textsc{xvii}, desqualificou o estilo de poetas italianos
e espanhóis famosos pela acumulação de agudezas, pelas dificuldades elocutivas, pelo
excesso no emprego de figuras etc. Ficou, por isso, conhecido como ``teórico'' da
poesia dita ``neoclássica'', talvez o mais importante autor moderno para as reformas
no estilo da poesia e da eloquência portuguesa no período que ficou
conhecido como restauração das letras em Portugal, desde a publicação do \textit{Verdadeiro
método de estudar}, de Verney, no final dos anos de 1740, e depois principalmente por
efeito da política pombalina contra os hábitos e métodos empregados pelos jesuítas
e substituição por modelos prediletos dos padres oratorianos}

\newcommand{\Marc}{\small\textbf{Marcgrave} (Georg Markgraf, 1610--1648): autor, com Guillelmo Piso, de
\textit{Historia Naturalis Brasiliae} (1648), livro dedicado a Maurício de Nassau,
que representa as plantas e animais do Brasil, além dos costumes dos indígenas  (ver \textbf{Lucrécio})}

\newcommand{\Marfi}{\small\textbf{Marfisa}: personagem do \textit{Orlando Furioso}, de Ariosto, e do \textit{Orlando Enamorado}, de Boiardo.
No poema, Marfisa, que tem o peito de bronze para o Amor, só é mencionada para colocar o Cupido em cena
treinando suas vinganças contra ela e, para isso, usando os corações de Doroteia e de Cosme como alvos,
o que terá consequências penosas para ambos e para a própria companhia (ver \textbf{Amor})}

\newcommand{\Margi}{\textit{\textbf{Margites}}: na \textit{Poética}, de Aristóteles, preservam"-se alguns dos poucos
excertos que se conhecem desta sátira, referida, principalmente, como espécie
poética reconhecida pelo seu metro, jâmbico, por isso chamada poesia jâmbica.
Diz"-se que esse poema de natureza satírica, hoje perdido, foi atribuído por
Aristóteles a Homero e outros o atribuíram a Pigres, um ateniense anterior 
ao tempo de Xerxes}

\newcommand{\Marqpombal}{\small\textbf{Marquês de Pombal}: Sebastião José de Carvalho e Melo (1699--1782) foi embaixador em Londres
e em Viena, durante o reinado de Dom João \textsc{v}. Com a ascensão de Dom José \textsc{i}, foi nomeado 
ministro dos negócios estrangeiros, depois Ministro de Estado plenipotenciário. Governou
Portugal como um ditador, durante praticamente todo o reinado de D. José \textsc{i}, de 1750 a 1777.
Foi preso depois da ascensão de Dona Maria \textsc{i}. Tornou"-se Marquês de Pombal somente em 1769,
título acumulado sobre o de Conde de Oeiras, recebido em 1759 (ver ``Introdução'')}


\newcommand{\Marte}{\small\textbf{Marte}: Ares para os gregos, deus da guerra e da discórdia; conforme os
gêneros e as circunstâncias discursivas. No poema, é às vezes vituperado --- ``o homicida Marte'', no canto \textsc{iii}
--- pelos danos das \textit{tristia bella} (as tristes guerras), mas em outras circunstâncias, que previssem outros
decoros, poderia ser alegorizado para constituir o louvor na representação
de virtudes guerreiras, que  concernem às vitórias dos reis e grandes senhores,
as \textit{res gestae} (os feitos ilustres), matéria da poesia heroica}


\newcommand{\Melanc}{\small\textbf{melancólico}: tipo de constituição anímica e física que, segundo 
a fisiologia antiga, é causada pela bílis negra, que conformaria
as disposições do ânimo de quem sofre de melancolia. No poema
é sobretudo essa acepção patológica que se aplica tanto a Rodrigo,
como a Cosme, que sofrem disso por se deixarem sempre enamorar além
da medida da razão, como o poema judica mais de uma vez}

\newcommand{\Miner}{\small\textbf{Minerva}:  Palas Atena para os gregos; deusa da sabedoria, representada
com elmo e lança. Aparece no poema sempre como alegoria das ciências reformadas
em Coimbra. Assim, ``os muros de Minerva'', de onde fugiram os desertores,
são os muros da Universidade}

\newcommand{\Mocho}{\small\textbf{mocho}: ave noturna e carnívora como as corujas}

\newcommand{\Monde}{\small\textbf{Mondego}: rio de Coimbra, que nasce na serra da Estrela e desagua junto à Figueira da Foz}

\newcommand{\Morg}{\small\textbf{morgado}: herança patrimonial exclusiva do primogênito. Criado pelo tio,
Gonçalo mente descaradamente a sua própria condição para a amante}

\newcommand{\Netoi}{\small\textbf{neto imortal}: Dom José de Bragança é referido como herdeiro de Dom José \textsc{i},
que não tivera filho varão. O neto morreria uma década depois do avô, sem
deixar herdeiro, o que faria de seu irmão, o futuro Dom João \textsc{vi}, o sucessor de Dona Maria \textsc{i}}

\newcommand{\Niso}{\small\textbf{Niso} e \textbf{Euríalo}: duas personagens troianas representadas
no canto \textsc{ix} da \textit{Eneida} de Virgílio} 

\newcommand{\Ovid}{\small\textbf{Ovídio}: poeta elegíaco romano, do século \versal{I} a.C. Foi banido de Roma por Augusto,
a quem dedica seu principal livro, as \textbf{Metamorfoses}. Tanto pelas \textit{Metamorfoses},
quanto pela \textit{Arte de amar}, Ovídio foi sempre muito lido, mesmo durante a chamada
Idade Média. No costume poético em que se inscrevem os poemas de Silva Alvarenga, as
\textit{Metamorfoses} são principalmente fonte de fábulas mitológicas que podiam ser
usadas para ornamento dos poemas, não como sinal de paganismo dos autores, que eram
sempre católicos e súditos do rei de Portugal.}

\newcommand{\Paida}{\small\textbf{pai da Pátria} ou \textit{pai do Povo}: designação de Dom José \textsc{i} nos discursos
encomiásticos e em documentos oficiais}

\newcommand{\Parag}{\small\textbf{Paraguai}: rio Paraguai, chamado no soneto final de \textsc{l.f.c.s.}, ``pátrio Paraguai''
porque o autor de \textit{O desertor} era americano. Na enumeração --- Paraguai, Zaire e Indo,
os três rios são mencionados em alusão aos três continentes por onde se estendiam os domínios de Portugal.
Seguindo os influxos do Mondego, os rios constituem um emblema dos
efeitos da Reforma da Universidade em todas as dominações portuguesas}

\newcommand{\Pego}{\small\textbf{pego}: variante de pélago, referindo"-se ao mar}

\newcommand{\Perip}{\small\textbf{peripatéticos}: da escola de Aristóteles. ``Filósofo peripatético''
equivale a dizer ``filósofo aristotélico''. Com a política pombalina, o
aristotelismo português foi desqualificado, e daí que o termo ``peripatético''
apareça nas letras pombalinas em sentido pejorativo. Ter se perdido nas
questões do \textit{Peripato} é causa do fracasso acadêmico da personagem
Tibúrcio, por exemplo, que estudara no tempo dos jesuítas.
Costuma"-se usar ``peripatético'' no século \textsc{xviii} para distinguir no vitupério
os maus seguidores e a verdadeira doutrina do Filósofo, lido por muitos
santos padres da Igreja (ver \textbf{Aristóteles})}

\newcommand{\Pichel}{\small\textbf{pichel}: recipiente grosseiro, caneca}

\newcommand{\Plat}{\small\textbf{Platão} (século \textsc{v} e \textsc{vi} a.C.): filósofo referido no século \textsc{xviii} como  doutrinador dos preceitos
morais de Sócrates, como fundador da \textit{Academia} em Atenas e como mestre
de Aristóteles. Este último teria dissentido dos princípios do mestre, mas as duas
doutrinas foram harmonizadas por mais de um intérprete e comentador entre os 
Padres da Igreja. O mais célebre arranjador da tese da harmonia entre
Platão e Aristóteles, foi Boécio (século \textsc{v}--\textsc{vi} d.C.). }

\newcommand{\Poet}{\small\textbf{Poética}: as Poéticas são textos de doutrina prática que ensinam os princípios da
arte e os procedimentos técnicos que regram as várias espécies de poesia. Em textos impressos
até o fim do século \textsc{xviii}, encontram"-se referências a conceitos retórico"-poéticos gregos
e latinos que prescreviam procedimentos para os efeitos da poesia, conforme os fins
de cada gênero e de cada espécie de poemas. A primeira autoridade conhecida em Poética é
Aristóteles, que define a Poesia como imitação de caracteres, afetos e ações, mas a sua
poética só foi conhecida em âmbito europeu, entre os séculos \textsc{xv} e \textsc{xvi}. Platão fala de poesia
mas esparsamente em diversos diálogos que mencionam ora uma ora outra espécie poética,
tratando"-as como um costume. Como arte imitadora, a Poética é para Platão
produtora de inverdades e causadora  de paixões; de ambos os efeitos deveriam fugir
os filósofos de sua escola, razão pela qual a poesia em geral é recusada pela filosofia
platônica (ver \textbf{Aristóteles}, \textbf{imitação da natureza}, \textbf{Platão} e \textbf{República})}

\newcommand{\Porf}{\small\textbf{pórfido}: cor púrpura, referindo"-se ao bronze da estátua}

\newcommand{\Prosop}{\small\textbf{prosopopéia}: Alegoria em que se personificam coisas concretas,
noções morais ou coletivas.}

\newcommand{\Repub}{\textit{\textbf{República}}: livro renomado de Platão, emulado por Cícero, cujo assunto é a
constituição da ideia de cidade perfeitamente governada, representação filosófica da \textit{pólis} melhor que o possível.
Nessa cidade filosófica, mesmo os poetas que pintam os homens melhor que o possível,
como a maior parte dos poetas trágicos, não teriam seu ofício reconhecido, porque nela
não deveriam entrar as artes miméticas, bem como não entrariam os imitadores de Homero,
ainda que, para o Sócrates de Platão, Homero fosse o melhor que era possível haver para a educação ateniense.
Não são expulsos da cidade os Poetas em geral, mas conforme as espécies
discursivas que produziam. Por exemplo, a poesia que produz o riso acerca do feio, do torpe,
do desprezível, para Platão, não ensina virtudes, como outras tradições de opiniões fariam
crer, e como \textit{O Desertor} também pressupõe. Na \textit{República}, a poesia \textit{lírica},
ou \textit{mélica}, entendida como
o louvor das virtudes dos heróis, é o único tipo de canto que a cidade perfeita admitiria;
porque aí o louvor dos feitos não inclui a \textit{mímesis}, isto é, o poeta não fala pelas
personagens, alterando o próprio \textit{ethos}, mas usa sempre a voz própria (ver \textbf{imitação da natureza}, \textbf{Platão})}

\newcommand{\Romvulg}{\small\textbf{romance vulgar}: no poema, designa"-se por essa expressão um
gênero de livros, quase sempre moralidades, que narrativamente ou
não ensinavam, embora com pouca arte, os bons costumes, a boa consciência,
os perigos das paixões etc. Mais de um título de livros assim é mencionado
no poema, em tom evidentemente desqualificador sempre; são obras
que tendo sido famosas entre o vulgo logo se tornam esquecidas.
Por exemplo, no embuste da prisão, o herói lembra"-se de passagens
de romances vulgares, como \textit{Alívio de tristes} ou \textit{Cristais
d'alma} para fazer suspirar a moça enganada, Doroteia, filha do carcereiro,
para que os libertasse. A arte desses livros é vituperada no próprio poema
como áspero estilo e hiperbólicas finezas}

\newcommand{\Secch}{\textit{\textbf{Secchia Rapita}}: poema em doze cantos de autoria incerta, atribuído a Tassoni,
primeira autoridade moderna na poesia heroi-cômica. O \textit{Secchia
rapita} tem por matéria heroica a guerra entre os bolonheses e os modenenses na
época do imperador Frederico \textsc{ii}. É tido por referência para a composição dos
cantos de \textit{Lutrin}, de Boileau, e de \textit{Rape of the Lock}, de Alexander Pope}

\newcommand{\Tasson}{\small\textbf{Tassoni}: Autor do livro \textit{La secchia rapita} emulado nos
séculos \textsc{xviii} e \textsc{xix}, como inventor moderno do poema heroi-cômico. As
circunstâncias de sua vida são obscuras, alguns referem a ele como sendo Torquato Tasso,
autor do poema heroico \textit{Jerusalém libertada}}

\newcommand{\Termin}{\small\textbf{Termindo Sipílio}: nome de Basílio da Gama na Arcádia de Roma. No primeiro soneto
que termina \textit{O desertor}, Termindo é citado como o que cantou a liberdade dos índios;
bem entendido, a libertação dos índios é narrada em \textit{O Uraguai} como a guerra que os dizima para os
tirar da custódia dos padres da Companhia de Jesus}

\newcommand{\Tiborn}{\small\textbf{tibornas e magustos}: a nota do autor aos dois termos
explica o que sejam. O fato de estarem em nota os dois termos
é um uso análogo das notas para Aiuruoca ou Tatu: trata"-se de
coisas insignificantes para os leitores mais ilustres do poema, que
no limite era até mesmo o Marquês de Pombal. Nos versos, os dois
termos, que têm provavelmente uma circulação vulgar, devia dar
comicidade à declaração de amor de Gonçalo, que depois de falar
em laços eternos de amor, pinta a felicidade como a mulher fazendo
pão com linguiça.}

\newcommand{\Tipos}{\small\textbf{tipos}: em textos poéticos como em textos históricos, os tipos são 
inventados como \textit{ethos}, caracteres, modelos de virtudes e vícios,
fingidos com palavras, com mais ou menos harmonia, números e tropos, imitados
conforme os seus costumes, os afetos e os feitos de homens que verdadeiramente existiram,
ou que foram concebidos pelo engenho de algum poeta (Ver \textbf{imitação da natureza})}

\newcommand{\Tiria}{\small\textbf{tíria}: feminino tírio, oriundo da cidade de Tiro, famosa pela cor escarlate dos
pigmentos que produzia}

\newcommand{\Trag}{\small\textbf{trágico} (poesia trágica): poesia puramente mimética, isto é, que imita os caracteres
agindo diretamente. Tendo como modelo principalmente Ésquilo, Sófocles e Eurípedes,
a tragédia está incluída no mesmo gênero de matéria da epopeia, e suas matérias particulares
costumaram ser tiradas das narrativas homéricas, imitando também ações, afetos e caracteres
de homens melhores (\textit{aristoi}). Se por um lado pertence ao mesmo gênero da epopeia, por outro
pertence ao mesmo gênero de enunciação que a comédia, por imitar as personagens
diretamente}

\newcommand{\Ulis}{\small\textbf{Ulisses}:  nome latino de Odisseu, rei de Ítaca, herói da \textit{Odisseia}, poema atribuído
a Homero que narra as peripécias de Ulisses após a guerra de Troia, retornando para
sua casa. Enfrentando a ira de Posídon, o deus dos mares, ajudado por outras divindades,
enfrentando monstros e outros perigos, perde todos os companheiros antes de ser reconduzido
à sua pátria, onde entra sob o disfarce de um mendigo para junto ao filho retomar a
ordem e o seu poder, ameaçado pelos pretendentes de Penélope, a esposa fiel, que não cedeu
o leito e o trono na ausência do marido. Como é um poema de viagem, e não de guerra, o poema heroi-cômico
de Alvarenga emula, mas comicamente, a espécie heroica da \textit{Odisseia}, assim como da \textit{Eneida}}

\newcommand{\Urag}{\textit{\textbf{O Uraguai}}: poema de Basílio da Gama que narra a guerra da aliança luso"-castelhana 
contra os Sete Povos das Missões, reduções jesuíticas juridicamente portuguesas até 1750 mas
que restaram no território que a partir do Tratado de Madrid passou a ser
possessão da Coroa espanhola. Com isso, o poema heroico em cinco cantos louva as ações
do Conde de Oeyras, de seu irmão e de seus lugares"-tenentes, na execução bélica do acordo diplomático.
Narrando a pacífica libertação da Universidade do jugo do ensino
jesuítico, \textit{O desertor} emula, mas com matéria cômica, \textit{O Uraguai} cujo
assunto é a sangrenta guerra, contra os padres jesuítas e os índios custodiados que se recusavam
a desocupar a terra, às margens do rio Uruguai, chamado Uraguai no poema de Basílio, seja por um
solecismo, seja deliberadamente pela eufonia do título.}

\newcommand{\Util}{\small\textbf{útil e agradável}: é tópica horaciana muito recorrente em textos setecentistas o
preceito do consórcio do \textit{útil} com o \textit{agradável} como definição do mais
apto na arte da poesia. Candido Lusitano assim comenta os versos de Horácio citados no
fim do ``Discurso sobre o poema heroi"-cômico'': \mbox{``O Poeta,} 
pois, que quiser ter os votos de todos, dos velhos e dos moços, há de em suas obras fazer
inseparável o instrutivo do deleitoso. Esta
é toda a força do \textit{pariter} (igualmente, ao mesmo tempo): isto é, não há de instruir em um lugar, e
deleitar em outro; há de o deleite acompanhar sempre a instrução. Os que sabem a
História Romana, bem alcançam que neste verso a palavra \textit{punctum} vale o
mesmo que \textit{suffragia} [votos], sendo costume dos Romanos dar os seus votos por
pontos.'' \cite[p. 158--159]{horacio}}

\newcommand{\Veros}{\small\textbf{verossimilhança}: Em \textit{Poéticas} de cunho aristotélico, 
como esta que Silva Alvarenga repõe, que entendem a poesia como imitação da natureza,
a verossimilhança deve ser produzida a partir das qualidades naturais das matérias em geral e das marcas
acidentais das matérias particulares. Nas pessoas, imita"-se sobretudo o caráter,
o \textit{ethos}, isto é, o que na imitação faz um velho, uma moça ou um escravo, parecerem realmente um
velho, uma moça e um escravo, para que um avarento possa parecer como costumam ser os avarentos,
assim como proporcionalmente os grandes homens do passado possam parecer os autênticos portadores
das virtudes que os imortalizaram, e assim por diante conforme as longas galerias de tipos que
a poesia sempre fez encenar. Estas qualidades são referidas nas tábuas e
preceitos que ensinavam imitação poética segundo categorias como: nascimento,
condição de vida, os diversos atributos das idades, a nação, a fortuna, o engenho
(ou inclinação particular do ânimo). Para isso, seguia"-se o que se doutrinava tanto
no \textit{Fedro}, de Platão, como na \textit{Retórica}, de Aristóteles: para
bem produzir os efeitos dos discursos era preciso conhecer antes de tudo a alma
humana, que será sempre o auditório e que quase sempre será o assunto dos discursos.
No \textit{Ad Herennium}, a definição das matérias está circunscrita às coisas
que os costumes e as leis instituíram para o uso civil; assim, a verossimilhança
tem suas condições determinadas pelos ofícios no uso civil.
Daí que antes de mais nada a persuasão do auditório esteja
diretamente subordinada à paciência dos ouvintes, e daí que a arte estivesse
condicionada à adequação aos ``receptores'' das palavras poéticas (ver \textbf{imitação da natureza}, \textbf{Poéticas} e \textit{tipos}.)}

\newcommand{\Xavec}{\small\textbf{xaveco}: embarcação mourisca que ficou conhecida pelo uso na pirataria,
devido à facilidade com que permitia a abordagem graças ao seu tamanho e à
disposição  das velas; pelo sentido negativo atribuído ao fato de sua
fabricação ser moura (ver \textbf{Ismaelita}), também pode significar simplesmente
embarcações inferiores e mal aparelhadas}

\newcommand{\Zaire}{\small\textbf{Zaire}: rio africano (ver \textbf{Paraguai})}

\newcommand{\Zefir}{\small\textbf{Zéfiro}: Figura mitológica que representa um dos ventos
filhos de Éolo com a Aurora. Vento oeste}


%\input{\printindex}


\printindex

\end{document}
