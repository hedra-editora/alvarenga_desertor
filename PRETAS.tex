\textbf{Manuel Inácio da Silva Alvarenga} (1749--1814) nasceu em Vila Rica,
mas viveu a maior parte da vida no Rio de Janeiro, capital da Colônia
e, a partir de 1808, sede da Corte Portuguesa.
Diz"-se que era pardo e de origem humilde, mas teria progredido
nos estudos graças ao empenho do pai e de uma subscrição de amigos
que teriam financiado sua ida ao Rio de Janeiro e depois a Coimbra,
onde se tornaria amigo de Basílio da Gama, autor de \textit{O Uraguai} (1769).
Permaneceu em Portugal enquanto durou seu curso em Coimbra,
entre 1773 e 1777. Voltando ao Rio de Janeiro, formado em Direito
Canônico, torna"-se advogado. 
De Dom Luís de Vasconcelos e Sousa, vice"-rei e capitão"-geral do Brasil,
obteve uma cadeira de Retórica e Poética. Na amizade deste integra mais
de uma agremiação literário"-científica na capital da Colônia. Com a
nomeação do Conde de Rezende, que proíbe essas associações,
ganha sua inimizade. Provavelmente por fazer circular sátiras contra seu
governo, é acusado de Inconfidência, perde os direitos civis e permanece
preso por dois anos, mas é indultado por decreto de Dona Maria \textsc{i}, com o
que readquire os direitos de súdito e aparentemente segue a carreira
que já cursava, sempre na cidade do Rio de Janeiro. Morre em 1814,
respeitado como advogado e dono de uma das maiores bibliotecas particulares
do Rio de Janeiro, a qual, após a sua morte, foi comprada pelo então príncipe
regente Dom João \textsc{vi} para a incorporá"-la à Biblioteca Real, que mais tarde
se tornaria a Biblioteca Nacional do Rio de Janeiro.

\textbf{\textit{O desertor: poema heroi"-cômico}} (1774) foi impresso pela Real
Oficina da Universidade de Coimbra, por ordem do Marquês de Pombal,
segundo informação do primeiro biógrafo de Silva Alvarenga, que o teria conhecido
como seu aluno nas lições de Retórica e Poética.
Quando \textit{O desertor} sai à luz, Silva Alvarenga tinha 24 anos de
idade e era aluno na Universidade recentemente reformada.
Com efeito, o argumento heroico do poema é a Reforma dos Estatutos da
Universidade de Coimbra, o que lhe dá sentido didático e encomiástico.
Por outro lado, a dissociação deliberada entre o assunto baixo e a elocução
ornada com palavras graves dignas de grandes feitos é o que fundamenta o
subtítulo do poema que o enquadra num gênero misto que então já tinha modelos
da poesia italiana, francesa e portuguesa que o autorizavam como tal.
A fábula cômica é constituída pelas peripécias de um grupo de estudantes
guiados pelo professor Tibúrcio, personificação da Ignorância, expulsa
de Coimbra pelo Marquês, que restituíra a Verdade ao trono na velha
instituição de ensino.
Aristotelicamente fundada, a fábula é cômica, por definição, porque imita
homens e ações \textit{piores}, descreve matérias baixas e dignas de opróbrio.
Assim, acumula tipos socialmente inferiores e/ou moralmente deformados,
relata brigas comezinhas, com unhas e dentes, tumultos e bebedeiras, em lugar
de triunfos da virtude.
A comicidade do poema foi quase sempre desmerecida pela crítica literária
dos séculos \textsc{xix} e \textsc{xx}, provavelmente porque a elocução do poema é alta,
imitando principalmente o estilo dos versos brancos heroicos de \textit{O Uraguai}.
Mas a graça do poema estava exatamente em narrar como grande coisa e com palavras
infladas, as bravatas risíveis de personagens dignos de desprezo.


\textbf{Ricardo Martins Valle} é doutor em Literatura Brasileira pela
\textsc{usp}, e professor de História Literária na Universidade Estadual do
Sudoeste da Bahia, \textsc{uesb}.


\textbf{Clara Carolina Sousa Santos} é professora, mestre em Memória e em Linguística pela Universidade Estadual do Sudoeste da Bahia, \textsc{uesb}. 


