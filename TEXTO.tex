\part[O desertor: poema heroi-cômico]{\textsc{o desertor}\break \textsc{poema heroi-cômico}} 

\chapter{Discurso sobre o poema heroi"-cômico}
\hedramarkboth{Discurso sobre o poema}{}

\begingroup
\linenumbers

A Imitação da Natureza, em que consiste toda força da Poesia, 				\index{\Iminat}
é o meio mais eficaz para mover e deleitar os homens; 					\index{\Util}
porque estes têm um inato amor à imitação, harmonia e ritmo.
Aristóteles, que bem tinha estudado a origem das paixões, assim o afirma no 
cap. 4 da \textit{Poética}. 							\index{\Arist}
Este inato amor foi o que logo ao princípio ensinou a imitar o Canto das Aves; ele 
depois foi o inventor da Flauta e da Poesia, como felizmente exprimiu Lucrécio
no liv. \versal{I}, v.~1378.								\index{\Lucre} \index{\Poet}


\begin{verse}
At liquidas avium voces imitarier ore \\
Ante fuit multo, quam l\ae via carmina cantu \\
Concelebrare homines possent, auresque \qb{}juvare. \\
Et Zephyri cava per calamorum sibila \qb{}primum \\
Agrestes docuere cavas instare cicutas.\footnote{ Segue uma tradução
setecentista dos versos: ``\textit{Das aves o terníssimo gorgeio/ Os homens
imitar co'a voz tentavam,/ Muito antes que cantando eles soubessem/ Articular
os versos sonorosos,/ Que hoje tanto os ouvidos nos encantam./ O silvo que dos
zéfiros se ouvia/ No oco das canas suscitar"-lhes pôde/ Dos cálamos agrestes a 
lembrança./ Pouco a pouco depois os sons maviosos/ Foi espalhando a cítara,
pulsada/ Por quem com doce voz a par lhe ia/ Nos bosques, selvas, brenhas, que
aos pastores,/ Por mudas solidões, por longos ócios,/ Da harmonia as primeiras
lições deram}.'' [p. 172] [\versal{N.} do org.] }\\[10pt] 				\index{\Zefir}
\end{verse}

O prazer que nos causam todas as artes imitadoras
é a mais segura prova deste princípio. Mas assim 
como o sábio Pintor para mover a compaixão não 						%gloss imitação
representa um quadro alegre e risonho; também o hábil 
Poeta deve escolher para a sua imitação ações 
conducentes ao fim que se propõe. Por isso o Épico, 					\index{\Epico}
que pretende inspirar a admiração e o amor da virtude, 
imita uma ação na qual possam aparecer brilhantes o valor,
a piedade, a constância, a prudência, o amor da Pátria, 
a veneração dos Príncipes, o respeito das Leis e os 
sentimentos da humanidade. O Trágico, que por meio do terror 				\index{\Trag}
e da compaixão deseja purgar o que há de mais violento em  
nossas paixões, escolhe ação onde possa ver"-se o horror 
do crime acompanhado da infâmia, do temor, do remorso, da desesperação 
e do castigo; enquanto o Cômico acha nas ações vulgares 				\index{\Comic}
um dilatado campo à irrisão, com que repreende os vícios.

Qual destas imitações consegue mais depressa o seu fim
é difícil o julgar; sendo tão diferentes os caracteres, 
como as inclinações; mas quase sempre o coração humano,
regido pelas leis do seu amor próprio, é mais fácil em ouvir 
a censura dos vícios, do que o louvor das virtudes alheias.

O poema chamado Heroi"-cômico, porque abraça ao mesmo
tempo uma e outra espécie de poesia, é a imitação 
de uma ação cômica heroicamente tratada. 
Este Poema pareceu monstruoso aos Críticos mais escrupulosos; 
porque se não pode (dizem eles) assinar o seu verdadeiro caráter. 
Isto é mais uma nota pueril, do que bem fundada crítica; 
pois a mistura do heroico e do cômico não envolve a
contradição, que se acha na Tragicomédia, onde o terror 
e o riso mutuamente se destroem. 							\index{\Comic}\index{\Heroic}

Não obsta a autoridade de Platão referida por muitos; 					\index{\Plat}
porque quando este Filósofo, no Diálogo 3 de sua \textit{República}, 			\index{\Trag}
parece dizer que são incompatíveis duas diversas imitações, 
fala expressamente dos Autores Trágicos e Cômicos, que jamais
serão perfeitos em ambas. 								\index{\Comic} \index{\Plat} \index{\Repub} \index{\Trag}

Esta Poesia não foi desconhecida dos Antigos.  Homero daria				\index{\Homer}
mais de um modelo digno da sua mão, se o tempo, que
respeitou a \textit{Batracomiomaquia}, deixasse chegar a nós o seu			\index{\Batr}
\textit{Margites}, de que fala								\index{\Margi}
Aristóteles no cap. 4 da \textit{Poética}, dizendo que este poema tinha com a		\index{\Arist}
Comédia a mesma relação que a \textit{Ilíada} com a Tragédia. O
\textit{Culex}, ou seja de Virgílio, ou de outro qualquer, não contribui 
pouco para confirmar a sua antiguidade.							\index{\Culex} \index{\Margi} \index{\Poet} \index{\Trag} %gloss cômico

Muitos são os poemas heroi"-cômicos modernos. A \textit{Secchia
rapita} de Tassoni é para os Italianos o mesmo que o \textit{Lutrin}	
de Boileau para os Franceses, e o \textit{Hudibraz} de Butler e o
\textit{Rape of the Lock} de Pope para os Ingleses.					\index{\Secch} \index{\Lutri} \index{\Tasson} %gloss, Boileau

Uns sujeitaram o poema heroi"-cômico a todos os preceitos da 
Epopeia e quiseram que só diferisse pelo cômico da ação,				\index{\Comic} 
e misturaram o ridículo e o sublime de tal sorte que servindo um 
de realce a outro, fizeram aparecer novas belezas em ambos os 
gêneros. Outros omitindo ou talvez desprezando algumas regras
abriram novos caminhos à sua engenhosa fantasia e mostraram
disfarçada com inocentes graciosidades a crítica mais 
insinuante, como M.~Gresset no seu \textit{Ververt}.

Não faltou quem tratasse comicamente uma ação heroica; 
mas esta imitação não foi tão bem recebida, ainda que a 
Paródia da \textit{Eneida}, de Scarron, possa servir de modelo. 

É desnecessário trazer à memória a autoridade e o 
sucesso de tão ilustres Poetas para justificar o Poema 
heroi"-cômico, quando não há quem duvide, que ele, 
porque imita, move e deleita: e porque mostra ridículo o 				\index{\Util}
vício, e amável a Virtude, consegue o fim da verdadeira poesia.

\begin{verse}
Omne tulit punctum, qui miscuit utile dulci\footnote{ Quem sabe pois
tecer ação, que instrua, juntamente agrade.}\\[10pt]

\hfill Horat. \textit{Poet}. v.~342 
\end{verse}

\medskip

\begin{verse}
Discit enim citius, meminitque libentius illud,\\[-15pt]
Quod quis deridet, quam quod probat, ac \qb{}veneratur. 
\\[10pt]

\hfill Horat. \textit{Epist}., 1, \versal{ii}. v. 262
\end{verse}

\endgroup

\pagebreak
\paginabranca

\mbox{}\vfill
\thispagestyle{empty}
\noindent Argumento do Canto \versal{i}
\medskip

{\footnotesize\noindent Invocação das Musas -- Dedicatória -- Chegada triunfal do	\index{\Dedic} \index{\Invoc}
Marquês de Pombal a Coimbra -- Memória do período áureo do reino de Portugal		\index{\Marqpombal}
interrompido pela morte de Dom Sebastião -- 
Apresentação da Ignorância, personificação dos		
hábitos do ensino jesuítico na Universidade -- Divulgação da Reforma pombalina pelo rio Mondego --		\index{\Monde}
A Ignorância lamenta seu Império perdido -- A 		
Preguiça e a Ociosidade alegorizadas como colegas
de pensão da Ignorância -- Transfiguração 			
da Ignorância em Tibúrcio, antiquário que vivia em
Coimbra -- Apresentação do herói, Gonçalo, o Desertor das Letras -- Conselhos da
Ignorância -- O letrado frustrado e vendedor de objetos usados vitupera as disciplinas
do estudo, mimetizando a linguagem difícil do método escolástico
-- Desmerece a carreira das letras naqueles
tempos, que já não conferiam distinção e obrigavam a uma longa carreira nas partes
distantes do Império -- Exorta Gonçalo a voltar para Mioselha, rever o tio, fazendo o louvor da
aldeia -- Tibúrcio constitui a companhia de desertores das letras -- Imprecação da Velha
Guiomar, mãe de Narcisa -- Ira de Narcisa, amante de Gonçalo, com a partida
do amante -- Gonçalo a consola com uma bolsa de dinheiro e a desculpa de uma nova herança
-- Tibúrcio, experiente, lembra ao herói as 
despesas da viagem -- Não valendo os seus argumentos e sua encenação, Tibúrcio
usa o braço arrancando"-o de Narcisa -- Briga na pensão -- O povo se ajunta
com paus e pedras -- Narcisa termina com a bolsa, recontando o dinheiro -- Guiomar,
insatisfeita de a filha ter sido rejeitada, planeja fazer prender a Gonçalo -- Rodrigo
o avisa das maldições da velha e o aconselha a fugir -- Partida de Coimbra.}

\chapter{Canto \versal{I}} 

\begingroup
\linenumbers

\begin{verse}
Musas, cantai o Desertor das letras\\
Que, depois dos estragos da Ignorância,\footnote{ Depois de abolidos os velhos estatutos pela criação da nova universidade.} \\		\index{\Ignor}
Por longos e duríssimos trabalhos, \\
Conduziu sempre firme os companheiros, \\
Desde o louro Mondego aos Pátrios montes. \\		\index{\Monde}
Em vão se opõem as luzes da Verdade \\
Ao fim que já na ideia tem proposto \\
E em vão do Tio as iras o ameaçam. \\[10pt]

E tu, que à sombra duma mão benigna,\\
Gênio da Lusitânia, no teu seio\\
De novo alentas as amáveis Artes;\\
Se ao surgir do letargo vergonhoso\\
Não receias pisar da Glória a estrada,\\
Dirige o meu batel, que as velas solta,\\
O porto deixa e rompe os vastos mares,\\
De perigosas Sirtes povoados. %gloss Sirtes

Quais seriam as causas, quais os meios\\
Por que Gonçalo renuncia os livros?\\
Os conselhos e indústrias da Ignorância\\	\index{\Ignor}		\index{\Indus}
O fizeram curvar ao peso enorme\\
De tão difícil e arriscada empresa.\\
E tanto pode a rústica progênie\footnote{ Virg. \AE n., 1.~\versal{I}:
\textit{``Tant\ae ne anmis c\oe lestibus ir\ae !''}. Despréaux no canto \versal{I} do
\textit{Lutrin}: \textit{``Tant de fiel ente"-t-il dans l'âme des dévots!''}.} 
\\[10pt]		\index{\Lutri}

A vós, por quem a Pátria altiva enlaça\\
Entre as penas vermelhas e amarelas\\
Honrosas palmas e sagrados louros,\\		\index{\Lour}
Firme coluna, escudo impenetrável\\
Aos assaltos do Abuso e da Ignorância,\\			\index{\Ignor}
A vós pertence o proteger meus versos.\\
Consenti que eles voem sem receio\\
Vaidosos de levar o vosso nome\\
Aos apartados climas onde chegam\\
Os ecos imortais da Lusa glória. \\[10pt]


Já o invicto Marquês com régia pompa\footnote{ O Ilustríssimo e
Excelentíssimo Senhor Marquês de Pombal entrou em Coimbra como Plenipotenciário		
e Lugar"-tenente de Sua Majestade Fidelíssima para a criação da Universidade em
22 de setembro de 1772.}\\		\index{\Marqpombal}
Da risonha Cidade avista os muros.\\
Já toca a larga ponte em áureo coche.\\
Ali junta a brilhante Infantaria,\\
Ao rouco som de música guerreira,\\
Troveja por espaços; a Justiça,\\
Fecunda mãe da Paz e da Abundância,\\
Vem a seu lado; as Filhas da Memória,\\			\index{\Filmemo}
Digna, imortal coroa lhe oferecem,\\
Prêmio de seus trabalhos; as Ciências\\ 
Tornam com ele aos ares do Mondego,\\		\index{\Monde}
E a Verdade entre júbilos o aclama\\
Restaurador do seu Império antigo.\\	
Brilhante luz, paterna liberdade,\\
Vós, que fostes num dia sepultadas,\\
C'o bravo Rei nos campos de Marrocos,\footnote{ O Senhor Rei D.~Sebastião
ficou em África no ano de 1578, e se perdeu com ele a liberdade
Portuguesa, de donde nasceram as funestas consequências que até agora se fizeram
sentir.}\\
Quando traidoras, ímpias mãos o armaram,\\
Vítima ilustre da ambição alheia,\\
Tornai, tornai a nós. Da régia estirpe\\
Renasce o vingador da antiga afronta:\footnote{ O Sereníssimo Senhor D.~José
Príncipe herdeiro.}\\
Assim o novo Cipião crescia\footnote{ Públio Cornélio Cipião vingou a
morte de seu Pai e Tio destruindo Cartago.}\\		\index{\Cipiao}
Para terror da bárbara Cartago.\\
Possam meus olhos ver o Ismaelita\footnote{Os Mouros são descendentes
de Ismael filho de Agar.}\\	\index{\Ismae}
Nadar em sangue e pálido de susto\\
Fugir da morte e mendigar cadeias;\\
E amontoando Luas sobre alfanges\\ %\footnote{ LUas sobre alfanges.}
Formar degraus ao Trono Lusitano.\\
Dissiparam"-se as trevas horrorosas\\
Que os belos horizontes assombravam\\
E a suspirada luz nos aparece.\\
Tal depois que, raivoso e sibilante,\\
Sobre o carro da Noite o Euro açoita\footnote{ Euro, o vento
vulgarmente chamado l'Este. Boótes, constelação na cauda da Ursa, ou a Guarda.}\\
Os tardios cavalos do Boótes,\footnote{ Juvenal, \textit{Sat}. \versal{V}, v. 23.:
\textit{Frigida circumagunt pigri Sarraça Boot\ae}.}\\
E insulta as terras e revolve os mares,\\
Raia a manhã serena entre douradas\\
E brancas nuvens; ri"-se o Céu e a Terra:\\
O Vento dorme e as Horas vigilantes,\\
Abrem ao claro Sol a azul campanha. \\[10pt]


A soberba Ignorância entanto observa,\\			\index{\Ignor}
E se confunde ao ver o próprio trono\\
Abalar"-se e cair; o seu ruído\\
Redobra os ecos nos opostos vales\\
E o Mondego feliz ao mar undoso\\			\index{\Monde}
Leva alegre a notícia, porque chegue\\
Das suas praias aos confins da Terra.\\
Ela abatida e só não acha abrigo,\\
E desta sorte em seu temor suspira. \\[10pt]


``Verei eu sepultar"-se entre ruínas\\
O meu reino, o meu nome e a minha glória,\\
Depois de ser temida, e respeitada?\\
Pobre resto de míseros vassalos\\
Não há mais que esperar. Já fui rainha:\\
Já fostes venturosos: não soframos\\
As injúrias que o vulgo nos prepara,\\
Injúrias mais cruéis do que a desgraça.\\
Deixemos para sempre estes terríveis\\
Climas de mágoa, susto, horror e estrago.\\
Mostrai"-me algum lugar desconhecido,\\
Onde oculta repouse até que possa\\
Tomar de quem me ofende alta vingança.\\
Mas onde se um Prelado formidável,\footnote{ O Ilustríssimo e
Excelentíssimo Senhor Bispo de Coimbra, Reitor e Reformador da Universidade.}\\
Esse Argos\footnote{ Fingiu a fábula ser Pastor de Tessália, que tinha
cem olhos, a quem Juno deu a guardar Io, filha de Ínaco, Rei dos Argivos.} que
me assusta vigilante\\			\index{\Fabula}
Ao lugar mais remoto estende a vista?\\
Monstros do cego abismo em meu socorro\\
Empenhai o poder do vosso braço;\\
Que se entre os homens me faltar asilo,\\
Ao triste vão dos ásperos rochedos,\\
Onde o Tenaro escuro e cavernoso\footnote{ Promontório de Lacônia,
onde há uma cova profundíssima, que os antigos chamaram a porta do Inferno.
Virg., \textit{Georg.}, liv. \versal{iv}, v. 467: \textit{T\ae narias etiam fauces alta
ostia Ditis}.}\\
Da morada sombria as portas abre,\\
Irei chorar meus dias sem ventura:\\
Irei''\ldots{} Assim falando misturava\\
Gemidos e soluços que sufocam\\
Dentro do peito a voz e umedecia\\
C'o pranto amargo a face descorada.\\
Mas logo, serenando o rosto aflito,\\
Corre por entre sustos e esperanças\\
Ao caro abrigo do fiel Gonçalo.\\
A sonolenta, a pigra Ociosidade\\
Por esta vez deixou de acompanhá"-la:\\
E a lânguida Preguiça forcejando\\
Pôde apenas segui"-la com os olhos. \\[10pt]


Toma a forma dum célebre Antiquário\\
Sebastianista acérrimo, incansável,\\
Libertino com capa de devoto.\\
Tem macilento o rosto, os olhos vivos,\\
Pesado o ventre, o passo vagaroso.\\
Nunca trajou à moda: uma casaca\\
Da cor da noite o veste e traz pendentes\\
Largos canhões do tempo dos Afonsos.\\
Dizem que o tempo da mais bela idade\\
Consagrou às questões do Peripato.\\ 	\index{\Perip}
Já viu passar dez lustros e experiente\\
Sabe enredos urdir e pôr"-se em salvo.\\
Entra por toda a parte e em toda a parte\\
é conhecido o nome de Tibúrcio. \\[10pt]


Gonçalo que foi sempre desejoso\\
Da mais bela instrução, lia e relia\\
Ora os longos acasos de Rosaura,\footnote{ \textit{Carlos} e
\textit{Rosaura}, \textit{Constante Florinda}, e \textit{Carlos Magno} são
romances muito conhecidos.}\\ 
Ora as tristes desgraças de Florinda,\\ 	\index{\Romvulg}
E sempre se detinha com mais gosto\\
Na cova Tristifeia e na passagem\\
Da perigosa ponte de Mantible.\\
Repetia de cor de Albano as queixas\\
Chamando a Damiana injusta, ingrata;\\
Quando Tibúrcio apaixonado e triste\\
Ralhando entrou. ``Que espera tu dos livros?\\
Crês que ainda apareçam grandes homens\\
Por estas invenções com que se apartam\\
Da profunda ciência dos antigos?\\
Morreram as \textit{postilas} e os \textit{Cadernos}:\\
Caiu de todo a \textit{Ponte}\footnote{ O método escolástico. Quem
conheceu a lógica peripatética, não ignora qual seja esta ponte.} e se acabaram\\		\index{\Perip}
As \textit{distinções} que tudo defendiam,\\
E o \textit{ergo}, que fará saudade a muitos!\\
Noutro tempo dos Sábios era a língua\\
\textit{Forma}, e mais \textit{forma}: tudo enfim se acaba,\\
Ou se muda em pior. Que alegres dias\\
Não foram os de Maio quando a estrada\\
Se enchia de Arrieiros e Estudantes!\\
Ó tempo alegre e bem"-aventurado!\\
Que fácil era então o azul Capelo,\\
Adornado de franjas e alamares,\\
O rico anel e flutuante borla,\\
Honra e fortuna que chegava a todos!\\
Hoje é grande a carreira e serão raros\\
Os que se atrevam a tocar a meta.\\
Ah Gonçalo! Gonçalo! que mais vale\\
Tirar co'a própria mão no fértil Souto\\
Moles castanhas do espinhoso ouriço!\\
Quanto é doce ao voltar da Primavera\\
O saboroso mel no louro favo!\\
Ó alegre e famosa Mioselha,\\
Fértil em queijos, fértil em tramoços!\\
Só lá de romaria em romaria\\
Podes viver feliz e descansado.\\ %\footnote{nOTA VOCABULAR + PROVÍNCIA.}
Quem te obriga a levar sobre os teus ombros\\
O desmedido peso, que te espera?\\
Não tenhas do bom Tio algum receio:\\
Comigo irás, bem sabes quanto posso.\\
Se te envergonhas de ser só, descansa;\\
Fiel parente, amigo inseparável,\\
Eu farei que, abraçando o mesmo exemplo,\\
Muitos se apressem a seguir teus passos.'' \\[10pt]


Assim falava, quando um ar de riso\\
Apareceu no rosto de Gonçalo.\\
Tudo o que se deseja se acredita;\\
Nem há quem o seu gosto desaprove.\\
Ele, porque já traz no pensamento\\
Poupar"-se dos estudos à fadiga,\\
Não vacila na escolha e se aproveita\\
Da feliz ocasião que lhe assegura\\
O meditado fim de seus desejos. \\[10pt]


Convocam"-se os heróis e deliberam\\
Em pleno consistório onde Gonçalo\\
Silêncio pede e assim a todos fala.\\
``Heróis, a quem uma alma livre anima,\\
Que desprezando as Artes e as Ciências,\\
Ides buscar da Pátria no regaço,\\
Longe da sujeição e da fadiga,\\
Doce descanso, amável liberdade:\\
Se algum de vós (o que eu não creio) ainda\\
Tem na alma o vão desejo dos estudos,\\
Levante o dedo ao alto.'' Uns para os outros\\ 
Olharam de repente e de repente\\
Rouco e brando sussurro ao ar se espalha:\\
Qual nos bosques de Tempe,\footnote{ Lugar de Tessália célebre pela
amenidade dos seus bosques.} ou nas \qb{}frondosas\\
Margens que banham o plácido Mondego,\\			\index{\Monde}
Costuma ouvir"-se o Zéfiro suave,\\		\index{\Zefir}
Quando meneia os álamos sombrios.\\
Nenhum alçou a mão e a Ignorância\\			\index{\Ignor}
Pareceu consolar"-se imaginando\\
Sonhadas glórias de futuro império. \\[10pt]


Dispõe"-se a companhia e se aparelha\\
Para partir antes que o Sol desate\\
Sobre a Terra orvalhada as tranças d'ouro.\\
Tibúrcio tudo apronta. Mas Janeiro\\ 		\index{\Jane}
Loquaz, traidor, doméstico inimigo\\
Voa de casa em casa publicando\\
Da forte esquadra a próxima partida. \\[10pt]


Guiomar, velha que há muito que insensível\\
às delícias do amor, aferrolhando\\
Emagrece nos míseros cuidados\\
Da faminta ambição e é na Cidade\\
Uma ave de rapina que entre as unhas\\
Leva tudo o que encontra aos ermos cumes\\
Da escalvada montanha onde a festejam\\
Co'a boca aberta os ávidos filhinhos:\\
Triste agora e infeliz ouve e se assusta\\
Das notícias cruéis que o Moço espalha.\\
``Ó Ama desgraçada! Ó dia infausto!\\ 
Agora que esperava mais sossego\\
Principiam de novo os meus trabalhos!''\\
Estas e outras palavras arrancava\\
Do peito descontente, enquanto a Filha\\
Amorosa e sagaz estuda os meios\\
Com que possa deter o ingrato amante.\\ 	\index{\Tipos}
Faz ajuntar de partes mil à pressa\\
Cordões e anéis e a pedra reluzente\\
Que os olhos desafia: os seus cabelos,\\
Que desconhecem o toucado, empasta\\
Co'a cheirosa pomada: a Mãe se lembra\\
Da própria mocidade e lhe vai pondo\\
Com a trêmula mão vermelhas fitas.\\
Simples noiva da aldeia que ao mover"-se\\
Teme perder o desusado adorno\\
Nunca formou mais vagarosa os passos.\\
Narcisa chega entre raivosa e triste,\\
E fingido"-se esquecer"-se da mantilha\\
Para mostrar"-se irada, desta sorte\\
Em alta voz lhe fala. ``Será certo\\
Que pretendes fugir, e que me deixas\\
Infeliz, enganada, e descontente?\\
Assim faltas cruel, pérfido, ingrato\\
Dum longo amor aos ternos juramentos?\\ 
Não disseste mil vezes\ldots{} mas que importa\\
Que os meus males recorde? enfim, perjuro,\\
As tuas vãs promessas me enganaram.\\
Justiça pedirei ao Céu e ao Mundo:\\ %footnote mulher ensandecida
O mundo tem prisões, o Céu tem raios.'' \\[10pt]


Falava e o Herói, que arrasta ainda\\
D'um incômodo amor os duros ferros,\\
Parece vacilar quando Tibúrcio\\
Dá conselhos a um, a outro ameaça,\\
Pondo irados os olhos em Narcisa.\\
Diz"-lhe que em vão suspira, que em vão chora\\
E que sempre tiveram as mulheres,\\
Para enganar aos míseros amantes,\\
As lágrimas no rosto, o riso na alma.\\
Gonçalo, então, que o seu dever conhece,\\
Dá provas de valor e de prudência.\\
``Ouve, Narcisa bela,'' (lhe dizia)\\
``Serena a tua dor e os teus queixumes;\\
O teu pranto me move, injusto pranto,\\
Que o meu constante amor de ingrato acusa.\\
Sossega: a nova herança dum morgado\\ 		\index{\Morg}
É quem me chama, a ausência será breve.\\
Tempo depois virá que em doces laços\\
Eterno amor as nossas almas prenda,\\
E então farás tibornas\footnote{ Comida feita de pão e azeite novo.}
e magustos.\footnote{ Castanhas assadas e vinho.}\\		\index{\Tiborn}
Nem sempre cobre o mar a longa praia:\\
Nem sempre o vento com furor raivoso\\
Do robusto pinheiro o tronco açoita.'' \\[10pt]


Acaba de falar e lhe oferece\\
A leve bolsa, que Narcisa aceita,\\
Como penhor sincero de amizade,\\
Bolsa que deve ser na dura ausência\\
Breve consolação de tristes mágoas. \\[10pt]


O experto Amigo, que se mostra em tudo\\
Companheiro fiel, c'os olhos tristes,\\
Pondera os longos e ásperos caminhos:\\
Lembra funestas noites de estalagem,\\
E adverte, em vão, que ao menos por cautela\\
Deve fazer"-lhe a bolsa companhia.\\
Deixando enfim inúteis argumentos\\
Remete a decisão ao próprio braço.\\
Não se esquecem das unhas, nem dos dentes,\\
Armas que a todos deu a Natureza.\\
Ouvem"-se pela casa em som confuso\\
As troncadas injúrias e os queixumes.\\
Assim dois cães se o hóspede imprudente\\
Lança da mesa os ossos esburgados\\
Prontos avançam: duma e doutra parte,\\
Se vê firme o valor; mordem"-se e rosnam,\\ 
Mas não cessa a contenda. Amigo e amante,\\
Que farias, Gonçalo, em tanto aperto?\\
Concorre a plebe e o férvido tumulto\\
Vai pelas negras fúrias conduzido\\ 
Despertando nos peitos a desordem.\\
Ninguém sabe por quê, mas todos gritam.\\
Já voam as cadeiras pelos ares:\\
Pedras e paus de longe se arremessam.\\
E se a cândida Paz com rosto alegre\\
Serenou as desgraças deste dia,\\
Os teus dentes, intrépido Gonçalo,\\
Viste voar em negro sangue envoltos. \\[10pt]


Torna alegre Narcisa, e cinco vezes\\
Abriu a bolsa e numerou a prata.\\
Fez diversas porções que num momento\\
Tornou a confundir: não doutra sorte\\
O menino impaciente e cobiçoso,\\
Quando alcança o que há muito lhe negavam,\\
Repara, volta, move, ajunta, espalha,\\
E neste giro o seu prazer sustenta. \\[10pt] 


Entanto, a mãe que já por experiência\\
Os enganos conhece mais ocultos\\
Busca novos pretextos de vingança,\\
Fingindo torpes e horrorosos crimes.\\
E espera ouvir gemer em poucas horas\\
O mancebo infeliz em prisão dura.\\
Mas Rodrigo que ouviu o rumor vago,\\
À pressa chega, e desta sorte fala. \\[10pt]


``Que desgraças te esperam! foge, foge,\\
Gonçalo, enquanto há tempo: gente armada\\
Vem logo contra ti. Guiomar convoca\\
Todo o poder do mundo: um só momento\\
Não percas, caro amigo; os companheiros\\
Com alvoroço esperam. Ah, deixemos,\\
Deixemos duma vez estas paredes,\\
Onde c'o próprio sangue escrita deixas\\
De teu trágico amor a breve história.\\		
É já outro o Mondego: a liberdade\\			\index{\Monde}
Destes campos fugiu e só ficaram\\
A dura sujeição e o triste estudo.\\
Enfim hei de apartar"-me desta sorte?\\ 
Ó sempre tristes, sempre amargos sejam\\
Os teus últimos dias, velha infame.''\\
Gonçalo sim chorando, monta e parte. \\[10pt]
\end{verse}
\pagebreak 

\endgroup

\thispagestyle{empty}
\movetoevenpage
\mbox{}\vfill
\thispagestyle{empty}
\noindent Argumento do Canto \versal{ii}
\medskip

{\footnotesize\noindent Catálogo e descrição dos tipos que formam a companhia de desertores, conforme os respectivos vícios:  \index{\Tipos}
Gonçalo, o mais destro, é um jovem que poderia ter sido promissor, mas
não aguentou os esforços das letras; Tibúrcio, a Ignorância, leva um lenço pardo amarrado
a um galho, como a bandeira da companhia; Cosme é enamorado; Rodrigo é
rústico; Bertoldo se diz fidalgo de antiga linhagem improvável; Gaspar
é iracundo; Alberto, o alegre, em Coimbra aplicou"-se às festas --
Chegada à estalagem -- Comem, bebem, brindam e brigam sobre mesas de tosco pinho --
Rodrigo brinda à vitória da viagem, imitando convenções da poesia de banquete; Tibúrcio diz palavras
lascivas para Rodrigo, também conforme foi costume em banquetes -- Fúria de Rodrigo -- Briga na estalagem --
Exortação do velho Ambrósio: pergunta"-lhes se aquilo era aprendido com as letras e os convoca à moderação;
narra o próprio exemplo demonstrando na sua miséria presente o efeito dos
seus vícios de juventude; exorta os jovens a retornarem para os estudos,  do contrário,
os amaldiçoa -- Gaspar, ofendido, ataca o velho -- Gaspar e Gonçalo armados
atacam a multidão que invadira o estabelecimento aos gritos do velho --
Levante geral contra os desertores -- Fuga dos companheiros através do mato espesso.}

\chapter{Canto \versal{ii}}
%\setlinenum{0}

\begingroup
\linenumbers

\begin{verse}

Com largo passo longe do Mondego,\\			\index{\Monde}
Alegre a forte gente caminhava.\\
Gonçalo excede a todos na estatura,\\
Na força, no valor e na destreza.\\
Sobre um magro jumento se escarrancha\\
Tibúrcio, e já dum ramo de salgueiro\\
Desata ao Norte fresco que assobia\\
Por vistoso estandarte um lenço pardo.\\ 
Cosme, infeliz e sempre namorado\\
Sem ser correspondido, vai saudoso,\\
Ama e não sabe a quem: vive penando\\
E se consola só porque imagina\\
Que tem de conseguir melhor ventura.\\ 
Rodrigo, que de todos desconfia,\\
é de índole grosseira e gênio bruto,\\
Não conhece os perigos, nem os teme:\\
Melancólico sempre, vai por gosto\\			\index{\Melanc}
Viver na choça aonde foi criado,\\ 		\index{\Choca}
Qual o Tatu, que o destro Americano\footnote{ Lin. \textit{Sys. nat.}, \textit{Zool.}, edic. 10, t.~\versal{I}, p. 50. \textit{Dasypus}.}\\
Vivo prendeu e em vão depois se cansa\\
Por fazê"-lo doméstico, que sempre\\
Temeroso nas conchas se recolhe\\
E parece fugir à luz do dia.\\ 
Também vinha Bertoldo e traz consigo\\
Carunchosos papéis por onde afirma\\
Vir do sétimo Rei dos Longobardos.\footnote{ Povos de Escandinávia e Pomerânia, que se apoderaram 
da parte da Gália Cisalpina em 568.}\\		\index{\Galia}	\index{\Longob}
Grita contra as riquezas, a Fortuna\\		\index{\Fortu}
Segundo o que ele diz não muda o sangue:\\
Pisa com força o chão e empavesado\\			\index{\Empav}
De ações que ele não pode chamar suas\\
Aos outros trata com feroz desprezo.\\ 
Iracundo Gaspar, que te enfureces\\
No jogo e quando perdes não duvidas\\
Meter a mão à ferrugenta espada,\\
Tu não ficaste: as noites sobre os livros\\
Não queres suportar, porque não temes\\
Da já viúva mãe as frouxas iras.\\
Nem tu, Alberto, alegre e desejado\\
Nas vistosas funções das romarias,\\
Que és vivo, pronto e ágil, e nos bailes\\
Tens fama de engraçado e garganteias\\
Co'a viola na mão trocando as pernas.\\ 
Os que aprendem o nome dos autores,\\
Os que leem só o prólogo dos livros,\\
E aqueles cujo sono não perturba\\
O côncavo metal que as horas conta,\\
Seguiram as bandeiras da ignorância\\
Nos incríveis trabalhos desta empresa. \\[10pt] 


O Sol já sobre os campos de Anfitrite\\		\index{\Anfit}
Inclina o carro e as nuvens carregadas\\	\index{\Carro}
Importunos chuveiros ameaçam\\
Quando a velha estalagem os recebe. \\[10pt]


Mesa de tosco pinho se povoa\\
De negras azeitonas e salgado\\
Queijo que estima a gente que mais bebe.\\
De um lado e de outro lado se levantam\\
Pichéis e copos em que o vinho abunda.\\
Corriam para aqui desafiados\\
Rodrigo, o triste, e o glutão Tibúrcio.\\
Este instante fatal é que decide\\
Da dúbia sorte dos heróis, cobrindo\\
Um de eterna vergonha, outro de glória. \\[10pt]


A feia Noite que aborrece as luzes,\\
Desce dos altos montes com mais pressa\\
Por ver este combate e afugentada\\
Pela sombria luz de uma candeia\\
De longe observa o novo desafio.\\
Um e outro, ocupando as mãos, e a boca\\
Avidamente a devorar começa:\\ 
Assim esse animal grosseiro e pingue,\\
Que de alpestres bolotas se sustenta,\\
À pressa come e tendo uma nos dentes,\\
Noutra tem o desejo e noutra a vista.\\
Rodrigo, quase certo da vitória,\\
Co'as mãos ambas levanta um grande copo,\\
Copo digno de Alcides, e à saúde\\ 			\index{\Alcid}	%\footnote{Hércules}
De todos os famosos Desertores\\
De uma vez esgotou: então Tibúrcio,\\
Cheio de nobre ardor, fechando os olhos\\ 
Toma um largo pichel e assim lhe fala. \\[10pt]	\index{\Pichel}


``Vasilha da minha alma, tu que guardas\\
A alegria dos homens no teu seio,\\
E tu, filho da cepa generoso,\\		\index{\Cepa}
Se estimas e recebes os meus votos,\\
Derrama sobre mim os teus encantos.''\\
Já tinha dito muito e enquanto bebe\\
Voa a cega Discórdia que se nutre\\
De sangue e de vingança e sobre os copos\\
Três vezes sacudiu as negras asas.\\
Viam"-se já, nos lívidos semblantes,\\
A raiva sanguinosa, a má tristeza.\\
A Noite, a quem o Acaso favorece,\\		\index{\Acaso}
Estende a fusca mão e a luz abafa.\\
Veloz passa o furor de peito em peito,\\
Perturba os corações e inspira o ódio. \\[10pt]


Só tu, Gonçalo, descrever puderas\\
Os terríveis estragos desta noite,\\
Tu, que posto debaixo duma banca\\
(Por manchar as mãos no sangue amigo),\\
Sentiste pela casa e pelos ares\\
Rolar os pratos e tinir os copos.\\
Range os dentes Gaspar e pelo escuro\\
Não acerta co'a espada, nem co'a porta:\\
Quando Ambrósio, que tinha envelhecido\\
Da Estalagem na mísera oficina,\\
Co'a candeia na mão assim falava.\\
``É crível, que entre vós jamais se encontre\\
Um gênio dócil, sério e moderado?\\
Isto deveis às letras? respondei"-me,\\
Ou insultai também os meus cabelos\\
Da triste e longa idade embranquecidos.\\
Julgais acaso que o saber se infunde\\
Deixando o nosso nome assinalado\\
Pelos muros e portas na Estalagem?\\
Ó néscia mocidade! é necessário\\
Muito tempo sofrer, gastando a vista\\
Na contínua lição e sobre os livros\\
Passar do frio Inverno as longas noites.\\
E quando já tivésseis conseguido\\
De tão bela carreira os dignos prêmios,\\
Muito pouco sabeis, se inda vos falta\\
Essa grande Arte de viver no mundo,\\
Essa, que em todo o estado nos ensina\\
A ter moderação, honra e prudência.\\
Eu também já na flor da mocidade\\
Varri co'a minha capa o pó da sala:\\
Eu também fui \textit{rancho da carqueja},\footnote{ Esta companhia de Estudantes cometeu muitos crimes e foi dispersa e castigada.}\\
Digno de fama e digno de castigo.\\
Era então como vós. Jamais os livros\\
Me deveram cuidado e me alegrava\\
Das noturnas empresas, dos distúrbios:\\
Os dias se passavam quase inteiros\\
Nos jogos, nos passeios, nas intrigas,\\
Que fomentam os ódios e as vinganças.\\
Por isso estou no seio da miséria,\\
Por isso arrasto uma infeliz velhice,\\
Sem honra, sem proveito, sem abrigo.\\
Tempo feliz da alegre mocidade!\\
Hoje encurvado sobre a sepultura,\\
Eu choro em vão de vos haver perdido!''\\
Assim suspira e geme e continua.\\
``Conservai, sempre firme, na memória,\\
De um velho desgraçado o triste exemplo,\\
E aprendei a ser bons, que a vossa idade\\
As indignas ações não justifica.\\
Mas se vós desprezais os meus conselhos,\\
Nunca gozeis o prêmio dos estudos:\\
Aflições e trabalhos vos oprimam,\\
Enquanto o mar das Índias vos espera.'' \\[10pt] %\footnote{mar das índias e moral da história}


Então Gaspar, tomando o caso em brio,\\
Aceso de ira com valor responde,\\
Traça o capote e tira pela espada.\\
O velho grita e foge: às suas vozes,\\
De rústico um povo se enfurece,\\
E toma as armas e bradando avança.\\
Qual nos imensos e profundos mares\\
O voraz Tubarão entre o cardume\\
De argentadas Sardinhas; elas fogem,\\
Deixam o campo, e nada lhe resiste:\\
Assim Gonçalo, a quem já todos temem,\\
Faz espalhar a turba que o rodeia,\\
E só deixa a quem foge de encontrá"-lo. \\[10pt]


Gaspar, que o rosto nunca viu ao medo,\\
A todos desafia e não perdoa\\
De uma oliveira ao carcomido tronco,\\
Que ele julga broquel impenetrável,\\
Vendo estalar da sua espada a folha. \\[10pt]


Da noite a densa névoa favorece.\\
Receosos de nova tempestade,\\
Salvam as vidas os Heróis fugindo\\
Por entre o mato espesso. Ouvem ao longe\\
Da vingativa plebe a voz irada.\\
À clara luz das pinhas resinosas\footnote{ Costumam os rústicos acender de noite as pinhas.}\\
Aparecem as foices e aparecem\\
Chuços, cacheiras, trancas e machados.\\
Levanta"-se o clamor e a crua guerra,\\
Que o sangue dos mortais derrama e bebe,\\
Gira por toda a parte e move as armas.\\
Entanto a valerosa companhia,\\
Amparada da sombra feia e triste,\\
Voa por longo espaço sobre as asas\\
Do pálido terror. Não de outra sorte\\
Rasos xavecos de piratas Mouros,\\		\index{\Xavec}
Quando aos ecos do bronze fulminante\\
Vêem tremular as vencedoras Quinas\\
Sobre a possante Nau, que oprime os mares,\\
Fogem à vela e remo, e não descansam\\
Sem ter beijado as Argelinas praias.\\
Ouvem"-se  então diversos sentimentos.\\
Chora Gaspar de se não ter vingado,\\
E ainda aqui colérico assevera\\		\index{\Coler}
Que a não faltar"-lhe a espada não fugira.\\
Espada, que ao romper as linhas d'Elvas,\footnote{ Gloriosa batalha,
que ganhou D.~Antonio Luiz de Menezes, Excelentíssimo Conde de Castanhede, no
ano de 1658. A este herói também se deve o triunfo de Montes Claros.}\\		\index{\Elvas}
Se dos velhos Avós não mente a história,\\
Abriu de meio a meio um Castelhano. \\[10pt] 


Teme Bertoldo que o encontre o Povo\\ 
E no meio daquela escuridade\\
Chega"-se aos mais com pânico receio.\\
Cosme, quase insensível aos perigos\\
E aos amargos momentos desta noite,\\
Aproveita o silêncio, o sítio, a hora,\\
Para chorar saudades sem motivo.\\
Só Gonçalo pensava cuidadoso\\
Em salvar os aflitos companheiros:\\
Assim o astuto assolador de Troia,\footnote{ Ulisses, cujos companheiros foram transformados por Circe. Homr. \textit{Odiss.}, \versal{I}, \versal{X}, v.~238.}\\ 	\index{\Ulis}
Quando os Gregos heróis ouviu cerdosos\\
Grunhir nos bosques da encantada Circe,\\
Ou quando viu a detestável mesa\footnote{ Polifemo devorou dois Gregos
em presença de Ulisses. \textit{Odyss.}, \versal{i}. \versal{ix}, v.289.}\\	\index{\Ulis}
Na vasta cova do Ciclope horrendo.\\
``Onde estarás fiel e caro amigo?''\\
(Dizia o condutor da estulta gente)\\
``Se tu me faltas como irei meter"-me\\
Nas mãos dum Tio rústico, inflexível?\\
Voltarei? mas ó Céus! quem me assegura\\
Que essa velha cruel, nefanda harpia\\
Não tenha urdido algum funesto engano?\\
E se o Povo indignado e ofendido\\
Nos vem seguindo, e ao surgir da Aurora\\
Neste inculto deserto\ldots{}  Céu piedoso,\\
Longe, longe de nós tão graves danos.'' \\[10pt]


Gonçalo assim falava e vigilante\\
Tristes horas passou até que o dia\\
Apareceu entre rosadas nuvens\\
Sobre as altas montanhas do horizonte. \\[10pt]

\end{verse}
\pagebreak

\endgroup

\mbox{}\vfill
\thispagestyle{empty}
\noindent Argumento do Canto \versal{iii}
\medskip


{\footnotesize\noindent A fama leva no vento os louvores do rei de Portugal pela restauração dos
estatutos em Coimbra -- Louvor do edifício da Universidade -- Semelhante
à fama, a infâmia leva as incógnitas notícias, como em bandos de papagaios
-- Espalha"-se assim a murmuração indignada contra a companhia de Gonçalo --
A multidão ainda busca o grupo de estudantes -- Descrição do \textit{locus
horrendus} em que dorme Rufino, que vive nas brenhas daquela serra por causa
de um amor não correspondido -- Vendo que os vassalos de seu novo reino
estavam perdidos nas montanhas e preocupada com o destino deles, que acabariam
presos, a Ignorância aparece em sonho ao triste Rufino na forma de sua amada
Doroteia, filha do velho Amaro, guardador da cadeia local -- No sonho,
Doroteia faz promessas de amor a Rufino e o aconselha a mudar"-se junto com
a companhia de desertores, para guiá"-los até Mioselha -- Iludido, Rufino
acorda confiante quanto à predição da Ignorância -- Levados pelo acaso,
os desertores encontram Rufino tão logo ele acorda -- Gonçalo solenemente
pede a orientação ao jovem habitador daquelas brenhas -- Rufino se apresenta,
refere o próprio sofrimento e o presságio que o animava a viajar -- Abandonando
a caverna onde queria enterrar o amor não correspondido, assume o serviço de
guiar os desertores --  Seguem viagem -- Quando Gonçalo pensa que o bom sucesso 
da viagem estava garantido, o grupo é repentinamente cercado pelo povo armado
de foices e outras armas de serviços mecânicos -- Descrição de combate entre
a multidão furiosa e a companhia de desertores -- Na briga, diante de um jovem
conhecido como gigante Ferrabrás, Gonçalo tropeça e cai -- Gaspar tenta em vão
salvar o grupo mas perde a espada -- Os desertores das letras são levados presos.} \enlargethispage{\baselineskip}


\chapter{Canto \versal{iii}}
%\setlinenum{0}

\begingroup
\linenumbers

\begin{verse}

A Fama sobre o carro transparente,\\		\index{\Fama}
Que arrastam através do espaço imenso\\
O sonoro Aquilon e o veloz Austro,\footnote{ Aquilon vento setentrional, e Austro meridional.}\\		\index{\Aquil}
Cantava o caro nome, a imortal glória\\
Do Augusto Pai do Povo. Entre milhares\\ \index{\Aug} 	\index{\Paida}
De ações dignas dum Rei, Europa admira\\
O soberbo Edifício levantado\\
Que o saudoso Mondego abraça e adora:\\ 		\index{\Monde}
Edifício que o tempo doravante\\
Vê de longe, rodeia, teme e foge:\\
Que sustenta em firmíssimas colunas\\
De ciência imortal o Régio Trono. \\[10pt]


Se longe da feroz barbaridade\\
Os olhos abre a forte Lusitânia,\\
Grande Rei, esta ação é toda vossa. \\[10pt]


Entanto a Fama heroica vão seguindo\\		\index{\Fama}
As velozes e incógnitas notícias,\\
Que trazem e que levam os sucessos\\
De país em país, de clima em clima.\\
Elas voam em turba, enchendo os ares\\
Dos ecos dissonantes a que atendem\\
Crédulas velhas e homens ociosos.\\
Qual no fértil Sertão da Aiuruoca\footnote{ Aiuruoca na língua dos índios		
soa o mesmo que \textit{casa de papagaios}. Este vasto país nas minas do Rio das Mortes
é abundantíssimo destas aves.}\\		\index{\Aiur}
Vaga nuvem de verdes Papagaios,\\
Que encobre a luz do Sol e que em seus gritos\\
É semelhante a um povo amotinado:\\
Assim vão as Notícias e estas vozes\\
Pelo campo entre os rústicos semeiam. \\[10pt]


Gente inexperta, alegre e sem cuidados,\\
Fero esquadrão que os vossos campos tala,\\
Vem destruindo as terras e os lugares.\\
O povo indócil, cego e receoso,\\
Que as funestas palavras acredita,\\
Toma os caminhos e os oiteiros cobre.\\
Por onde irás, intrépido Gonçalo,\\
Que escapes ao furor da plebe armada?\\
Mas já os desgraçados companheiros\\
Desciam por incógnitas veredas\\
Para o fundo dum vale cavernoso,\\
Que o Zêrere veloz lavando insulta\footnote{ Este pequeno e arrebatado
rio perde o nome no Tejo, e faz a maior parte do seu curso por penhascos inacessíveis.}\\
Co'as turvas águas do gelado Inverno.\\
Há um lugar nunca dos homens visto,\\
Na raiz de dois montes sobranceiros.\\
Suam as frias e musgosas pedras,\\
Que dos altos cabeços penduradas\\
Ameaçam ruína há tempo imenso.\\
Jamais do Cão feroz o ardor maligno\footnote{ A constelação chamada Canícula.}\\		\index{\Canic}
Desfez a neve eterna destas grutas.\\
Árvores, que se firmam sobre a rocha,\\
Famintas de sustento, à terra enviam\\
As tortas e longuíssimas raízes.\\
Pendentes caracóis co'a frágil concha\\
Adornam as abóbadas sombrias.\\
Neste lugar se esconde temerosa\\
A Noite envolta em longo e negro manto\\
Ao ver do Sol os lúcidos cavalos,\\
Fúnebre, eterno abrigo aos tristes mochos,\\		\index{\Mocho}
Às velhas, às fatídicas corujas,\\
Que com medonha voz gemendo aumentam\\
O rouco som do rio acantilado. \\[10pt]


Rufino por seu mal sempre extremoso\\
E sempre escarnecido, suspirando\\
Aqui se entrega ao pálido ciúme,\\
De um puro amor ingrata recompensa.\\
Contam que nestas hórridas cavernas\\
De míseras angústias rodeado,\\
Vinha exalar os últimos suspiros\\
Queixando"-se do Amor e da Fortuna.\\		\index{\Amor} \index{\Fortu}
Entre os braços do sono repousava,\\
Este infeliz já de chorar cansado;\\
Quando a inquieta Ignorância, que se aflige\\		\index{\Ignor}
De ver nestas montanhas escabrosas\\
Os tímidos amigos, em que funda\\
De novo império a única esperança,\\
Por que Rufino os acompanhe e guie\\
À pingue e suspirada Mioselha,\\
Que é de tantos heróis Pátria famosa,\\
Finge o rosto da bela Doroteia,\\
Doroteia a mais nova, a mais humana,\\
De quantas filhas teve o velho Amaro.\\
Ela a roca na cinta, as mãos no fuso,\\
Em sonhos lhe aparece e mais corada\\
Que a rosa na manhã da Primavera\\
A falar principia. ``Se até agora\\
Ingrata me mostrei a teus amores,\\
Se inconstante e perjura me chamaste,\\
Dá"-me nomes mais doces e ouve atento\\
De uma alma amante a confissão sincera.\\
Sempre te amei e espero ver unidos\\
Os nossos corações em fortes laços\\
Do casto amor que o Céu não desaprova.\\
Mas eu sem nada mais, que a lã, que fio,\\
Tu rico só de afetos e palavras,\\
Onde iremos que a sórdida miséria\\
Não seja em nossos males companheira?\\
Vai"-te e longe de mim segue a ventura,\\
Que firme te hei de ser em toda a idade.\\
Do velho Afonso o triste e pobre filho,\\
Pela dura madrasta afugentado,\\
Também deixou a suspirada Pátria,\\
E veio em poucos anos o mais rico\\
Dos bens imensos que o Brasil encerra.\\
Vês tu quanto cresceu que não cabendo\\
No paterno casal, ergue as paredes,\\
Até chegar ao Céu que testemunha\\
A ditosa união com que ele paga\\
O firme amor da venturosa Ulina?\\
Vai pois, Rufino meu, que muitas vezes\\
Muda"-se a terra e muda"-se a Fortuna.'' \\[10pt] 		\index{\Fortu}


Assim falando os braços lhe oferece.\\
Ó que instante feliz, se Amor perverso,\\		\index{\Amor}
Dos últimos favores sempre avaro,\\
Não firmasse esta sombra de ventura\\
Sobre as asas de um sonho lisonjeiro!\\
Desperta o triste e desgostoso amante,\\
E não duvida que a pressaga imagem\\
Noutro lugar tesouros lhe promete.\\
Futuros bens na ideia se apresentam,\\
E ele crê possuí"-los. Ó dos homens\\
Contínuo delirar sem fundamento!\\
Que bela e fácil se nos pinta a posse\\
Dum incógnito bem, que desejamos! \\[10pt]


Já se ajuntava o esquadrão famoso\\
Pela mesma Ignorância conduzido,\\			\index{\Ignor}
E Gonçalo primeiro assim falando,\\
Os mais em roda todos escutavam. \\[10pt]


``Benigno habitador de incultas brenhas,\\ 
Se um desgraçado errante e peregrino\\
Dentro em tua alma a compaixão desperta,\\
Os meus passos dirige, antes que a fome\\
Com ímpia mão nos deixe frio pasto\\
Às bravas feras, às famintas aves.'' \\[10pt] 


Falava ainda: alguns estremeceram,\\
Outros amargo pranto derramaram.\\
Da boca de Rufino todos pendem.\\
Ele os lânguidos olhos levantando\\
Já do longo chorar enfraquecidos,\\
Estas vozes soltou do rouco peito.\\         
``Que Fortuna cruel, maligna, incerta\\			\index{\Fortu}
Vos trouxe a penetrar o intacto abrigo\\
Destes lugares ermos e escabrosos?\\
Vós em mim achareis amigo e guia:\\
Que pode dar alguma vez socorro\\
Um desgraçado a outro desgraçado.\\
Duros casos de amor me conduziram\\
A acabar nesta gruta os tristes dias;\\
Mas hoje volto por feliz presságio\\
A tentar noutra parte a desventura.'' \\[10pt]


Acaba de falar movendo os passos\\
Pelo torcido vão das nuas pedras.\\
Todos o seguem com trabalho imenso. \\[10pt]


Depois que largo tempo caminharam\\
Por ásperas montanhas, aparecem\\
Ao longe a estrada e o lugar vizinho.\\
Qual a nau sofredora das tormentas,\\
Que, depois de tocar o porto amigo,\\
Sente fugir"-lhe as arenosas praias,\\
E dos hórridos ventos açoitada\\
Volta a lutar c'o pélago profundo:\\
Assim Gonçalo, quando ver espera\\
Tranquilo fim de míseros trabalhos,\\
O povo o cerca e dos confusos gritos\\
As montanhas ao longe retumbaram.\\
Vós, ó Musas, dizei como a Discórdia\\
Com o negro tição que acende os peitos,\\
Mostra o rosto de sangue e pó coberto,\\
Seguindo os passos do homicida Marte.\\		\index{\Marte}
Aqui não aparecem refulgentes\\
Escudos de aço e bronze triplicado,\\
Não assombram a testa dos guerreiros\\
Flutuantes penachos que ameaçam,\\
Como tu viste, ó Troia, ante os teus muros;\\
Mas o valor intrépido aparece\\
A peito descoberto. O povo armado\\
De choupas, longos paus e curvas foices,\\
É semelhante a um bosque de pinheiros,\\
Que o fogo devorou, deixando nuas\\
As elevadas pontas. Animoso\\ 
Dispõe Gonçalo a forma de batalha\\
Posto na frente: à sua voz a um tempo\\
Todos avançam, todos se aproveitam\\
Das perigosas e terríveis armas,\\
Que o terreno oferece em larga cópia.\\
Voa a cega Desordem e aparece\\
No meio do combate. Por um lado\\
Gaspar se opõe arremessando pedras\\
Com força tal que atroam os ouvidos.\\
Gonçalo doutra parte invicto e forte\\
Abre com ferro agudo amplo caminho.\\
Já pendia a balança da vitória\\
Contra a tímida gente que se espalha;\\
Quando chega atrevido Brás, o forte.\\
(Gigante Ferrabrás lhe chama o povo\\
Pela enorme estatura e força incrível)\\
Ergue a pesada maça sem trabalho,\\
Qual nos montes de Lerne o fero Alcides:\footnote{ Lerne, lago de Acaia, onde Hércules matou a Hidra.}\\	\index{\Alcid}
Gonçalo evita a morte com destreza.\\
Ele renova os formidáveis golpes;\\
Mas o irado mancebo ao desviar"-se\\
Tropeça e cai. Neste arriscado instante\\
Serias morto, intrépido Gonçalo,\\
Se Gaspar com um rochedo áspero e rombo\\
Não atalhasse do inimigo a fúria,\\
Quebrando"-lhe com golpe repentino\\
Ambas as canas do direito braço.\\
Rangem os ossos e a terrível maça\\
Caindo sobre a terra ao longe soa.\\
Torna a juntar"-se a fugitiva plebe,\\
E o prudente Gonçalo que deseja\\
Mostrar o seu valor noutros perigos,\\
Finge"-se morto: a turba irada o pisa,\\
Mas ele não se move. Contra todos\\
Então Gaspar em cólera se acende:\\
Ameaça, derriba, ataca e fere;\\
Até que já sem forças, rodeado\\
Vê de seus companheiros os opróbrios. \\[10pt]


Soa nas costas dos heróis valentes\\
O duro azambujeiro e são levados\\
Ao som terrível de insultantes gritos\\
Para a escura prisão que os esperava.\\
Gonçalo, o bom Gonçalo as mãos atadas,\\
Os olhos para o chão, porque era terno\\
Não refreou o compassivo pranto.\\
A par dele Bertoldo em vão lamenta\\
A falta de respeito que devia\\
Rústica plebe ao neto de Alarico\footnote{ Alarico, Rei dos Godos, que
alcançou muitas vitórias contra os Romanos no tempo de Honório.}\\
Com vagaroso passo todos marcham,\\
Como as ovelhas por caminho estreito.\\
Tal depois da ruína de um Quilombo\footnote{ Fortificação de escravos
rebelados, que muitas vezes se fazem temidos pelas suas hostilidades.}\\
Vem a indômita plebe da Etiópia,\\
Quando rico dos louros da vitória\\			\index{\Lour}
O velho Chagas sempre valeroso\footnote{ Este famoso Índio foi dos que
mais se assinalaram nas ocasiões de ataques contra os escravos.}\\
Cobre o fuzil da pele da Guariba,\footnote{ Guariba, espécie de mono, 
cuja pele serve aos viajantes dos sertões para livrar o fuzil da umidade, e
costumam estes homens forrar"-se com a pele dos animais que matam. Pode ver"-se
M. Buffon no tom. \versal{iv}, edic. de 4 vol., p. 378. Lin., \textit{Sys. nat. anim},
ed. 10, tom. \versal{i}, p. 26, \textit{Paniscus}, \textit{Marcgrave}, 226.}\\		\index{\Marc}
E forra o largo peito c'os despojos\\
Da malhada Pantera\footnote{ Lin. \textit{Sys. nat. anim.}, ed.~10, p.~41. \textit{Pardus}.} e do escamoso\\
Jacaré\footnote{ Crocodilo brasiliense. \textit{Marcgrave}, 242. Lin.		
\textit{Sys. nat.}, p. 200,  \textit{Crocodilus}.} nadador, que infesta as
águas. \\[10pt] 	\index{\Marc}  %para sobre a edição}

\end{verse}

\endgroup

\pagebreak
\thispagestyle{empty}

\movetoevenpage
\mbox{}\vfill
\thispagestyle{empty}
\noindent Argumento do Canto \versal{iv}
\medskip

{\footnotesize\noindent Episódios na prisão onde o velho Amaro é carcereiro -- Tibúrcio faz passar"-se por um
monge eremita que se hospeda com Amaro -- Embuste sobre Doroteia: primeiro
Tibúrcio, fingindo"-se monge, e depois Marcela, uma vidente, lançam o falso
presságio de que Doroteia é a prometida de Gonçalo, conforme combinado com todos,
inclusive com o herói -- Como parte do embuste, Gonçalo deixa palavras apaixonadas num papel 
-- Esperanças de Doroteia sobre o bem vindo esposo -- Trazendo vinho e presunto aos
presos, a moça leva a resposta que Gonçalo e companhia esperavam: promete soltá"-los
na calada da noite com as chaves do pai -- Sonho  de Gonçalo na cadeia: a Verdade,
a Justiça e a Paz se apresentam alegorizadas lado a lado -- No sonho, louvam"-se os
progressos da Reforma da Universidade -- Mesmo sentindo o suave efeito da Verdade, 
Gonçalo denega do seu apelo e de sua advertência -- Doroteia, enganada pelos companheiros
e pela vidente, planeja libertar os prisioneiros -- Ao fazê"-lo, tropeça e acorda o pai,
mas Tibúrcio improvisa um fantasma com o lençol do carcereiro, fingindo ser seu pai
que o vinha buscar do outro mundo -- Enquanto o velho treme de medo, Doroteia liberta os companheiros
e o presumido esposo a quem se entrega, fazendo sofrer Rufino, enganado pela Ignorância,
e também Cosme, enganado pelo Amor -- Fuga dos amigos pela floresta --  Combate entre os amigos por		\index{\Amor}
conta de Doroteia, que notara tarde o embuste que sofrera e pelo qual tivera culpa -- Doroteia
tenta matar Gonçalo com a espada do herói -- Fim da luta, com a imobilização violenta de Doroteia
por três dos companheiros.}


\chapter{Canto \versal{iv}}
%\setlinenum{0}

\begingroup
\linenumbers

\begin{verse}

Tibúrcio, que nas guerras da estalagem\\
Soube abrandar os inimigos peitos,\\
Pondo"-se como em êxtase profundo\\
Com os olhos no Céu e as mãos no peito,\\
Vem a empenhar a força das intrigas.\\
Que não farás, intrépida Ignorância,\\			\index{\Ignor}
Por libertar os tristes prisioneiros! \\[10pt]


Tem o cuidado das ferradas portas,\\
Amaro, vigilante inexorável;\\
Mas crédulo e medroso; e tem ouvido\\
Não sem horror pela calada noite\\
Grasnar nos ares e mugir nos campos\\
Feias bruxas e vagos lobisomens.\\
Com ele o Antiquário se acredita\\
Por um devoto e santo Anacoreta,\\		\index{\Anac}
Que passa os breves dias deste mundo\\
Entre os rigores duma austera vida.\\
Amaro, que se fia de aparências,\\
Para nutrir o frágil penitente\\
Vai degolando os patos e as galinhas.\\
Entanto (quem dissera!) a própria filha\\
Inocente era o móvel deste enredo,\\
Seu nome é Doroteia e no semblante\\
Gênio se lhe descobre inquieto e leve.\\
E como estes momentos preciosos\\
Não se devem perder, depois que a fome\\
Afugentou do estômago vazio,\\
Com branda voz em tom de profecia,\\
Humildade afetando assim começa. \\[10pt]


``Pois tanta caridade usais comigo\\
O Senhor, que reparte os seus tesouros,\\ 
Vos encherá de mil prosperidades.\\
A vossa filha\ldots{} mas convém que eu cale\\
Os segredos que o Céu me comunica.\\
Inda vereis nascer, entre riquezas,\\
Os venturosos netos, doce arrimo\\
Aos fracos dias da caduca idade.''\\
O velho então co'as lágrimas nos olhos\\
Assim falou: ``ó filho Abençoado,\\
Que pela débil voz já me pareces\\
Habitador do Céu, quanto consolas\\
As pecadoras cãs que te estão vendo!\\
Assim talvez seria o meu Leandro,\\
Se as bexigas em flor o não roubassem!\\	\index{\Bexig}
Dez anos tinha, quando a morte avara\\
Cortou co'a dura mão seus tenros dias.''\\
Então suspira e segue passo a passo\\
A longa enfermidade; e enquanto narra,\\
Aparece Marcela, conhecida\\
Entre todas as velhas por mais sábia\\
Em penetrar, olhando para os dedos,\footnote{ Esta superstição tem tido
grande uso, vulgarmente \textit{dizer a buena dicha}.}\\
Tudo quanto já dantes lhe contaram.\\
Sobre o pequeno pau a que se encosta,\\
Ela vem debruçada pouco a pouco,\\
O semblante enrugado, os olhos fundos,\\
Contra o nariz oposta a barba aguda:\\
Os dois últimos dentes balanceiam\\
Com pestífero alento, que respira.\\
Em segredo lhe mostra Doroteia\\
A esquerda mão por que ela decifrasse\\
As confusas palavras de Tibúrcio.\\
Ela observa e, depois de mil trejeitos,\\
Franzindo a testa, arcando as sobrancelhas,\\
Com voz trêmula e fraca assim dizia. \\[10pt]


``Ó que grande ventura o Céu te guarda!\\
Por esposo terás um cavalheiro\\
Que te ama e te deseja. Mas ai triste!\\
Em vão chora infeliz o terno amante\\
Nesta escura prisão desconhecido\\
Por casos de Fortuna. Criai filhos,\\			\index{\Fortu}
Ó desgraçadas mães, para que um dia\\
Longe de vós padeçam mil trabalhos!''\\
Aqui suspira a boa velha e chora.\\
Duas vezes começa e depois fala.\\
``O seu nome é Gonçalo: é rico e nobre,\\
E mancebo gentil, robusto e louro.''\\
Estas e outras palavras lhe dizia,\\
E Doroteia já se sente amante,\\
Excogitando os mais seguros meios\\
De abrir a porta e dar"-lhe a liberdade.\\
Na molesta prisão, o novo engano,\\
De imperceptível arte pronto efeito,\\
Sabe o Herói e assim consigo fala.\\
``Ó amigo tão raro como a Fênix,\\
Que podendo deixar"-me entre estes ferros,\\
Vens encher"-me de alívios e esperanças!''\\
Valentes expressões em crespa frase,\\
Que ao \textit{Alívio de tristes}\footnote{ Romance vulgar.} rouba a glória,\\		\index{\Romvulg}
Pensando, felizmente ressuscita\\
Aquelas hiperbólicas finezas,\\
Que em seus escritos prodigou Gerardo,\footnote{ Gerardo de Escobar fez
uma obra que intitulou \textit{Cristais d'alma}, cheia de ridículas
hipérboles.}\\ %\textbf{Inserir aqui uma nota sobre este uso do vitupério.}. 
Num pequeno papel como convinha\\
A triste e desgraçado prisioneiro,\\
Viu Doroteia as letras amorosas,\\
Que os ditos confirmaram de Marcela;\\
E dois grandes presuntos, que jaziam\\
Intactos na despensa do bom velho,\\
Vão levar a resposta, acompanhados\\
Do roxo néctar, que dissipa os males.\\
Mensageira fiel, então afirma,\\
Que virá Doroteia abrir"-lhe as portas\\
Nas horas em que o plácido sossego\\
Dos cansados mortais os olhos cerra.\\
Gonçalo espera tímido e confuso,\\
Vem"-lhe à memória o seu antigo afeto;\\
Qual leve sombra, escuta, arde e deseja\\
Sentir no coração novas cadeias. \\[10pt]


Já com a fria mão a noite escura\\
Entre o miúdo orvalho derramava\\
Papoilas soporíferas, que inspiram\\
O brando sono e o doce esquecimento.\\
Reina o vago silêncio que acompanha\\
De amor furtivo os trágicos transportes.\\
Gonçalo então, cansada a fantasia\\
Sobre os meios e os fins de seus projetos,\\ %FANTASIA, MEIOS E FINS
Pouco a pouco se esquece, e pouco a pouco\\
Cerra os olhos, boceja, dorme e sonha.\\
Quando voa do leito, onde deixava\\
Nos braços do Descanso ao Pai da Pátria,\\ \index{\Paida} %\Dom José I	
A brilhante Verdade, e lhe aparece\\		%esse "e lhe" soa estranho
Numa nuvem azul bordada d'ouro.\\
A Deusa ocupa o meio, um lado, e outro\\
A severa Justiça, a Paz ditosa. \\[10pt]


``Benignos Céus, enchei meus puros votos:\\
Fazei que esta celeste companhia,\\
Como do terno Avô rodeia o trono,\footnote{ O Augusto e Fidelíssimo Rei de Portugal.}\\		\index{\Aug}
De seu Neto imortal orne a Coroa.''\footnote{ O Sereníssimo príncipe
herdeiro.}\\[10pt] 		\index{\Netoi}


Gonçalo viu e pondo as mãos nos olhos\\
Receia e teme de encarar as luzes.\\
``Abre os olhos, mortal,'' (assim lhe fala\\
Do claro Céu a preciosa filha)\\
``Abre os olhos, verás como se eleva\\
Do meu nascente Império, a nova glória.\footnote{ A Universidade de Coimbra novamente criada.}\\
Esses muros, que a pérfida Ignorância\\			\index{\Ignor}
Infamou temerária com seus erros,\\
Cobertos hão de ser em poucos dias\\
Com eternos sinais de meus triunfos.\\
Eu sou quem de intrincados Labirintos\footnote{ A Filosofia Racional
sem os enredos dos silogismos Peripatéticos.}\\	\index{\Perip}
Pôs em salvo a Razão ilesa e pura.\\		\index{\Fisic}
Eu abri aos mortais os 
meus tesouros:\footnote{ A física.}\\	\index{\Fisic}
Fiz chegar aos seus olhos quanto esconde\footnote{ A história natural.}\\		\index{\Fisic}
No seio imenso a fértil Natureza.\\
Pode uma destra mão por mim guiada\\
Descrever o caminho dos Planetas;\\
O mar descobre as causas do seu fluxo;\\
A Terra\ldots{} mas que digo? Que ciência\\
Não fiz tornar às margens do Mondego,\\		\index{\Monde}
Ou dentre os braços da Latina Gente,\footnote{ Os ótimos e famosos
Professores, que El Rei Fidelíssimo atraiu de diversas partes da Europa.}\\
Ou dos belos países, cujas praias\\
O mar azul por toda a parte lava?\\
Se são firmes por mim o Estado, a Igreja,\\
Se é no seio da paz feliz o Povo,\\
Dizei"-o vós, ó Ninfas do Parnaso,\\
Ilustres, imortais, vós que ditastes\\
As poderosas leis a vez primeira,\\
Vós que ouvistes da lira de Mercúrio\\
Os úteis meios de alongar a vida.\\
Eu vejo renascer um Povo ilustre\\
Nas armas e nas letras respeitado.\\
O seu nome vai já de boca em boca\\
A tocar os limites do Universo.\\
O pacífico Rei lhe traz os dias\\
Dignos de Manuel,\footnote{ O senhor rei D.~Manuel, chamado o Feliz.}
dignos de Augusto.\\			\index{\Aug}
E tu enquanto a Pátria se levanta,\\
Sacudindo os vestidos empoados\\
Co'a cinza vil dum ócio entorpecido,\\
Enquanto corre a mocidade alegre\\
A colher louros ávidos de glória,\\			\index{\Lour}
Serás o frouxo, o estúpido, o insensível?\\
Sacrificas o nome, a honra, a Pátria\\
Aos moles dias de uma vida escura?\\
Cego, errado mortal, vê que te enganas.''\\
Disse: e cerrada a nuvem luminosa,\\
Estremece Gonçalo; foge o sono;\\
Por toda a parte lança incerto a vista,\\
Busca assustado, mas já nada encontra.\\
As mesmas impressões em seus sentidos\\
Vivas imagens pintam e não sabe\\
Se então dormia, ou se inda agora sonha.\\
Sente a suave força da Verdade;\\
Mas recusa abraçá"-la. Triste sorte\\
D'alma infeliz que ao erro se acostuma! \\[10pt]


Entanto sem receio o Velho dorme,\\
E a filha vem as sombras apalpando\\
Com as chaves na mão; e quantas vezes\\
Segue, vacila e para, e lhe parece\\
Ouvir a voz do Pai; escuta e treme;\\
Move os passos, tropeça e ao ruído\\
Acorda Amaro e grita. Ela se apressa,\\
E torna a tropeçar. Aqui Tibúrcio\\
Em casos repentinos pronto e destro\\
Em um lençol se embrulha e corre ao leito\\
Onde jazia o Velho espavorido,\\
Que cuida que vê bruxas e fantasmas:\\
Então lhe diz em tom medonho: ``Ó filho,\\
Ingrato filho, que dum Pai te esqueces!\\
Que mal, que mal cumpriste os meus legados!\\
Hoje comigo irás\ldots{}'' Ao Velho o medo\\
Corre as medulas dos cansados ossos;\\
A voz lhe falta, eriça"-se o cabelo.\\
Entanto as portas Doroteia abrindo\\
(Amor a fez intrépida) abraçava\\		\index{\Amor}
O prometido esposo; ele se apressa,\\
Acorda os miserandos companheiros,\\
Que se alegram, deixando solitárias\\
As vagas sombras da prisão funesta.\\
Passa o resto da noite entre temores\\
Amaro, quanto pode o prejuízo! \\[10pt]


Apenas matizava a branca aurora\\
Da Tíria cor o véu açafroado,\\ 		\index{\Acaf}	\index{\Tiria}
Quando o Velho ao través da luz escassa\\
Viu abertas as portas. ``Doroteia,\\
Doroteia onde estás?'' Assim clamava,\\
E entregue à sua dor consulta os olhos\\
Do profeta que pronto a pôr"-se em marcha\\
Com rosto de candura e de inocência\\
Brandamente o consola. ``O Céu, Amigo,\\
Tudo faz por melhor e muitas vezes\\
Com trabalhos cruéis aos bons aflige.''\\
Disse e deixando ao Pai desconsolado,\\
Caminha na esperança de encontrar"-se\\
C'o valente esquadrão dos fugitivos.\\
O Sol já com seus raios luminosos\\
Tinha roubado às folhas dos arbustos\\
O frio gelo do noturno orvalho.\\
Eis à sombra de fúnebre arvoredo\\
Rufino, o melancólico, chorando.\\		\index{\Melanc}
``Quem és, que em tua mágoa inconsolável,\\
Pareces abalar estas montanhas?''\\
Compassivo, pergunta o Antiquário.\\
E depois de chorar por largo tempo,\\
Estas vozes o triste lhe tornava.\\
``Eu sou aquele amante sem ventura,\\
Sempre extremoso e sempre escarnecido,\\
Sofredor das ingratas esquivanças\\
Que vi (ai dura vista!) face a face\\
Do tardo Desengano o feio rosto.\\
Ah Doroteia, um sonho lisonjeiro\\
Meus dias dilatou para que agora\\
Te visse em outros braços insultando\\
O meu fiel amor? Ó noite infausta,\\
Noite terrível, noite acerba e dura!\\
Quanto eu fora feliz se a tua sombra\\
Eternamente os olhos me cobrisse!'' \\[10pt]


Tibúrcio, que já tudo penetrava,\\
Do caminho se informa e dos lugares,\\
Por onde fora a incerta companhia,\\
Que em tanto risco o seu conselho espera. \\[10pt]


Não distante se eleva antigo bosque\\
Horroroso por fama: já nos tempos,\\
Em que torrente Bárbara saindo\footnote{ A irrupção dos bárbaros foi no século \versal{V}.}\\
Do seio da Meótis inundava\\                   %\Meotis
As províncias de Europa, aqui se via\\
Arruinado Templo. Os vivedouros\\
Ciprestes se levantam sobre os pinhos;\\
Heras e madressilvas enlaçadas\\
Ali fazem curvar a crespa rama\\
Dos velhos e infrutíferos carrascos.\\
Três fontes misturando as puras águas\\
Mansamente se envolvem e oferecem\\
À vista cobiçosa os alvos seixos\\
E os verdes limos que no fundo nascem.\\
Os amigos fiéis aqui se encontram.\\
Qual, em noite funesta e pavorosa,\\
Perdido caminhante que receia\\
Achar em cada passo um precipício,\\
Se acaso a dúbia luz divisa ao longe\\
A esperança renasce e de alegria\\
Sente pular o coração no peito:\\
Assim o Desertor, constante e forte,\\
Ao ver o companheiro que prudente\\
Sabe evitar e prevenir os males.\\
Eles se reconhecem e derramam\\
De alegria e ternura o doce pranto.\\
Ó vínculos do sangue e da amizade!\\
Menos unidos viu o Lácio antigo\\
Aos dois Troianos, que uma cega noite,\footnote{ Niso e Euríalo. \textit{Virg}.}\\		\index{\Niso}
Espalhando o terror no campo adverso,\\
Levou às turvas margens de Aqueronte.\\		\index{\Aque}
Gonçalo se retira pelo bosque;\\
Com ele vai Tibúrcio e mil projetos\\
Formavam sobre o fim da grande empresa;\\
E a muito fácil e infeliz donzela\\
Do seu profeta o rosto e a voz conhece,\\
E pensa e teme de se achar culpada. \\[10pt]


Então o Amor, que na sonora aljava\\		\index{\Amor}
Esconde setas de mortal veneno,\\
E setas doutro ardor mais grato e puro,\\
Fazia escolha das terríveis armas,\\
Para vingar"-se da cruel Marfisa:\\			\index{\Marfi}
Marfisa ingrata, pérfida, inconstante,\\	\index{\Marfi}
Peito de bronze, a quem a natureza\\
Não formou para ternos sentimentos.\\
E por ver se os seus tiros correspondem\\
Sempre fiéis à mão e ao desejo,\\
Faz no teu peito, ó Doroteia, o alvo,\\
As forças prova e a destreza ensaia.\\
Encurva o arco ebúrneo, solta e voa\\
Sequiosa de sangue a ponta aguda\\
Tinta no Averno. Ao golpe inevitável\\
Tremeu o coração e um vivo lume\\
Nos olhos aparece: do seu braço\\
Admira a força Amor. Vai outra seta\\		\index{\Amor}
Ao brando peito incauto e descoberto\\
Do mancebo infeliz. A vez primeira\\
Soube de amor o namorado Cosme.\\
Que violenta paixão pode encobrir"-se!\\
Os olhos falam: seguem as palavras;\\
E depois o delírio. O tempo é surdo\\
Aos votos dos amantes. Eles viam\\
Crescer ditoso em rápidos momentos\\
De uma nova esperança o belo fruto;\\
Mas Gonçalo a favor dos arvoredos\\
Oculto chega, para e ceva as iras.\\
Tal pode ver"-se o rápido Jaguará\footnote{ Marcgrave \textit{Hist. Brasil.}, p. 235.}\\ \index{\Marc}
Do fértil Ingaí\footnote{ Rio da América, nas Minas do Rio das Mortes.}
nos vastos campos,\\
Se tem defronte o cervo temeroso.\\
Encolhe"-se torcendo a hirsuta cauda,\\
Tenta, vigia, espera e lambe os beiços\\
Formando o salto sobre a incauta presa.\\
Cegos amantes, aprendei agora\\
Os perigos da nímia confiança.\\
O zeloso Gonçalo investe, acodem\\
Os companheiros duma e doutra parte.\\
Triste ruído! pedras contra pedras\\
Ali se despedaçam; ao seu lado\\
Acha Cosme a Rodrigo, acha a Bertoldo.\\
Enquanto dura o férvido combate,\\
Doroteia, que vê sem uso a espada,\\
De que o Herói em fúria se não lembra,\\
(Que não farás, Amor, tu que transformas\\		\index{\Amor}
Uma donzela num feroz guerreiro!)\\
Desembainha; a Morte insaciável\\
Lhe afia o gume e o furor sanguíneo\\
Ergue e dirige o ferro; já pendente\\
Sobre Gonçalo o golpe, salta e chega\\
O amigo a tempo de salvar"-lhe a vida,\\
Pelos braços a aperta e neles grava\\
Roxos sinais dos dedos. Em derrota,\\
Correm os três e o campo desamparam.\\
O mísero, infeliz e novo amante\\
As negras fúrias levam que despertam\\
No aflito coração desesperado\\
Ciúme, raiva, amor, ódio e vingança.\\
Assim o invicto domador dos monstros,\footnote{ Hércules, que recebeu
de Dejanira o vestido tinto no sangue do centauro Nesso, e agitado das Fúrias se
lançou no fogo.}\\
Quando por mão da crédula consorte\\
Recebeu o vestido envenenado\\
No sangue infausto do biforme Nesso,\\
Os rochedos e os montes abalava\\ 
Soaram os seus fúnebres gemidos\\
Por longo tempo nas Ismárias grutas.\footnote{ Ismaro, monte de Trácia.}\\
Valentes e indiscretos vencedores\\
Tarde conhecereis e muito tarde,\\
Que um amigo ultrajado é perigoso. \\[10pt]


Para soltar os oprimidos braços\\
Doroteia se empenha; mas Tibúrcio,\\
Lançando a esquerda mão à ruiva trança,\\
A fez voltar, torcendo"-lhe o pescoço,\\
Ao claro Céu a vista ameaçante.\\
Gaspar o ferro dentre as mãos lhe arranca;\\
Este um braço sustenta, outro Gonçalo,\\
E ela presa e sem forças grita e geme.\\
Não doutra sorte o touro da Chamusca,\footnote{ Todos sabem que desta
vila são bravíssimos os touros.}\\
Quando três cães o cercam atrevidos,\\
Dois pendem das orelhas e um da cauda.\\
A cornígera testa em vão sacode:\\
Contra a terra se arroja a um lado e outro,\\
E depois que não pode defender"-se,\\
Mugindo exala a indômita fereza. \\[10pt]
\end{verse}

\endgroup

\pagebreak
\thispagestyle{empty}

\movetoevenpage
\mbox{}\vfill
\thispagestyle{empty}
\noindent Argumento do Canto \versal{V}
\medskip

{\footnotesize\noindent Conselho dos heróis sobre o destino de Doroteia prisioneira --
Sofrimento de Doroteia -- Os efeitos da imprudência no Amor			\index{\Amor}
que leva além dos limites da razão -- Para que Rufino parasse de
queixar"-se da Fortuna, o Acaso, filho dela, conduz o amante não					\index{\Acaso}		\index{\Fortu}
correspondido aonde Doroteia ficara amarrada -- Chegada dos companheiros a Mioselha --
Por ciúmes de Gonçalo, Cosme vai ao tio contar as causas da derrota nos estudos 
-- Gaspar determina que os amigos se abriguem na casa de sua mãe --
Jantam levemente -- Tibúrcio chora de fome e lamenta o tempo em
que tivera uma ocupação como tesoureiro de uma irmandade -- Descrição da biblioteca
do tio de Gaspar, que se gabava de livros que para o estilo da Universidade restaurada
eram já obras de mau gosto -- Amaro segue o grupo que lhe roubara a filha -- O povo
cerca a casa -- A Ignorância fala pela boca de Gonçalo, exortando os amigos
a mais uma vez lutar contra a multidão -- Gonçalo precipita um vaso grande
do alto do sobrado sobre a multidão, fazendo muitos feridos -- Começa assim a penúltima
pragmatografia de guerra, descrição de homens em ação bélica -- Os desertores
fogem por um postigo em que Tibúrcio fica entalado -- Encontro de Gonçalo com o tio,
cujas iras se anunciaram no primeiro canto do poema -- O tio apela para que Gonçalo
volte aos estudos -- Gonçalo denega -- Pragmatografia final da surra de pau que Gonçalo
leva do tio -- A Ignorância goza o Império que ainda tem sobre aqueles espíritos fracos
condenados ao ostracismo na província -- Peroração: a voz poética do poema
heroi"-cômico pede que as mesmas mãos e cetro que a expulsaram de Coimbra confinem
o monstro da Ignorância definitivamente nas montanhas mais ermas onde o céu
não cansa de lançar raios. }


\chapter{Canto \versal{V}} 
%\setlinenum{0}

\begingroup
\linenumbers

\begin{verse}

Alto conselho aqui se faz, aonde,\\
Infeliz Doroteia, o teu destino\\
Cruel e dúbio dum só voto pende.\\
Dos três heróis discordam as sentenças.\\
Um deseja que fique em liberdade\\
E do Pai ultrajado exposta às iras.\\
Inexorável outro pensa e julga\\
Que a sua morte deve dar exemplo,\\
Que encha d'horror as pérfidas amantes.\\
Gonçalo, que era o único ofendido,\\
Consulta o coração e se enternece.\\
Mas o ardente Ciúme, que se alegra\\
De pintar como crimes horrorosos,\\
Inocentes ações, então lhe mostra\\
A feia Ingratidão e o torpe Engano.\\
A Vingança cruel e o vil Desprezo,\\
Ainda mais terrível que a Vingança,\\
Ganham do coração ambas as portas.\\
Mimosa Doroteia, e como ficas\\
Co'as mãos ligadas a um pinheiro bronco\\
Sem outra companhia que os teus males!\\
É este o prêmio, filhas namoradas,\\
Este o prêmio de Amor, quando imprudente\\		\index{\Amor}
Os termos passa que a razão prescreve.\\
De quando em quando um ai do peito arranca,\\
Que ao longe os tristes magoados Ecos\\
Desperta e faz sentir os duros troncos.\\
E espera sem defesa (sorte ingrata!)\\
Que a devorem os lobos carniceiros.\\
Assim ligada aos ásperos rochedos\\
A filha de Cefeu\footnote{ Andrômeda foi exposta a um monstro marinho.
Ovi, \textit{Metamorphose.}} ao mar lançava\\
A temerosa vista e lhe parece\\
A cada instante ver surgir das ondas\\
A verde espalda do marinho monstro. \\[10pt]


Sem esposo, sem pai, sem liberdade,\\
Mísera Doroteia chora e geme.\\
``Ai, Marcela cruel, que me enganaste\\
Com teus belos, fantásticos agouros!\\
Queira o Céu que outras lágrimas sem fruto\\
Mil vezes tresdobradas te consumam\\
Os encovados olhos! Que inda a Morte\\
Às tuas vozes surda correr deixe\\
Piorando em seu curso vagaroso\\
Os momentos de dor e de amargura!'' \\[10pt]


Assim falava. A leve Fantasia\\
Com as cores mais vivas lhe apresenta,\\
De escarpados rochedos no alto cume,\\
O palácio da cândida Inocência,\\
Cercado de funestos precipícios.\\
Ó morada feliz, onde não torna\\
Quem uma vez rodou entre as ruínas!\\
Giram no plano do elevado monte\\
Cruas dores, remorsos devorantes,\\
As três Irmãs, a Peste, a Fome, a Guerra,\\
O pálido Receio, o Crime, a Morte,\\
As Fúrias e as Harpias, que se envolvem\\
No turbilhão dos míseros cuidados. \\[10pt]


Então, de tantas lágrimas movida\\
A mãe soberba do propício Acaso,\\      \index{\Acaso}
A mudável Fortuna e já cansada\\           \index{\Fortu}
De ouvir as tristes queixas de Rufino,\\
Tais palavras ao filho dirigia. \\[10pt]


``Esse amante infeliz que em vão suspira,\\
Ache a dita uma vez e enxugue o pranto.''\\
Acaba de falar e ao mesmo tempo\\
Rufino para o bosque se encaminha,\\
E o Acaso o conduz por entre as sombras\\		\index{\Acaso}
Da pavorosa Noite que já desce.\\
À rouca voz da mísera donzela\\
Palpita o coração: o Amor e o Susto\\			\index{\Amor}
Quiméricas imagens lhe afiguram;\\
Mas ele chega: o próprio crime e o pejo\\
Cobrem de roxas nuvens o semblante\\
De Doroteia ao ver"-se ainda amada\\
Por aquele que foi há poucas horas\\
Alvo de seus insultos e desprezos.\\
A mole vista, as lágrimas em fio,\\
Que aos corações indômitos abrandam,\\
Que fariam num peito namorado?\\
Tu lhe ensinas c'o fraco rendimento\\
Os meios de vencer. Ó sete vezes\\
Venturoso Rufino, se ela um dia\\
Não quiser renovar os seus triunfos\\
E medir a fraqueza do teu peito\\
Pelo grande poder das suas armas! \\[10pt]


Depois de longa e trabalhosa marcha,\\
Cansado de sofrer enfim respira\\
O Desertor e mostra aos companheiros\\
Os conhecidos montes. Fuma ao longe\\
A fértil Mioselha e pouco a pouco\\
Os oiteiros e as casas aparecem. \\[10pt]


Tibúrcio, que uma antiga e voraz fome\\
Sofreu nestes aspérrimos trabalhos,\\
Com gosto espera de afogá"-la em vinho,\\
E já se apressa alegre e transportado.\\
Qual o novilho que perdeu nos bosques\\
A doce vista do rebanho amigo,\\
E depois devagar a noite e o dia\\
Por vales sem caminho a Mãe conhece,\\
Alegre salta e berra e por momentos\\
Espera umedecer entre carícias\\
C'o leite represado a boca ardente. \\[10pt]


Mas Cosme, que conserva na memória\\
As passadas injúrias, por vingar"-se,\\
Ao Tio de Gonçalo, narra as causas\\
Da funesta derrota. Determina\\
Gaspar que os fatigados companheiros\\
Achem na própria casa um doce abrigo.\\
De os ver a Mãe se aflige; mas espera\\
Que obrigados da fome se retirem.\\
Leve foi o Jantar, mais leve a Ceia,\\
E Tibúrcio com pena assim chorava\\
Os dias, em que fora Tesoureiro\\
Duma rica e devota Confraria.\\
``Ó santa Ocupação, tu nunca viste\\
A magra mão da pálida Miséria,\\
Que os fracos membros do mendigo apalpa.\\
Sem trabalho em teus próvidos Celeiros\\
A ditosa Abundância se recolhe.\\
Se torno a possuir"-te, quantas vezes\\
Dos cuidados tenazes e importunos\\
Lavarás a minha alma nas perenes\\
Purpúreas fontes do espremido cacho!'' \\[10pt]


Mostra Gaspar vaidoso a livraria,\\
Donde o Tio Doutor sermões tirava.\\
Mau Gosto, que à razão não dás ouvidos,\\
Vem numerar as obras, que ditaste:\\
Seja a última vez e eu te asseguro\\
Que não vejas fumar nos teus altares\\
Do Gênio Português jamais o incenso. \\[10pt]


Geme infeliz a carunchosa Estante\\
C'o peso de indulgentes \textit{Casuístas},\footnote{ Pode ver"-se o
que deles diz Concina, \textit{Appar. ad Theol. christ.}, c.~\versal{iv}, cap. 5.}\\		\index{\Casui}
\textit{Dianas, Bonacinas, Tamburinos,}\\
\textit{Moias, Sanches, Molinas e Lagarras}.\\ %\footnote
Criminosa Moral, que em surdo ataque\\
Fez nos muros da Igreja horrível brecha,\\
Moral, que tudo encerra e tudo inspira,\\
Menos o puro amor que a Deus se deve.\\
Aparecei famosa \textit{Academia}\\
\textit{De humildes e ignorantes}, \textit{Eva e ave},\\
\textit{Báculo pastoral}, e \textit{Flos sanctorum},\\
E vós, ó \textit{Teoremas predicáveis},\footnote{ Coleção de sermões.}\\
Não tomeis o lugar, que é bem devido\\
Ao \textit{Kess}, ao \textit{Bem Ferreira}, ao \textit{Baldo}, ao \textit{Pegas},\\
Grão"-Mestre de forenses subterfúgios.\\ %\footnote
Aqui Tibúrcio vê o amado \textit{Aranha},\\
O \textit{Reis}, o bom \textit{Supico} e os dois
\textit{Suares}:\footnote{ Lusitano e Granatense.}\\ %footnte
Dum lado o \textit{Sol nascido no Ocidente},\\
E a \textit{Mística Cidade}, doutro lado,\\
Cedem ao pó e à roedora traça.\\
Por cima o \textit{Lavatório da consciência},\\
\textit{Peregrino da América}, os \textit{Segredos}\\
\textit{Da Natureza}, a \textit{Fênix renascida,}\\
\textit{Lenitivos da dor} e os \textit{Olhos de água}:\footnote{ Obra
que tem este título --- Fluxo Breve, desengano perene, que o Pégaso da Morte
abriu no monte da contemplação em nove olhos de água para refrescar a alma das
securas do espírito etc. Todas as obras nomeadas neste lugar são conhecidas, e
quando o não fossem bastaria ver os títulos para julgar do seu merecimento, e da
barbaridade do século em que foram escritas. Talvez não sejam estas as mais
extravagantes à vista do \textit{Chrysol Seraphico, da Tuba concionatoria,
Syntagma comparistico, Primavera Sagrada etc}.}\\
Por baixo está de \textit{São Patrício a cova},\\
A \textit{Imperatriz Porcina} e quantos \textit{Autos}\\
A miséria escreveu do Limoeiro\footnote{ A cadeia pública da Corte.}\\
Para entreter os cegos e os rapazes.\\
Rudes montões de Gótica escritura,\\		\index{\Gotic}
Quanto cheirais aos séculos de barro!\\
Falta ainda uma Estante; mas Amaro\\
Seguindo os passos da roubada filha\\
Caminha aflito e de encontrar receia\\
O valente esquadrão que procurava.\\
Tanto a fama das bélicas proezas\\
O seu nome fazia respeitado! \\[10pt]


Que novas desventuras se preparam!\\
O povo cerca da Viúva as portas;\\
Quando a triste Ignorância, que deseja\\			\index{\Ignor}
Arrancar dentre os ásperos perigos\\
Aos seus Heróis, por boca de Gonçalo\\
Começou a falar. ``Se tantas vezes\\
Mais que heroico valor tendes mostrado,\\		\index{\Heroic}
É este o campo, ide a cortar os louros\\		\index{\Lour}
Para cingir a vencedora frente.\\
Não se diga que fostes oprimidos\\
Por fraca e rude plebe; este combate\\
Não se pode evitar: só dois caminhos\\
Em tanto aperto aos olhos se oferecem.\\
Escolhei ou a Índia, ou a Vitória.'' \\[10pt]


Disse, e depois abrindo uma janela,\\
Arroja de improviso sobre o povo\\
De informe barro uma espantosa talha.\\
Seco trovão que faz gemer os Polos\\
Quando vomitam as pesadas nuvens\\
Do oculto seio a negra tempestade,\\
Não causa mais pavor: ao golpe horrendo\\
Muitos feridos, muitos assombrados\\
Mancham de negro pó as mãos e o rosto.\\
Amaro anima aos seus e enquanto voam\\
Contra a janela mil pesados seixos,\\
(Que novo estratagema!) o Antiquário\\
Finge da capa um vulto, que aparece\\
De quando em quando, com que atrai as \qb{}armas,\\
Que hão de servir depois para a defesa. \\[10pt]


Novo furor os corações acende.\\
Qual a grossa saraiva ao sopro horrível\\
Do Bóreas turbulento embravecido\\		\index{\Borea}
As searas derrota, os troncos despe,\\
E o triste lavrador contempla e chora\\
A perdida esperança de seus frutos:\\
Assim de pedras vaga e densa nuvem\\
Sai da janela a devastar o campo:\\
As que arroja o Herói já se distinguem\\
Pelo som entre as mais, já pelo estrago.\\
A confusão e o susto ao mesmo instante\\
Pelo povo se espalha: então Gonçalo\\
Valeroso saiu por um postigo;\\
Depois Gaspar; o intrépido Tibúrcio,\\
Metendo o braço e a cabeça, clama\\
Que o não deixem ficar naquele estado.\\
O Herói as mãos firmando na orelha\\
Ainda mais o aperta e deixa exposto\\
Da plebe ao riso, à cólera de Amaro.\\
Quantas vezes Tibúrcio desejaste\\
Não ser de grosso peito e largo ventre! \\[10pt]


O Desertor enfim cansado chega\\
À presença do Tio formidável,\\
E a teimosa Ignorância, que se aferra\\			\index{\Ignor}
E que afirma somente porque afirma,\\
O coração de novo lhe endurece.\\
A sofrer o trabalho dos estudos\\
O Tio o anima e roga e ameaça,\\
Mas o Herói inflexível só responde,\\
Que não há de mudar do seu projeto.\\
Não é mais firme a carrancuda roca,\\
Com que Sintra\footnote{ Serra, que acaba na foz do Tejo com nome do
cabo da Roca.} soberba enfreia os mares;\\
Nem tu, ó Pão de Açúcar,\footnote{ Grande rochedo na barra da baía do
Rio de Janeiro.} namorado\\
Da formosa Cidade, Velho e forte,\\
Que dás repouso às nuvens e te avanças\\
Por defendê"-la do furor das ondas. \\[10pt]


Então falando o Tio em torpes crimes,\\
E em furtadas Donzelas, ergue irado\\
Co'a mão inda robusta o pau grosseiro,\\
E a paixão desabafa: a longa idade\\
Proíbe"-lhe o correr; mas não proíbe\\
Que o pau com força ao longe o acompanhe.\\
Ai, Gonçalo infeliz, que dura estrela\\
Maligna cintilou quando nasceste!\\
Depois de mil trabalhos insofríveis,\\
Onde o gosto esperavas e o sossego,\\
Viste nascer estragos e ruínas.\\
Assim depois dos últimos combates,\\
Que as margens do Escamandro \qb{}ensanguentaram,\\
O Rei potente\footnote{ Agamenon, que voltando do cerco de Troia foi
assassinado por Egisto.} d'Argos e Micenas,\\
Esperando abraçar saudoso os Lares,\\
Abraça o ferro duma mão traidora.\\
Fechadas tem o experto Tio as portas:\\
Volta Gonçalo, encontra novos golpes\\
E jaz enfim  por terra. Ferve o sangue\\
Da boca e dos ouvidos; sem acordo,\\
Apenas se conhece que inda vive;\\
Mas tem glória de trazer consigo\\
A derrotada estúpida Ignorância.\\			\index{\Ignor}
Ela reina em seu peito e se contenta\\
De ter roubado aos muros de Minerva\\		\index{\Miner}
De fracos Cidadãos o preço inútil. \\[10pt]


Goza, Monstro orgulhoso, o antigo Império\\
Sobre espíritos baixos que te adoram;\\
Enquanto à vista dum Prelado ilustre,\\
Prudente, Pio, Sábio, e Justo e Firme\\ % assim mesma a pontuaçao e as conjunções
Defensor das Ciências que renascem,\\
Puras as águas cristalinas correm\\
A fecundar os aprazíveis campos.\\
Brotam as flores e aparecem frutos.\\
Que hão de encurvar com próprio peso os \qb{}ramos\\
Nos belos dias da estação dourada.\\
Possa a robusta mão, que o Cetro empunha,\\
Lançar"-te num lugar tão desabrido,\\
Que te sejam amáveis os rochedos\footnote{ Os montes Acroceraunos de
Epiro, onde frequentemente caem raios.}\\
Onde os coriscos de contínuo chovem. \\[10pt]
\end{verse}

\endgroup

\pagebreak

\chapter{Soneto}
%\setlinenum{0}

\begingroup
\linenumbers

\begin{verse}


A Terra oprima pórfido, luzente\\		\index{\Porf}
E brilhante metal, que ao Céu erguidos\\
Os altos feitos mostrem esculpidos\\
Do Rei que mais amou a Lusa Gente. \\[10pt]


Esteja aos Régios pés Dragão potente,\\
Que tanto os povos teve espavoridos,\\
C'os tortuosos colos suspendidos\\
No gume cortador da espada ardente. \\[10pt]


Juntas as castas filhas da Memória,\\		\index{\Filmemo}
As brancas asas sobre o Trono abrindo,\\
Assombrem a dourada e muda História. \\[10pt]


Ao Índio livre já cantou Termindo.\\		\index{\Termin}
Que falta, Grande Rei, à tua Glória,\\
Se os louros de Minerva canta Alcindo? \\[10pt]		\index{\Lour}		\index{\Miner}

\vspace*{2em}\versal{E.~G.~P.}
\end{verse}

\endgroup

\pagebreak
\oneside
\chapter{Soneto}
%\setlinenum{0}

\begingroup
\linenumbers

\begin{verse}

Enquanto o Grande Rei co'a mão potente\\
Quebra os grilhões do Erro e da Ignorância,\\			\index{\Ignor}
E enquanto firma, com igual constância,\\
À Ciência imortal, Trono luzente, \\[10pt]


Nova Musa de clima diferente\\
Canta do Pai da Pátria a vigilância,\\		\index{\Paida}
Vingando a Mãe das luzes, da arrogância\\
Com que a despreza o estúpido indolente. \\[10pt]


O Monstro de mil bocas sem sossego,\\
Que a Glória de José vai repetindo\\
Ou sobre a Terra ou sobre o imenso Pego: \\[10pt]		\index{\Pego}


Com ela o nome levará de Alcindo\\
Desde a invejada margem do Mondego\\		\index{\Monde}
Ao pátrio Paraguai, ao Zaire, ao Indo. \\[10pt]		\index{\Indo}	\index{\Parag}	\index{\Zaire}

\vspace*{2em}\versal{L.~F.~C.~S.}
\end{verse}

\twoside

\endgroup

